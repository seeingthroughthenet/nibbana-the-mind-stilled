\chapter{Sermon 13}

\NibbanaOpeningQuote

With the permission of the Most Venerable Great Preceptor and the assembly of the venerable meditative monks. This is the thirteenth sermon in the series of sermons on Nibbāna.

In our last sermon we attempted an exposition under the topic \emph{sabbadhammamūlapariyāya}, ``the basic pattern of behaviour of all mind objects'', which constitutes the theme of the very first sutta of the \emph{Majjhima Nikāya}, namely the \emph{Mūlapariyāyasutta}.

We happened to mention that the discourse describes three different attitudes regarding twenty-four concepts such as earth, water, fire and air. We could however discuss only two of them the other day, namely the world view, or the attitude of the untaught ordinary person, and the attitude of the noble one, who is in higher training.

So today, to begin with, let us bring up the third type of attitude given in the discourse, that is, the attitude of \emph{arahants} and that of the Tathāgata, both being similar. It is described in these words:

\begin{quote}
\emph{Paṭhaviṁ paṭhavito abhijānāti, paṭhaviṁ paṭhavito abhiññāya paṭhaviṁ na maññati, paṭhaviyā na maññati, paṭhavito na maññati, `paṭhaviṁ me'ti na maññati, paṭhaviṁ nābhinandati. Taṁ kissa hetu? `Pariññātaṁ tassā'ti vadāmi}.\footnote{M I 1, \emph{Mūlapariyāyasutta}}

The \emph{arahant} (as well as the Tathāgata) understands through higher knowledge earth as `earth', having understood through higher knowledge earth as `earth', he does not imagine earth to be `earth', he does not imagine `on the earth', he does not imagine `from the earth', he does not imagine `earth is mine', he does not delight in earth. Why is that? I say, it is because it has been well comprehended by him.
\end{quote}

Let us now try to compare and contrast these three attitudes, so that we can understand them in greater detail. The attitude of the untaught ordinary person in regard to any of the twenty-four concepts like earth, water, fire, air (the twenty-four cited being illustrations), is so oriented that he perceives it as such.

For instance in the case of earth, he perceives a real earth, that is, takes it as earth per se. It may sometimes be only a block of ice, but because it is hard to the touch, he grasps it as `earth'. Thus the ordinary person, the worldling, relies only on perception in his pursuit of knowledge. Having perceived earth as `earth', he imagines it to be `earth'. The peculiarity of \emph{maññanā}, or `me'-thinking, is that it is an imagining in terms of `I' and `mine'.

So he first imagines it as `earth', then he imagines `on the earth', `from the earth', `earth is mine' and delights in the earth. Here we find various flexional forms known to grammar.

As a matter of fact, grammar itself is a product of the worldlings for purposes of transaction in ideas bound up with defilements. Its purpose is to enable beings, who are overcome by the personality view, to communicate with their like-minded fellow beings. Grammar, therefore, is something that caters to their needs. As such, it embodies certain misconceptions, some of which have been highlighted in this context.

For instance, \emph{paṭhaviṁ maññati} could be interpreted as an attempt to imagine an earth -- as a full-fledged noun or substantive. It is conceived as something substantial. By \emph{paṭhaviyā maññāti}, ``he imagines `on the earth'\,'', the locative case is implied; while \emph{`paṭhaviṁ me'ti maññati}, ``he imagines `earth is mine'\,'', is an instance of the genitive case, expressing the idea of possession.

Due to such imaginings, a reality is attributed to the concept of `earth' and its existence is taken for granted. In other words, these various forms of imaginings go to confirm the notion already aroused by the concept of `earth'. Once it is confirmed one can delight in it, \emph{paṭhaviṁ abhinandati}. This, then, is the worldview of the untaught ordinary person.

The other day we mentioned that the monk who is in higher training understands through higher knowledge, not through perception, earth as `earth'. Though it is a higher level of understanding, he is not totally free from imaginings. That is why certain peculiar expressions are used in connection with him, such as \emph{paṭaviṁ mā maññi, paṭhaviyā mā maññi, paṭhavito mā maññi, `paṭhaviṁ me'ti mā maññi, paṭhaviṁ mā abhinandi.}

Here we have to call in question the commentarial explanation. According to the commentary, this peculiar expression had to be used as a dilly dally phrase, because the monk in higher training could not be said to imagine or not imagine.\footnote{Ps I 41} But it is clear enough that the particle \emph{mā} in this context is used in its prohibitive sense. \emph{Mā maññi} means ``do not imagine!'', and \emph{mā abhinandi} means ``do not delight!''.

What is significant about the \emph{sekha}, the monk in higher training, is that he is in a stage of voluntary training. In fact, the word \emph{sekha} literally means a `learner'. That is to say, he has obtained a certain degree of higher understanding but has not attained as yet full comprehension.

It is precisely for that reason that the section about him is summed up by the statement:

\enlargethispage{\baselineskip}

\begin{quote}
\emph{Taṁ kissa hetu? Pariññeyyaṁ tassā'ti vadāmi.}

Why is that? Because, I say, that it should be comprehended by him.
\end{quote}

Since he has yet to comprehend it, he is following that course of higher training. The particle \emph{mā} is therefore a pointer to that effect. For example, \emph{mā maññi} ``do not imagine!'', \emph{mā abhinandi} ``do not delight!''.

In other words, the monk in higher training cannot help using the grammatical structure in usage among the worldlings and as his latencies are not extinct as yet, he has to practise a certain amount of restraint. By constant employment of mindfulness and wisdom he makes an attempt to be immune to the influence of the worldling's grammatical structure.

There is a possibility that he would be carried away by the implications of such concepts as earth, water, fire and air, in his communications with the world regarding them. So he strives to proceed towards full comprehension with the help of the higher understanding already won, keeping mindfulness and wisdom before him. That is the voluntary training implied here.

The monk in higher training is called \emph{attagutto}, in the sense that he tries to guard himself.\footnote{A III 6, \emph{Kāmasutta}; see also Dhp 379, \emph{Bhikkhuvagga}} Such phrases like \emph{mā maññi} indicate that voluntary training in guarding himself. Here we had to add something more to the commentarial explanation. So this is the situation with the monk in higher training.

Now as to the \emph{arahant} and the Tathāgata, the world views of both are essentially the same. That is to say, they both have a higher knowledge as well as a full comprehension with regard to the concept of earth, for instance. \emph{Pariññātaṁ tassā'ti vadāmi}, ``I say it has been comprehended by him''.

As such, they are not carried away by the implications of the worldlings' grammatical structure. They make use of the worldly usage much in the same way as parents do when they are speaking in their child's language. They are not swept away by it. There is no inner entanglement in the form of imagining. There is no attachment, entanglement and involvement by way of craving, conceit and view, in regard to those concepts.

All this goes to show the immense importance of the \emph{Mūlapariyāyasutta}. One can understand why this sutta came to be counted as the first among the suttas of the \emph{Majjhima Nikāya}. It is as if this sutta was intended to serve as the alphabet in deciphering the words used by the Buddha in his sermons delivered in discursive style. As a matter of fact the \emph{Majjhima Nikāya} in particular is a text abounding in deep suttas. This way we can understand why both higher knowledge and full comprehension are essential.

We have shown above that this discourse bears some relation to the grammatical structure. Probably due to a lack of recognition of this relationship between the modes of imagining and the grammatical structure, the commentators were confronted with a problem while commenting upon this discourse.

Such phrases as \emph{paṭhaviṁ maññati} and \emph{paṭhaviyā maññati} occur all over this discourse in referring to various ways of imagining. The commentator, however, always makes it a point to interpret these ways of imagining with reference to craving, conceit and views. So when he comes to the phrase \emph{mā abhinandi}, he finds it to be superfluous. That is why Venerable Buddhaghosa treats it as a repetition and poses a possible question as follows:

\begin{quote}
\emph{`Paṭhaviṁ maññatī'ti' eteneva etasmiṁ atthe siddhe kasmā evaṁ vuttanti ce. Avicāritaṁ etaṁ porāṇehi. Ayaṁ pana me attano mati, desanāvilāsato vā ādīnavadassanato vā.}\footnote{Ps I 28}
\end{quote}

Now this is how the commentator poses his own problem: When the phrase \emph{paṭhaviṁ maññati} by itself fulfils the purpose, why is it that an additional phrase like \emph{paṭhaviṁ abhinandati} is brought in? That is to say, if the imagining already implies craving, conceit and views, what is the justification for the concluding phrase \emph{paṭhaviṁ abhinandati}, ``he delights in earth'', since craving already implies a form of delighting?

So he takes it as a repetition and seeks for a justification. He confesses that the ancients have not handed down an explanation and offers his own personal opinion on it, \emph{ayaṁ pana me attano mati,} ``but then this is my own opinion''.

And what does his own explanation amount to? \emph{Desanāvilāsato vā ādīnavadassanato vā}, ``either as a particular style in preaching, or by way of showing the perils of the ways of imagining''. He treats it as yet another way of preaching peculiar to the Buddha, or else as an attempt to emphasize the perils of imagining.

However, going by the explanation we have already given above, relating these modes of imagining to the structure of grammar, we can come to a conclusion as to why the phrase \emph{mā abhinandi} was brought in.

The reason is that each of those concepts crystallized into a real thing as a result of imagining, based on the framework of grammar. It received real object status in the world of imagination. Once its object status got confirmed, one can certainly delight in it. It became a thing in truth and fact. The purpose of these ways of imagining is to mould it into a thing.

Let us go deeper into this problem. There is, for instance, a certain recurrent passage in the discourses on the subject of sense restraint.\footnote{E.g. D I 70, \emph{Sāmaññaphalasutta}} The gist of that passage amounts to this: A person with defilements takes in signs and features through all the six sense doors, inclusive of the mind.

Due to that grasping at signs and features, various kinds of influxes are said to flow in, according to the passages outlining the practice of sense restraint. From this we can well infer that the role of \emph{maññanā}, or imagining, is to grasp at signs with regard to the objects of the mind.

That is to say, the mind apperceives its object as `something', \emph{dhammasaññā}. The word \emph{dhamma} in the opening sentence of this sutta, \emph{sabbadhammamūlapariyāyaṁ vo, bhikkhave, desessāmi}, means a `thing', since every-thing is an object of the mind in the last analysis.

\emph{Paṭhaviṁ maññati}, ``he imagines earth as earth'', is suggestive of a grasping at the sign in regard to objects of the mind. Thinking in such terms as \emph{paṭhaviyā maññati, paṭhavito maññāti}, and \emph{`paṭhaviṁ me'ti maññati}, ``he imagines `on the earth', he imagines `from the earth', he imagines `earth is mine'\,'', are like the corroborative features that go to confirm that sign already grasped.

The two terms \emph{nimitta,} sign, and \emph{anuvyañjana}, feature, in the context of sense restraint have to be understood in this way. Now the purpose of a \emph{nimitta}, or sign, is to give a hazy idea like `this may be so'.

It receives confirmation with the help of corroborative features, \emph{anuvyañjana}, all the features that are accessory to the sign. The corroboration comes, for instance, in this manner: `This goes well with this, this accords with this, therefore the sign I took is right'. So even on the basis of instructions on sense restraint, we can understand the special significance of this \emph{maññanā}, or `me'-thinking.

The reason for the occurrence of these different ways of me-thinking can also be understood. In this discourse the Buddha is presenting a certain philosophy of the grammatical structure. The structure of grammar is a contrivance for conducting the worldlings' thought process, characterised by the perception of permanence, as well as for communication of ideas arising out of that process.

The grammatical structure invests words with life, as it were. This mode of hypostasizing is revealed in the nouns and substantives implying such notions as `in it', `by it' and `from it'. The last of the flexional forms, the vocative case, \emph{he paṭhavi}, ``hey earth'', effectively illustrates this hypostasizing character of grammar. It is even capable of infusing life into the concept of `earth' and arousing it with the words ``hey earth''.

In an earlier sermon we had occasion to refer to a legend in which a tiger was reconstituted and resurrected out of its skeletal remains.\footnote{See \emph{Sermon 11}} The structure of grammar seems to be capable of a similar feat. The \emph{Mūlapariyāyasutta} gives us an illustration of this fact.

It is because of the obsessional character of this \emph{maññanā}, or me-thinking, that the Buddha has presented this \emph{Mūlapariyāyasutta} to the world as the basic pattern or paradigm representing three types of world views, or the world views of three types of persons.

This discourse deals with the untaught ordinary person, who is obsessed by this grammatical structure, the disciple in higher training, who is trying to free himself from its grip, and the emancipated one, completely free from it, at the same time giving their respective world views as well.

The other day we enumerated the list of twenty-four concepts, presented in that discourse. Out of these concepts, we have to pay special attention to the fact that Nibbāna is counted as the last, since it happens to be the theme of all our sermons.

Regarding this concept of Nibbāna too, the worldling is generally tempted to entertain some kind of \emph{maññanā}, or me-thinking. Even some philosophers are prone to that habit. They indulge in some sort of prolific conceptualisation and me-thinking on the basis of such conventional usages as `in Nibbāna', `from Nibbāna', `on reaching Nibbāna' and `my Nibbāna'. By hypostasizing Nibbāna they develop a substance view, even of this concept, just as in the case of \emph{paṭhavi}, or earth. Let us now try to determine whether this is justifiable.

The primary sense of the word Nibbāna is `extinction', or `extinguishment'. We have already discussed this point with reference to such contexts as \emph{Aggivacchagottasutta}.\footnote{See \emph{Sermon 1}} In that discourse the Buddha explained the term Nibbāna to the wandering ascetic Vacchagotta with the help of a simile of the extinction of a fire. Simply because a fire is said to go out, one should not try to trace it, wondering where it has gone.

The term Nibbāna is essentially a verbal noun. We also came across the phrase \emph{nibbuto tveva saṅkhaṁ gacchati}, ``it is reckoned as `extinguished'\,''.\footnote{M I 487, \emph{Aggivacchagottasutta}}

As we have already pointed out in a previous sermon, \emph{saṅkhā}, \emph{samaññā} and \emph{paññatti}, `reckoning', `appellation' and `designation' are more or less synonymous.\footnote{See \emph{Sermon 12}}

\emph{Saṅkhaṁ gacchati} only means ``comes to be reckoned''. Nibbāna is therefore some sort of reckoning, an appellation or designation. The word Nibbāna, according to the \emph{Aggivacchagottasutta}, is a designation or a concept.

But the commentator takes much pains to prove that the Nibbāna mentioned at the end of the list in the \emph{Mūlapariyāyasutta} refers not to our orthodox Nibbāna, but to a concept of Nibbāna upheld by heretics.\footnote{Ps I 38} The commentator, it seems, is at pains to salvage our Nibbāna, but his attempt is at odds with the trend of this discourse, because the \emph{sekha}, or the monk in higher training, has no need to train himself in refraining from delighting in any heretical Nibbāna. So here too, the reference is to our orthodox Nibbāna.

Presumably the commentator could not understand why the \emph{arahants} do not delight in Nibbāna. For instance, in the section on the Tathāgata one reads:

\begin{quote}
\emph{Nibbānaṁ nābhinandati. Taṁ kissa hetu? Nandi dukkhassa mūlan'ti iti viditvā, bhavā jāti, bhūtassa jarāmaraṇaṁ.}

He does not delight in Nibbāna. Why so? Because he knows that delighting is the root of suffering, and from becoming comes birth and to the one become there is decay-and-death.
\end{quote}

It seems, then, that the Tathāgata does not delight in Nibbāna, because delighting is the root of suffering. Now \emph{nandi} is a form of grasping, \emph{upādāna}, impelled by craving. It is sometimes expressly called an \emph{upādāna}:

\begin{quote}
\emph{Yā vedanāsu nandi tadupādānaṁ,}

whatever delighting there is in feeling, that is a grasping.\footnote{M I 266, \emph{Mahātaṇhāsaṅkhayasutta}}
\end{quote}

Where there is delighting, there is a grasping. Where there is grasping, there is \emph{bhava}, becoming or existence. From becoming comes birth, and to the one who has thus come to be there is decay-and-death.

It is true that we project the concept of Nibbāna as an objective to aim at in our training. But if we grasp it like the concept of earth and start indulging in me-thinkings or imaginings about it, we would never be able to realize it. Why? Because what we have here is an extraordinary path leading to an emancipation from all concepts:

\clearpage

\begin{quote}
\emph{nissāya nissāya oghassa nittharaṇā,}

``crossing over the flood with relative dependence''.\footnote{M II 265, \emph{Āneñjasappāyasutta}}
\end{quote}

Whatever is necessary is made use of, but there is no grasping in terms of craving, conceits and views. That is why even with reference to the Tathāgata the phrase \emph{Nibbānaṁ nābhinandati}, ``he does not delight in Nibbāna'', occurs in this discourse.

One might ask: ``What is wrong in delighting in Nibbāna?'' But then we might recall a pithy dialogue already quoted in an earlier sermon.\footnote{See \emph{Sermon 2}} A deity comes and accosts the Buddha: ``Do you rejoice, recluse?'' And the Buddha responds: ``On getting what, friend?'' Then the deity asks: ``Well then, recluse, do you grieve?'' And the Buddha retorts: ``On losing what, friend?'' The deity now mildly remarks: ``So then, recluse, you neither rejoice nor grieve!'' And the Buddha confirms it with the assent: ``That is so, friend.''\footnote{S I 54, \emph{Kakudhasutta}}

This then is the attitude of the Buddha and the \emph{arahants} to the concept of Nibbāna. There is nothing to delight in it, only equanimity is there.

Seen in this perspective, the word Nibbāna mentioned in the \emph{Mūlapariyāyasutta} need not be taken as referring to a concept of Nibbāna current among heretics. The reference here is to our own orthodox Nibbāna concept. But the attitude towards it must surely be changed in the course of treading the path to it.

If, on the contrary, one grasps it tenaciously and takes it to be substantial, presuming that the word is a full fledged noun, and goes on to argue it out on the basis of logic and proliferate on it conceptually, it will no longer be our Nibbāna. There one slips into wrong view. One would never be able to extricate oneself from wrong view that way. Here then is an issue of crucial importance.

Many philosophers start their exposition with an implicit acceptance of conditionality. But when they come to the subject of Nibbāna, they have recourse to some kind of instrumentality. ``On reaching Nibbāna, lust and delight are abandoned.''\footnote{Vibh-a 53}

Commentators resort to such explanations under the influence of \emph{maññanā}. They seem to imply that Nibbāna is instrumental in quenching the fires of defilement. To say that the fires of defilements are quenched by Nibbāna, or on arriving at it, is to get involved in a circular argument. It is itself an outcome of \emph{papañca}, or conceptual prolificity, and betrays an enslavement to the syntax.

When one says `the river flows', it does not mean that there is a river quite apart from the act of flowing. Likewise the idiom `it rains' should not be taken to imply that there is something that rains. It is only a turn of speech, fulfilling a certain requirement of the grammatical structure.

On an earlier occasion we happened to discuss some very important aspects of the \emph{Poṭṭhapādasutta}.\footnote{See \emph{Sermon 12}} We saw how the Buddha presented a philosophy of language, which seems so extraordinary even to modern thinkers. This \emph{Mūlapariyāyasutta} also brings out a similar attitude to the linguistic medium.

Such elements of a language as nouns and verbs reflect the worldling's mode of thinking. As in the case of a child's imagination, a noun appears as a must. So it has to rain for there to be rain. The implicit verbal sense becomes obscured, or else it is ignored. A periphrastic usage receives acceptance. So the rain rains, and the river flows. A natural phenomenon becomes mystified and hypostasized.

Anthropomorphism is a characteristic of the pre-historic man's philosophy of life. Wherever there was an activity, he imagined some form of life. This animistic trend of thought is evident even in the relation between the noun and the verb. The noun has adjectives as attributes and the verb has adverbs to go with it. Particles fall in between, and there we have what is called grammar. If one imagines that the grammar of language must necessarily conform to the grammar of nature, one falls into a grievous error.

Now the commentators also seem to have fallen into such an error in their elaborate exegesis on Nibbāna, due to a lack of understanding of this philosophy of language. That is why the \emph{Mūlapariyāyasutta} now finds itself relegated, though it is at the head of the suttas of the \emph{Majjhima Nikāya}.

It is in the nature of concepts that nouns are invested with a certain amount of permanence. Even a verbal noun, once it is formed, gets a degree of permanence more or less superimposed on it. When one says `the river flows', one somehow tends to forget the flowing nature of the so-called river. This is the result of the perception of permanence.

As a matter of fact, perception as such carries with it the notion of permanence, as we mentioned in an earlier sermon.\footnote{See sermons 9 and 12} To perceive is to grasp a sign. One can grasp a sign only where one imagines some degree of permanence.

The purpose of perception is not only to recognize for oneself, but also to make it known to others. The Buddha has pointed out that there is a very close relationship between recognition and communication. This fact is expressly stated by the Buddha in the following quotation from the Sixes of the \emph{Aṅguttara Nikāya}:

\begin{quote}
\emph{Vohāravepakkaṁ ahaṁ, bhikkhave, saññaṁ vadāmi. Yathā yathā naṁ sañjānāti, tathā tathā voharati, evaṁ saññī ahosin'ti.}\footnote{A III 413, \emph{Nibbedhikasutta}}

Monks, I say that perception has linguistic usage as its result. In whatever way one perceives, so one speaks out about it, saying: `I~was of such a perception'.
\end{quote}

The word \emph{vepakka} is a derivative from the word \emph{vipāka}, which in the context of \emph{kamma}, or ethically significant action, generally means the result of that action. In this context, however, its primary sense is evident, that is, as some sort of a ripening. In other words, what this quotation implies is that perception ripens or matures into verbal usage or convention.

So here we see the connection between \emph{saññā}, perception, and \emph{saṅkhā}, reckoning. This throws more light on our earlier explanation of the last line of a verse in the \emph{Kalahavivādasutta}, namely:

\begin{quote}
\emph{saññānidānā hi papañcasaṅkhā,}

for reckonings born of prolificity have perception as their source.\footnote{Sn 874, \emph{Kalahavivādasutta}; see \emph{Sermon 11}}
\end{quote}

So now we are in a better position to appreciate the statement that linguistic usages, reckonings and designations are the outcome of perception. All this goes to show that an insight into the philosophy of language is essential for a proper understanding of this Dhamma. This is the moral behind the \emph{Mūlapariyāyasutta}.

Beings are usually dominated by these reckonings, appellations and designations, because the perception of permanence is inherent in them. It is extremely difficult for one to escape it. Once the set of such terms as milk, curd and butter comes into vogue, the relation between them becomes an insoluble problem even for the great philosophers.

Since we have been talking about the concept of Nibbāna so much, one might ask: ``So then, Nibbāna is not an absolute, \emph{paramattha}?'' It is not a \emph{paramattha} in the sense of an absolute. It is a \emph{paramattha} only in the sense that it is the highest good, \emph{parama attha}.

This is the sense in which the word was used in the discourses,\footnote{E.g. at Sn 219, \emph{Munisutta}; and Th 748, \emph{Telakānittheragāthā}} though it has different connotations now. As exemplified by such quotations as \emph{āraddhaviriyo paramatthapattiyā},\footnote{Sn 68, \emph{Khaggavisāṇasutta}} ``with steadfast energy for the attainment of the highest good'', the suttas speak of Nibbāna as the highest good to be attained.

In later Buddhist thought, however, the word \emph{paramattha} came to acquire absolutist connotations, due to which some important discourses of the Buddha on the question of worldly appellations, worldly expressions and worldly designations fell into disuse. This led to an attitude of dwelling in the scaffolding, improvised just for the purpose of constructing a building.

As a postscript to our exposition of the \emph{Mūlapariyāyasutta} we may add the following important note: This particular discourse is distinguished from all other discourses in respect of one significant feature. That is, the concluding statement to the effect that the monks who listened to the sermon were not pleased by it.

Generally we find at the end of a discourse a more or less thematic sentence like:

\begin{quote}
\emph{attamanā te bhikkhū Bhagavato bhāsitaṁ abhinanduṁ,}

those monks were pleased and they rejoiced in the words of the Exalted~One.\footnote{E.g. at M I 12, \emph{Sabbāsavasutta}}
\end{quote}

But in this sutta we find the peculiar ending:

\begin{quote}
\emph{idaṁ avoca Bhagavā, na te bhikkhū Bhagavato bhāsitaṁ abhinanduṁ,}

the Exalted One said this, but those monks did not rejoice in the words of the Exalted~One.\footnote{M I 6, \emph{Mūlapariyāyasutta}}
\end{quote}

Commentators seem to have interpreted this attitude as an index to the abstruseness of the discourse.\footnote{Ps I 56} This is probably why this discourse came to be neglected in the course of time.

But on the basis of the exposition we have attempted, we might advance a different interpretation of the attitude of those monks. The declaration that none of the concepts, including that of Nibbāna, should be egoistically imagined, could have caused displeasure in monks, then as now. So much, then, for the \emph{Mūlapariyāyasutta}.

The Buddha has pointed out that this \emph{maññanā}, or egoistic imagining, or me-thinking, is an extremely subtle bond of Māra.

A discourse which highlights this fact comes in the \emph{Saṁyutta Nikāya} under the title \emph{Yavakalāpisutta}.\footnote{S IV 201, \emph{Yavakalāpisutta}} In this discourse the Buddha brings out this fact with the help of a parable. It concerns the battle between gods and demons, which is a theme that comes up quite often in the discourses.

In a war between gods and demons, the gods are victorious and the demons are defeated. The gods bind Vepacitti, the king of the demons, in a fivefold bondage, that is, hands and feet and neck, and bring him before Sakka, the king of the gods.

This bondage has a strange mechanism about it. When Vepacitti thinks ``gods are righteous, demons are unrighteous, I will go to the \emph{deva} world'', he immediately finds himself free from that bondage and capable of enjoying the heavenly pleasures of the five senses.

But as soon as he slips into the thought ``gods are unrighteous, demons are righteous, I will go back to the \emph{asura} world'', he finds himself divested of the heavenly pleasures and bound again by the fivefold bonds.

After introducing this parable, the Buddha comes out with a deep disquisition of Dhamma for which it serves as a simile.

\begin{quote}
\emph{Evaṁ sukhumaṁ kho, bhikkhave, Vepacittibandhanaṁ. Tato sukhumataraṁ Mārabandhanaṁ. Maññamāno kho, bhikkhave, baddho Mārassa, amaññamāno mutto pāpimato.}

\emph{Asmī'ti, bhikkhave, maññitaṁ etaṁ, `ayaṁ ahaṁ asmī'ti maññitaṁ etaṁ, `bhavissan'ti maññitaṁ etaṁ, `na bhavissan'ti maññitaṁ etaṁ, `rūpī bhavissan'ti maññitaṁ etaṁ, `arūpī bhavissan'ti maññitaṁ etaṁ, `saññī bhavissan'ti maññitaṁ etaṁ, `asaññī bhavissan'ti maññitaṁ etaṁ, `nevasaññīnāsaññī bhavissan'ti maññitaṁ etaṁ.}

\emph{Maññitaṁ, bhikkhave, rogo, maññitaṁ gaṇḍo, maññitaṁ sallaṁ. Tasmātiha, bhikkhave, `amaññamānena cetasā viharissāmā'ti evañhi vo, bhikkhave, sikkhitabbaṁ.}

So subtle, monks, is the bondage of Vepacitti. But more subtle still is the bondage of Māra. Imagining, monks, one is bound by Māra, not imagining one is freed from the Evil One.

`Am', monks, is an imagining, `this am I' is an imagining, `I shall be' is an imagining, `I shall not be' is an imagining, `I shall be one with form' is an imagining, `I shall be formless' is an imagining, `I~shall be percipient' is an imagining, `I shall be non-percipient' is an imagining, `I shall be neither-percipient-nor-non-percipient' is an imagining.

Imagining, monks, is a disease, imagining is an abscess, imagining is a barb, therefore, monks, should you tell yourselves: `We shall dwell with a mind free from imaginings, thus should you train yourselves'.
\end{quote}

First of all, let us try to get at the meaning of this exhortation. The opening sentence is an allusion to the simile given above. It says that the bondage in which Vepacitti finds himself is of a subtle nature, that is to say, it is a bondage connected with his thoughts. Its very mechanism is dependent on his thoughts.

But then the Buddha declares that the bondage of Māra is even subtler. And what is this bondage of Māra? ``Imagining, monks, one is bound by Māra, not imagining one is freed from that Evil One.'' Then comes a list of nine different ways of imaginings.

In the same discourse the Buddha goes on to qualify each of these imaginings with four significant terms, namely: \emph{iñjitaṁ} agitation, \emph{phanditaṁ} palpitation, \emph{papañcitaṁ} proliferation and \emph{mānagataṁ} conceit.

\emph{Iñjitaṁ} is an indication that these forms of imaginings are the outcome of craving, since \emph{ejā} is a synonym for \emph{taṇhā}, or craving.

\emph{Phanditaṁ} is an allusion to the fickleness of the mind, as for instance conveyed by the first line of a verse in the \emph{Dhammapada}, \emph{phandanaṁ capalaṁ cittaṁ}, ``the mind, palpitating and fickle''.\footnote{Dhp 33, \emph{Cittavagga}} The fickle nature of the mind brings out those imaginings.

They are also the products of proliferation, \emph{papañcita}. We have already discussed the meaning of the term \emph{papañca}.\footnote{See sermons 11 and 12} We happened to point out that it is a sort of straying away from the proper path.

\emph{Mānagataṁ} is suggestive of a measuring. \emph{Asmi}, or `am', is the most elementary standard of measurement. It is the peg from which all measurements take their direction. As we pointed out in an earlier sermon, the grammatical structure of language is based on this peg `am'.\footnote{See \emph{Sermon 10}}

In connection with the three persons, first person, second person and third person, we happened to mention that as soon as one grants `I am', a `here' is born. It is only after a `here' is born, that a `there' and a `yonder' come to be. The first person gives rise to the second and the third person, to complete the basic framework for grammar.

So \emph{asmi}, or `am', is itself a product of proliferation. In fact, the deviation from the proper path, implied by the proliferation in \emph{papañca}, is a result of these multifarious imaginings.

It is in the nature of these imaginings that as soon as an imagining or a me-thinking occurs, a thing is born as a matter of course. And with the birth of a thing as `something', impermanence takes over. That is to say, it comes under the sway of impermanence.

This is a very strange phenomenon. It is only after becoming a `something' that it can become `another thing'. \emph{Aññathābhāva}, or otherwiseness, implies a change from one state to another. A change of state already presupposes some state or other, and that is what is called a `thing'.

Now where does a `thing' arise? It arises in the mind. As soon as something gets hold of the mind, that thing gets infected with the germ of impermanence.

The modes of imagining listed above reveal a double bind. There is no freedom either way. Whether one imagines `I shall be with form' or `I shall be formless', one is in a dichotomy. It is the same with the two ways of imagining `I shall be percipient', `I shall be non-percipient'.

We had occasion to refer to this kind of dichotomy while explaining the significance of quite a number of discourses. The root of all this duality is the thought `am'.

The following two verses from the \emph{Dvayatānupassanāsutta} throw light on some subtle aspects of \emph{maññanā}, or imagining:

\begin{quote}
\emph{Yena yena hi maññanti,}\\
\emph{tato taṁ hoti aññathā,}\\
\emph{taṁ hi tassa musā hoti,}\\
\emph{mosadhammaṁ hi ittaraṁ.}

\emph{Amosadhammaṁ Nibbānaṁ,}\\
\emph{tad ariyā saccato vidū,}\\
\emph{te ve saccābhisamayā,}\\
\emph{nicchātā parinibbutā.}\footnote{Sn 757-758, \emph{Dvayatānupassanāsutta}}

In whatever way they imagine,\\
Thereby it turns otherwise,\\
That itself is the falsity\\
Of this puerile deceptive thing.

Nibbāna is unfalsifying in its nature,\\
That they understood as the truth,\\
And indeed by the higher understanding of that truth\\
They have become hungerless and fully appeased.
\end{quote}

The first verse makes it clear that imagining is at the root of \emph{aññathābhāva}, or otherwiseness, in so far as it creates a thing out of nothing. As soon as a thing is conceived in the mind by imagining, the germ of otherwiseness or change enters into it at its very conception.

So a thing is born only to become another thing, due to the otherwiseness in nature. To grasp a thing tenaciously is to exist with it, and birth, decay and death are the inexorable vicissitudes that go with it.

The second verse says that Nibbāna is known as the truth, because it is of an unfalsifying nature. Those who have understood it are free from the hunger of craving. The word \emph{parinibbuta} in this context does not mean that those who have realized the truth have passed away. It only conveys the idea of full appeasement or a quenching of that hunger.

Why is Nibbāna regarded as unfalsifying? Because there is no `thing' in it. It is so long as there is a thing that all the distress and misery follow. Nibbāna is called \emph{animitta}, or the signless, precisely because there is no-thing in it.

Because it is signless, it is unestablished, \emph{appaṇihita}. Only where there is an establishment can there be a dislodgement. Since it is not liable to dislodgement or disintegration, it is unshakeable. It is called \emph{akuppā cetovimutti}, unshakeable deliverance of the mind,\footnote{E.g. at D III 273, \emph{Dasuttarasutta}} because of its unshaken and stable nature. Due to the absence of craving there is no directional apsiration, or \emph{paṇidhi}.

Similarly \emph{suññata}, or voidness, is a term implying that there is no essence in Nibbāna in the substantial sense in which the worldlings use that term. As mentioned in the \emph{Mahāsāropamasutta}, deliverance itself is the essence.\footnote{M I 197, \emph{Mahāsāropamasutta}} Apart from that, there is nothing essential or substantial in Nibbāna. In short, there is no thing to become otherwise in Nibbāna.

On an earlier occasion, too, we had to mention the fact that there is quite a lot of confusion in this concern.\footnote{See \emph{Sermon 2}} \emph{Saṅkhata}, the compounded, is supposed to be a thing. And \emph{asaṅkhata}, or the uncompounded, is also a thing. The compounded is an impermanent thing, while the uncompounded is a permanent thing. The compounded is fraught with suffering, and the uncompounded is blissful. The compounded is not self, but the uncompounded is \ldots{} At this point the line of argument breaks off.

Some of those who attempt this kind of explanation find themselves in a quandary due to their lack of understanding of the issues involved. The two verses quoted above are therefore highly significant.

Because of \emph{maññanā}, worldlings tend to grasp, hold on and adhere to mind-objects. The Buddha has presented these concepts just for the purpose of crossing over the flood,

\begin{quote}
\emph{desitā nissāya nissāya oghassa nittharaṇā,}\footnote{M II 265, \emph{Āneñjasappāyasutta}}

the process of crossing over the flood with relative dependence has been preached.
\end{quote}

All the \emph{dhammas} that have been preached are for a practical purpose, based on an understanding of their relative value, and not for grasping tenaciously, as illustrated by such discourses like the \emph{Rathavinītasutta} and the \emph{Alagaddūpamasutta}.\footnote{M I 145, \emph{Rathavinītasutta}; M I 130, \emph{Alagaddūpamasutta}}

Let alone other concepts, not even Nibbāna as a concept is to be grasped. To grasp the concept of Nibbāna is to slip into an error. So from the couplet quoted above we clearly understand how subtle this \emph{maññanā} is and why it is called an extremely subtle bondage of Māra.

It might be recalled that while discussing the significance of the \emph{Brahmanimantanikasutta} we mentioned that the non-manifestative consciousness described in that discourse does not partake of the earthiness of earth.\footnote{See \emph{Sermon 8}; M I 329, \emph{Brahmanimantanikasutta}} That is to say, it is not under the sway of the earth quality of earth.

In fact as many as thirteen out of the twenty-four concepts mentioned in the \emph{Mūlapariyāyasutta} come up again in the \emph{Brahmanimantanikasutta}. The implication therefore is that the non-manifestative consciousness is not subject to the influence of any of those concepts. It does not take any of those concepts as substantial or essential, and that is why it is beyond their power.

For the same reason it is called the non-manifestative consciousness. Consciousness as a rule takes hold of some object or other. This consciousness, however, is called non-manifestative in the sense that it is devoid of the nature of grasping any such object. It finds no object worthy of grasping.

What we have discussed so far could perhaps be better appreciated in the light of another important sutta in the \emph{Majjhima Nikāya}, namely the \emph{Cūḷataṇhāsaṅkhayasutta}. A key to the moral behind this discourse is to be found in the following dictum occurring in it: \emph{sabbe dhammā nālaṁ abhinivesāya,} ``nothing is worth entering into dogmatically''.\footnote{M I 251, \emph{Cūḷataṇhāsaṅkhayasutta}}

The word \emph{abhinivesa}, suggestive of dogmatic adherence, literally means `entering into'. Now based on this idea we can bring in a relevant metaphor.

We happened to mention earlier that as far as concepts are concerned, the \emph{arahants} have no dogmatic adherence. Let us take, for instance, the concept of `a house'. \emph{Arahants} also enter a house, but they do not enter into the concept of `a house'. This statement might appear rather odd, but what we mean is that one can enter a house without entering into the concept of `a house'.

Now leaving this as something of a riddle, let us try to analyse a certain fairy tale-like episode in the \emph{Cūḷataṇhāsaṅkhayasutta}, somewhat as an interlude.

The main theme of the \emph{Cūḷataṇhāsaṅkhayasutta} is as follows: Once Sakka, the king of the gods, came to see the Buddha when he was staying at Pubbārāma and asked the question:

\begin{quote}
``How does a monk attain deliverance by the complete destruction of craving?''
\end{quote}

The quintessence of the Buddha's brief reply to that question is the above mentioned dictum,

\begin{quote}
\emph{sabbe dhammā nālaṁ abhinivesāya,}

``nothing is worth entering into dogmatically.''
\end{quote}

Sakka rejoiced in this sermon approvingly and left. Venerable Mahā Moggallāna, who was seated near the Buddha at that time, had the inquisitive thought:

\begin{quote}
``Did Sakka rejoice in this sermon having understood it, or did he rejoice without understanding it?''
\end{quote}

Being curious to find this out he vanished from Pubbārāma and appeared in the Tāvatiṁsa heaven as quickly as a strong man might stretch out his bent arm and bend back his outstretched arm.

At that time Sakka was enjoying heavenly music. On seeing Venerable Mahā Moggallāna coming at a distance he stopped the music and welcomed the latter, saying:

\begin{quote}
``Come good sir Moggallāna, welcome good sir Moggallāna! It is a long time, good sir Moggallāna, since you found an opportunity to come here.''
\end{quote}

\clearpage

He offered a high seat to Venerable Mahā Moggallāna and took a low seat at one side. Then Venerable Mahā Moggallāna asked Sakka what sort of a sermon the Buddha had preached to him on his recent visit, saying that he himself is curious on listening to it.

Sakka's reply was:

\begin{quote}
``Good sir Moggallāna, we are so busy, we have so much to do, not only with our own business, but also with the business of other gods of Tāvatiṁsa. So it is not easy for us to remember such Dhamma discussions.''
\end{quote}

Then Sakka goes on to relate some other episode, which to him seems more important:

\begin{quote}
``After winning the war against the \emph{asuras}, I had the Vejayanti palace built. Would you like to see it, good sir Moggallāna?''
\end{quote}

Probably as a part of etiquette, binding on a visitor, Venerable Mahā Moggallāna agreed and Sakka conducted him around the Vejayanti palace in the company of his friend, king Vessavaṇa. It was a wonderful palace with hundreds of towers. Sakka's maids, seeing Venerable Mahā Moggallāna coming in the distance, were embarrassed out of modest respect and went into their rooms. Sakka was taking Venerable Mahā Moggallāna around, saying:

\begin{quote}
``See, good sir, how lovely this palace is.''
\end{quote}

Venerable Mahā Moggallāna also courteously responded, saying that it is a fitting gift for his past merit. But then he thought of arousing a sense of urgency in Sakka, seeing how negligent he has become now. And what did he do? He shook the Vejayanti palace with the point of his toe, using his supernormal power.

Since Sakka had `entered into' the Vejayanti palace with his craving, conceit and views, he also was thoroughly shaken, along with the palace. That is to say, a sense of urgency was aroused in him, so much so that he remembered the sermon the Buddha had preached to him.

\clearpage

It was then that Venerable Mahā Moggallāna asked Sakka pointedly:

\begin{quote}
``How did the Exalted One state to you in brief the deliverance through the destruction of craving?''
\end{quote}

Sakka came out with the full account, creditably.

So after all it seems that the Venerable Mahā Moggallāna took all this trouble to drive home into Sakka the moral of the sermon \emph{sabbe dhammā nālaṁ abhinivesāya}, ``nothing is worth clinging onto''.

If one goes through this discourse ignoring the deeper aspects of it, it appears merely as a fairy tale. Even as those heavenly maidens entered their rooms, Sakka also had entered into this Vejayanti palace of his own creation, while showing his distinguished visitor around, like a rich man these days after building his mansion.

So from this we can see the nature of these worldly concepts. For instance, in the case of the concept of `a house', entering the house physically does not necessarily mean that one is `in it'. Only if one has entered into the concept of a house is he `in it'.

Let us take a simply analogy. Little children sometimes build a little hut, out of fun, with a few sticks and shady leaves. They might even invite their mother for the house-warming. When the mother creeps into the improvised hut, she does not seriously entertain the concept of `a house' in it, as the children would~do.

It is the same in the case of Buddhas and \emph{arahants.} To the Emancipated Ones, who have fully understood and comprehended the true meaning of concepts like `house', `mansion' and `palace', the sandcastles of adults appear no better than the playthings of little children. We have to grant it, therefore, that \emph{tathāgatas}, or Such-like Ones, cannot help making use of concepts in worldly usage.

As a matter of fact, once a certain deity even raised the question whether the emancipated \emph{arahant} monks, when they use such expressions as `I speak' and `they speak to me', do so out of conceit. The Buddha's reply was:

\clearpage

\begin{quote}
\emph{Yo hoti bhikkhu arahaṁ katāvī,}\\
\emph{khīṇāsavo antimadehadhārī,}\\
\emph{`ahaṁ vadāmī'ti pi so vadeyya,}\\
\emph{`mamaṁ vadantī'ti pi so vadeyya}\\
\emph{loke samaññaṁ kusalo viditvā,}\\
\emph{vohāramattena so vohareyyā.}\footnote{S I 14, \emph{Arahantasutta}}

That monk, who is an \emph{arahant,} who has finished his task,\\
Whose influxes are extinct and who bears his final body,\\
Might still say `I speak',\\
He might also say `they speak to me',\\
Being skilful, knowing the world's parlance,\\
He uses such terms merely as a convention.
\end{quote}

In the case of an \emph{arahant}, who has accomplished his task and is influx-free, a concept like `house', `mansion', or `palace' has no influence by way of craving, conceit and views. He might say `I speak' or `I preach', he might even say `they speak to me', but since he has understood the nature of worldly parlance, he uses such expressions as mere turns of speech. Therefore the Buddhas and \emph{arahants}, though they may enter a house, do not entertain the concept of `a house' in it.

Some might think that in order to destroy the concept of `a house', one has to break up the tiles and bricks into atoms. But that is not the way to deliverance. One has to understand according to the law of dependent arising that not only is a house dependent on tiles and bricks, but the tiles and bricks are themselves dependent on a house. Very often philosophers forget about the principle of relativity involved here.

Tiles and bricks are dependent on a house. This is a point worth considering. One might think that a house is made up of tiles and bricks, but tiles and bricks themselves come to be because of a house. There is a mutual relationship between them.

If one raises the question: ``What is a tile?'', the answer will be: ``It is an item used for building the roof of a house''. Likewise a brick is an item used in building a wall. This shows the relativity between a house and a tile as well as between a house and a brick. So there is no need to get down to an atomistic analysis like nuclear physicists. Wisdom is something that enables one to see this relativity penetratively, then and there.

Today we happened to discuss some deep sections of the Dhamma, particularly on the subject of \emph{maññanā}. A reappraisal of some of the deep suttas preached by the Buddha, now relegated into the background as those dealing with conventional truth, will be greatly helpful in dispelling the obsessions created by \emph{maññanā}. What the \emph{Mūlapariyāyasutta} offers in this respect is of utmost importance.

In fact, the Buddha never used a language totally different from the language of the worldlings. Now, for instance, chemists make use of a certain system of symbolic formulas in their laboratories, but back at home they revert to another set of symbols. However, both are symbols. There is no need to discriminate between them as higher or lower, so long as they serve the purpose at hand.

Therefore it is not proper to relegate some sermons as discursive or conventional in style. Always it is a case of using concepts in worldly parlance. In the laboratory one uses a particular set of symbols, but on returning home he uses another. In the same way, it is not possible to earmark a particular bundle of concepts as absolute and unchangeable.

As stated in the \emph{Poṭṭhapādasutta}, already discussed, all these concepts are worldly appellations, worldly expressions, worldly usages, worldly designations, which the Tathāgata makes use of without tenacious grasping.\footnote{D I 202, \emph{Poṭṭhapādasutta}} However philosophical or technical the terminology may be, the \emph{arahants} make use of it without grasping it tenaciously.

What is of importance is the function it fulfils. We should make use of the conceptual scaffolding only for the purpose of putting up the building. As the building comes up, the scaffolding has to leave. It has to be dismantled. If one simply clings onto the scaffolding, the building would never come~up.
