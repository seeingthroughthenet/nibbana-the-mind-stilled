\chapter{Sermon 33}

\NibbanaOpeningQuote

With the permission of the assembly of the venerable meditative monks. This is the thirty-third sermon in the series of sermons on Nibbāna.

Towards the end of our last sermon, the other day, we happened to mention that in developing the noble eightfold path fully intent on Nibbāna, all the other enlightenment factors, namely the four foundations of mindfulness, the four right endeavours, the four bases for success, the five spiritual faculties, the five powers and the seven factors of enlightenment go to fulfilment by development.

Though we started analyzing the way in which the Buddha clarified the above-mentioned peculiarity of the noble eightfold path in the \emph{Mahāsaḷāyatanikasutta} of the \emph{Majjhima Nikāya}, we could not finish it.

From the sutta passage we quoted the other day, we could see how the lack of knowledge of things as they are in regard to the six-fold sense-sphere gives rise to attachments, entanglements and delusions. As a result of it, the five aggregates of grasping get accumulated, leading to an increase in craving that makes for re-becoming, as well as an increase in bodily stresses and torment, mental stresses and torment, bodily fevers and mental fevers, and bodily and mental suffering.

Today, to begin with, let us discuss the rest of that discourse.

\begin{quote}
\emph{Cakkhuñca kho, bhikkhave, jānaṁ passaṁ yathābhūtaṁ, rūpe jānaṁ passaṁ yathābhūtaṁ, cakkhuviññāṇaṁ jānaṁ passaṁ yathābhūtaṁ, cakkhusamphassaṁ jānaṁ passaṁ yathābhūtaṁ, yampidaṁ cakkhusamphassapaccayā uppajjati vedayitaṁ sukhaṁ vā dukkhaṁ vā adukkhamasukhaṁ vā tampi jānaṁ passaṁ yathābhūtaṁ, cakkhusmiṁ na sārajjati, rūpesu na sārajjati, cakkhuviññāṇe na sārajjati, cakkhusamphasse na sārajjati, yampidaṁ cakkhusamphassapaccayā uppajjati vedayitaṁ sukhaṁ vā dukkhaṁ vā adukkhamasukhaṁ vā tasmimpi na sārajjati.}

\emph{Tassa asārattassa asaṁyuttasa asammūḷhassa ādīnavānupassino viharato āyatiṁ pañcupādānakkhandhā apacayaṁ gacchanti. Taṇhā cassa ponobhavikā nandirāgasahagatā tatratatrābhinandinī sā cassa pahīyati. Tassa kāyikāpi darathā pahīyanti, cetasikāpi darathā pahīyanti, kāyikāpi santāpā pahīyanti, cetasikāpi santāpā pahīyanti, kāyikāpi pariḷāhā pahīyanti, cetasikāpi pariḷāhā pahīyanti. So kāyasukhampi cetosukhampi paṭisaṁvedeti.}\footnote{M III 288, \emph{Mahāsaḷāyatanikasutta}}

Monks, knowing and seeing the eye as it actually is, knowing and seeing forms as they actually are, knowing and seeing eye-consciousness as it actually is, knowing and seeing eye-contact as it actually is, whatever is felt, pleasant or unpleasant or neither-unpleasant-nor-pleasant, arising in dependence on eye-contact, knowing and seeing that too as it actually is, one does not get lustfully attached to the eye, to forms, to eye-consciousness, to eye-contact, and to whatever is felt as pleasant or unpleasant or neither-unpleasant-nor-pleasant, arising in dependence on eye-contact.

And for him, who is not lustfully attached, not fettered, not infatuated, contemplating danger, the five aggregates of grasping get diminished for the future and his craving, which makes for re-becoming, which is accompanied by delight and lust, delighting now here now there, is abandoned, his bodily stresses are abandoned, his mental stresses are abandoned, his bodily torments are abandoned, his mental torments are abandoned, his bodily fevers are abandoned, his mental fevers are abandoned, and he experiences bodily and mental happiness.
\end{quote}

Then the Buddha goes on to point out how the noble eightfold path gets developed in this noble disciple by this training in regard to the six spheres of sense.

\begin{quote}
\emph{Yā tathābhūtassa diṭṭhi sāssa hoti sammā diṭṭhi, yo tathābhūtassa saṅkappo svāssa hoti sammā saṅkappo, yo tathābhūtassa vāyāmo svāssa hoti sammā vāyāmo, yā tathābhūtassa sati sāssa hoti sammā sati, yo tathābhūtassa samādhi svāssa hoti sammā samādhi, Pubbeva kho panassa kāyakammaṁ vacīkammaṁ ājīvo suparisuddho hoti. Evamassāyaṁ ariyo aṭṭhaṅgiko maggo bhāvanāpāripūriṁ gacchati.}

The view of a person such as this is right view. The intention of a person such as this is right intention. The effort of a person such as this is right effort. The mindfulness of a person such as this is right mindfulness. The concentration of a person such as this is right concentration. But his bodily action, his verbal action and his livelihood have already been purified earlier. Thus this noble eightfold path comes to fulfilment in him by development.
\end{quote}

It is noteworthy that in this context the usual order in citing the factors of the path is not found. But at the end we are told that bodily action, verbal action and livelihood have already been purified.

This is reminiscent of the explanation given in the \emph{Mahācattārīsakasutta}, in the previous sermon. That is to say, when the noble eightfold path is perfected at the supramundane level, the three factors right speech, right action and right livelihood are represented by the very thought of abstaining.

Now the Buddha proclaims how all the enlightenment factors reach fulfilment by development when one develops the noble eightfold path in this way.

\begin{quote}
\emph{Tassa evaṁ imaṁ ariyaṁ aṭṭhaṅgikaṁ maggaṁ bhāvayato cattāropi satipaṭṭhānā bhāvanāpāripūriṁ gacchanti, cattāropi sammappadhānā bhāvanāpāripūriṁ gacchanti, cattāropi iddhipādā bhāvanāpāripūriṁ gacchanti, pañcapi indriyāni bhāvanāpāripūriṁ gacchanti, pañcapi balāni bhāvanāpāripūriṁ gacchanti, sattapi bojjhaṅgā bhāvanā-\\ pāripūriṁ gacchanti. Tass'ime dve dhammā yuganaddhā vattanti, samatho ca vipassanā ca.}

When he develops this noble eightfold path in this way, the four foundations of mindfulness also come to fulfilment by development, the four right endeavours also come to fulfilment by development, the four bases for success also come to fulfilment by development, the five faculties also come to fulfilment by development, the five powers also come to fulfilment by development and the seven factors of enlightenment also come to fulfilment by development. These two things, namely serenity and insight, occur in him yoked evenly together.
\end{quote}

The net result of perfecting all the enlightenment factors is summed up by the Buddha in the following declaration:

\begin{quote}
\emph{So ye dhammā abhiññā pariññeyyā te dhamme abhiññā parijānāti, ye dhammā abhiññā pahātabbā te dhamme abhiññā pajahati, ye dhammā abhiññā bhāvetabbā te dhamme abhiññā bhāveti, ye dhammā abhiññā sacchikātabbā te dhamme abhiññā sacchikaroti.}

He comprehends by direct knowledge those things that should be comprehended by direct knowledge, he abandons by direct knowledge those things that should be abandoned by direct knowledge, he develops by direct knowledge those things that should be developed by direct knowledge, he realizes by direct knowledge those things that should be realized by direct knowledge.
\end{quote}

The things that should be comprehended by direct knowledge are explained in the sutta itself as the five aggregates of grasping. The things that should be abandoned by direct knowledge are ignorance and craving. The things that should be developed by direct knowledge are serenity and insight. The things that should be realized by direct knowledge are true knowledge and deliverance.

So then, as we have already mentioned, the orderly arrangement in these thirty-seven enlightenment factors is well illustrated in this discourse. It is because of this orderliness that even in a stream-winner, who is well established in the noble eightfold path, other enlightenment factors are said to be present as if automatically.

Simply because the phrase \emph{ekāyano ayaṁ, bhikkhave, maggo} occurs in the \emph{Satipaṭṭhānasutta}, some are tempted to interpret the four foundations of mindfulness as `the only way'.\footnote{M I 55, \emph{Satipaṭṭhānasutta}}

We have pointed out, with valid reasons on an earlier occasion, that such a conclusion is unwarranted. \emph{Ekāyano} does not mean `the only way', it means `directed to one particular destination', that is, to Nibbāna. That is why the following words occur later on in the same sentence:

\begin{quote}
\emph{ñāyassa adhigamāya Nibbānassa sacchikiriyāya},

for the attainment of the supramundane path, for the realizing of Nibbāna.
\end{quote}

The four foundations of mindfulness are the preliminary training for the attainment of the supramundane path and realization of Nibbāna. The initial start made by the four foundations of mindfulness is carried over by the four right endeavours, the four bases for success, the five faculties, the five powers and the seven enlightenment factors, to reach the acme of perfection in the noble eightfold path.

In the \emph{Mahāsaḷāyatanikasutta} we came across the repetitive phrase:

\begin{quote}
\emph{jānaṁ passaṁ yathābhūtaṁ}

knowing and seeing as it actually is
\end{quote}

Used in connection with the eye, forms, eye-consciousness, eye-contact and whatever is felt due to eye-contact. Let us examine what this knowing and seeing as it actually is amounts to.

Perception has been compared to a mirage.\footnote{S III 142, \emph{Pheṇapiṇḍūpamasutta}} This mirage nature of perception has to be understood. A deer which sees a mirage in a plain from a distance in the dry season has a perception of water in it. In other words, it imagines water in the mirage. Impelled by that imagining, it runs towards the mirage with the idea that by running it can do away with the gap between itself and the water, and reach that water.

But there is something that the deer is not aware of, and that is that this gap can never be reduced by running.

So long as there are two ends, there is a middle. This is a maxim worth emphasizing. Where there are two ends, there is a middle. If the eye is distinguished as one end and what appears in the distance is distinguished as water, there is an intervening space, a gap between the two. All these three factors are integral in this perceptual situation. That is why the gap can never be done away with.

The emancipated one, who has understood that this can never be eliminated, does not run after the mirage. That one with discernment, that \emph{arahant}, stops short at the seen, true to the aphorism \emph{diṭṭhe diṭṭhamattaṁ}, ``in the seen just the seen''.\footnote{Ud 8, \emph{Bāhiyasutta}}

He stops at the heard in the heard, he stops at the sensed in the sensed, he stops at the cognized in the cognized. He does not go on imagining like that deer, taking his stand on perception. He does not imagine a thing seen or one who sees. Nor does he entertain imaginings in regard to the heard, the sensed and the cognized.

\enlargethispage{\baselineskip}

The fact that this freedom from imaginings is there in an \emph{arahant} is clear from the statement we quoted from the \emph{Chabbisodhanasutta} on an earlier occasion. According to that discourse, a monk rightly claiming \emph{arahanthood}, one who declares himself to be an \emph{arahant}, should be able to make the following statement in respect of the seen, the heard, the sensed and the cognized.

\begin{quote}
\emph{Diṭṭhe kho ahaṁ, āvuso, anupāyo anapāyo anissito appaṭibaddho vippamutto visaṁyutto vimariyādikatena cetasā viharāmi.}\footnote{M III 30, \emph{Chabbisodhanasutta}; see \emph{Sermon 15}}

Friends, with regard to the seen, I dwell unattracted, unrepelled, independent, uninvolved, released, unshackled, with a mind free from barriers.
\end{quote}

Now let us try to understand this statement in the light of what we have already said about the mirage. One can neither approach nor retreat from a mirage. Generally, when one sees a mirage in the dry season, one imagines a perception of water in it and runs towards it due to thirst.

But let us, for a moment, think that on seeing the mirage one becomes apprehensive of a flood and turns and runs away to escape it. Having run some far, if he looks back he will still see the mirage behind him.

So in the case of a mirage, the more one approaches it, the farther it recedes, the more one recedes from it, the nearer it appears. So in regard to the mirage of percepts, such as the seen and the heard, the \emph{arahant} neither approaches nor recedes. Mentally he neither approaches nor recedes, though he may appear to do both physically, from the point of view of the worldling -- \emph{anupāyo anapāyo}, unattracted, unrepelled.

It is the same with regard to the term \emph{anissito}, independent. He does not resort to the mirage with the thought ``Ah, here is a good reservoir''.

\emph{Appaṭibaddho}, uninvolved, he is not mentally involved in the mirage.

\emph{Vippamutto}, released, he is released from the perception of water in the mirage, from imagining water in it.

\emph{Visaṁyutto,} unshackled, he is not bound by it.

\emph{Vimariyādikatena cetasā}, with a mind free from barriers. What are these barriers? The two ends and the middle. The demarcation mentioned above by distinguishing eye as distinct from form, with the intervening space or the gap as the `tertium quid'. So for the \emph{arahant} there are no barriers by taking the eye, the forms and the gap as discrete.

Now from what we have already discussed, it should be clear that by \emph{maññanā} or imagining a thing-hood is attributed to the seen, the heard, the sensed and the cognized. One imagines a thing in the seen, heard etc. By that very imagining as a thing it becomes another thing, true to the dictum expressed in the line of that verse from the \emph{Dvayatānupassanāsutta} we had quoted earlier,

\begin{quote}
\emph{yena yena hi maññanti, tato taṁ hoti aññāthā},\footnote{Sn 757, \emph{Dvayatānupassanāsutta}; see \emph{Sermon 13}}

in whatever egoistic terms they imagine, thereby it turns otherwise.
\end{quote}

That is why we earlier said that a thing has to be there first for it to become another thing, for there to be an otherwiseness. The more one tries to approach the thing imagined, the more it recedes.

In our analysis of the \emph{Mūlapariyāyasutta}, we discussed at length about the three levels of knowledge mentioned there, namely \emph{saññā, abhiññā} and \emph{pariññā}.\footnote{M I 1, \emph{Mūlapariyāyasutta}; see \emph{Sermon 12}}

The untaught worldling is bound by sense-perception and goes on imagining according to it. Perceiving earth in the earth element, he imagines `earth' as a thing, he imagines `in the earth', `earth is mine', `from the earth' etc. So also with regard to the seen, \emph{diṭṭha}.

But the disciple in training, \emph{sekha}, since he has a higher knowledge of conditionality, although he has not exhausted the influxes and latencies, trains in resisting from the tendency to imagine. An emancipated one, the \emph{arahant}, has fully comprehended the mirage nature of perception.

It seems, therefore, that these forms of \emph{maññanā} enable one to imagine things, attributing a notion of substantiality to sense data. In fact, what we have here is only a heap of imaginings. There is also an attempt to hold on to things imagined. Craving lends a hand to it, and so there is grasping, \emph{upādāna}. Thereby the fact that there are three conditions is ignored or forgotten.

In our analysis of the \emph{Madhupiṇḍikasutta} we came across a highly significant statement:

\begin{quote}
\emph{cakkhuñc'āvuso paṭicca rūpe ca uppajjati cakkhuviññāṇaṁ,}\footnote{M I 111, \emph{Madhupiṇḍikasutta}; see \emph{Sermon 11}}

dependent on the eye and forms, friends, arises eye-consciousness.
\end{quote}

The deepest point in sense perception is already implicit there. This statement clearly indicates that eye-consciousness is dependently arisen. Thereby we are confronted with the question of the two ends and the middle, discussed above.

In fact, what is called eye-consciousness is the very discrimination between eye and form. At whatever moment the eye is distinguished as the internal sphere and form is distinguished as the external sphere, it is then that eye-consciousness arises. That itself is the gap in the middle, the intervening space. Here, then, we have the two ends and the middle.

To facilitate understanding this situation, let us hark back to the simile of the carpenter we brought up in an earlier sermon.\footnote{See \emph{Sermon 10}}

We mentioned that a carpenter, fixing up a door by joining two planks, might speak of the contact between the two planks when his attention is turned to the intervening space, to see how well one plank touches the other. The concept of touching between the two planks came up because the carpenter's attention picked up the two planks as separate and not as one board.

A similar phenomenon is implicit in the statement \emph{cakkhuñca paṭicca rūpe ca uppajjati cakkhuviññāṇaṁ}, ``dependent on eye and forms arises eye-consciousness''. In this perceptual situation, the eye is distinguished from forms. That discrimination itself is consciousness. That is the gap or the interstice, the middle. So here we have the two ends and the middle.

\enlargethispage{\baselineskip}

Eye-contact, from the point of view of Dhamma, is an extremely complex situation. As a matter of fact, it is something that has two ends and a middle. The two ends and the middle belong to it. However, there is a tendency in the world to ignore this middle.

The attempt to tie up the two ends by ignoring the middle is \emph{upādāna} or grasping. That is impelled by craving, \emph{taṇhā}. Due to craving, grasping occurs as a matter of course. It is as if the deer, thinking ``I am here and the water is there, so let me get closer'', starts running towards it. The gap is ignored.

A similar thing happens in the case of sense perception. What impels one to ignore that gap is craving. It is sometimes called \emph{lepa} or glue. With that agglutinative quality in craving the gap is continually sought to be glued up and ignored.

The Buddha has compared craving to a seamstress. The verb \emph{sibbati} or \emph{saṁsibbati} is used to convey the idea of sewing and weaving both. In sewing as well as in weaving, there is an attempt to reduce a gap by stitching up or knitting up. What is called \emph{upādāna}, grasping or holding on, is an attempt to tie up two ends with the help of \emph{taṇhā}, craving or thirst.

In the \emph{Tissametteyyasutta} of the \emph{Pārāyaṇavagga} in the \emph{Sutta Nipāta}, the Buddha shows how one can bypass this seamstress or weaver that is craving and attain emancipation in the following extremely deep verse.

\begin{quote}
\emph{Yo ubh' anta-m-abhiññāya}\\
\emph{majjhe mantā na lippati,}\\
\emph{taṁ brūmi mahāpuriso'ti}\\
\emph{so 'dha sibbanim accagā.} \footnote{Sn 1042, \emph{Tissametteyyamāṇavapucchā}}

He who, having known both ends,\\
With wisdom does not get attached to the middle,\\
Him I call a great man,\\
He has gone beyond the seamstress in this {[}world{]}.
\end{quote}

This verse is so deep and meaningful that already during the lifetime of the Buddha, when he was dwelling at Isipatana in Benares, a group of Elder Monks gathered at the assembly hall and held a symposium on the meaning of this verse.

In the Buddha's time, unlike today, for deep discussions on Dhamma, they took up such deep topics as found in the \emph{Aṭṭhakavagga} and \emph{Pārāyaṇavagga} of the \emph{Sutta Nipāta}. In this case, the topic that came up for discussion, as recorded among the Sixes in the \emph{Aṅguttara Nikāya}, is as follows:

\begin{quote}
\emph{Katamo nu kho, āvuso, eko anto, katamo dutiyo anto, kiṁ majjhe, kā sibbani?}\footnote{A III 399, \emph{Majjhesutta}}

What, friends, is the one end, what is the second end, what is in the middle and who is the seamstress?
\end{quote}

The first venerable Thera, who addressed the assembly of monks on this topic, offered the following explanation:

\begin{quote}
Contact, friends, is one end, arising of contact is the second end, cessation of contact is in the middle, craving is the seamstress, for it is craving that stitches up for the birth of this and that specific existence.

In so far, friends, does a monk understand by higher knowledge what is to be understood by higher knowledge, comprehend by full understanding what is to be comprehended by full understanding. Understanding by higher knowledge what is to be understood by higher knowledge, comprehending by full understanding what is to be comprehended by full understanding, he becomes an ender of suffering in this very life.
\end{quote}

Craving, according to this interpretation, is a seamstress, because it is craving that puts the stitch for existence.

Then a second venerable Thera puts forth his opinion. According to his point of view, the past is one end, the future is the second end, the present is the middle, craving is the seamstress.

A third venerable Thera offered his interpretation. For him, one end is pleasant feeling, the second end is unpleasant or painful feeling, and the middle is neither-unpleasant-nor-pleasant feeling. Craving is again the seamstress.

A fourth venerable Thera opines that the one end is name, the second end is form, the middle is consciousness and the seamstress is craving.

A fifth venerable Thera puts forward the view that the one end is the six internal sense-spheres, the second end is the six external sense-spheres, consciousness is the middle and craving is the seamstress.

A sixth venerable Thera is of the opinion that the one end is \emph{sakkāya}, a term for the five aggregates of grasping, literally the `existing body'. The second end, according to him, is the arising of \emph{sakkāya}. The middle is the cessation of \emph{sakkāya}. As before, the seamstress is craving.

When six explanations had come up before the symposium, one monk suggested, somewhat like a point of order, that since six different interpretations have come up, it would be best to approach the teacher, the Fortunate One, and report the discussion for clarification and correct judgement.

Approving that suggestion, they all went to the Buddha and asked:

\begin{quote}
\emph{Kassa nu kho, bhante, subhāsitaṁ?}

Venerable sir, whose words are well spoken?
\end{quote}

The Buddha replied:

\begin{quote}
Monks, what you all have said is well said from some point of view or other. But that for which I preached that verse in the \emph{Metteyyapañha} is this.
\end{quote}

Quoting the verse in question the Buddha explains:

\begin{quote}
Monks, contact is one end, the arising of contact is the second end, the cessation of contact is in the middle, craving is the seamstress, for it is craving that puts the stitch for the birth of this or that existence.

In so far, monks, does a monk understand by higher knowledge what is to be understood by higher knowledge, and comprehend by full understanding what is to be comprehended by full understanding. Understanding by higher knowledge what is to be understood by higher knowledge, and comprehending by full understanding what is to be comprehended by full understanding, he becomes an ender of suffering in this very life.
\end{quote}

The Buddha's explanation happens to coincide with the interpretation given by the first speaker at the symposium. However, since he ratifies all the six interpretations as well said, we can see how profound and at the same time broad the meaning of this cryptic verse is.

Let us now try to understand these six explanations. One can make use of these six as meditation topics. The verse has a pragmatic value and so also the explanations given. What is the business of this seamstress or weaver?

According to the first interpretation, craving stitches up the first end, contact, with the second end, the arising of contact, ignoring the middle, the cessation of contact. It is beneath this middle, the cessation of contact, that ignorance lurks.

As the line implies: \emph{majjhe mantā na lippati}, ``with wisdom does not get attached to the middle'', when what is in the middle is understood, there is emancipation. One is released from craving. So our special attention should be directed to what lies in the middle, the cessation of contact.

\begin{enumerate}
\def\labelenumi{\arabic{enumi}.}
\item
  Therefore, according to the first interpretation, the seamstress, craving, stitches up contact and the arising of contact, ignoring the cessation of contact.
\item
  According to the second interpretation, the past and the future are stitched up, ignoring the present.
\item
  The third interpretation takes it as a stitching up of unpleasant feeling and pleasant feeling, ignoring the neither-unpleasant-nor-pleasant feeling.
\item
  The fourth interpretation speaks of stitching up name and form, ignoring consciousness.
\item
  For the fifth interpretation, it is a case of stitching up the six internal sense-spheres with the six external sense-spheres, ignoring consciousness.
\item
  In the sixth interpretation, we are told of a stitching up of \emph{sakkāya}, or `existing-body', with the arising of the existing-body, ignoring the cessation of the existing-body.
\end{enumerate}

We mentioned above that in sewing as well as in weaving there is an attempt to reduce a gap by stitching up or knitting up. These interpretations show us that ignoring the middle is a common trait in the worldling. It is there that ignorance lurks. If one rightly understands this middle dispassion sets in, leading to disenchantment, relinquishment and deliverance.

Let us now turn our attention to a few parallel discourses that throw some light on the depth of these meditation topics. We come across two verses in the \emph{Dvayatānupassanāsutta} of the \emph{Sutta Nipāta}, which are relevant to the first interpretation, namely that which concerns contact, the arising of contact and the cessation of contact.

\begin{quote}
\emph{Sukhaṁ vā yadi va dukkhaṁ,}\\
\emph{adukkhamasukhaṁ sahā,}\\
\emph{ajjhattañ ca bahiddhā ca}\\
\emph{yaṁ kiñci atthi veditaṁ,}\\
\emph{etaṁ `dukkhan'ti ñatvāna,}

\emph{mosadhammaṁ palokinaṁ},\\
\emph{phussa phussa vayaṁ passaṁ}\\
\emph{evaṁ tattha virajjati,}\\
\emph{vedanānaṁ khayā bhikkhu,}\\
\emph{nicchāto parinibbuto.}\footnote{Sn 738-739, \emph{Dvayatānupassanāsutta}}

Be it pleasant or unpleasant,\\
Or neither-unpleasant-nor-pleasant,\\
Inwardly or outwardly,\\
All what is felt,\\
Knowing it as `pain',

Delusive and brittle,\\
Touch after touch, seeing how it wanes,\\
That way he grows dispassionate therein,\\
By the extinction of feeling it is\\
That a monk becomes hungerless and fully appeased.
\end{quote}

The following two lines are particularly significant, as they are relevant to the knowledge of `breaking up' in the development of insight.

\begin{quote}
\emph{phussa phussa vayaṁ passaṁ}\\
\emph{evaṁ tattha virajjati.}

Touch after touch, seeing how it wanes,\\
that way he grows dispassionate therein.
\end{quote}

It seems, therefore, that generally the cessation of contact is ignored or slurred over by the worldling's mind, busy with the arising aspect. Therefore the seeing of cessation comes only with the insight knowledge of seeing the breaking up, \emph{bhaṅgañāṇa}.

As an illustration in support of the second interpretation we may quote the following verses from the \emph{Bhaddekarattasutta} of the \emph{Majjhima Nikāya}:

\begin{quote}
\emph{Atītaṁ nānvāgameyya,}\\
\emph{nappaṭikaṅkhe anāgataṁ}\\
\emph{yad atītaṁ pahīnaṁ taṁ}\\
\emph{appattañ ca anāgataṁ.}\\
\emph{Paccuppannañ ca yo dhammaṁ}\\
\emph{tattha tattha vipassati,}\\
\emph{asaṁhīraṁ asaṁkuppaṁ}\\
\emph{taṁ vidvā-m-anubrūhaye.}\footnote{M III 187, \emph{Bhaddekarattasutta}}

Let one not trace back whatever is past,\\
Nor keep on hankering for the not yet come,\\
Whatever is past is gone for good,\\
That which is future is yet to come.\\
But {[}whoever{]} sees that which rises up,\\
As now with insight as and when it comes,\\
Neither `drawing in' nor `pushing on',\\
That kind of stage should the wise cultivate.
\end{quote}

In the reflection on preparations, \emph{saṅkhārā}, in deep insight meditation, it is the present preparations that are presented to reflection. That is why we find the apparently unusual order \emph{atīta -- anāgata -- paccuppanna}, `past -- future -- present', mentioned everywhere in the discourses.

To reflect on past preparations is relatively easy, so also are the future preparations. It is the present preparations that are elusive and difficult to muster. But in deep insight meditation the attention should be on the present preparations. So much is enough for the second interpretation.

The third interpretation has to do with the three grades of feeling, the pleasant, unpleasant and the neither-unpleasant-nor-pleasant. About these we have already discussed at length, on an earlier occasion, in connection with the long dialogue between the Venerable \emph{arahant} nun Dhammadinnā and the lay disciple Visākha on the question of those three grades of feeling. Suffice it for the present to cite the following relevant sections of that dialogue.

\begin{quote}
\emph{Sukhāya vedanāya dukkhā vedanā paṭibhāgo \ldots{}}\\
\emph{dukkhāya vedanāya sukhā vedanā paṭibhāgo \ldots{}}\\
\emph{adukkhamasukhāya vedanāya avijjā paṭibhāgo \ldots{}}\\
\emph{avijjāya vijjā paṭibhāgo \ldots{}}\\
\emph{vijjāya vimutti paṭibhāgo \ldots{}}\\
\emph{vimuttiyā Nibbānaṁ paṭibhāgo.}\footnote{M I 304, \emph{Cūḷavedallasutta}}

Unpleasant feeling is the counterpart of pleasant feeling \ldots{}\\
pleasant feeling is the counterpart of unpleasant feeling \ldots{}\\
ignorance is the counterpart of neither-unpleasant-\\ nor-pleasant feeling \ldots{}\\
knowledge is the counterpart of ignorance \ldots{}\\
deliverance is the counterpart of knowledge \ldots{}\\
Nibbāna is the counterpart of deliverance.
\end{quote}

The counterpart or the `other half' of pleasant feeling is unpleasant feeling. The counterpart of unpleasant feeling is pleasant feeling. Between these two there is a circularity in relationship, a seesawing. There is no way out.

But there is in the middle neither-unpleasant-nor-pleasant feeling. The counterpart of neither-unpleasant-nor-pleasant feeling is ignorance. So we see how the neutrality and indifference of equanimity has beneath it ignorance.

But luckily there is the good side in this pair of counterparts. Deliverance lies that way, for knowledge is the counterpart of ignorance. When ignorance is displaced, knowledge surfaces. From knowledge comes deliverance, and from deliverance Nibbāna or extinction. This much is enough for the third interpretation.

Now for the fourth interpretation. Here we have consciousness between name-and-form. Let us remind ourselves of the two verses quoted in an earlier sermon from the \emph{Dvayatānupassanāsutta} of the \emph{Sutta Nipāta}.

\begin{quote}
\emph{Ye ca rūpūpagā sattā}\\
\emph{ye ca arūpaṭṭhāyino,}\\
\emph{nirodhaṁ appajānantā}\\
\emph{āgantāro punabbhavaṁ.}

\emph{Ye ca rūpe pariññāya,}\\
\emph{arūpesu asaṇṭhitā,}\\
\emph{nirodhe ye vimuccanti,}\\
\emph{te janā maccuhāyino.}\footnote{Sn 754-755, \emph{Dvayatānupassanāsutta}, see \emph{Sermon 15}}

Those beings that go to realms of form,\\
And those who are settled in formless realms,\\
Not understanding the fact of cessation,\\
Come back again and again to existence.

Those who, having comprehended realms of form,\\
Do not settle in formless realms,\\
Are released in the experience of cessation,\\
It is they that are the dispellers of death.
\end{quote}

The cessation here referred to is the cessation of consciousness, or the cessation of becoming. Such emancipated ones are called `dispellers of death', \emph{maccuhāyino}.

\enlargethispage{\baselineskip}

We have mentioned earlier that, before the advent of the Buddha and even afterwards, sages like Āḷāra Kālāma tried to escape form, \emph{rūpa}, by grasping the formless, \emph{arūpa}. But only the Buddha could point out that one cannot win release from form by resorting to the formless. Release from both should be the aim.

How could that come about? By the cessation of consciousness which discriminates between form and formless. It is tantamount to the cessation of existence, \emph{bhavanirodha}.

As a little hint to understand this deep point, we may allude to that simile of the dog on the plank across the stream which we brought up several times. Why does that dog keep on looking at the dog it sees in the water, its own reflection? Because it is unaware of the reflexive quality of the water.

Consciousness is like that water which has the quality of reflecting on its surface. What is there between the \textbf{seen} dog and the \textbf{looking} dog as the middle is consciousness itself. One can therefore understand why consciousness is said to be in the middle between name and form.

Generally, in the traditional analysis of the relation between name-and-form and consciousness, this fact is overlooked. True to the simile of the magical illusion, given to consciousness, its middle position between name and form is difficult for one to understand. Had the dog understood the reflective quality of water, it would not halt on that plank to gaze down and growl.

The fifth interpretation puts the six internal sense-spheres and the six external sense-spheres on either side, to have consciousness in the middle. A brief explanation would suffice.

\begin{quote}
\emph{Dvayaṁ, bhikkhave, paṭicca viññāṇaṁ sambhoti},\footnote{S IV 67, \emph{Dutiyadvayasutta}}

monks, dependent on a dyad consciousness arises,
\end{quote}

\ldots{} says the Buddha. That is to say, dependent on internal and external sense-spheres consciousness arises. As we have already pointed out, consciousness is the very discrimination between the two. Therefore consciousness is the middle.

So at the moment when one understands consciousness, one realizes that the fault lies in this discrimination itself. The farther limit of the internal is the nearer limit of the external. One understands then that the gap, the interstice between them, is something imagined.

Then as to the sixth interpretation, we have the \emph{sakkāya}, the `existing body', and \emph{sakkāyasamudaya}, the arising of the existing body, as the two ends.

Because the term \emph{sakkāya} is not often met with, it might be difficult to understand what it means. To be brief, the Buddha has defined the term as referring to the five aggregates of grasping.\footnote{S III 159, \emph{Sakkāyasutta}} Its derivation, \emph{sat + kāya}, indicates that the term is suggestive of the tendency to take the whole group as existing, giving way to the perception of the compact, \emph{ghanasaññā}.

The arising of this notion of an existing body is \emph{chandarāga} or desire and lust. It is due to desire or craving that one grasps a heap as a compact whole. The cessation of the existing body is the abandonment of desire and lust. This, then, is a summary of the salient points in these six interpretations as meditation topics for realization.

Let us now turn our attention to the sewing and weaving spoken of here. We have mentioned above that both in sewing and weaving a knotting comes in, as a way of reducing the gap. This knotting involves some kind of attracting, binding and entangling. In the case of a sewing machine, every time the needle goes down, the shuttle hastens to put a knot for the stitch. So long as this attraction continues, the stitching goes on.

There is some relation between sewing and weaving. Sewing is an attempt to put together two folds. In weaving a single thread of cotton or wool is looped into two folds. In both there is a formation of knots. As already mentioned, knots are formed by some sort of attraction, binding and entangling.

Now craving is the seamstress who puts the stitches to this existence, \emph{bhava}. She has a long line of qualifications for it. \emph{Ponobhavikā nandirāgasahagatā tatratatrābhinandinī} are some of the epithets for craving.

She is the perpetrator in re-becoming or renewed existence, \emph{ponobhavikā}, bringing about birth after birth. She has a trait of delighting and lusting, \emph{nandirāgasahagatā}. Notoriously licentious she delights now here now there, \emph{tatratatrābhinandinī}. Like that seamstress, craving puts the stitches into existence, even as the needle and the shuttle.

Craving draws in with \emph{upādāna}, grasping, while conceit binds and views complete the entanglement. That is how existence gets stitched up.

At whatever moment the shuttle runs out of its load of cotton, the apparent stitches do not result in a seam. Similarly, in a weaving, if instead of drawing in the thread to complete the knot it is drawn out, all what is woven will be undone immediately. This is the difference between existence and its cessation. Existence is a formation of knots and stress. Cessation is an unravelling of knots and rest.

\textbf{Existence is a formation of knots and stress. Cessation is an unravelling of knots and rest.}

The following verse in the \emph{Suddhaṭṭhakasutta} of the \emph{Sutta Nipāta} seems to put in a nutshell the philosophy behind the simile of the seamstress.

\begin{quote}
\emph{Na kappayanti na purekkharonti}\\
\emph{`accantasuddhī' ti na te vadanti,}\\
\emph{ādānaganthaṁ gathitaṁ visajja,}\\
\emph{āsaṁ na kubbanti kuhiñci loke.}\footnote{Sn 794, \emph{Suddhaṭṭhakasutta}}

They fabricate not, they proffer not,\\
Nor do they speak of a `highest purity',\\
Unravelling the tangled knot of grasping,\\
They form no desire anywhere in the world.
\end{quote}

The comments we have presented here, based on the verse beginning with \emph{yo ubh' anta-m-abhiññāya} could even be offered as a synopsis of the entire series of thirty-three sermons.

All what we brought up in these sermons concerns the question of the two ends and the middle. The episode of the two ends and the middle enshrines a profound insight into the law of dependent arising and the Buddha's teachings on the middle path. That is why we said that the verse in question is both profound and broad, as far as its meaning is concerned.

So now that we have presented this synoptic verse, we propose to wind up this series of sermons.

As a matter of fact, the reason for many a misconception about Nibbāna is a lack of understanding the law of dependent arising and the middle path. For the same reason, true to the Buddha's description of beings as taking delight in existence, \emph{bhavarāmā}, lusting for existence, \emph{bhavaratā}, and rejoicing in existence, \emph{bhavasammuditā},\footnote{It 43, \emph{Diṭṭhigatasutta}} Nibbāna came to be apprehensively misconstrued as tantamount to annihilation.

Therefore even commentators were scared of the prospect of a cessation of existence and tried to explain away Nibbāna through definitions that serve to perpetuate craving for existence.

If by this attempt of ours to clear the path to Nibbāna, overgrown as it is through neglect for many centuries, due to various social upheavals, any store of merit accrued to us, may it duly go to our most venerable Great Preceptor, who so magnanimously made the invitation to deliver this series of sermons. As he is staying away for medical treatment at this moment, aged and ailing, let us wish him quick recovery and long life. May all his Dhamma aspirations be fulfilled!

May the devoted efforts in meditation of all those fellow dwellers in this holy life, who listened to these sermons and taped them for the benefit of those who would like to lend ear to them, be rewarded with success! Let a myriad \emph{arahant} lotuses, unsmeared by water and mud, bright petalled and sweet scented, bloom all over the forest hermitage pond. May the merits accrued by giving these sermons be shared by my departed parents, who brought me up, my teachers, who gave me vision, and my friends, relatives and lay supporters, who helped keep this frail body alive. May they all attain the bliss of Nibbāna!

May all gods and Brahmās and all beings rejoice in the merits accrued by these sermons! May it conduce to the attainment of that peaceful and excellent Nibbāna! May the dispensation of the Fully Enlightened One endure long in this world! Let this garland of well preached Dhamma words be a humble offering at the foot of the Dhamma shrine, which received honour and worship even from the Buddha himself.
