\chapter{Sermon 4}

\NibbanaOpeningQuote

With the permission of the Most Venerable Great Preceptor and the assembly of the venerable meditative monks.

Towards the end of the last sermon, we were trying to explain how the process of the \emph{saṁsāric} journey of beings could be understood even with the couple of terms \emph{itthabhāva} and \emph{aññatthābhāva}, or this-ness and otherwise-ness.\footnote{See \emph{Sermon 3}} On an earlier occasion, we happened to quote the following verse in the \emph{Sutta Nipāta}:

\begin{quote}
\emph{Taṇhā dutiyo puriso,}\\
\emph{dīghamaddhāna saṁsāraṁ,}\\
\emph{itthabhāvaññathābhāvaṁ,}\\
\emph{saṁsāraṁ nātivattati}.\footnote{\href{https://suttacentral.net/snp3.12/pli/ms}{Snp 3.12 / Sn 740}, \emph{Dvayatānupassanāsutta}; see also \emph{Sermon 2}, \emph{Taṇhā dutiyo puriso\ldots{}}}
\end{quote}

It means: ``The man with craving as his second'', or ``as his companion'', ``faring on for a long time in \emph{saṁsāra}, does not transcend the round, which is of the nature of a this-ness and an otherwise-ness.''

This is further proof that the two terms imply a circuit. It is a circuit between a `here' and a `there', or a `this-ness' and an `otherwise-ness'. It is a turning round, an alternation or a circuitous journey. It is like a rotation on the spot. It is an ambivalence between a here and a there.

It is the relationship between this this-ness and otherwise-ness that we tried to illustrate with quotations from the suttas. We mentioned in particular that consciousness, when it leaves this body and gets well established on a preconceived object, which in fact is its name-and-form object, that name-and-form attains growth and maturity there itself.\footnote{See \emph{Sermon 3}} Obviously, therefore, name-and-form is a necessary condition for the sustenance and growth of consciousness in a mother's womb.

It should be clearly understood that the passage of consciousness from here to a mother's womb is not a movement from one place to another, as in the case of the body. In reality, it is only a difference of point of view, and not a transmigration of a soul. In other words, when consciousness leaves this body and comes to stay in a mother's womb, when it is fully established there, `that' place becomes a `this' place. From the point of view of that consciousness, the `there' becomes a `here'. Consequently, from the new point of view, what was earlier a `here', becomes a `there'. What was formerly `that place' has now become `this place' and vice versa. That way, what actually is involved here, is a change of point of view. So it does not mean completely leaving one place and going to another, as is usually meant by the journey of an individual.

The process, then, is a sort of going round and round. This is all the more clear by the Buddha's statement that even consciousness is dependently arisen. There are instances in which the view that this selfsame consciousness fares on in \emph{saṁsāra} by itself, \emph{tadevidaṁ viññāṇaṁ sandhāvati saṁsarati, anaññaṁ}, is refuted as a wrong view.\footnote{\href{https://suttacentral.net/mn38/pli/ms}{MN 38 / M I 256}, \emph{Mahātaṇhāsaṅkhayasutta}}

On the one hand, for the sustenance and growth of name-and-form in a mother's womb, consciousness is necessary. On the other hand, consciousness necessarily requires an object for its stability. It could be some times an intention, or else a thought construct. In the least, it needs a trace of latency, or \emph{anusaya}. This fact is clear enough from the sutta quotations we brought up towards the end of the previous sermon. From the \emph{Cetanāsutta}, we happened to quote on an earlier occasion, it is obvious that at least a trace of latency is necessary for the sustenance of consciousness.\footnote{See \emph{Sermon 3}}

When consciousness gets established in a mother's womb, with this condition in the least, name-and-form begins to grow. It grows, at it were, with a flush of branches, in the form of the six sense bases, to produce a fresh tree of suffering. It is this idea that is voiced by the following well known verse in the \emph{Dhammapada}:

\begin{quote}
\emph{Yathāpi mūle anupaddave daḷhe}\\
\emph{chinno pi rukkho punareva rūhati}\\
\emph{evam pi taṇhānusaye anūhate}\\
\emph{nibbattati dukkham idaṁ punappunaṁ}.\footnote{\href{https://suttacentral.net/dhp334-359/pli/ms}{Dhp 338}, \emph{Taṇhāvagga}}

Just as a tree, so long as its root is unharmed and firm,\\
Though once cut down, will none the less grow up again,\\
Even so, when craving's latency is not yet rooted out,\\
This suffering gets reborn again and again.
\end{quote}

It is clear from this verse too that the latency to craving holds a very significant place in the context of the \emph{saṁsāric} journey of a being. In the \emph{Aṅguttara Nikāya} one comes across the following statement by the Buddha:

\begin{quote}
\emph{Kammaṁ khettaṁ, viññāṇaṁ bījaṁ, taṇhā sineho}.\footnote{\href{https://suttacentral.net/an3.76/pli/ms}{AN 3.76 / A I 223}, \emph{Paṭhamabhavasutta}}

\emph{Kamma} is the field, consciousness is the seed, craving is the moisture.
\end{quote}

This, in effect, means that consciousness grows in the field of \emph{kamma} with craving as the moisture.

It is in accordance with this idea and in the context of this particular simile that we have to interpret the reply of Selā Therī to a question raised by Māra. In the \emph{Sagātha Vagga} of the \emph{Saṁyutta Nikāya} one comes across the following riddle put by Māra to the \emph{arahant} nun Selā:

\begin{quote}
\emph{Ken'idaṁ pakataṁ bimbaṁ,}\\
\emph{ko nu bimbassa kārako,}\\
\emph{kvannu bimbaṁ samuppannaṁ,}\\
\emph{kvannu bimbaṁ nirujjhati?}\footnote{\href{https://suttacentral.net/sn5.9/pli/ms}{SN 5.9 / S I 134}, \emph{Selāsutta}}

By whom was this image wrought,\\
Who is the maker of this image,\\
Where has this image arisen,\\
And where does the image cease?
\end{quote}

The image meant here is one's body, or one's outward appearance which, for the conventional world, is name-and-form. Selā Therī gives her answer in three verses:

\begin{quote}
\emph{Nayidaṁ attakataṁ bimbaṁ,}\\
\emph{nayidaṁ parakataṁ aghaṁ,}\\
\emph{hetuṁ paṭicca sambhūtaṁ,}\\
\emph{hetubhaṅgā nirujjhati.}

\emph{Yathā aññataraṁ bījaṁ,}\\
\emph{khette vuttaṁ virūhati,}\\
\emph{pathavīrasañcāgamma,}\\
\emph{sinehañca tadūbhayaṁ.}

\emph{Evaṁ khandhā ca dhātuyo,}\\
\emph{cha ca āyatanā ime,}\\
\emph{hetuṁ paṭicca sambhūtā,}\\
\emph{hetubhaṅgā nirujjhare.}

Neither self-wrought is this image,\\
Nor yet other-wrought is this misery,\\
By reason of a cause, it came to be,\\
By breaking up the cause, it ceases to be.

Just as in the case of a certain seed,\\
Which when sown on the field would feed\\
On the taste of the earth and moisture,\\
And by these two would grow.

Even so, all these aggregates\\
Elements and bases six,\\
By reason of a cause have come to be,\\
By breaking up the cause will cease to be.
\end{quote}

The first verse negates the idea of creation and expresses the conditionally arisen nature of this body. The simile given in the second verse illustrates this law of dependent arising. It may be pointed out that this simile is not one chosen at random. It echoes the idea behind the Buddha's statement already quoted, \emph{kammaṁ khettaṁ, viññāṇaṁ bījaṁ, taṇhā sineho}. \emph{Kamma} is the field, consciousness the seed, and craving the moisture.

Here the venerable Therī is replying from the point of view of Dhamma, which takes into account the mental aspect as well. It is not simply the outward visible image, as commonly understood by \emph{nāma-rūpa}, but that image which falls on consciousness as its object. The reason for the arising and growth of \emph{nāma-rūpa} is therefore the seed of consciousness. That consciousness seed grows in the field of \emph{kamma}, with craving as the moisture. The outgrowth is in terms of aggregates, elements and bases. The cessation of consciousness is none other than Nibbāna.

Some seem to think that the cessation of consciousness occurs in an \emph{arahant} only at the moment of his \emph{parinibbāna}, at the end of his life span. But this is not the case. Very often, the deeper meanings of important suttas have been obliterated by the tendency to interpret the references to consciousness in such contexts as the final occurrence of consciousness in an \emph{arahant's} life -- \emph{carimaka viññāṇa}.\footnote{E.g. at Sv-pṭ I 513}

What is called the cessation of consciousness has a deeper sense here. It means the cessation of the specifically prepared consciousness, \emph{abhisaṅkhata viññāṇa}. An \emph{arahant's} experience of the cessation of consciousness is at the same time the experience of the cessation of name-and-form. Therefore, we can attribute a deeper significance to the above verses.

In support of this interpretation, we can quote the following verse in the \emph{Munisutta} of the \emph{Sutta Nipāta}:

\begin{quote}
\emph{Saṅkhāya vatthūni pamāya bījaṁ,}\\
\emph{sineham assa nānuppavecche,}\\
\emph{sa ve munī jātikhayantadassī,}\\
\emph{takkaṁ pahāya na upeti saṅkhaṁ}.\footnote{\href{https://suttacentral.net/snp1.12/pli/ms}{Snp 1.12 / Sn 209}, \emph{Munisutta}}

Having surveyed the field and measured the seed,\\
He waters it not for moisture,\\
That sage in full view of birth's end,\\
Lets go of logic and comes not within reckoning.
\end{quote}

By virtue of his masterly knowledge of the fields and his estimate of the seed of consciousness, he does not moisten it with craving. Thereby he sees the end of birth and transcends logic and worldly convention. This too shows that the deeper implications of the \emph{Mahānidānasutta}, concerning the descent of consciousness into the mother's womb, have not been sufficiently appreciated so far.

\emph{Anusaya}, or latency, is a word of special significance. What is responsible for rebirth, or \emph{punabbhava}, is craving, which very often has the epithet \emph{ponobhavikā} attached to it. The latency to craving is particularly instrumental in giving one yet another birth to fare on in \emph{saṁsāra}. There is also a tendency to ignorance, which forms the basis of the latency to craving. It is the tendency to get attached to worldly concepts, without understanding them for what they are. That tendency is a result of ignorance in the worldlings and it is in itself a latency. In the sutta terminology the word \emph{nissaya} is often used to denote it. The cognate word \emph{nissita} is also used alongside. It means `one who associates something', while \emph{nissaya} means `association'.

As a matter of fact, here it does not have the same sense as the word has in its common usage. It goes deeper, to convey the idea of `leaning on' something. Leaning on is also a form of association. Worldlings have a tendency to tenaciously grasp the concepts in worldly usage, to cling to them dogmatically and lean on them. They believe that the words they use have a reality of their own, that they are categorically true in their own right. Their attitude towards concepts is tinctured by craving, conceit and views.

We come across this word \emph{nissita} in quite a number of important suttas. It almost sounds like a topic of meditation. In the \emph{Channovādasutta} of the \emph{Majjhima Nikāya} there is a cryptic passage, which at a glance looks more or less like a riddle:

\begin{quote}
\emph{Nissitassa calitaṁ, anissitassa calitaṁ natthi. Calite asati passaddhi, passaddhiyā sati nati na hoti, natiyā asati āgatigati na hoti, āgatigatiyā asati cutūpapāto na hoti, cutūpapāte asati nev'idha na huraṁ na ubhayamantare. Es' ev' anto dukhassa}.\footnote{\href{https://suttacentral.net/mn144/pli/ms}{MN 144 / M III 266}, \emph{Channovādasutta}}

To the one attached, there is wavering. To the unattached one, there is no wavering. When there is no wavering, there is calm. When there is calm, there is no inclination. When there is no inclination, there is no coming and going. When there is no coming and going, there is no death and birth. When there is no death and birth, there is neither a `here' nor a `there' nor a `between the two'. This itself is the end of suffering.
\end{quote}

It looks as if the ending of suffering is easy enough. On the face of it, the passage seems to convey this much. To the one who leans on something, there is wavering or movement. He is perturbable. Though the first sentence speaks about the one attached, the rest of the passage is about the unattached one. That is to say, the one released.

So here we see the distinction between the two. The one attached is movable, whereas the unattached one is not. When there is no wavering or perturbation, there is calm. When there is calm, there is no inclination. The word \emph{nati} usually means `bending'. So when there is calm, there is no bending or inclination. When there is no bending or inclination, there is no coming and going. When there is no coming and going, there is no passing away or reappearing. When there is neither a passing away nor a reappearing, there is neither a `here', nor a `there', nor any position in between. This itself is the end of suffering.

The sutta passage, at a glance, appears like a jumble of words. It starts by saying something about the one attached, \emph{nissita}. It is limited to just one sentence: `To one attached, there is wavering.' But we can infer that, due to his wavering and unsteadiness or restlessness, there is inclination, \emph{nati}. The key word of the passage is \emph{nati}. Because of that inclination or bent, there is a coming and going. Given the twin concept of coming and going, there is the dichotomy between passing away and reappearing, \emph{cuti/uppatti}. When these two are there, the two concepts `here' and `there' also come in. And there is a `between the two' as well. Wherever there are two ends, there is also a middle. So it seems that in this particular context the word \emph{nati} has a special significance.

The person who is attached is quite unlike the released person. Because he is not released, he always has a forward bent or inclination. In fact, this is the nature of craving. It bends one forward. In some suttas dealing with the question of rebirth, such as the \emph{Kutūhalasālāsutta}, craving itself is sometimes called the grasping, \emph{upādāna}.\footnote{\href{https://suttacentral.net/sn44.9/pli/ms}{SN 44.9 / S IV 400}, \emph{Kutūhalasālāsutta}: `\emph{taṇhupādāna}'} So it is due to this very inclination or bent that the two concepts of coming and going, come in. Then, in accordance with them, the two concepts of passing away and reappearing, fall into place.

The idea of a journey, when viewed in the context of \emph{saṁsāra}, gives rise to the idea of passing away and reappearing. Going and coming are similar to passing away and reappearing. So then, there is the implication of two places, all this indicates an attachment. There is a certain dichotomy about the terms here and there, and passing away and reappearing. Due to that dichotomous nature of the concepts, which beings tenaciously hold on to, the journeying in \emph{saṁsāra} takes place in accordance with craving. As we have mentioned above, an alternation or transition occurs.

As for the released person, about whom the passage is specially concerned, his mind is free from all those conditions. To the unattached, there is no wavering. Since he has no wavering or unsteadiness, he has no inclination. As he has no inclination, there is no coming and going for him. As there is no coming and going, he has no passing away or reappearing. There being no passing away or reappearing, there is neither a here, nor a there, nor any in between. That itself is the end of suffering.

The \emph{Udāna} version of the above passage has something significant about it. There the entire sutta consists of these few sentences. But the introductory part of it says that the Buddha was instructing, inciting and gladdening the monks with a Dhamma talk connected with Nibbāna:

\begin{quote}
\emph{Tena kho pana samayena Bhagavā bhikkhū nibbānapaṭisaṁyuttāya dhammiyā kathāya sandasseti samādapeti samuttejeti sampahaṁseti}.\footnote{\href{https://suttacentral.net/ud8.4/pli/ms}{Ud 8.4 / Ud 81}, \emph{Catutthanibbānapaṭisaṁyuttasutta}}
\end{quote}

This is a pointer to the fact that this sermon is on Nibbāna. So the implication is that in Nibbāna the \emph{arahant's} mind is free from any attachments.

There is a discourse in the \emph{Nidāna} section of the \emph{Saṁyutta Nikāya}, which affords us a deeper insight into the meaning of the word \emph{nissaya}. It is the \emph{Kaccāyanagottasutta}, which is also significant for its deeper analysis of right view. This is how the Buddha introduces the sermon:

\begin{quote}
\emph{Dvayanissito khvāyaṁ, Kaccāyana, loko yebhuyyena: atthitañceva natthitañca. Lokasamudayaṁ kho, Kaccāyana, yathābhūtaṁ sammappaññāya passato yā loke natthitā sā na hoti. Lokanirodhaṁ kho, Kaccāyana, yathābhūtaṁ sammappaññāya passato yā loke atthitā sā na hoti}.\footnote{\href{https://suttacentral.net/sn12.15/pli/ms}{SN 12.15 / S II 17}, \emph{Kaccāyanagottasutta}}

This world, Kaccāyana, for the most part, bases its views on two things: on existence and non-existence. Now, Kaccāyana, to one who with right wisdom sees the arising of the world as it is, the view of non-existence regarding the world does not occur. And to one who with right wisdom sees the cessation of the world as it really is, the view of existence regarding the world does not occur.
\end{quote}

The Buddha comes out with this discourse in answer to the following question raised by the \emph{brahmin} Kaccāyana:

\begin{quote}
\emph{Sammā diṭṭhi, sammā diṭṭhī'ti, bhante, vuccati. Kittāvatā nu kho, bhante, sammā diṭṭhi hoti?}

Lord, `right view', `right view', they say. But how far, Lord, is there `right view'?
\end{quote}

In his answer, the Buddha first points out that the worldlings mostly base themselves on a duality, the two conflicting views of existence and non-existence, or `is' and `is not'. They would either hold on to the dogmatic view of eternalism, or would cling to nihilism. Now as to the right view of the noble disciple, it takes into account the process of arising as well as the process of cessation, and thereby avoids both extremes. This is the insight that illuminates the middle path.

Then the Buddha goes on to give a more detailed explanation of right view:

\begin{quote}
\emph{Upayupādānābhinivesavinibandho khvāyaṁ, Kaccāyana, loko yebhuyyena. Tañcāyaṁ upayupādānaṁ cetaso adhiṭṭhānaṁ abhinivesānusayaṁ na upeti na upādiyati nādhiṭṭhāti: `attā me'ti. `Dukkham eva uppajjamānaṁ uppajjati, dukkhaṁ nirujjhamānaṁ nirujjhatī'ti na kaṅkhati na vicikicchati aparapaccayā ñāṇam ev' assa ettha hoti. Ettāvatā kho, Kaccāyana, sammā diṭṭhi hoti.}

The world, Kaccāyana, for the most part, is given to approaching, grasping, entering into and getting entangled as regards views. Whoever does not approach, grasp, and take his stand upon that proclivity towards approaching and grasping, that mental standpoint, namely the idea: `This is my soul', he knows that what arises is just suffering and what ceases is just suffering. Thus, he is not in doubt, is not perplexed, and herein he has the knowledge that is not dependent on another. Thus far, Kaccāyana, he has right view.
\end{quote}

The passage starts with a string of terms which has a deep philosophical significance. \emph{Upaya} means `approaching', \emph{upādāna} is `grasping', \emph{abhinivesa} is `entering into', and \emph{vinibandha} is the consequent entanglement. The implication is that the worldling is prone to dogmatic involvement in concepts through the stages mentioned above in an ascending order.

The attitude of the noble disciple is then outlined in contrast to the above dogmatic approach, and what follows after it. As for him, he does not approach, grasp, or take up the standpoint of a self.

The word \emph{anusaya}, latency or `lying dormant', is also brought in here to show that even the proclivity towards such a dogmatic involvement with a soul or self, is not there in the noble disciple. But what, then, is his point of view? What arises and ceases is nothing but suffering. There is no soul or self to lose, it is only a question of arising and ceasing of suffering. This, then, is the right view.

Thereafter the Buddha summarizes the discourse and brings it to a climax with an impressive declaration of his via media, the middle path based on the formula of dependent arising:

\begin{quote}
\emph{`Sabbam atthī'ti kho, Kaccāyana, ayam eko anto. `Sabbaṁ natthī'ti ayaṁ dutiyo anto. Ete te, Kaccāyana, ubho ante anupagamma majjhena Tathāgato Dhammaṁ deseti:}

\emph{Avijjāpaccayā saṅkhārā, saṅkhārapaccayā viññāṇaṁ, viññāṇapaccayā nāmarūpaṁ, nāmarūpapaccayā saḷāyatanaṁ, saḷāyatanapaccayā phasso, phassapaccayā vedanā, vedanāpaccayā taṇhā, taṇhāpaccayā upādānaṁ, upādānapaccayā bhavo, bhavapaccayā jāti, jātipaccayā jarāmaraṇaṁ sokaparidevadukkhadomanassūpāyāsā sambhavanti. Evametassa kevalassa dukkhakkhandhassa samudayo hoti.}

\emph{Avijjāya tveva asesavirāganirodhā saṅkhāranirodho, saṅkharanirodhā viññāṇanirodho, viññāṇanirodhā nāmarūpanirodho, nāmarūpanirodhā saḷāyatananirodho, saḷāyatananirodhā phassanirodho, phassanirodhā vedanānirodho, vedanānirodhā taṇhānirodho, taṇhānirodhā upādānanirodho, upādānanirodhā bhavanirodho, bhavanirodhā jātinirodho, jātinirodhā jarāmaraṇaṁ sokaparidevadukkhadomanassūpāyāsā nirujjhanti. Evametassa kevalassa dukkhakkhandhassa nirodho hoti}.

`Everything exists', Kaccāyana, is one extreme. `Nothing exists' is the other extreme. Not approaching either of those extremes, Kaccāyana, the Tathāgata teaches the Dhamma by the middle way:

From ignorance as condition, preparations come to be; from preparations as condition, consciousness comes to be; from consciousness as condition, name-and-form comes to be; from name-and-form as condition, the six sense-bases come to be; from the six sense-bases as condition, contact comes to be; from contact as condition, feeling comes to be; from feeling as condition, craving comes to be; from craving as condition, grasping comes to be; from grasping as condition, becoming comes to be; from becoming as condition, birth comes to be; and from birth as condition, decay-and-death, sorrow, lamentation, pain, grief and despair come to be. Such is the arising of this entire mass of suffering.

From the complete fading away and cessation of that very ignorance, there comes to be the cessation of preparations; from the cessation of preparations, there comes to be the cessation of consciousness; from the cessation of consciousness, there comes to be the cessation of name-and-form; from the cessation of name-and-form, there comes to be the cessation of the six sense-bases; from the cessation of the six sense-bases, there comes to be the cessation of contact; from the cessation of contact, there comes to be the cessation of feeling; from the cessation of feeling, there comes to be the cessation of craving; from the cessation of craving, there comes to be the cessation of grasping; from the cessation of grasping, there comes to be the cessation of becoming; from the cessation of becoming, there comes to be the cessation of birth; and from the cessation of birth, there comes to be the cessation of decay-and-death, sorrow, lamentation, pain, grief and despair. Such is the cessation of this entire mass of suffering.
\end{quote}

It is clear from this declaration that in this context the law of dependent arising itself is called the middle path. Some prefer to call this the Buddha's metaphysical middle path, as it avoids both extremes of `is' and `is not'. The philosophical implications of the above passage lead to the conclusion that the law of dependent arising enshrines a certain pragmatic principle, which dissolves the antinomian conflict in the world.

It is the insight into this principle that basically distinguishes the noble disciple, who sums it up in the two words \emph{samudayo}, arising, and \emph{nirodho}, ceasing. The arising and ceasing of the world is for him a fact of experience, a knowledge. It is in this light that we have to understand the phrase:

\begin{quote}
\emph{aparappaccayā ñāṇam ev'assa ettha hoti}

herein he has a knowledge that is not dependent on another.
\end{quote}

In other words, he is not believing in it out of faith in someone, but has understood it experientially. The noble disciple sees the arising and the cessation of the world through his own six sense bases.

In the \emph{Saṁyutta Nikāya} there is a verse which presents this idea in a striking manner:

\begin{quote}
\emph{Chasu loko samuppanno,}\\
\emph{chasu kubbati santhavaṁ,}\\
\emph{channam eva upādāya,}\\
\emph{chasu loko vihaññati}.\footnote{\href{https://suttacentral.net/sn1.70/pli/ms}{SN 1.70 / S I 41}, \emph{Lokasutta}}

In the six the world arose,\\
In the six it holds concourse,\\
On the six themselves depending,\\
In the six it has its woes.
\end{quote}

The verse seems to say that the world has arisen in the six, that it has associations in the six, and that depending on those very six, the world comes to grief.

Though the commentators advance an interpretation of this six, it does not seem to get the sanction of the sutta as it is. According to them, the first line speaks of the six internal sense bases, such as the eye, ear and nose.\footnote{Spk I 96} The world is said to arise in these six internal sense bases. The second line is supposed to refer to the six external sense bases. Again the third line is interpreted with reference to the six internal sense bases, and the fourth line is said to refer to the six external sense bases.

In other words, the implication is that the world arises in the six internal sense bases and associates with the six external sense bases, and that it holds on to the six internal sense bases and comes to grief in the six external sense bases.

This interpretation seems to miss the point. Even the grammar does not allow it, for if it is a case of associating `with' the external sense bases, the instrumental case would have been used instead of the locative case, that is, \emph{chahi} instead of \emph{chasu}. On the other hand, the locative \emph{chasu} occurs in all the three lines in question. This makes it implausible that the first two lines are referring to two different groups of sixes.

It is more plausible to conclude that the reference is to the six sense bases of contact, \emph{phassāyatana}, which include both the internal and the external. In fact, at least two are necessary for something to be dependently arisen. The world does not arise in the six internal bases in isolation. It is precisely in this fact that the depth of this Dhamma is to be seen.

In the \emph{Samudayasutta} of the \emph{Saḷāyatana} section in the \emph{Saṁyutta Nikāya} this aspect of dependent arising is clearly brought out:

\begin{quote}
\emph{Cakkhuñca paṭicca rūpe ca uppajjati cakkhuviññāṇaṁ, tiṇṇaṁ saṅgati phasso, phassapaccayā vedanā, vedanāpaccayā taṇhā, taṇhāpaccayā upādānaṁ, upādānapaccayā bhavo, bhavapaccayā jāti, jātipaccayā jarāmaraṇaṁ sokaparidevadukkhadomanassūpāyāsā sambhavanti. Evametassa kevalassa dukkhakkhandhassa samudayo hoti}.\footnote{\href{https://suttacentral.net/sn35.106/pli/ms}{SN 35.106 / S IV 86}, \emph{Dukkhasutta}}

Dependent on the eye and forms arises eye consciousness; the coming together of the three is contact; with contact as condition, arises feeling; conditioned by feeling , craving; conditioned by craving, grasping; conditioned by grasping, becoming; conditioned by becoming, birth; and conditioned by birth, decay-and-death, sorrow, lamentation, pain, grief and despair. Thus is the arising of this entire mass of suffering.
\end{quote}

Here the sutta starts with the arising of contact and branches off towards the standard formula of \emph{paṭicca samuppāda}. Eye consciousness arises dependent on, \emph{paṭicca}, two things, namely eye and forms. And the concurrence of the three is contact. This shows that two are necessary for a thing to be dependently arisen.

So in fairness to the sutta version, we have to conclude that the reference in all the four lines is to the bases of contact, comprising both the internal and the external. That is to say, we cannot discriminate between them and assert that the first line refers to one set of six, and the second line refers to another. We are forced to such a conclusion in fairness to the sutta.

So from this verse also we can see that according to the usage of the noble ones the world arises in the six sense bases. This fact is quite often expressed by the phrase \emph{ariyassa vinaye loko}, the world in the noble one's discipline.\footnote{\href{https://suttacentral.net/sn35.116/pli/ms}{SN 35.116 / S IV 95}, \emph{Lokakāmaguṇasutta}} According to this noble usage, the world is always defined in terms of the six sense bases, as if the world arises because of these six sense bases. This is a very deep idea. All other teachings in this Dhamma will get obscured, if one fails to understand this basic fact, namely how the concept of the world is defined in this mode of noble usage.

This noble usage reveals to us the implications of the expression \emph{udayatthagāminī paññā}, the wisdom that sees the rise and fall. About the noble disciple it is said that he is endowed with the noble penetrative wisdom of seeing the rise and fall, \emph{udayatthagāminiyā paññāya sammanāgato ariyāya nibbhedikāya}.\footnote{E.g. at \href{https://suttacentral.net/dn33/pli/ms}{DN 33 / D III 237}, \emph{Sangītisutta}} The implication is that this noble wisdom has a penetrative quality about it. This penetration is through the rigidly grasped almost impenetrable encrustation of the two dogmatic views in the world, existence and non-existence.

Now, how does that penetration come about? As already stated in the above quoted \emph{Kaccāyanasutta}, when one sees the arising aspect of the world, one finds it impossible to hold the view that nothing exists in the world. His mind does not incline towards a dogmatic involvement with that view. Similarly, when he sees the cessation of the world through his own six sense bases, he sees no possibility to go to the other extreme view in the world: `Everything exists'.

The most basic feature of this principle of dependent arising, with its penetrative quality, is the breaking down of the power of the above concepts. It is the very inability to grasp these views dogmatically that is spoken of as the abandonment of the personality view, \emph{sakkāyadiṭṭhi}. The ordinary worldling is under the impression that things exist in truth and fact, but the noble disciple, because of his insight into the norm of arising and cessation, understands the arising and ceasing nature of concepts and their essencelessness or insubstantiality.

Another aspect of the same thing, in addition to what has already been said about \emph{nissaya}, is the understanding of the relatedness of this to that, \emph{idappaccayatā}, implicit in the law of dependent arising. In fact, we began our discussion by highlighting the significance of the term \emph{idappaccayatā}.\footnote{See \emph{Sermon 1}} The basic principle involved, is itself often called \emph{paṭicca samuppāda}.

\begin{quote}
This being, this comes to be, with the arising of this, this arises. This not being, this does not come to be. With the cessation of this, this ceases.
\end{quote}

This insight penetrates through those extreme views. It resolves the conflict between them. But how? By removing the very premise on which it rested, and that is that there are two things. Though logicians might come out with the law of identity and the like, according to right view, the very bifurcation itself is the outcome of a wrong view. That is to say, this is only a conjoined pair. In other words, it resolves that conflict by accepting the worldly norm.

Now this is a point well worth considering. In the case of the twelve links of the formula of dependent arising, discovered by the Buddha, there is a relatedness of this to that, \emph{idappaccayatā}.

As for instance already illustrated above by the two links birth and decay-and-death.\footnote{See \emph{Sermon 3}} When birth is there, decay-and-death come to be, with the arising of birth, decay-and-death arise (and so on). The fact that this relatedness itself is the eternal law, is clearly revealed by the following statement of the Buddha in the \emph{Nidānasaṁyutta} of the \emph{Saṁyutta Nikāya}:

\begin{quote}
\emph{Avijjāpaccayā, bhikkhave, saṅkhārā. Ya tatra tathatā avitathatā anaññathatā idappaccayatā, ayaṁ vuccati, bhikkhave, paṭiccasamuppādo}.\footnote{\href{https://suttacentral.net/sn12.20/pli/ms}{SN 12.20 / S II 26}, \emph{Paccayasutta}}

From ignorance as condition, preparations come to be. That suchness therein, the invariability, the not-otherwiseness, the relatedness of this to that, this, monks, is called dependent arising.
\end{quote}

Here the first two links have been taken up to illustrate the principle governing their direct relation. Now let us examine the meaning of the terms used to express that relation. \emph{Tathā} means `such' or `thus', and is suggestive of the term \emph{yathābhūtañāṇadassana}, the knowledge and vision of things as they are. The correlatives \emph{yathā} and \emph{tathā} express between them the idea of faithfulness to the nature of the world.

So \emph{tathatā} asserts the validity of the law of dependent arising, as a norm in accordance with nature. \emph{Avitathatā}, with its double negative, reaffirms that validity to the degree of invariability. \emph{Anaññathatā}, or not-otherwiseness, makes it unchallengeable, as it were. It is a norm beyond contradiction.

When a conjoined pair is accepted as such, there is no conflict between the two. But since this idea can well appear as some sort of a puzzle, we shall try to illustrate it with a simile. Suppose two bulls, a black one and a white one, are bound together at the neck and allowed to graze in the field as a pair. This is sometimes done to prevent them from straying far afield. Now out of the pair, if the white bull pulls towards the stream, while the black one is pulling towards the field, there is a conflict. The conflict is not due to the bondage, at least not necessarily due to the bondage. It is because the two are pulling in two directions.

Supposing the two bulls, somehow, accept the fact that they are in bondage and behave amicably. When then the white bull pulls towards the stream, the black one keeps him company with equanimity, though he is not in need of a drink. And when the black bull is grazing, the white bull follows him along with equanimity, though he is not inclined to eat.

Similarly, in this case too, the conflict is resolved by accepting the pair-wise combination as a conjoined pair. That is how the Buddha solved this problem. But still the point of this simile might not be clear enough.

So let us come back to the two links, birth and decay-and-death, which we so often dragged in for purposes of clarification. So long as one does not accept the fact that these two links, birth and decay-and-death, are a conjoined pair, one would see between them a conflict. Why? Because one grasps birth as one end, and tries to remove the other end, which one does not like, namely decay-and-death. One is trying to separate birth from decay-and-death. But this happens to be a conjoined pair. ``Conditioned by birth, monks, is decay-and-death.'' This is the word of the Buddha. Birth and decay-and-death are related to each other.

The word \emph{jarā}, or decay, on analysis would make this clear. Usually by \emph{jarā} we mean old age. The word has connotations of senility and decrepitude, but the word implies both growth and decay, as it sets in from the moment of one's birth itself. Only, there is a possible distinction according to the standpoint taken. This question of a standpoint or a point of view is very important at this juncture. This is something one should assimilate with a meditative attention. Let us bring up a simile to make this clear.

Now, for instance, there could be a person who makes his living by selling the leaves of a particular kind of tree. Suppose another man sells the flowers of the same tree, to make his living. And yet another sells the fruits, while a fourth sells the timber. If we line them up and put to them the question, pointing to that tree: `Is this tree mature enough?', we might sometimes get different answers. Why? Each would voice his own commercial point of view regarding the degree of maturity of the tree. For instance, one who sells flowers would say that the tree is too old, if the flowering stage of the tree is past.

Similarly, the concept of decay or old age can change according to the standpoint taken up. From beginning to end, it is a process of decay. But we create an artificial boundary between youth and old age. This again shows that the two are a pair mutually conjoined. Generally, the worldlings are engaged in an attempt to separate the two in this conjoined pair. Before the Buddha came into the scene, all religious teachers were trying to hold on to birth, while rejecting decay-and-death. But it was a vain struggle. It is like the attempt of the miserly millionaire Kosiya to eat rice-cakes alone, to cite another simile.

According to that instructive story, the millionaire Kosiya, an extreme miser, once developed a strong desire to eat rice-cakes.\footnote{\href{https://www.digitalpalireader.online/_dprhtml/index.html?loc=k.1.0.0.4.4.0.a}{Dhp 49 Commentary: Dhp-a I 367}, \emph{Macchariyakosiyaseṭṭhivatthu}} As he did not wish to share them with anyone else, he climbed up to the topmost storey of his mansion with his wife and got her to cook rice-cakes for him.

To teach him a lesson, Venerable Mahā Moggallāna, who excelled in psychic powers, went through the air and appeared at the window as if he is on his alms round. Kosiya, wishing to dismiss this intruder with a tiny rice-cake, asked his wife to put a little bit of cake dough into the pan. She did so, but it became a big rice-cake through the venerable \emph{thera's} psychic power. Further attempts to make tinier rice-cakes ended up in producing ever bigger and bigger ones. In the end, Kosiya thought of dismissing the monk with just one cake, but to his utter dismay, all the cakes got joined to each other to form a string of cakes. The couple then started pulling this string of cakes in either direction with all their might, to separate just one from it. But without success. At last they decided to let go and give up, and offered the entire string of cakes to the venerable \emph{Thera.}

The Buddha's solution to the above problem is a similar let go-ism and giving up. It is a case of giving up all assets, \emph{sabbūpadhipaṭinissagga}. You cannot separate these links from one another. Birth and decay-and-death are intertwined. This is a conjoined pair. So the solution here, is to let go. All those problems are due to taking up a standpoint. Therefore the kind of view sanctioned in this case, is one that leads to detachment and dispassion, one that goes against the tendency to grasp and hold on. It is by grasping and holding on that one comes into conflict with Māra.

Now going by the story of the millionaire Kosiya, one might think that the Buddha was defeated by Māra. But the truth of the matter is that it is Māra who suffered defeat by this sort of giving up. It is a very subtle point.

Māra's forte lies in seizing and grabbing. He is always out to challenge. Sometimes he takes delight in hiding himself to take one by surprise, to drive terror and cause horripilation. So when Māra comes round to grab, if we can find some means of foiling his attempt, or make it impossible for him to grab, then Māra will have to accept defeat.

Now let us examine the Buddha's solution to this question. There are in the world various means of preventing others from grabbing something we possess. We can either hide our property in an inaccessible place, or adopt security measures, or else we can come to terms and sign a treaty with the enemy. But all these measures can sometimes fail. However, there is one unfailing method, which in principle is bound to succeed. A method that prevents all possibilities of grabbing. And that is -- letting go, giving up.

When one lets go, there is nothing to grab. In a tug-of-war, when someone is pulling at one end with all his might, if the other suddenly lets go of its hold, one can well imagine the extent of the former's discomfiture, let alone victory. It was such a discomfiture that fell to Māra's lot, when the Buddha applied this extraordinary solution. All this goes to show the importance of such terms as \emph{nissaya} and \emph{idappaccayatā} in understanding this Dhamma.

We have already taken up the word \emph{nissaya} for comment. Another aspect of its significance is revealed by the \emph{Satipaṭṭhānasutta}. Some parts of this sutta, though well known, are wonderfully deep. There is a certain thematic paragraph, which occurs at the end of each subsection in the \emph{Satipaṭṭhānasutta}. For instance, in the section on the contemplation relating to body, \emph{kāyānupasssanā}, we find the following paragraph:

\begin{quote}
\emph{Iti ajjhattaṁ vā kāye kāyānupassī viharati, bahiddhā vā kāye kāyānupassī viharati, ajjhattabahiddhā vā kāye kāyānupassī viharati; samudayadhammānupassī vā kāyasmiṁ viharati, vayadhammānupassī vā kāyasmiṁ viharati, samudayavayadhammānupassī vā kāyasmiṁ viharati; `atthi kāyo'ti vā pan'assa sati paccupaṭṭhitā hoti, yāvadeva ñāṇamattāya paṭissatimattāya; anissito ca viharati, na ca kiñci loke upādiyati}.\footnote{\href{https://suttacentral.net/mn10/pli/ms}{MN 10 / M I 56}, \emph{Satipaṭṭhānasutta}}

In this way he abides contemplating the body as a body internally, or he abides contemplating the body as a body externally, or he abides contemplating the body as a body internally and externally. Or else he abides contemplating the arising nature in the body, or he abides contemplating the dissolving nature in the body, or he abides contemplating the arising and dissolving nature in the body. Or else the mindfulness that `there is a body' is established in him only to the extent necessary for just knowledge and further mindfulness. And he abides independent and does not cling to anything in the world.
\end{quote}

A similar paragraph occurs throughout the sutta under all the four contemplations, body, feeling, mind and mind objects. As a matter of fact, it is this paragraph that is called \emph{satipaṭṭhāna bhāvanā}, or meditation on the foundation of mindfulness.\footnote{\href{https://suttacentral.net/sn47.40/pli/ms}{SN 47.40 / S V 183}, \emph{Vibhaṅgasutta}}

The preamble to this paragraph introduces the foundation itself, or the setting up of mindfulness as such. The above paragraph, on the other hand, deals with what pertains to insight. It is the field of insight proper. If we examine this paragraph, here too we will find a set of conjoined or twin terms:

\begin{quote}
In this way he abides contemplating the body as a body internally, or he abides contemplating the body externally,
\end{quote}

And then:

\begin{quote}
he abides contemplating the body both internally and externally.
\end{quote}

Similarly:

\begin{quote}
He abides contemplating the arising nature in the body, or he abides contemplating the dissolving nature in the body,
\end{quote}

And then:

\begin{quote}
he abides contemplating both the arising and dissolving nature in the body.

Or else the mindfulness that `there is a body' is established in him only to the extent necessary for knowledge and remembrance.
\end{quote}

This means that for the meditator even the idea `there is a body', that remembrance, is there just for the purpose of further development of knowledge and mindfulness.

\begin{quote}
And he abides independent and does not cling to anything in the world.
\end{quote}

Here too, the word used is \emph{anissita}, independent, or not leaning towards anything. He does not cling to anything in the world. The word \emph{nissaya} says something more than grasping. It means `leaning on' or `associating'.

This particular thematic paragraph in the \emph{Satipaṭṭhānasutta} is of paramount importance for insight meditation. Here, too, there is the mention of internal, \emph{ajjhatta}, and external, \emph{bahiddhā}.

When one directs one's attention to one's own body and another's body separately, one might sometimes take these two concepts, internal and external, too seriously with a dogmatic attitude. One might think that there is actually something that could be called one's own or another's. But then the mode of attention next mentioned unifies the two, as internal-external, \emph{ajjhattabahiddhā}, and presents them like the conjoined pair of bulls. And what does it signify? These two are not to be viewed as two extremes, they are related to each other.

Now let us go a little deeper into this interrelation. The farthest limit of the internal is the nearest limit of the external. The farthest limit of the external is the nearest limit of the internal. More strictly rendered, \emph{ajjhatta} means inward and \emph{bahiddhā} means outward. So here we have the duality of an inside and an outside.

One might think that the word \emph{ajjhattika} refers to whatever is organic. Nowadays many people take in artificial parts into their bodies. But once acquired, they too become internal. That is why, in this context \emph{ajjhattika} has a deeper significance than its usual rendering as `one's own'.

Whatever it may be, the farthest limit of the \emph{ajjhatta} remains the nearest limit of the \emph{bahiddhā}. Whatever portion one demarcates as one's own, just adjoining it and at its very gate is \emph{bahiddhā}. And from the point of view of \emph{bahiddhā}, its farthest limit and at its periphery is \emph{ajjhatta}. This is a conjoined pair. These two are interrelated. So the implication is that these two are not opposed to each other. That is why, by attending to them both together, as \emph{ajjhattabahiddhā}, that dogmatic involvement with a view is abandoned. Here we have an element of reconciliation, which prevents adherence to a view. This is what fosters the attitude of \emph{anissita}, unattached.

So the two, \emph{ajjhatta} and \emph{bahiddhā}, are neighbours. Inside and outside as concepts are neighbours to each other. It is the same as in the case of arising and ceasing, mentioned above. This fact has already been revealed to some extent by the \emph{Kaccāyanagottasutta}.

Now if we go for an illustration, we have the word \emph{udaya} at hand in \emph{samudaya}. Quite often this word is contrasted with \emph{atthagama}, going down, in the expression \emph{udayatthagaminī paññā}, the wisdom that sees the rise and fall. We can regard these two as words borrowed from everyday life. \emph{Udaya} means sunrise, and \emph{atthagama} is sunset. If we take this itself as an illustration, the farthest limit of the forenoon is the nearest limit of the afternoon. The farthest limit of the afternoon is the nearest limit of the forenoon. And here again we see a case of neighbourhood.

When one understands the neighbourly nature of the terms \emph{udaya} and \emph{atthagama}, or \emph{samudaya} and \emph{vaya}, and regards them as interrelated by the principle of \emph{idappaccayatā}, one penetrates them both by that mode of contemplating the rise and fall of the body together, \emph{samudayavayadhammānupassī vā kāyasmiṁ viharati}, and develops a penetrative insight.

What comes next in the \emph{satipaṭṭhāna} passage, is the outcome or net result of that insight.

\begin{quote}
The mindfulness that `there is a body' is established in him only to the extent necessary for pure knowledge and further mindfulness,

\emph{`atthi kāyo'ti vā pan'assa sati pacupaṭṭhitā hoti, yāvadeva ñāṇamattāya paṭissatimattāya}.
\end{quote}

At that moment one does not take even the concept of body seriously. Even the mindfulness that `there is a body' is established in that meditator only for the sake of, \emph{yavadeva}, clarity of knowledge and accomplishment of mindfulness. The last sentence brings out the net result of that way of developing insight:

\begin{quote}
He abides independent and does not cling to anything in the world.
\end{quote}

Not only in the section on the contemplation of the body, but also in the sections on feelings, mind, and mind objects in the \emph{Satipaṭṭhānasutta}, we find this mode of insight development. None of the objects, taken up for the foundation of mindfulness, is to be grasped tenaciously. Only their rise and fall is discerned. So it seems that, what is found in the \emph{Satipaṭṭhānasutta}, is a group of concepts. These concepts serve only as a scaffolding for the systematic development of mindfulness and knowledge. The Buddha often compared his Dhamma to a raft:

\begin{quote}
\emph{nittharaṇatthāya no gahaṇatthāya}

for crossing over and not for holding on to.\footnote{\href{https://suttacentral.net/mn22/pli/ms}{MN 22 / M I 134}, \emph{Alagaddūpamasutta}}
\end{quote}

Accordingly, what we have here are so many scaffoldings for the up-building of mindfulness and knowledge.

Probably due to the lack of understanding of this deep philosophy enshrined in the \emph{Satipaṭṭhānasutta}, many sects of Buddhism took up these concepts in a spirit of dogmatic adherence. That dogmatic attitude of clinging on is like the attempt to cling on to the scaffoldings and to live on in them. So with reference to the \emph{Satipaṭṭhānasutta} also, we can understand the importance of the term \emph{nissaya}.
