\chapter{Sermon 19}

\NibbanaOpeningQuote

With the permission of the Most Venerable Great Preceptor and the assembly of the venerable meditative monks. This is the nineteenth sermon in the series of sermons on Nibbāna.

Towards the end of our last sermon, we started commenting on the two terms \emph{sa-upādisesā Nibbānadhātu} and \emph{anupādisesā Nibbānadhātu}. Our discussion was based on a discourse, which we quoted from the \emph{Itivuttaka}. We also drew attention to a certain analogy found in the discourses, which shows that the two Nibbāna elements actually represent two stages of the extinguishment implicit in the term Nibbāna.

When no more firewood is added to a blazing fire, flames would subside and the logs of wood already burning go on smouldering as embers. After some time, they too get extinguished and become ashes. With regard to the \emph{arahant}, too, we have to think in terms of this analogy.

It can be taken as an illustration of the two Nibbāna elements. To the extent the living \emph{arahant} is free from fresh graspings, lust, hate and delusions do not flare up. But so long as he has to bear the burden of this organic combination, this physical frame, the \emph{arahant} has to experience certain afflictions and be receptive to likes and dislikes, pleasures and pains.

In spite of all that, mentally he has access to the experience of the extinguishment he has already won. It is in that sense that the \emph{arahant} is said to be in the Nibbāna element with residual clinging in his everyday life, while taking in the objects of the five senses.

At the last moment of the \emph{arahant's} life, even this organic body that had been grasped as \emph{upādiṇṇa} has to be abandoned. It is at that moment, when he is going to detach his mind from the body, that \emph{anupādisesā parinibbānadhātu} comes in. A brief hint to this effect is given in one of the verses occurring in the \emph{Nāgasutta} referred to earlier. The verse runs thus:

\begin{quote}
\emph{Vītarāga vītadoso}\\
\emph{vītamoho anāsavo}\\
\emph{sarīraṁ vijahaṁ nāgo}\\
\emph{parinibbissati anāsavo.}\footnote{\href{https://suttacentral.net/an6.43/pli/ms}{AN 6.43 / A III 347}, \emph{Nāgasutta}}

The one who has abandoned lust,\\
Hate and delusion and is influx-free,\\
That elephant of a man, on giving up his body,\\
Will attain full appeasement, being influx-free.
\end{quote}

If we define in brief the two Nibbāna elements this way, a more difficult problem confronts us relating to the sense in which they are called \emph{diṭṭhadhammika} and \emph{samparāyika}. \emph{Diṭṭhadhammika} means what pertains to this life and \emph{samparāyika} refers to what comes after death. What is the idea in designating \emph{sa-upādisesā Nibbānadhātu} as \emph{diṭṭhadhammika} and \emph{anupādisesā Nibbānadhātu} as \emph{samparāyika}?

In the context of \emph{kamma}, the meaning of these two terms is easily understood. But when it comes to Nibbāna, such an application of the terms would imply two types of \emph{Nibbānic} bliss, one to be experienced here and the other hereafter.

But that kind of explanation would not accord with the spirit of this Dhamma, because the Buddha always emphasizes the fact that Nibbāna is something to be realized here and now in toto. It is not a piecemeal realization, leaving something for the hereafter. Such terms like \emph{diṭṭheva dhamme}, in this very life, \emph{sandiṭṭhika}, here and now, and \emph{akālika}, timeless, emphasize this aspect of Nibbāna.

In the context of Nibbāna, these two terms have to be understood as representing two aspects of a perfect realization attainable in this very life. Briefly stated, \emph{anupādisesā Nibbānadhātu} is that which confers the certitude, well in time, that the appeasement experienced by an \emph{arahant} during this life time remains unchanged even at death.

To say that there is a possibility of realizing or ascertaining one's state after death might even seem contradictory. How can one realize one's after death state?

We get a clear-cut answer to that question in the following passage in the \emph{Dhātuvibhaṅgasutta} of the \emph{Majjhima Nikāya}.

\begin{quote}
\emph{Seyyathāpi, bhikkhu, telañca paṭicca vaṭṭiñca paṭicca telappadīpo jhāyati, tasseva telassa ca vaṭṭiyā ca pariyādānā aññassa ca anupahārā anāhāro nibbāyati, evameva kho, bhikkhu, kāyapariyantikaṁ vedanaṁ vediyamāno `kāyapariyantikaṁ vedanaṁ vedayāmī'ti pajānati, jīvitapariyantikaṁ vedanaṁ vediyamāno `jīvitapariyantikaṁ vedanaṁ vedayāmī'ti pajānati, `kāyassa bhedā paraṁ maraṇā uddhaṁ jīvitapariyādānā idheva sabbavedayitāni anabhinanditāni sītībhavissantī'ti pajānati.}\footnote{\href{https://suttacentral.net/mn140/pli/ms}{MN 140 / M III 245}, \emph{Dhātuvibhaṅgasutta}}

Just as, monk, an oil lamp burns depending on oil and the wick, and when that oil and the wick are used up, if it does not get any more of these, it is extinguished from lack of fuel, even so, monk, when he feels a feeling limited to the body, he understands `I feel a feeling limited to the body', when he feels a feeling limited to life, he understands `I feel a feeling limited to life', he understands `on the breaking up of this body, before life becomes extinct, even here itself, all that is felt, not being delighted in, will become cool'.
\end{quote}

The last sentence is particularly noteworthy in that it refers to an understanding well beforehand that all feelings, not being delighted in, will become cool at death. The futuristic ending signifies an assurance, here and now, as the word \emph{idheva}, even here itself, clearly brings out. The delighting will not be there, because all craving for a fresh existence is extirpated.

The \emph{arahant} has won this assurance already in his \emph{arahattaphalasamādhi}, in which he experiences the cooling off of all feelings. That is why we find the \emph{arahants} giving expression to their \emph{Nibbānic} bliss in the words \emph{sītibhūto'smi nibbuto}, ``gone cool am I, yea, extinguished''.\footnote{\href{https://suttacentral.net/thag4.8/pli/ms}{Thag 4.8 / Th 298}, \emph{Rāhulatheragāthā}}

Since for the \emph{arahant} this cooling off of feelings is a matter of experience in this very life, this realization is referred to as \emph{anupādā parinibbāna} in the discourses. Here we seem to have fallen into another track. We opened our discussion with an explanation of what \emph{anupādisesa parinibbāna} is, now we are on \emph{anupādā parinibbāna}. How are we to distinguish between these two?

\emph{Anupādisesa parinibbāna} comes at the last moment of the \emph{arahant's} life, when this organic combination of elements, grasped par excellence, \emph{upādiṇṇa}, is discarded for good. But \emph{anupādā parinibbāna} refers to the \emph{arahattaphalasamādhi} as such, for which even other terms like \emph{anupādā vimokkha} are also applied on occasion.\footnote{E.g. \href{https://suttacentral.net/mn106/pli/ms}{MN 106 / M II 265}, \emph{Āneñjasappāyasutta}}

As the term \emph{anupādā parinibbāna} signifies, the \emph{arahant} experiences, even in this very life, that complete extinguishment, \emph{parinibbāna}, in his \emph{arahatta phalasamādhi}. This fact is clearly brought out in the dialogue between Venerable Sāriputta and Venerable Puṇṇa Mantāṇiputta in the \emph{Rathavinītasutta} of the \emph{Majjhima Nikāya}.

Venerable Sāriputta's exhaustive interrogation ending with

\begin{quote}
\emph{kim atthaṁ carahāvuso, bhagavati brahmacariyaṁ vussati?}\footnote{M I 147, \emph{Rathavinītasutta}}

For the sake of what then, friend, is the holy life lived under the Exalted One?
\end{quote}

gets the following conclusive answer from Venerable Puṇṇa Mantāṇiputta:

\begin{quote}
\emph{anupādāparinibbānatthaṁ kho, āvuso, bhagavati brahmacariyaṁ vussati},

Friend, it is for the sake of perfect Nibbāna without grasping that the holy life is lived under the Exalted One.
\end{quote}

As the goal of endeavour, \emph{anupādā parinibbāna} surely does not mean the ending of life. What it implies is the realization of Nibbāna. It is that experience of the cooling off of feelings the \emph{arahant} goes through in the \emph{arahattaphalasamādhi}.

It is sometimes also called \emph{nirupadhi}, the `asset-less'.\footnote{S I 194, \emph{Moggallānasutta}} Here we have a problem of a semantic type. At a later date, even the term \emph{nirupadhisesa} seems to have come into vogue, which is probably a cognate formed after the term \emph{anupādisesa}.\footnote{Bv-a 252}

Nowhere in the discourses one comes across the term \emph{nirupadhisesa parinibbāna}. Only such terms as \emph{nirupadhi}, \emph{nirūpadhiṁ,} \emph{nirupadhi dhammaṁ} are met with. They all refer to that \emph{arahattaphalasamādhi}, as for instance in the following verse, which we had occasion to quote earlier too:

\enlargethispage{\baselineskip}

\begin{quote}
\emph{Kāyena amataṁ dhātuṁ,}\\
\emph{phusayitvā nirūpadhiṁ,}\\
\emph{upadhipaṭinissaggaṁ,}\\
\emph{sacchikatvā anāsavo,}\\
\emph{deseti sammāsambuddho,}\\
\emph{asokaṁ virajaṁ padaṁ.}\footnote{\href{https://suttacentral.net/iti73/pli/ms}{Iti 73 / It 62}, \emph{Santatarasutta}, see \emph{Sermon 17}}

Having touched with the body,\\
The deathless element, which is asset-less,\\
And realized the relinquishment of assets,\\
Being influx-free, the perfectly enlightened one,\\
Proclaims the sorrow-less, taintless state.
\end{quote}

To proclaim, one has to be alive. Therefore \emph{nirupadhi} is used in the discourses definitely for the \emph{arahattaphalasamādhi}, which is a living experience for the \emph{arahant}.

\emph{Anupādā parinibbāna, anupādā vimokkha} and \emph{nirupadhi} all refer to that experience of the cooling off of feelings. This fact is clearly revealed by the following two verses in the \emph{Vedanāsaṁyutta} of the \emph{Saṁyutta Nikāya}:

\begin{quote}
\emph{Samāhito sampajāno,}\\
\emph{sato Buddhassa sāvako,}\\
\emph{vedanā ca pajānāti,}\\
\emph{vedanānañca sambhavaṁ.}

\emph{Yattha cetā nirujjhanti,}\\
\emph{maggañca khayagāminaṁ,}\\
\emph{vedanānaṁ khayā bhikkhu,}\\
\emph{nicchāto parinibbuto.}\footnote{S IV 204, \emph{Samādhisutta}}
\end{quote}

In this couplet, the experience of the fruit of \emph{arahanthood} is presented under the heading of feeling. The disciple of the Buddha, concentrated, fully aware and mindful, understands feelings, the origin of feelings, and the point at which they surcease and the way leading to their extinction.

With the extinction of feelings, that monk is hunger-less and perfectly extinguished. The reference here is to that bliss of Nibbāna which is devoid of feeling, \emph{avedayita sukha}.\footnote{Ps III 115, \emph{aṭṭhakathā} on MN 59, \emph{Bahuvedanīyasutta}} It is hunger-less because it is free from craving.

The perfect extinguishment mentioned here is not to be understood as the death of the \emph{arahant}. In the discourses the term \emph{parinibbuta} is used as such even with reference to the living \emph{arahant}. Only in the commentaries we find a distinction made in this respect. The \emph{parinibbāna} of the living \emph{arahant} is called \emph{kilesaparinibbāna}, the perfect extinguishment of the defilements, while what comes at the last moment of an \emph{arahant's} life is called \emph{khandhaparinibbāna}, the perfect extinguishment of the groups or aggregates.\footnote{E.g. at Mp I 91} Such a qualification, however, is not found in the discourses.

The reason for this distinction was probably the semantic development the term \emph{parinibbāna} had undergone in the course of time. The fact that this perfect extinguishment is essentially psychological seems to have been ignored with the passage of time. That is why today, on hearing the word \emph{parinibbāna}, one is immediately reminded of the last moment of the life of the Buddha or of an \emph{arahant}. In the discourses, however, \emph{parinibbāna} is clearly an experience of the living \emph{arahant} in his \emph{arahattaphalasamādhi}.

This fact is clearly borne out by the statement in the \emph{Dhātuvibhaṅgasutta} already quoted:

\begin{quote}
\emph{idheva sabbavedayitāni anabhinanditāni sītībhavissantī'ti pajānati},\footnote{\href{https://suttacentral.net/mn140/pli/ms}{MN 140 / M III 245}, \emph{Dhātuvibhaṅgasutta}}

he understands that all what is felt will cool off here itself.
\end{quote}

It is this very understanding that is essential. It gives the certitude that one can defeat Māra at the moment of death through the experience of the cooling off of feelings.

The phrase \emph{jīvitapariyantikaṁ vedanaṁ} refers to the feeling which comes at the termination of one's life. For the \emph{arahant}, the \emph{arahattaphalasamādhi} stands in good stead, particularly at the moment of death. That is why it is called \emph{akuppā cetovimutti}, the unshakeable deliverance of the mind.

All other deliverances of the mind get shaken before the pain of death, but not this unshakeable deliverance of the mind, which is the REAL-ization of extinguishment that is available to the \emph{arahant} already in the \emph{arahattaphalasamādhi}, in the experience of the cooling off of feelings. It is this unshakeable deliverance of the mind that the Buddha and the \emph{arahants} resort to at the end of their lives, when Māra comes to grab and seize.

So now we can hark back to that verse which comes as the grand finale in the long discourse from the \emph{Itivuttaka} we have already quoted.

\begin{quote}
\emph{Ye etad aññāya padaṁ asaṅkhataṁ,}\\
\emph{vimuttacittā bhavanettisaṅkhayā,}\\
\emph{te dhammasārādhigamā khaye ratā,}\\
\emph{pahaṁsu te sabbabhavāni tādino.}\footnote{It 39, \emph{Nibbānadhātusutta}, see \emph{Sermon 18}}
\end{quote}

This verse might appear problematic, as it occurs at the end of a passage dealing with the two Nibbāna elements.

\begin{quote}
\emph{Ye etad aññāya padaṁ asaṅkhataṁ},

those who having fully comprehended this unprepared state,

\emph{vimuttacittā bhavanettisaṅkhayā},

are released in mind by the cutting off of tentacles to becoming,

\emph{te dhammasārādhigamā khaye ratā},

taking delight in the extirpation of feelings due to their attainment to the essence of \emph{dhamma},

(that is the unshakeable deliverance of the mind),

\emph{pahaṁsu te sabbabhavāni tādino},

being steadfastly such like, they have given up all forms of becoming.
\end{quote}

The last line is an allusion to the experience of the cessation of existence here and now, which in effect is the realization of Nibbāna, true to the definition \emph{bhavanirodho nibbānaṁ}, ``cessation of existence is Nibbāna''.\footnote{\href{https://suttacentral.net/an10.7/pli/ms}{AN 10.7 / A V 9}, \emph{Sāriputtasutta}}

It is that very cessation of existence that is called \emph{asaṅkhata dhātu}, the `unprepared element'. If \emph{bhava}, or existence, is to be called \emph{saṅkhata}, the `prepared', the cessation of existence has to be designated as \emph{asaṅkhata}, the `unprepared'. Here lies the difference between the two.

So we have here two aspects of the same unprepared element, designated as \emph{sa-upādisesā parinibbānadhātu} and \emph{anupādisesā parinibbānadhātu}. The mind is free even at the stage of \emph{sa-upādisesa}, to the extent that the smouldering embers do not seek fresh fuel.

\emph{Anupādisesa} refers to the final experience of extinguishment. There the relevance of the term \emph{parinibbāna} lies in the fact that at the moment of death the \emph{arahants} direct their minds to this unshakeable deliverance of the mind. This is the `island' they resort to when Māra comes to grab.

The best illustration for all this is the way the Buddha faced death, when the time came for it. Venerable Anuruddha delineates it beautifully in the following two verses:

\begin{quote}
\emph{Nāhu assāsapassāso,}\\
\emph{ṭhitacittassa tādino,}\\
\emph{anejo santimārabbha,}\\
\emph{yaṁ kālamakarī muni.}

\emph{Asallīnena cittena,}\\
\emph{vedanaṁ ajjhavāsayi,}\\
\emph{pajjotass'eva nibbānaṁ,}\\
\emph{vimokkho cetaso ahu.}\footnote{\href{https://suttacentral.net/dn16/pli/ms}{DN 16 / D II 157}, \emph{Mahāparinibbānasutta}}

Adverting to whatever peace,\\
The urgeless sage reached the end of his life span,\\
There were no in-breaths and out-breaths,\\
For that steadfastly such-like one of firm mind.

With a mind fully alert,\\
He bore up the pain,\\
The deliverance of the mind was like\\
The extinguishment of a torch.
\end{quote}

The allusion here is to the deliverance of the mind. This is a description of how the Buddha attained \emph{parinibbāna}. Though there is a great depth in these two verses, the commentarial exegesis seems to have gone at a tangent at this point. Commenting on the last two lines of the first verse, the commentary observes: \emph{Buddhamuni santiṁ gamissāmīti, santiṁ ārabbha kālamakari}, ``the Buddha, the sage, passed away for the sake of that peace with the idea `I will go to that state of peace'\,''.\footnote{Sv II 595}

There is some discrepancy in this explanation. Commentators themselves usually give quite a different sense to the word \emph{ārabbha} than the one implicit in this explanation. Here it means ``for the sake of''. It is for the sake of that peace that the Buddha is said to have passed away.

In such commentaries as \emph{Jātaka-aṭṭhakathā} and \emph{Dhammapada-aṭṭhakathā}, commentators do not use the word \emph{ārabbha} in the introductory episodes in this sense. There it only means ``in connection with'', indicating the origin of the story, as suggested by the etymological background of the word itself.

When for instance it is said that the Buddha preached a particular sermon in connection with Devadatta Thera, it does not necessarily mean that it was meant for him.\footnote{\emph{Devadattaṁ ārabbha} at Dhp-a I 133 and Ja I 142} He may not have been there at all, it may be that he was already dead by that time. The term \emph{ārabbha} in such contexts only means that it was in connection with him. It can refer to a person or an incident, as the point of origin of a particular sermon.

Granted this, we have to explain the verse in question not as an allusion to the fact that the Buddha, the sage, passed away for the sake of that peace with the idea `I will attain to that state of peace'. It only means that the Buddha, the sage, passed away having brought his mind into that state of peace. In other words, according to the commentary the passing away comes first and the peace later, but according to the sutta proper, peace comes first and the passing away later.

There is a crucial point involved in this commentarial divergence. It has the presumption that the Buddha passed away in order to enter into `that Nibbāna element'. This presumption is evident quite often in the commentaries. When hard put to it, the commentaries sometimes concede the sutta's standpoint, but more often than otherwise they follow a line of interpretation that comes dangerously close to an eternalist point of view, regarding Nibbāna.

Here too the commentarial exegesis, based on the term \emph{ārabbha}, runs the same risk. On the other hand, as we have pointed out, the reference here is to the fact that the Buddha adverted his mind to that peace well before the onset of death, whereby Māra's attempt is foiled, because feelings are already cooled off. It is here that the unshakeable deliverance of the mind proves its worth.

As a `real'-ization it is already available to the Buddha and the \emph{arahants} in the \emph{arahattaphalasamādhi}, and when the time comes, they put forward this experience to beat off Māra. That is why we find a string of epithets for Nibbāna, such as \emph{tāṇaṁ, leṇaṁ, dīpaṁ, saraṇaṁ, parāyanaṁ, khemaṁ} and \emph{amataṁ}.\footnote{S IV 371, \emph{Asaṅkhatasaṁyutta}}

When faced with death, or the pain of death, it gives `protection'\emph{, tānaṁ}.

It provides shelter, like a `cave', \emph{leṇaṁ}.

It is the `island', \emph{dīpaṁ}, within easy reach.

It is the `refuge', \emph{saraṇaṁ}, and the `resort', \emph{parāyanaṁ}.

It is the `security', \emph{khemaṁ}, and above all the `deathless', \emph{amataṁ}.

This deathlessness they experience in this very world, and when death comes, this realization stands them in good stead.

Why Venerable Anuruddha brought in the profane concept of death with the expression \emph{kālamakari} into this verse, describing the Buddha's \emph{parinibbāna}, is also a question that should arrest our attention.

This particular expression is generally used in connection with the death of ordinary people. Why did he use this expression in such a hallowed context? It is only to distinguish and demarcate the deliverance of the mind, couched in the phrase \emph{vimokkho cetaso ahu}, from the phenomenon of death itself.

The Buddhas and \emph{arahants} also abandon this body, like other beings. The expression \emph{kālamakari}, ``made an end of time'', is an allusion to this phenomenon. In fact, it is only the Buddhas and \emph{arahants} who truly make an `end' of time, being fully aware of it. Therefore the most important revelation made in the last two lines of the first verse, \emph{anejo santimārabbha, yaṁ kālamakarī muni}, is the fact that the Buddha passed away having brought his mind to the peace of Nibbāna.

All this goes to prove that an \emph{arahant}, even here and now in this very life, has realized his after death state, which is none other than the birthless cessation of all forms of existence that amounts to deathlessness itself.

In all other religions immortality is something attainable after death. If one brings down the Buddha's Dhamma also to that level, by smuggling in the idea of an everlasting Nibbāna, it too will suffer the same fate. That would contradict the teachings on impermanence, \emph{aniccatā}, and insubstantiality, \emph{anattatā}.

But here we have an entirely different concept. It is a case of overcoming the critical situation of death by directing one's mind to a concentration that nullifies the power of Māra. So it becomes clear that the two terms \emph{sa-upādisesā parinibbānadhātu} and \emph{anupādisesā parinibbānadhātu} stand for two aspects of the same \emph{asaṅkhatadhātu}, or the unprepared element.

As a matter of fact, \emph{arahants} have already directly realized, well in time, their after death state. That is to say, not only have they gone through the experience of extinguishment here and now, but they are also assured of the fact that this extinguishment is irreversible even after death, since all forms of existence come to cease.

This is an innovation, the importance of which can hardly be overestimated. Here the Buddha has transcended even the dichotomy between the two terms \emph{sandiṭṭhika} and \emph{samparāyika}. Generally, the world is inclined to believe that one can be assured only of things pertaining to this life. In fact, the word \emph{sandiṭṭhika} literally means that one can be sure only of things visible here and now. Since one cannot be sure of what comes after death, worldlings are in the habit of investing faith in a particular teacher or in a god.

To give a clearer picture of the principle involved in this statement, let us bring up a simple episode, concerning the general Sīha, included among the Fives of the \emph{Aṅguttara Nikāya}. It happens to centre on \emph{dānakathā}, or talks on liberality. Let it be a soft interlude -- after all these abstruse discourses.

Sīha, the general, is a wealthy benefactor, endowed with deep faith in the Buddha. One day he approaches the Buddha and asks the question:

\clearpage

\begin{quote}
\emph{sakkā nu kho, bhante, sandiṭṭhikaṁ dānaphalaṁ paññāpetuṁ?}\footnote{A III 39, \emph{Sīhasenāpatisutta}}

Is it possible, Lord, to point out an advantage or fruit of giving visible here and now?
\end{quote}

What prompted the question may have been the usual tendency to associate the benefits of giving with the hereafter. Now the Buddha, in his answer to the question, gave four advantages visible here and now and one advantage to come hereafter. The four fruits of giving visible here and now are stated as follows:

\begin{enumerate}
\def\labelenumi{\arabic{enumi}.}
\tightlist
\item
  \emph{dāyako, sīha, dānapati bahuno janassa piyo hoti manāpo}, ``Sīha, a benevolent donor is dear and acceptable to many people''.
\item
  \emph{dāyakaṁ dānapatiṁ santo sappurisā bhajanti}, ``good men of integrity resort to that benevolent donor''.
\item
  \emph{dāyakassa dānapatino kalyāṇo kittisaddo abbhuggacchati}, ``a good report of fame goes in favour of that benevolent donor''.
\item
  \emph{dāyako dānapati yaṁ yadeva parisaṁ upasaṅkamati, yadi khattiyaparisaṁ yadi brāhmaṇaparisaṁ yadi gahapatiparisaṁ yadi samaṇaparisaṁ, visārado va upasaṅkamati amaṅkubhūto}, ``whatever assembly that benevolent donor approaches, be it an assembly of kings, or brahmins, or householders, or recluses, he approaches with self confidence, not crestfallen''.
\end{enumerate}

These four fruits or advantages are reckoned as \emph{sandiṭṭhika}, because one can experience them here and now. In addition to these, the Buddha mentions a fifth, probably by way of encouragement, though it is outside the scope of the question.

\begin{enumerate}
\def\labelenumi{\arabic{enumi}.}
\setcounter{enumi}{4}
\tightlist
\item
  \emph{dāyako, sīha, dānapati kāyassa bhedā paraṁ maraṇā sugatiṁ saggaṁ lokaṁ upapajjati}, ``the benevolent donor, Sīha, when his body breaks up after death is reborn in a happy heavenly world.''
\end{enumerate}

This is a fruit of giving that pertains to the next world, \emph{samparāyikaṁ dānaphalaṁ}. Then Sīha the general makes a comment, which is directly relevant to our discussion:

\begin{quote}
\emph{Yānimāni, bhante, bhagavatā cattāri sandiṭṭhikāni dānaphalāni akkhātāni, nāhaṁ ettha bhagavato saddhāya gacchāmi, ahaṁ petāni jānāmi. Yañca kho maṁ, bhante, bhagavā evamāha `dāyako, sīha, dānapati kāyassa bhedā paraṁ maraṇā sugatiṁ saggaṁ lokaṁ upapajjatī'ti, etāhaṁ na jānāmi, ettha ca panāhaṁ bhagavato saddhāya gacchāmi}.

Those four fruits of giving, visible here and now, which the Lord has preached, as for them, I do not believe out of faith in the Exalted One, because I myself know them to be so. But that about which the Exalted One said: `Sīha, a benevolent donor, when the body breaks up after death is reborn in a happy heavenly world', this I do not know. As to that, however, I believe out of faith in the Exalted One.
\end{quote}

Regarding the first four advantages of giving, Sīha says ``I do not believe out of faith in the Exalted One, because I myself know them to be so'', \emph{nāhaṁ ettha bhagavato saddhāya gacchāmi, ahaṁ petāni jānāmi}. It is because he knows out of his own experience that they are facts that he does not believe out of faith in the Exalted One. There is something deep, worth reflecting upon, in this statement.

Then with regard to the fruit of giving, mentioned last, that is to say the one that concerns the hereafter, \emph{samparāyika}, Sīha confesses that he does not know it as a fact, but that he believes it out of faith in the Exalted One, \emph{etāhaṁ na jānāmi, ettha ca panāhaṁ bhagavato saddhāya gacchāmi}. It is because he does not know, that he believes out of faith in the Exalted One.

Here then we have a good illustration of the first principle we have outlined earlier. Where there is knowledge born of personal experience, there is no need of faith. Faith is displaced by knowledge of realization. It is where one has no such experiential knowledge that faith comes in. That is why Sīha confesses that he has faith in the fifth fruit of giving. With regard to the first four, faith is something redundant for him.

Now that we have clarified for ourselves this first principle, there is a certain interesting riddle verse in the \emph{Dhammapada}, to which we may apply it effectively, not out of a flair for riddles, but because it is relevant to our topic.

\begin{quote}
\emph{Assaddho akataññū ca,}\\
\emph{sandhicchedo ca yo naro,}\\
\emph{hatāvakāso vantāso,}\\
\emph{sa ve uttamaporiso.}\footnote{Dhp 97, \emph{Arahantavagga}}
\end{quote}

This is a verse attributed to the Buddha that comes in the \emph{Arahantavagga} of the \emph{Dhammapada}, which puns upon some words. Such riddle verses follow the pattern of a figure of speech called double entendre, which makes use of ambiguous words. The above verse sounds blasphemous on the first hearing. The Buddha is said to have employed this device to arrest the listener's attention. The surface meaning seems to go against the Dhamma, but it provokes deeper reflection.

For instance, \emph{assaddho} means faithless, to be \emph{akataññū} is to be ungrateful, \emph{sandhicchedo} is a term for a housebreaker, \emph{hatāvakāso} is a hopeless case with no opportunities, \emph{vantāso} means greedy of vomit. So the surface meaning amounts to this:

\begin{quote}
That faithless ungrateful man,\\
Who is a housebreaker,\\
Who is hopeless and greedy of vomit,\\
He indeed is the man supreme.
\end{quote}

For the deeper meaning the words have to be construed differently. \emph{Assaddho} implies that level of penetration into truth at which faith becomes redundant. \emph{Akata}, the unmade, is an epithet for Nibbāna, and \emph{akataññū} is one who knows the unmade. \emph{Sandhicchedo} means one who has cut off the connecting links to \emph{saṁsāra}. \emph{Hatāvakāso} refers to that elimination of opportunities for rebirth. \emph{Vantāso} is a term for one who has vomited out desires. The true meaning of the verse, therefore, can be summed up as follows:

\enlargethispage{\baselineskip}

\begin{quote}
That man who has outgrown faith,\\
\vin as he is a knower of the unmade,\\
Who has sundered all shackles to existence\\
\vin and destroyed all possibilities of rebirth,\\
Who has spewed out all desires,\\
He indeed is the man supreme.
\end{quote}

The description, then, turns out to be that of an \emph{arahant}. \emph{Assaddho} as an epithet for the \emph{arahant} follows the same norm as the epithet \emph{asekho}. \emph{Sekha}, meaning `learner', is a term applied to those who are training for the attainment of \emph{arahanthood}, from the stream-winner, \emph{sotāpanna}, upwards.

Literally, \emph{asekha} could be rendered as `unlearned' or `untrained'. But it is certainly not in that sense that an \emph{arahant} is called \emph{asekha}. He is called \emph{asekha} in the sense that he is no longer in need of that training, that is to say, he is an adept. \emph{Assaddho}, too, has to be construed similarly.

As we have mentioned before, the \emph{arahant} has already realized the cessation of existence in his \emph{arahattaphalasamādhi}, thereby securing the knowledge of the unmade, \emph{akata}, or the unprepared, \emph{asaṅkhata}. The term \emph{akataññū} highlights that fact of realization.

The most extraordinary and marvellous thing about the realization of Nibbāna is that it gives an assurance not only of matters pertaining to this life, \emph{sandiṭṭhika}, but also of what happens after death, \emph{samparāyika} -- in other words, the realization of the cessation of existence.

Nibbāna as the realization here and now of the cessation of existence, \emph{bhavanirodho nibbānaṁ}, carries with it the assurance that there is no more existence after death. So there is only one \emph{asaṅkhatadhātu}. The verse we already quoted, too, ends with the words \emph{pahaṁsu te sabbabhavāni tādino}, ``those steadfastly such like ones have given up all forms of existence''.\footnote{It 39, \emph{Nibbānadhātusutta}}

One thing should be clear now. Though there are two Nibbāna elements called \emph{sa-upādisesā Nibbānadhātu} and \emph{anupādisesā Nibbānadhātu}, there is no justification whatsoever for taking \emph{anupādisesā Nibbānadhātu} as a place of eternal rest for the \emph{arahants} after death -- an everlasting immortal state.

The deathlessness of Nibbāna is to be experienced in this world itself. That is why an \emph{arahant} is said to feast on ambrosial deathlessness, \emph{amataṁ paribhuñjati}, when he is in \emph{arahattaphalasamādhi}. When it is time for death, he brings his mind to this \emph{samādhi}, and it is while he is partaking of ambrosial deathlessness that Māra quietly takes away his body.

An \emph{arahant} might even cremate his own body, as if it is another's.

Now we are at an extremely deep point in this Dhamma. We have to say something in particular about the two terms \emph{saṅkhata} and \emph{asaṅkhata}. In our last sermon, we happened to give a rather unusual explanation of such pair-wise terms like the `hither shore' and the `farther shore', as well as the `mundane' and the `supramundane'.

The two terms in each pair are generally believed to be far apart and the gap between them is conceived in terms of time and space. But we compared this gap to that between the lotus leaf and the drop of water on it, availing ourselves of a simile offered by the Buddha himself.

The distance between the lotus leaf and the drop of water on it is the same as that between the hither shore and the farther shore, between the mundane and the supramundane. This is no idle sophistry, but a challenge to deeper reflection.

The \emph{Dhammapada} verse we quoted earlier beginning with \emph{yassa pāraṁ apāraṁ vā, pārāpāraṁ na vijjati},\footnote{\href{https://suttacentral.net/dhp383-423/pli/ms}{Dhp 385}, \emph{Brāhmaṇavagga}; see \emph{Sermon 18}} ``to whom there is neither a farther shore nor a hither shore nor both'', is puzzling enough. But what it says is that the \emph{arahant} has transcended both the hither shore and the farther shore. It is as if he has gone beyond this shore and the other shore as well, that is to say, he has transcended the dichotomy.

We have to say something similar with regard to the two terms \emph{saṅkhata} and \emph{asaṅkhata}. \emph{Saṅkhata}, or the prepared, is like a floral design. This prepared floral design, which is \emph{bhava}, or existence, is made up, as it were, with the help of the glue of craving, the tangles of views and the knots of conceits.

If one removes the glue, disentangles the tangles and unties the knots, the \emph{saṅkhata}, or the prepared, itself becomes \emph{asaṅkhata}, the unprepared, then and there. The same floral design, which was the \emph{saṅkhata}, has now become the \emph{asaṅkhata}. This itself is the cessation of existence, \emph{bhavanirodho}. When one can persuade oneself to think of Nibbāna as an extinguishment, the term \emph{parinibbāna} can well be understood as `perfect extinguishment'.

The \emph{parinibbāna} of the \emph{arahant} Dabba Mallaputta is recorded in the \emph{Udāna} as a special occasion on which the Buddha uttered a paean of joy. Venerable Dabba Mallaputta was an \emph{arahant}, gifted with marvellous psychic powers, specializing in miracles performed by mastering the fire element, \emph{tejo dhātu}. His \emph{parinibbāna}, too, was a marvel in itself.

When he found himself at the end of his life span, he approached the Buddha and informed him of it, as if begging permission, with the words:

\begin{quote}
\emph{parinibbāna kālo me dāni, sugata},\footnote{\href{https://suttacentral.net/ud8.9/pli/ms}{Ud 8.9 / Ud 92}, \emph{Paṭhamadabbasutta}}

it is time for me to attain \emph{parinibbāna}, O well-gone one.
\end{quote}

And the Buddha too gave permission with the words:

\begin{quote}
\emph{yassa dāni tvaṁ, Dabba, kālaṁ maññasi},

Dabba, you may do that for which the time is fit.
\end{quote}

As soon as the Buddha uttered these words, Venerable Dabba Mallaputta rose from his seat, worshipped the Buddha, circumambulated him, went up into the sky and, sitting cross-legged, aroused the concentration of the fire element and, rising from it, attained \emph{parinibbāna}. As his body thus miraculously self-cremated burnt in the sky, it left no ashes or soot.

This was something significant that fits in with the definition of Nibbāna so far given. That is probably why the Buddha is said to have uttered a special verse of uplift or paean of joy at this extinguishment, which was perfect in every sense.

\begin{quote}
\emph{Abhedi kāyo, nirodhi saññā,}\\
\emph{vedanā sītirahaṁsu sabbā,}\\
\emph{vūpasamiṁsu saṅkhārā,}\\
\emph{viññānaṁ attham agamā.}

Body broke up, perceptions ceased,\\
All feelings cooled off,\\
Preparations calmed down,\\
Consciousness came to an end.
\end{quote}

This event was of such a great importance that, though it occurred at Veḷuvana Ārāma in Rājagaha, the Buddha related the event to the congregation of monks when he returned to Sāvatthī.

It was not an incidental mention in reply to a particular question, but a special peroration recounting the event and commemorating it with the following two \emph{Udāna} verses, which so aptly constitute the grand finale to our \emph{Udāna} text.

\begin{quote}
\emph{Ayoghanahatass'eva,}\\
\emph{jalato jātavedaso,}\\
\emph{anupubbūpasantassa,}\\
\emph{yathā na ñāyate gati.}

\emph{Evaṁ sammāvimuttānaṁ,}\\
\emph{kāmabandhoghatārinaṁ,}\\
\emph{paññāpetuṁ gatī natthi,}\\
\emph{pattānaṁ acalaṁ sukhaṁ.}\footnote{\href{https://suttacentral.net/ud8.10/pli/ms}{Ud 8.10 / Ud 93}, \emph{Dutiyadabbasutta}}

Just as in the case of a fire\\
Blazing like a block of iron in point of compactness,\\
When it gradually calms down,\\
No path it goes by can be traced.

Even so of those who are well released,\\
Who have crossed over the floods of shackles of sensuality,\\
And reached Bliss Unshaken,\\
There is no path to be pointed out.
\end{quote}

We have deviated from the commentarial interpretation in our rendering of the first two lines of the verse. The commentary gives two alternative meanings, probably because it is in doubt as to the correct one. Firstly it brings in the idea of a bronze vessel that is being beaten at the forge with an iron hammer, giving the option that the gradual subsidence mentioned in the verse may apply either to the flames or to the reverberations of sound arising out of it.\footnote{Ud-a 435} Secondly, as a `some say so' view, \emph{kecidvāda}, it gives an alternative meaning, connected with the ball of iron beaten at the forge.

In our rendering, however, we had to follow a completely different line of interpretation, taking the expression \emph{ayoghanahatassa} as a comparison, \emph{ayoghanahatassa + iva}, for the blazing fire, \emph{jalato jātavedaso}. On seeing a fire that is ablaze, one gets a notion of compactness, as on seeing a red hot block of solid iron.

In the \emph{Dhammapada} verse beginning with \emph{seyyo ayogulo bhutto, tatto aggisikhūpamo},\footnote{Dhp 308, \emph{Nirayavagga}} ``better to swallow a red hot iron ball, that resembles a flame of fire'', a cognate simile is employed somewhat differently. There the ball of iron is compared to a flame of fire. Here the flame of fire is compared to a block of iron.

All in all, it is highly significant that the Buddha uttered three verses of uplift in connection with the \emph{parinibbāna} of the \emph{arahant} Venerable Dabba Mallaputta. The most important point that emerges from this discussion is that Nibbāna is essentially an extinction or extinguishment.

An extinguished fire goes nowhere. In the case of other \emph{arahants}, who were cremated after their \emph{parinibbāna}, there is a left over as ashes for one to perpetuate at least the memory of their existence. But here Venerable Dabba Mallaputta, as if to drive a point home, through his psychic powers based on the fire element, saw to it that neither ashes nor soot will mar his perfect extinguishment in the eyes of the world. That is why the Buddha celebrated it with these special utterances of joy.

So then the cessation of existence is itself Nibbāna. There is no everlasting immortal Nibbāna awaiting the \emph{arahants} at their \emph{parinibbāna}.

That kind of argument the commentaries sometimes put forward is now and then advanced by modern day writers and preachers, too, in their explanations. When it comes to Nibbāna, they resort to two pet parables of recent origin, the parable of the tortoise and the parable of the frog.

In the former, a tortoise goes down into the water and the fishes ask him where he came from. The tortoise replies that he came from land. In order to determine what sort of a thing land is, the fishes go on asking the tortoise a number of questions based on various qualities of water. To each question the tortoise has to reply in the negative, since land has none of the qualities of water.

The parable of the frog is much the same. When it gets into water it has to say `no no' to every question put by the toad, still unfamiliar with land. To make the parables convincing, those negative answers, the `no-nos', are compared to the strings of negative terms that are found in the sutta passages dealing with the \emph{arahattaphalasamādhi}, which we have already quoted.

For instance, to prove their point those writers and teachers would resort to the famous \emph{Udāna} passage beginning with:

\begin{quote}
\emph{Atthi, bhikkhave, tad āyatanaṁ, yattha n'eva pathavī na āpo na tejo na vāyo na ākāsānañcāyatanaṁ na viññāṇānañcāyatanaṁ na ākiñcaññāyatanaṁ na nevasaññānāsaññāyatanaṁ na ayaṁ loko na paraloko na ubho candimasūriyā \ldots{}}\footnote{\href{https://suttacentral.net/ud8.2/pli/ms}{Ud 8.2 / Ud 80}, \emph{Paṭhamanibbānapaṭisaṁyuttasutta}, see \emph{Sermon 17}}

There is, monks, that sphere, in which there is neither earth, nor water, nor fire, nor air; neither the sphere of infinite space, nor the sphere of infinite consciousness, nor the sphere of nothingness, nor the sphere of neither-perception-nor-\\ non-perception; neither this world nor the world beyond, nor the sun and the moon \ldots{}
\end{quote}

But we have reasonably pointed out that those passages do not in any way refer to a non-descript realm into which the \emph{arahants} enter after their demise, a realm that the tortoise and the frog cannot describe. Such facile explanations contradict the deeper teachings on the cessation of existence, dependent arising and not self. They create a lot of misconceptions regarding Nibbāna as the ultimate aim.

The purpose of all those arguments is to assert that Nibbāna is definitely not an annihilation. The ideal of an everlasting Nibbāna is held out in order to obviate nihilistic notions. But the Buddha himself has declared that when he is preaching about the cessation of existence, those who held on to eternalist views wrongly accused him for being an annihilationist, who teaches about the annihilation, destruction and non-existence of a truly existing being, \emph{sato satassa ucchedaṁ vināsaṁ vibhavaṁ paññāpeti.}\footnote{\href{https://suttacentral.net/mn22/pli/ms}{MN 22 / M I 140}, \emph{Alagaddūpamasutta}}

On such occasions, the Buddha did not in any way incline towards eternalism in order to defend himself. He did not put forward the idea of an everlasting Nibbāna to counter the accusation. Instead, he drew attention to the three signata and the four noble truths and solved the whole problem. He maintained that the charge is groundless and utterly misconceived, and concluded with the memorable declaration:

\begin{quote}
\emph{pubbe cāhaṁ, bhikkhave, etarahi ca dukkhañceva paññāpemi, dukkhassa ca nirodhaṁ},

formerly as well as now, O monks, I point out only a suffering and a cessation of that suffering.
\end{quote}

Even the term \emph{tathāgata}, according to him, is not to be conceived as a self. It is only a mass of suffering that has come down through \emph{saṁsāra}, due to ignorance. The so-called existence, \emph{bhava}, is an outcome of grasping, \emph{upādāna}. When grasping ceases, existence comes to an end. That itself is the cessation of existence, \emph{bhavanirodha}, which is Nibbāna.

As the term \emph{anupādā parinibbāna} suggests, there is no grasping in the experience of the cessation of existence. It is only when one is grasping something that he can be identified with it, or reckoned by it. When one lets go of everything, he goes beyond reckoning. Of course, even the commentaries sometimes use the expression \emph{apaññattikabhāvaṁ gatā},\footnote{Sv II 635} ``gone to the state beyond designation'' with regard to the \emph{parinibbāna} of \emph{arahants}.

Nevertheless, they tacitly grant a destination, which in their opinion defies definition. Such vague arguments are riddled with contradictions. They obfuscate the deeper issues of the Dhamma, relating to \emph{paṭicca samuppāda} and \emph{anattā}, and seek to perpetuate personality view by slanting towards eternalism.

It is to highlight some extremely subtle aspects of the problem of Nibbāna that we brought out all these arguments today.
