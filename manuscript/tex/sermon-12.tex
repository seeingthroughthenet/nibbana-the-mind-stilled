\chapter{Sermon 12}

\NibbanaOpeningQuote

With the permission of the Most Venerable Great Preceptor and the assembly of the venerable meditative monks.

This is the twelfth sermon in the series of sermons on Nibbāna. At the beginning of our last sermon, we brought up the two terms \emph{papañca} and \emph{nippapañca}, which help us rediscover quite a deep dimension in Buddhist philosophy, hidden under the sense of time. In our attempt to clarify the meaning of these two terms, initially with the help of the \emph{Madhupiṇḍikasutta}, what we could determine so far is the fact that \emph{papañca} signifies a certain gross state in sense-perception.

Though in ordinary linguistic usage \emph{papañca} meant `elaboration', `circumlocution', and `verbosity', the \emph{Madhupiṇḍikasutta} has shown us that in the context of sensory perception it has some special significance. It portrays how a person, who directed sense perception, is overwhelmed by \emph{papañcasaññāsaṅkhā} with regard to sense-objects relating to the three periods of time, past, present, and future, as a result of his indulging in \emph{papañca} based on reasoning about percepts.

All this goes to show that \emph{papañca} has connotations of some kind of delusion, obsession, and confusion arising in a man's mind due to sense perception.

In explaining the meaning of this term, commentators very often make use of words like \emph{pamatta}, `excessively intoxicated', `indolent', \emph{pamāda}, `headlessness', and \emph{madana}, `intoxication'. For example:

\begin{quote}
\emph{Kenaṭṭhena papañco? Mattapamattākārapāpanaṭṭhena papañco}.\footnote{Sv III 721}

\emph{Papañca} in what sense? In the sense that it leads one on to a state of intoxication and indolence.
\end{quote}

Sometimes it is commented on as follows:

\begin{quote}
\emph{papañcitā ca honti pamattākārapattā.}\footnote{Spk III 73}

They are subject to \emph{papañca}, that is, they become more or less inebriated or indolent.
\end{quote}

Or else it is explained as:

\begin{quote}
\emph{madanākārasaṇṭhito kilesapapañco.}\footnote{Mp III 348}

\emph{Papañca} of a defiling nature which is of an inebriating character.
\end{quote}

On the face of it, \emph{papañca} looks like a term similar in sense to \emph{pamāda}, indolence, heedlessness. But there is a subtle difference in meaning between them.

\emph{Pamāda}, even etymologically, conveys the basic idea of `excessive intoxication'. It has a nuance of inactivity or inefficiency, due to intoxication. The outcome of such a state of affairs is either negligence or heedlessness.

But as we have already pointed out, \emph{papañca} has an etymological background suggestive of expansion, elaboration, verbosity and circumlocution. Therefore, it has no connotations of inactivity and inefficiency. On the other hand, it seems to imply an inability to reach the goal due to a deviation from the correct path.

Let us try to understand the distinction in meaning between \emph{pamāda} and \emph{papañca} with the help of an illustration. Suppose we ask someone to go on an urgent errant to Colombo. If instead of going to Colombo, he goes to the nearest tavern and gets drunk and sleeps there -- that is a case of \emph{pamāda}. If, on the other hand, he takes to a long labyrinthine road, avoiding the shortest cut to Colombo, and finally reaches Kandy instead of Colombo -- that is \emph{papañca}.

There is such a subtle difference in the nuances associated with these two terms. Incidentally, there is a couplet among the Sixes of the \emph{Aṅguttara Nikāya}, which sounds like a distant echo of the illustration we have already given.

\begin{quote}
\emph{Yo papañcam anuyutto}\\
\emph{papañcābhirato mago,}\\
\emph{virādhayī so Nibbānaṁ,}\\
\emph{yogakkhemaṁ anuttaraṁ.}

\emph{Yo ca papañcaṁ hitvāna,}\\
\emph{nippapañca pade rato,}\\
\emph{ārādhayī so Nibbānaṁ,}\\
\emph{yogakkhemaṁ anuttaraṁ.}\footnote{A III 294, \emph{Bhaddakasutta} and \emph{Anutappiyasutta}}

The fool who indulges in \emph{papañca},\\
Being excessively fond of it,\\
Has missed the way to Nibbāna,\\
The incomparable freedom from bondage.

He who, having given up \emph{papañca},\\
delights in the path to \emph{nippapañca},\\
Is well on the way to Nibbāna,\\
The incomparable freedom from bondage.
\end{quote}

In this way we can understand the difference between the two words \emph{papañca} and \emph{pamāda} in respect of the nuances associated with them.

Commentaries very often explain the term \emph{papañca} simply as a synonym of craving, conceit, and views, \emph{taṇhādiṭṭhimānānam etaṁ adhivacanaṁ}.\footnote{Ps II 10} But this does not amount to a definition of \emph{papañca} as such. It is true that these are instances of \emph{papañca,} for even in the \emph{Madhupiṇḍikasutta} we came across the three expressions \emph{abhinanditabbaṁ, abhivaditabbaṁ,} and \emph{ajjhositabbaṁ}, suggestive of them.\footnote{M I 109, \emph{Madhupiṇḍikasutta}}

\emph{Abhinanditabbaṁ} means `what is worth delighting in', \emph{abhivaditabbaṁ} means `what is worth asserting', \emph{ajjhositabbaṁ} means `what is worth clinging on to'. These three expressions are very often used in the discourses to denote the three defilements craving, conceit and views.

That is to say, `delighting in' by way of craving with the thought `this is mine'; `asserting' by way of conceit with the thought `this am I'; and `clinging on to' with the dogmatic view `this is my soul'.

Therefore the commentarial exegesis on \emph{papañca} in terms of craving, conceit and views is to a great extent justifiable. However, what is particularly significant about the term \emph{papañca} is that it conveys the sense of proliferation and complexity of thought, on the lines of those three basic tendencies. That is why the person concerned is said to be `overwhelmed by \emph{papañcasaññāsaṅkhā}'.\footnote{M I 112, \emph{Madhupiṇḍikasutta}}

Here we need to clarify for ourselves the meaning of the word \emph{saṅkhā.} According to the commentary, it means `parts', \emph{papañcasaññāsaṅkhā'ti ettha saṅkhā'ti koṭṭhāso,}\footnote{Ps II 75} ``\,`\emph{papañcasaññāsaṅkhā}', herein `\emph{saṅkhā}' means parts''. In that case \emph{papañcasaṅkhā} could be rendered as `parts of \emph{papañca}', which says nothing significant about \emph{saṅkhā} itself. On the other hand, if one carefully examines the contexts in which the terms \emph{papañcasaññāsaṅkhā} and \emph{papañcasaṅkhā} are used in the discourses, one gets the impression that \emph{saṅkhā} means something deeper than `part' or `portion'.

\emph{Saṅkhā, samaññā} and \emph{paññatti} are more or less synonymous terms. Out of them, \emph{paññatti} is fairly well known as a term for `designation'.

\emph{Saṅkhā} and \emph{samaññā} are associated in sense with \emph{paññatti}. \emph{Saṅkhā} means `reckoning' and \emph{samaññā} is `appellation'. These three terms are often used in connection with worldly usage.

We come across quite a significant reference, relevant to this question of \emph{papañca}, in the \emph{Niruttipathasutta} of the \emph{Khandhasaṁyutta} in the \emph{Saṁyutta Nikāya.} It runs:

\begin{quote}
\emph{Tayome, bhikkhave, niruttipathā, adhivacanapathā, paññattipathā asaṅkiṇṇā asaṅkiṇṇapubbā, na saṅkīyanti, na saṅkīyissanti, appaṭikuṭṭhā samaṇehi brāhmaṇehi viññūhi. Katame tayo? Yaṁ, bhikkhave, rūpaṁ atītaṁ niruddhaṁ vipariṇataṁ 'ahosī'ti tassa saṅkhā, 'ahosī'ti tassa samaññā, 'ahosī'ti tassa paññatti, na tassa saṅkhā 'atthī'ti, na tassa saṅkhā 'bhavissatī'ti.}\footnote{S III 71, \emph{Niruttipathasutta}}

Monks, there are these three pathways of linguistic usage, of synonyms and of designation, that are not mixed up, have never been mixed up, that are not doubted and will not be doubted, and are undespised by intelligent recluses and brahmins. What are the three? Whatever form, monks, that is past, ceased, transformed, `it was' is the reckoning for it, `it was' is its appellation, `it was' is its designation, it is not reckoned as `it is', it is not reckoned as `it will be'.
\end{quote}

The burden of this discourse, as it proceeds in this way, is the maxim that the three periods of time should never be mixed up or confounded. For instance, with regard to that form that is past, a verb in the past tense is used. One must not imagine what is past to be existing as something present. Nor should one imagine whatever belongs to the future as already existing in the present.

Whatever has been, is past. Whatever is, is present. It is a common mistake to conceive of something that is yet to come as something already present, and to imagine whatever is past also as present. This is the confusion the world is in. That is why those recluses and brahmins, who are wise, do not mix them up.

Just as the above quoted paragraph speaks of whatever is past, so the discourse continues to make similar statements with regard to whatever is present or future. It touches upon all the five aggregates, for instance, whatever form that is present is reckoned as `it is', and not as `it was' or `it will be'. Similarly, whatever form that is yet to come is reckoned as `it will be', and not as `it was' or `it is'. This is how the \emph{Niruttipathasutta} lays down the basic principle of not confounding the linguistic usages pertaining to the three periods of time.

Throughout this discourse, the term \emph{saṅkhā} is used in the sense of `reckoning'. In fact, the three terms \emph{saṅkhā, samaññā} and \emph{paññatti} are used somewhat synonymously in the same way as \emph{nirutti, adhivacana} and \emph{paññatti}. All these are in sense akin to each other in so far as they represent the problem of worldly usage.

This makes it clear that the intriguing term \emph{papañcasaññāsaṅkhā} has a relevance to the question of language and modes of linguistic usages. The term could thus be rendered as `reckonings born of prolific perceptions'.

If we are to go deeper into the significance of the term \emph{saṅkhā}, we may say that its basic sense in linguistic usage is connected with numerals, since it means `reckoning'. As a matter of fact, numerals are more primitive than letters, in a language.

To perceive is to grasp a sign of permanence in something. Perception has the characteristic of grasping a sign. It is with the help of signs that one recognizes. Perceptions of forms, perceptions of sounds, perceptions of smells, perceptions of tastes, etc., are so many ways of grasping signs.

Just as a party going through a forest would blaze a trail with an axe in order to find their way back with the help of notches on the trees, so does perception catch a sign in order to be able to recognize.

\enlargethispage{2\baselineskip}

This perception is like the groping of a blind man, fumbling in the dark. There is a tendency in the mind to grasp a sign after whatever is felt. So it gives rise to perceptions of forms, perceptions of sounds, etc. A sign necessarily involves the notion of permanence. That is to say, a sign stands for permanence. A sign has to remain unchanged until one returns to it to recognize it. That is also the secret behind the mirage nature of perception as a whole.\footnote{\emph{Marīcikūpamā saññā} at S III 142, \emph{Pheṇapiṇḍūpamasutta}}

As a matter of fact, the word \emph{saññā}, used to denote perception as such, primarily means the `sign', `symbol', or `mark', with which one recognizes. But recognition alone is not enough. What is recognized has to be made known to the world, to the society at large. That is why \emph{saññā}, or perception, is followed by \emph{saṅkhā}, or reckoning.

The relationship between \emph{saṅkhā, samaññā} and \emph{paññatti} in this connection could also be explained. \emph{Saṅkhā} as `reckoning' or `counting' totals up or adds up into groups of, say, five or six. It facilitates our work, particularly in common or communal activities. So the most primitive symbol in a language is the numeral.

\emph{Samaññā}, or appellation, is a common agreement as to how something should be known. If everyone had its own may of making known, exchange of ideas would be impossible. \emph{Paññatti}, or designation, determines the pattern of whatever is commonly agreed upon. This way we can understand the affinity of meaning between the terms \emph{saṅkhā, samaññā} and \emph{paññatti}.

Among them, \emph{saṅkhā} is the most primitive form of reckoning. It does not simply mean reckoning or adding up in terms of numerals. It is characteristic of language too, as we may infer from the occurrence of the expression \emph{saṅkhaṁ gacchati} in many discourses. There the reckoning meant is a particular linguistic usage. We come across a good illustration of such a linguistic usage in the \emph{Mahāhatthipadopamasutta}, where Venerable Sāriputta is addressing his fellow monks.

\begin{quote}
\emph{Seyyathāpi, āvuso, kaṭṭhañca paṭicca valliñca paṭicca tiṇañca paṭicca mattikañca paṭicca ākāso parivārito agāraṁ tveva saṅkhaṁ gacchati; evameva kho, āvuso, aṭṭhiñca paṭicca nahāruñca paṭicca maṁsañca paṭicca cammañca paṭicca ākāso parivārito rūpaṁ tveva saṅkhaṁ gacchati.}\footnote{M I 190, \emph{Mahāhatthipadopamasutta}}

Friends, just as when space is enclosed by timber and creepers, grass and clay, it comes to be reckoned as `a house'; even so, when space is enclosed by bones and sinews, flesh and skin, it comes to be reckoned as `material form'.
\end{quote}

Here the expression \emph{saṅkhaṁ gacchati} stands for a designation as a concept. It is the way something comes to be known.

Let us go for another illustration from a sermon by the Buddha himself. It is one that throws a flood of light on some deep aspects of Buddhist philosophy, relating to language, grammar and logic. It comes in the \emph{Poṭṭhapādasutta} of the \emph{Dīgha Nikāya}, where the Buddha is exhorting Citta Hatthisāriputta.

\begin{quote}
\emph{Seyyathāpi, Citta, gavā khīraṁ, khīramhā dadhi, dadhimhā navanītaṁ, navanītamhā sappi, sappimhā sappimaṇḍo. Yasmiṁ samaye khīraṁ hoti, neva tasmiṁ samaye dadhī'ti saṅkhaṁ gacchati, na navanītan'ti saṅkhaṁ gacchati, na sappī'ti saṅkhaṁ gacchati, na sappimaṇḍo'ti saṅkhaṁ gacchati, khīraṁ tveva tasmiṁ samaye saṅkhaṁ gacchati.}\footnote{D I 201, \emph{Poṭṭhapādasutta}}

Just, Citta, as from a cow comes milk, and from milk curds, and from curds butter, and from butter ghee, and from ghee junket. But when it is milk, it is not reckoned as curd or butter or ghee or junket, it is then simply reckoned as milk.
\end{quote}

We shall break up the relevant quotation into three parts, for facility of comment. This is the first part giving the introductory simile. The simile itself looks simple enough, though it is suggestive of something deep. The simile is in fact extended to each of the other stages of milk formation, namely curd, butter, ghee, and junket, pointing out that in each case, it is not reckoned otherwise. Now comes the corresponding doctrinal point.

\begin{quote}
\emph{Evameva kho, Citta, yasmiṁ samaye oḷāriko attapaṭilābho hoti, neva tasmiṁ samaye manomayo attapaṭilābho'ti saṅkhaṁ gacchati, na arūpo attapaṭilābho'ti saṅkhaṁ gacchati, oḷāriko attapaṭilābho tveva tasmiṁ samaye saṅkhaṁ gacchati.}

Just so, Citta, when the gross mode of personality is going on, it is not reckoned as `the mental mode of personality', nor as `the formless mode of personality', it is then simply reckoned as `the gross mode of personality'.
\end{quote}

These three modes of personality correspond to the three planes of existence, the sensuous, the form, and the formless. The first refers to the ordinary physical frame, sustained by material food, \emph{kabaḷīkārāhārabhakkho}, enjoying the sense pleasures.\footnote{D I 195, \emph{Poṭṭhapādasutta}} At the time a person is in this sensual field, possessing the gross mode of personality, one must not imagine that the mental mode or the formless mode of personality is hidden in him.

This is the type of confusion the ascetics entrenched in a soul theory fell into. They even conceived of self as fivefold, encased in concentric shells. Whereas in the \emph{Taittirīya Upaniṣad} one comes across the \emph{pañcakośa} theory, the reference here is to three states of the self, as gross, mental and formless modes of personality. Out of the five selves known to \emph{Upaniṣadic} philosophy, namely \emph{annamaya, prāṇamaya, saṁjñāmaya, vijñāṇamaya} and \emph{ānandamaya}, only three are mentioned here, in some form or other. The gross mode of personality corresponds to \emph{annamayātman}, the mental mode of personality is equivalent to \emph{saṁjñāmayātman}, while the formless mode of personality stands for \emph{vijñāṇamayātman}.

The correct perspective of understanding this distinction is provided by the milk simile. Suppose someone gets a \emph{jhāna} and attains to a mental mode of personality. He should not imagine that the formless mode of personality is already latent in him. Nor should he think that the former gross mode of personality is still lingering in him. They are just temporary states, to be distinguished like milk and curd. This is the moral the Buddha is trying to drive home.

Now we come to the third part of the quotation, giving the Buddha's conclusion, which is extremely important.

\begin{quote}
\emph{Imā kho, Citta, lokasamaññā lokaniruttiyo lokavohārā lokapaññattiyo, yāhi Tathāgato voharati aparāmasaṁ.}

For all these, Citta, are worldly apparitions, worldly expressions, worldly usages, worldly designations, which the Tathāgata makes use of without tenacious grasping.
\end{quote}

It is the last word in the quotation, \emph{aparāmasaṁ,} which is extremely important. There is no tenacious grasping. The Buddha uses the language much in the same way as parents make use of a child's homely prattle, for purpose of meditation.

He had to present this Dhamma, which goes against the current,\footnote{\emph{Paṭisotagāmi} at M I 168, \emph{Ariyapariyesanasutta}} through the medium of worldly language, with which the worldlings have their transaction in defilements. That is probably the reason why the Buddha at first hesitated to preach this Dhamma. He must have wondered how he can convey such a deep Dhamma through the terminology, the grammar and the logic of worldlings.

All this shows the immense importance of the \emph{Poṭṭhapādasutta}. If the ordinary worldling presumes that ghee is already inherent in the milk obtained from the cow, he will try to argue it out on the grounds that after all it is milk that becomes ghee. And once it becomes ghee, he might imagine that milk is still to be found in ghee, in some latent form.

As a general statement, this might sound ridiculous. But even great philosophers were unaware of the implications of their theories. That is why the Buddha had to come out with this homely milk simile, to bring them to their senses. Here lies the secret of the soul theory. It carried with it the implication that past and future also exist in the same sense as the present.

The Buddha, on the other hand, uses the verb \emph{atthi}, `is', only for what exists in the present. He points out that, whatever is past, should be referred to as \emph{ahosi}, `was', and whatever is yet to come, in the future, should be spoken of as \emph{bhavissati}, `will be'. This is the fundamental principle underlying the \emph{Niruttipathasutta} already quoted. Any departure from it would give rise to such confusions as referred to above.

Milk, curd, butter and ghee are merely so many stages in a certain process. The worldlings, however, have put them into watertight compartments, by designating and circumscribing them. They are caught up in the conceptual trap of their own making.

When the philosophers started working out the logical relationship between cause and effect, they tended to regard these two as totally unrelated to each other. Since milk becomes curd, either the two are totally different from each other, or curd must already be latent in milk for it to become curd. This is the kind of dilemma their logic posed for them.

Indian philosophical systems reflect a tendency towards such logical subtleties. They ended up with various extreme views concerning the relation between cause and effect. In a certain school of Indian philosophy, known as \emph{ārambhavāda}, effect is explained as something totally new, unrelated to the cause. Other schools of philosophy, such as \emph{satkāriyavāda} and \emph{satkaraṇavāda}, also arose by confusing this issue. For them, effect is already found hidden in the cause, before it comes out. Yet others took only the cause as real. Such extreme conclusions were the result of forgetting the fact that all these are mere concepts in worldly usage. Here we have a case of getting caught up in a conceptual trap of one's own making.

This confusion regarding the three periods of time, characteristic of such philosophers, could be illustrated with some folk tales and fables, which lucidly bring out a deep truth.

There is, for instance, the tale of the goose that lays golden eggs, well known to the West. A certain goose used to lay a golden egg every day. Its owner, out of excessive greed, thought of getting all the as yet ones. He killed the goose and opened it up, only to come to grief. He had wrongly imagined the future to be already existing in the present.

This is the kind of blunder the soul theorists also committed. In the field of philosophy, too, the prolific tendency led to such subtle complications. It is not much different from the proliferations indulged in by the ordinary worldling in his daily life. That is why reckonings born of prolific perception are said to be so overwhelming. One is overwhelmed by one's own reckonings and figurings out, under the influence of prolific perceptions.

An Indian poet once spotted a ruby, shining in the moon light, and eagerly approached it, enchanted by it, only to find a blood red spittle of beetle. We often come across such humorous stories in literature, showing the pitfalls of prolific conceptualisation.

The introductory story, leading up to the \emph{Dhammapada} verse on the rambling nature of the mind, \emph{dūraṅgamaṁ ekacaraṁ, asarīraṁ guhāsayaṁ}, as recorded in the commentary to the \emph{Dhammapada,} is very illustrative.\footnote{Dhp 37, \emph{Cittavagga}; Dhp-a I 301}

The pupil of venerable Saṅgharakkhita Thera, a nephew of his, indulged in a \emph{papañca} while fanning his teacher. In his imagination, he disrobed, got married, had a child, and was coming in a chariot with his wife and child to see his former teacher. The wife, through carelessness, dropped the child and the chariot run away. So he whipped his wife in a fit of anger, only to realize that he had dealt a blow on his teacher's head with the fan still in his hand. Being an \emph{arahant} with psychic powers, his teacher immediately understood the pupil's state of mind, much to the latter's discomfiture.

A potter in Sanskrit literature smashed his pots in a sort of business \emph{papañca} and was remorseful afterwards. Similarly the proud milk maid in English literature dropped a bucket of milk on her head in a day dream of her rosy future. In all these cases one takes as present something that is to come in the future. This is a serious confusion between the three periods of time. The perception of permanence, characteristic of concepts, lures one away from reality into a world of fantasy, with the result that one is overwhelmed and obsessed by it.

So this is what is meant by \emph{papañcasaññāsaṅkhasamudācāra}. So overwhelming are reckonings born of prolific perception. As we saw above, the word \emph{saṅkhā} is therefore nearer to the idea of reckoning than that of part or portion.

\emph{Tathāgatas} are free from such reckonings born of prolific perception, \emph{papañcasaññāsaṅkhā}, because they make use of worldly linguistic usages, conventions and designation, being fully aware of their worldly origin, as if they were using a child's language.

When an adult uses a child's language, he is not bound by it. Likewise, the Buddhas and \emph{arahants} do not forget that these are worldly usages. They do not draw any distinction between the relative and the absolute with regard to those concepts. For them, they are merely concepts and designations in worldly usage. That is why the \emph{tathāgatas} are said to be free from \emph{papañca}, that is to say they are \emph{nippapañca}, whereas the world delights in \emph{papañca.} This fact is clearly expressed in the following verse in the \emph{Dhammapada}.

\begin{quote}
\emph{Ākāse va padaṁ natthi}\\
\emph{samaṇo natthi bāhire,}\\
\emph{papañcābhiratā pajā,}\\
\emph{nippapañcā Tathāgatā.}\footnote{Dhp 254, \emph{Malavagga}}

No track is there in the air,\\
And no recluse elsewhere,\\
This populace delights in prolificity,\\
But `Thus-gone-ones' are non-prolific.
\end{quote}

It is because the \emph{tathāgatas} are non-prolific that \emph{nippapañca} is regarded as one of the epithets of Nibbāna in a long list of thirty-three.\footnote{S IV 370, \emph{Asaṅkhatasaṁyutta}}

Like \emph{dukkhūpasama}, quelling of suffering, \emph{papañcavūpasama,} `quelling of prolificity', is also recognized as an epithet of Nibbāna. It is also referred to as \emph{papañcanirodha}, `cessation of prolificity'. We come across such references to Nibbāna in terms of \emph{papañca} quite often.

The \emph{tathāgatas} are free from \emph{papañcasaññāsaṅkhā}, although they make use of worldly concepts and designations. In the \emph{Kalahavivādasutta} we come across the dictum \emph{saññānidānā hi papañcasaṅkhā},\footnote{Sn 874, \emph{Kalahavivādasutta}} according to which reckonings through prolificity arise from perception. Now the \emph{tathāgatas} have gone beyond the pale of perception in attaining wisdom. That is why they are free from \emph{papañcasaññāsaṅkhā}, reckonings born of prolific perception.

Such reckonings are the lot of those who grope in the murk of ignorance, under the influence of perception. Since Buddhas and \emph{arahants} are enlightened with wisdom and released from the limitations of perception, they do not entertain such reckonings born of prolific perception.

\clearpage

Hence we find the following statement in the \emph{Udāna}:

\begin{quote}
\emph{Tena kho pana samayena Bhagavā attano papañcasaññāsaṅkhāpahānaṁ paccavekkhamāno nisinno hoti.}\footnote{Ud 77, \emph{Papañcakhayasutta}}

And at that time the Exalted One was seated contemplating his own abandonment of reckonings born of prolific perception.
\end{quote}

The allusion here is to the bliss of emancipation. Quite a meaningful verse also occurs in this particular context.

\begin{quote}
\emph{Yassa papañcā ṭhiti ca natthi,}\\
\emph{sandānaṁ palighañca vītivatto,}\\
\emph{taṁ nittaṇhaṁ muniṁ carantaṁ,}\\
\emph{nāvajānāti sadevako pi loko.}\footnote{Ud 77, \emph{Papañcakhayasutta}}

To whom there are no proliferations and standstills,\\
Who has gone beyond the bond and the deadlock,\\
In that craving-free sage, as he fares along,\\
The world with its gods sees nothing to decry.
\end{quote}

The two words \emph{papañca} and \emph{ṭhiti} in juxtaposition highlight the primary sense of \emph{papañca} as a `rambling' or a `straying away'. According to the \emph{Nettippakaraṇa}, the idiomatic standstill mentioned here refers to the latencies, \emph{anusaya}.\footnote{Nett 37}

So the rambling \emph{papañcas} and doggedly persisting \emph{anusayas} are no longer there. The two words \emph{sanḍānaṁ} and \emph{palighaṁ} are also metaphorically used in the Dhamma. Views, \emph{diṭṭhi}, are the bond, and ignorance, \emph{avijjā}, is the deadlock.\footnote{Ud-a 373}

The fact that \emph{papañca} is characteristic of worldly thoughts, connected with the household life, emerges from the following verse in the \emph{Saḷāyatanasaṁyutta} of the \emph{Saṁyutta Nikāya}.

\clearpage

\begin{quote}
\emph{Papañcasaññā itarītarā narā,}\\
\emph{papañcayantā upayanti saññino,}\\
\emph{manomayaṁ gehasitañca sabbaṁ,}\\
\emph{panujja nekkhammasitaṁ irīyati.}\footnote{S IV 71, \emph{Adanta-aguttasutta}}

The common run of humanity, impelled by prolific perception,\\
Approach their objects with rambling thoughts, limited by perception as they are,\\
Dispelling all what is mind-made and connected with the household,\\
One moves towards that which is connected with renunciation.
\end{quote}

The approach meant here is comparable to the approach of that imaginative poet towards the ruby shining in moonlight, only to discover a spittle of beetle. The last two lines of the verse bring out the correct approach of one who is aiming at Nibbāna. It requires the dispelling of such daydreams connected with the household as entertained by the nephew of Venerable Saṅgharakkhita Thera.

Worldlings are in the habit of constructing speculative views by taking too seriously linguistic usage and grammatical structure. All pre-Buddhistic philosophers made such blunders as the confusion between milk and curd. Their blunders were mainly due to two reasons, namely, the persistent latency towards perception and the dogmatic adherence to views. It is precisely these two points that came up in the very first statement of the \emph{Madhupiṇḍikasutta}, discussed in our previous sermon.

That is to say, they formed the gist of the Buddha's cursory reply to the Sakyan Daṇḍapāṇi's question. For the latter it was a riddle and that is why he raised his eyebrows, wagged his tongue and shook his head. The question was:

\begin{quote}
What does the recluse assert and what does he proclaim?\footnote{M I 108, \emph{Madhupiṇḍikasutta}}
\end{quote}

The Buddha's reply was:

\begin{quote}
According to whatever doctrine one does not quarrel or dispute with anyone in the world, such a doctrine do I preach. And due to whatever statements, perceptions do not underlie as latencies, such statements do I proclaim.
\end{quote}

This might well appear a strange paradox. But since we have already made some clarification of the two terms \emph{saññā} and \emph{paññā}, we might as well bring up now an excellent quotation to distinguish the difference between these two. It is in fact the last verse in the \emph{Māgandiyasutta} of the \emph{Sutta Nipāta}, the grand finale as it were.

\begin{quote}
\emph{Saññāviratassa na santi ganthā,}\\
\emph{paññāvimuttassa na santi mohā,}\\
\emph{saññañca diṭṭhiñca ye aggahesuṁ,}\\
\emph{te ghaṭṭhayantā vicaranti loke.}\footnote{Sn 847, \emph{Māgandiyasutta}}

To one unattached to percepts no bonds exist,\\
In one released through wisdom no delusions persist,\\
But they that cling to percepts and views,\\
Go about rambling in this world.
\end{quote}

In the \emph{Pupphasutta} of the \emph{Khandhasaṁyutta} one comes across the following declaration of the Buddha.

\begin{quote}
\emph{Nāhaṁ, bhikkhave, lokena vivadāmi, loko va mayā vivadati.}\footnote{S III 138, \emph{Pupphasutta}}

Monks, I do not dispute with the world, it is the world that is disputing with me.
\end{quote}

This looks more or less like a contradictory statement, as if one would say ``he is quarrelling with me but I am not quarrelling with him''. However, the truth of the statement lies in the fact that the Buddha did not hold on to any view. Some might think that the Buddha also held on to some view or other. But he was simply using the child's language, for him there was nothing worth holding on to in it.

There is a Canonical episode which is a good illustration of this fact. One of the most well-known among the debates the Buddha had with ascetics of other sects is the debate with Saccaka, the ascetic. An account of it is found in the \emph{Cūḷasaccakasutta} of the \emph{Majjhima Nikāya}.

The debate had all the outward appearance of a hot dispute. However, towards the end of it, the Buddha makes the following challenge to Saccaka:

\begin{quote}
As for you, Aggivessana, drops of sweat have come down from your forehead, soaked through your upper robe and reached the ground. But, Aggivessana, there is no sweat on my body now.
\end{quote}

So saying he uncovered his golden-hued body in that assembly,

\begin{quote}
\emph{iti bhagavā tasmiṁ parisatiṁ suvaṇṇavaṇṇaṁ kāyaṁ vivari}.\footnote{M I 233, \emph{Cūḷasaccakasutta}}
\end{quote}

Even in the midst of a hot debate, the Buddha had no agitation because he did not adhere to any views. There was for him no bondage in terms of craving, conceit and views. Even in the thick of a heated debate the Buddha was uniformly calm and cool.

It is the same with regard to perception. Percepts do not persist as a latency in him. We spoke of name-and-form as an image or a reflection. Buddhas do no have the delusion arising out of name-and-form, since they have comprehended it as a self-image. There is a verse in the \emph{Sabhiyasutta} of the \emph{Sutta Nipāta} which puts across this idea.

\begin{quote}
\emph{Anuvicca papañca nāmarūpaṁ,}\\
\emph{ajjhattaṁ bahiddhā ca rogamūlaṁ,}\\
\emph{sabbarogamūlabandhanā pamutto,}\\
\emph{anuvidito tādi pavuccate tathattā.}\footnote{Sn 530, \emph{Sabhiyasutta}}

Having understood name-and-form,\\
\vin which is a product of prolificity,\\
And which is the root of all malady within and without,\\
He is released from bondage to the root of all maladies,\\
That Such-like-one is truly known as\\
\vin `the one who has understood'.
\end{quote}

Name-and-form is a product of \emph{papañca}, the worldling's prolificity. We spoke of the reflection of a gem in a pond and the image of a dog on a plank across the stream.\footnote{See sermons 6 and 7 (dog simile) and sermon 9 (gem simile).} One's grasp on one's world of name-and-form is something similar.

Now as for the Buddha, he has truly comprehended the nature of name-and-form. Whatever maladies, complications and malignant conditions there are within beings and around them, the root cause of all that malady is this \emph{papañca nāmarūpa}. To be free from it is to be `such'. He is the one who has really understood.

If we are to say something in particular about the latency of perception, we have to pay special attention to the first discourse in the \emph{Majjhima Nikāya}. The advice usually given to one who picks up the \emph{Majjhima Nikāya} these days is to skip the very first sutta. Why? Because it is not easy to understand it. Even the monks to whom it was preached could not understand it and were displeased. ``It is too deep for us, leave it alone.''

But it must be pointed out that such an advice is not much different from asking one to learn a language without studying the alphabet. This is because the first discourse of the \emph{Majjhima Nikāya}, namely the \emph{Mūlapariyāyasutta}, enshrines an extremely vital first principle in the entire field of Buddhist philosophy.

Just as much as the first discourse of the \emph{Dīgha Nikāya,} namely the \emph{Brahmajālasutta}, is of great relevance to the question of views, even so the \emph{Mūlapariyāyasutta} is extremely important for its relevance to the question of perception.

Now what is the basic theme of this discourse? There is a certain pattern in the way objects occur to the mind and are apperceived. This discourse lays bare that elementary pattern. The Buddha opens this discourse with the declaration,

\begin{quote}
\emph{sabbadhammamūlapariyāyaṁ vo, bhikkhave, desessāmi,}\footnote{M I 1, \emph{Mūlapariyāyasutta}}

monks, I shall preach to you the basic pattern of behaviour of all mind objects.
\end{quote}

In a nutshell, the discourse deals with twenty-four concepts, representative of concepts in the world. These are fitted into a schema to illustrate the attitude of four types of persons towards them.

The twenty-four concepts mentioned in the sutta are:

\begin{quote}
\emph{paṭhavi, āpo, tejo, vāyo, bhūta, deva, Pajāpati, Brahma, Ābhassara, Subhakinha, Vehapphala, abhibhū, ākāsānañcāyatanaṁ, viññāṇañcāyatanaṁ, ākiñcañāyatanaṁ, nevasaññānāsaññāyatanaṁ, diṭṭhaṁ, sutaṁ, mutaṁ, viññātaṁ, ekattaṁ, nānattaṁ, sabbaṁ, Nibbānaṁ.}

Earth, water, fire, air, beings, gods, Pajāpati, Brahma, the Abhassara Brahmas, the Subhakinha Brahmas, the Vehapphala Brahmas, the overlord, the realm of infinite space, the realm of infinite consciousness, the realm of nothingness, the realm of neither-perception-nor-non-perception, the seen, the heard, the sensed, the cognised, unity, diversity, all, Nibbāna.
\end{quote}

The discourse describes the differences of attitude in four types of persons with regard to each of these concepts. The four persons are:

\begin{enumerate}
\def\labelenumi{\arabic{enumi}.}
\item
  An untaught ordinary person, who has no regard for the Noble Ones and is unskilled in their Dhamma, \emph{assutavā puthujjana}.
\item
  A monk who is in higher training, whose mind has not yet reached the goal and who is aspiring to the supreme security from bondage, \emph{bhikkhu sekho appattamānaso.}
\item
  An \emph{arahant} with taints destroyed who has lived the holy life, done what has to be done, laid down the burden, reached the goal, destroyed the fetters of existence and who is completely liberated through final knowledge, \emph{arahaṁ khīṇāsavo}.
\item
  The Tathāgata, accomplished and fully enlightened, \emph{Tathāgato arahaṁ sammāsambuddho}.
\end{enumerate}

Out of these, the second category comprises the Stream-winner, the Once-returner and the Non-returner. Though there are four types, according to the analysis of their attitudes, the last two can be regarded as one type, since their attitudes to those concepts are the same. So we might as well speak of three kinds of attitudes. Let us now try to understand the difference between them.

What is the world-view of the untaught ordinary person, the worldling? The Buddha describes it as follows:

\begin{quote}
\emph{Paṭhaviṁ paṭhavito sañjānāti. Paṭhaviṁ paṭhavito saññatvā paṭhaviṁ maññati, paṭhaviyā maññati, paṭhavito maññati, 'paṭhaviṁ me'ti maññati, paṭhaviṁ abhinandati. Taṁ kissa hetu? Apariññātaṁ tassā'ti vadāmi.}

He perceives earth as `earth'. Having perceived earth as `earth', he imagines `earth' as such, he imagines `on the earth', he imagines `from the earth', he imagines `earth is mine', he delights in earth. Why is that? I say that it is because he has not fully comprehended it.
\end{quote}

The untaught ordinary person can do no better than to perceive earth as `earth', since he is simply groping in the dark. So he perceives earth as `earth' and goes on imagining, for which the word used here is \emph{maññati}, methinks. One usually methinks when a simile or a metaphor occurs, as a figure of speech. But here it is something more than that. Here it refers to an indulgence in a deluded mode of thinking under the influence of craving, conceit and views. Perceiving earth as `earth', he imagines earth to be substantially `earth'.

Then he resorts to inflection, to make it flexible or amenable to his methinking. `On the earth', `from the earth', `earth is mine', are so many subtle ways of methinking, with which he finally finds delight in the very concept of earth. The reason for all this is the fact that he has not fully comprehended it.

Then comes the world-view of the monk who is in higher training, that is, the \emph{sekha}.

\begin{quote}
\emph{Paṭhaviṁ paṭhavito abhijānāti. Paṭhaviṁ paṭhavito abhiññāya paṭhaviṁ mā maññi, paṭhaviyā mā maññi, paṭhavito mā maññi, 'paṭhaviṁ me'ti mā maññi, paṭhaviṁ mābhinandi. Taṁ kissa hetu? Pariññeyyaṁ tassā'ti vadāmi.}

He understands through higher knowledge earth as `earth'. Having known through higher knowledge earth as `earth', let him not imagine `earth' as such, let him not imagine `on the earth', let him not imagine `from the earth', let him not imagine `earth is mine', let him not delight in earth. Why is that? I say it is because it should be well comprehended by him.
\end{quote}

As for the monk who is in higher training, he does not merely perceive, but understands through higher knowledge.

Here we are against a peculiar expression, which is rather problematic, that is, \emph{mā maññi}.

The commentary simply glosses over with the words \emph{maññatī'ti maññi}, taking it to mean the same as \emph{maññati}, `imagines'.\footnote{Ps I 41} Its only explanation for the use of this peculiar expression in this context is that the \emph{sekha}, or the one in higher training, has already done away with \emph{diṭṭhimaññanā} or imagining in terms of views, though he still has imaginings through craving and conceit. So, for the commentary, \emph{mā maññi} is a sort of mild recognition of residual imagining, a dilly-dally phrase. But this interpretation is not at all convincing.

Obviously enough the particle \emph{mā} has a prohibitive sense here, and \emph{mā maññi} means `let one not imagine', or `let one not entertain imaginings', \emph{maññanā}.

A clear instance of the use of this expression in this sense is found at the end of the \emph{Samiddhisutta}, discussed in an earlier sermon.\footnote{See \emph{Sermon 9}} Venerable Samiddhi answered Venerable Sāriputta's catechism creditably and the latter acknowledged it with a `well-done', \emph{sādhu sādhu}, but cautioned him not to be proud of it, \emph{tena ca mā maññi}, ``but do not be vain on account of~it''.\footnote{A IV 386, \emph{Samiddhisutta}}

The use of the prohibitive particle with reference to the world-view of the monk in higher training is quite apt, as he has to train himself in overcoming the tendency to go on imagining. For him it is a step of training towards full comprehension. That is why the Buddha concludes with the words: ``Why is that? I say it is because it should be well comprehended by~him.''
