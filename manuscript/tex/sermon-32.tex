\chapter{Sermon 32}

\NibbanaOpeningQuote

With the permission of the assembly of the venerable meditative monks. This is the thirty-second sermon in the series of sermons on Nibbāna.

In the course of our last sermon, we took up the position that the seven groups of doctrinal categories collectively known as the thirty-seven participative factors of enlightenment follow an extremely practical and systematic order of arrangement. By way of proof, we discussed at some length the inner consistency evident within each group and the way the different groups are related to each other.

So far, we have pointed out how the setting up of mindfulness through the four foundations of mindfulness serves as a solid basis for the four ways of putting forth energy, by the four right endeavours; and how the progressive stages in putting forth energy, outlined by the four right endeavours, give rise to the four bases for success. It was while discussing the way in which the four bases for success are helpful in arousing the five faculties, like faith, that we had to stop our last sermon.

It should be sufficiently clear, after our discussion the other day, that the four factors desire, energy, determination and investigation could be made the base for success in any venture.

The five faculties, however, are directly relevant to Nibbāna. That is why faith is given pride of place among the faculties. \emph{Saddhindriya}, or the faculty of faith, takes the lead, which is obviously related to \emph{chanda}, desire or interest. But the element of faith in \emph{saddhindriya} is defined at a higher level. In this context, it is reckoned as the firm faith characteristic of the stream-winner.

Then comes the faculty of energy, \emph{viriyindriya}. Though apparently it is yet another occurrence of the term, \emph{viriya} in this context is that element of energy weathered and reinforced by its fourfold application as a base for success, \emph{iddhipāda}.

As for \emph{samādhi} or concentration, we already came across the terms \emph{chandasamādhi, viriyasamādhi, cittasamādhi} and \emph{vīmaṁsāsamādhi} in the description of the development of the bases for success. The concentration meant by \emph{samādhi} in that context is actually a one-pointedness of the mind, \emph{cittekaggatā}, which could be made the basis for arousing energy.

But the level of concentration envisaged by the concentration faculty, \emph{samādhindriya}, is of a higher grade as far as its potential is concerned. It is defined as the first four \emph{jhānas}, based on which one can develop insight and attain Nibbāna. In fact, there is a statement to that effect:

\begin{quote}
\emph{Idha, bhikkhave, ariyasāvako vossaggārammaṇam karitvā labhati samādhiṁ, labhati cittassa ekaggataṁ},\footnote{\href{https://suttacentral.net/sn48.10/pli/ms}{SN 48.10 / S V 197}, \emph{Paṭhamavibhaṅgasutta}}

herein, monks, a noble disciple gains concentration, gains one-pointedness of mind, having made the release {[}of Nibbāna{]} its object.
\end{quote}

The term \emph{vossagga} connotes Nibbāna as a giving up or relinquishment. So the concentration faculty is that concentration which is directed towards Nibbāna.

Similarly the wisdom faculty, as defined here, is of the highest degree, pertaining to the understanding of the four noble truths. Sometimes it is called the ``noble penetrative wisdom of rise and fall'', \emph{udayatthagāminī paññā ariyā nibbedhikā}. By implication, it is equivalent to the factor called \emph{vīmaṁsā}, investigation, we came across in our discussion of the bases for success. As a faculty, it comes out full-fledged in the guise of wisdom.

The mindfulness faculty, which stands in the middle, fulfils a very important function. Now in the context of the four foundations of mindfulness, the role of mindfulness is the simple task of being aware of the appropriate object presented to it. But here in this domain of faculties, mindfulness has attained lordship and fulfils an important function. It maintains the balance between the two sets of pair-wise faculties, by equalizing faith with wisdom and energy with concentration.

This function of balancing of faculties, which mindfulness fulfils, has a special practical value. To one who is striving for Nibbāna, balancing of faculties could sometimes be an intricate problem, since it is more easily said than done.

In order to unravel this problem, let us take up the simile of the rock, we employed the other day. We discussed the question of toppling a rock as an illustration to understand the various stages in the four-fold right endeavour. We distinguished the five stages in putting forth effort in the phrase \emph{chandaṁ janeti, vāyamati, viriyaṁ ārabhati, cittaṁ paggaṇhāti, padahati} with the help of that illustration. Out of these stages, the last one represented by the word \emph{padahati} shows the climax. \emph{Padhāna} or endeavour is the highest grade of effort.

Even verbally it implies something like toppling a rock, which requires a high degree of momentum. This momentum has to be built up mindfully and gradually. That rock, in our illustration, was levered up with great difficulty. After it was levered up, there came that dangerous situation, when it threatened to roll back. It called for that supreme purposeful effort, which required the zeal of self sacrifice. That zealous endeavour is made at the risk of one's body and life.

But even there, one has to be cautious and mindful. If excessive energy is applied in that last heave, one would be thrown off head over heels after the rock. If insufficient energy is applied the rock would roll back and one would get crushed. That is why a balancing is needed before the last spurt. Right endeavour has to be preceded by a balancing.

It is this preliminary balancing that finds mention in a certain highly significant statement in the \emph{Caṅkīsutta} of the \emph{Majjhima Nikāya}, where we are told how a person arouses faith in the Dhamma and gradually develops it and puts forth effort and endeavour and attains Nibbāna. To quote the relevant section of that long sentence:

\begin{quote}
\emph{chandajāto ussahati, ussahitvā tuleti, tulayitvā padahati, pahitatto samāno kāyena ceva paramasaccaṁ sacchikaroti, paññāya ca naṁ ativijjha passati},\footnote{\href{https://suttacentral.net/mn95/pli/ms}{MN 95 / M II 173}, \emph{Caṅkīsutta}}

having aroused a desire or keen interest, he strives; having strived, he balances; having balanced or equalized, he endeavours; and with that endeavour he realizes the highest truth by his body and penetrates into it with wisdom.
\end{quote}

Unfortunately, the key word here, \emph{tulayati} or \emph{tuleti}, is explained in a different way in the commentary. It is interpreted as a reference to contemplation on insight, \emph{aniccādivasena tīreti}, ``adjudges as impermanent, etc.''\footnote{Ps III 426}

But if we examine the word within the context here, as it occurs between \emph{ussahati}, `strives' (literally `bearing up' or `enduring'), and \emph{padahati}, `endeavours', the obvious meaning is `equalizing' or `balancing'.

\emph{Tuleti} has connotations of weighing and judging, and one who strives to lift up a rock needs to know how heavy it is and how much effort is required to topple it. By merely looking at the rock, without trying to lift it up, one cannot say how much effort is needed to topple it. One has to put one's shoulder to it. In fact the word \emph{ussahati} is suggestive of enduring effort with which one bears up.

\enlargethispage{\baselineskip}

Sometimes the Buddha uses the term \emph{ussoḷhī} to designate that steadily enduring effort -- literally, the bearing up. A clear instance of the occurrence of this term in this sense can be found among the Eights of the \emph{Aṅguttara Nikāya} in a discourse on the recollection of death, \emph{maraṇasati}. The sutta is an exhortation to the monks to make use of the recollection of death to reflect on one's unskilful mental states daily in the morning and in the evening with a view to strengthen one's determination to abandon them. For instance, we find the following exhortation:

\begin{quote}
\emph{Sace, bhikkhave, bhikkhu paccavekkhamāno evaṁ pajānāti: `atthi me pāpakā akusalā dhammā appahīnā ye me assu rattiṁ kālaṁ karontassa antarāyāyā'ti, tena, bhikkhave, bhikkhunā tesaṁ yeva pāpakānaṁ akusalānaṁ dhammānaṁ pahānāya adhimatto chando ca vāyāmo ca ussāho ca ussoḷhi ca appaṭivānī ca sati ca sampajaññañca karaṇīyaṁ.}\footnote{A IV 320, \emph{Dutiyamaraṇasatisutta}}

If, monks, upon reflection a monk understands: `There are in me unabandoned evil unskilful states which could spell danger to me if I die today', then, monks, for the abandonment of those very evil unskilful states that monk should arouse a high degree of desire, effort, striving, enduring effort, unremitting effort, mindfulness and full awareness.
\end{quote}

The sequence of terms \emph{chando, vāyāmo, ussāho, ussoḷhi, appaṭivānī, sati} and \emph{sampajañña} is particularly significant in this long sentence.

\emph{Chanda} is that desire to abandon evil unskilful states, \emph{vāyāma} is the initial effort, \emph{ussāha} is literally putting the shoulder to the task, \emph{ussoḷhi} is bearing it up with endurance, \emph{appaṭivānī} is unshrinking effort or unremitting effort. \emph{Sati} is that mindfulness and \emph{sampajañña} that full awareness which are indispensable in this sustained unremitting endeavour.

If a better illustration is needed to clarify the idea of balancing, prior to the final endeavour, we may take the case of lifting a log of wood. Here we have an actual lifting up or putting one's shoulder to it. Without lifting up a log of wood and putting one's shoulder to it, one cannot get to know the art of balancing.

\enlargethispage{-\baselineskip}

If, for instance, the log of wood is thick at one end and thin at the other end, one cannot locate the centre of gravity at a glance. So one puts one's shoulder to one end and goes on lifting it up. It is when one reaches the centre of gravity that one is able to balance it on one's shoulder and take it away. It is because we are looking at this question of balancing of faculties from a practical point of view that we made this detour in explanation.

So, then, the mindfulness faculty is also performing a very important function among these faculties. From the \emph{Saddhāsutta} we quoted the other day we could see that there is also a gradual arrangement in this group of five faculties. That is to say, in a person with faith, energy arises. One who is energetic is keen on developing mindfulness. In one who is mindful, concentration grows; and one who has concentration attains wisdom.

This gradual arrangement becomes all the more meaningful since the faculty of wisdom is declared the chief among the faculties. In the \emph{Indriya Saṁyutta} of the \emph{Saṁyutta Nikāya} the Buddha gives a number of similes to show that the wisdom faculty is supreme in this group.

Just as the lion is supreme among animals, and the footprint of the elephant is the biggest of all footprints, the wisdom faculty is supreme among faculties.\footnote{S V 227, \emph{Sālasutta}; and S V 231, \emph{Padasutta}} The Buddha even goes on to point out that until the wisdom faculty steps in, the other four faculties do not get established. This he makes clear by the simile of the gabled hall in the \emph{Mallikasutta} of the \emph{Indriya Saṁyutta}.

\begin{quote}
Just as, monks, in a gabled hall, so long as the roof peak has not been raised, the rafters are not conjoined, the rafters are not held in place, even so, as long as the noble knowledge has not arisen in a noble disciple, the four faculties are not conjoined, the four faculties are not held in place.\footnote{S V 228, \emph{Mallikasutta}}
\end{quote}

Until one becomes a stream-winner, the five faculties do not get established in him, since the wisdom faculty is so integral. At least one has to be on the path to attaining the fruit of a stream-winner.

It is said that the five faculties are to be found only in the eight noble persons, the four treading on the paths to the four fruits and the four who have attained the fruits of the path, \emph{cattāro ca paṭipannā, cattāro ca phale ṭhitā.}

In others, they are weak and not properly harnessed. It is in the \emph{arahant} that the wisdom faculty is found in its strongest form. In the other grades of supramundane attainment, they are weaker by degrees. The lowest grade is the one treading the path to stream-winning. In the worldling they are not at all to be found, in any way, \emph{sabbena sabbaṁ sabbathā sabbaṁ natthi.}\footnote{S V 202, \emph{Paṭipannasutta}}

Next comes the group of five powers. As to their function, some explanation might be necessary, though it seems simple enough. As we have already mentioned, the term \emph{indriya} connotes kingship or lordship. Faith, energy, mindfulness, concentration and wisdom were elevated to the position of a king or lord. They have attained sovereignty. So now they are exercising their power.

For what purpose? To put down the evil unskilful mental states that rise in revolt against Nibbāna. The noble disciple uses the same faculties as powers to fight out the hindrances and break the fetters. That is why among the participative factors of enlightenment they are represented as powers, by virtue of their special function.

Then we come to the category called seven factors of enlightenment. A high degree of importance is attached to this particular group. It has an orderly arrangement. The constituents are: \emph{sati}, mindfulness; \emph{dhammavicaya}, investigation of states; \emph{viriya}, energy; \emph{pīti}, joy; \emph{passaddhi}, calmness; \emph{samādhi}, concentration; \emph{upekkhā}, equanimity.

In this group of seven, mindfulness takes precedence. In fact, the arrangement resembles the mobilization for winning that freedom of Nibbāna. The \emph{bojjhaṅgā}, factors of enlightenment, are so-called because they are conducive to enlightenment, \emph{bodhāya saṁvattanti}.\footnote{S V 72, \emph{Bhikkhusutta}}

\emph{Sati} leads the way and at the same time marshals the squad. Three members of the group, namely \emph{dhammavicaya, viriya} and \emph{pīti} are by nature restless, while the other three, \emph{passaddhi, samādhi} and \emph{upekkhā} are rather slack. They have to be marshalled and properly aligned, and \emph{sati} comes to the forefront for that purpose.

At the same time, one can discern an orderly arrangement within this group. Right from the stage of the four foundations of mindfulness, the same term \emph{sati} seems to occur down the line, but its function differs in different contexts. Now in this context, it is specifically called a \emph{bojjhaṅga}, a factor of enlightenment. The phrase \emph{satisambojjhaṅgaṁ bhāveti}, ``he develops the enlightenment factor of mindfulness'', is directly used with reference to it here.

When one develops a particular meditation subject, whether it be mindfulness of breathing, \emph{ānāpānasati}, or even one of the four divine abidings of loving kindness, \emph{mettā}, compassion, \emph{karuṇā}, altruistic joy, \emph{muditā}, or equanimity, \emph{upekkhā}, one can arouse these enlightenment factors. That is why we come across, in the \emph{Indriya Saṁyutta}, for instance, such statements as the following:

\begin{quote}
\emph{Idha, bhikkhave, bhikkhu mettāsahagataṁ satisambojjhaṅgaṁ bhāveti vivekanissitaṁ virāganissitaṁ nirodhanissitaṁ vossaggapariṇāmiṁ.}\footnote{S V 119, \emph{Mettāsahagatasutta}}

Herein monks, a monk develops the enlightenment factor of mindfulness imbued with loving kindness, based upon seclusion, dispassion and cessation, maturing in release.
\end{quote}

All the four terms \emph{viveka}, seclusion, \emph{virāga}, dispassion, \emph{nirodha}, cessation, and \emph{vossagga}, release, are suggestive of Nibbāna. So, \emph{satisambojjhaṅga} implies the development of mindfulness as an enlightenment factor, directed towards the attainment of Nibbāna.

What follows in the wake of the enlightenment factor of mindfulness, once it is aroused, is the enlightenment factor of investigation of states, \emph{dhammavicayasambojjhaṅga}, which in fact is the function it fulfils. For instance, in the \emph{Ānandasutta} we read:

\begin{quote}
\emph{so tathā sato viharanto taṁ dhammaṁ paññāya pavicinati pavicarati parivīmaṁsamāpajjati},\footnote{S V 331, \emph{Paṭhamaānandasutta}}

dwelling thus mindfully, he investigates that mental state with wisdom, goes over it mentally and makes an examination of it.
\end{quote}

The mental state refers to the particular subject of meditation, and by investigating it with wisdom and mentally going over it and examining it, the meditator arouses energy. So, from this enlightenment factor one draws inspiration and arouses energy. It is also conducive to the development of wisdom.

This enlightenment factor of investigation of states gives rise to the enlightenment factor of energy since the mental activity implied by it keeps him wakeful and alert, as the following phrase implies:

\begin{quote}
\emph{āraddhaṁ hoti viriyaṁ asallīnaṁ},

energy is stirred up and not inert.
\end{quote}

To one who has stirred up energy, there arises a joy of the spiritual type,

\begin{quote}
\emph{āraddhaviriyassa uppajjati pīti nirāmisā.}
\end{quote}

Of one who is joyful in mind, the body also calms down,

\begin{quote}
\emph{pītimanassa kāyopi passambhati,}
\end{quote}

and so too the mind,

\begin{quote}
\emph{cittampi passambhati.}
\end{quote}

The mind of one who is calm in body and blissful gets concentrated,

\begin{quote}
\emph{passaddhakāyassa sukhino cittaṁ samādhiyati.}
\end{quote}

So now the enlightenment factor of concentration has also come up. What comes after the enlightenment factor of concentration is the enlightenment factor of equanimity. About it, it is said:

\begin{quote}
\emph{so tathāsamāhitaṁ cittaṁ sādhukaṁ ajjhupekkhitā hoti,}

he rightly looks on with equanimity at the mind thus concentrated.
\end{quote}

Once the mind is concentrated, there is no need to struggle or strive. With equanimity one has to keep watch and ward over it.

\clearpage

As an enlightenment factor, equanimity can be evalued from another angle. It is the proper basis for the knowledge of things as they are, \emph{yathābhūtañāṇa}. The neutrality that goes with equanimity not only stabilizes concentration, but also makes one receptive to the knowledge of things as they are. So here we have the seven factors conducive to enlightenment.

What comes next, as the last of the seven groups, is the noble eightfold path, \emph{ariyo aṭṭhaṅgiko maggo}, which is reckoned as the highest among them.

There is some speciality even in the naming of this group. All the other groups show a plural ending, \emph{cattāro satipaṭṭhānā, cattāro sammappadhānā, cattāro iddhipādā, pañc'indriyāni, pañca balāni, satta bojjhaṅgā}, but this group has a singular ending, \emph{ariyo aṭṭhaṅgiko maggo}. The collective sense is suggestive of the fact that this is the \emph{magga-samādhi}, the path concentration. The noble eightfold path is actually the presentation of that concentration of the supramundane path with its constituents. The singular ending is therefore understandable.

This fact comes to light particularly in the \emph{Mahācattārīsakasutta} of the \emph{Majjhima Nikāya}. It is a discourse that brings out a special analysis of the noble eightfold path. There, the Buddha explains to the monks the noble right concentration with its supportive conditions and requisite factors.

\begin{quote}
\emph{Katamo ca, bhikkhave, ariyo sammāsamādhi sa-upaniso saparikkhāro? Seyyathidaṁ sammā diṭṭhi, sammā saṅkappo, sammā vācā, sammā kammanto, sammā ājīvo, sammā vāyāmo sammā sati, yā kho, bhikkhave, imehi sattahaṅgehi cittassa ekaggatā parikkhatā, ayaṁ vuccati, bhikkhave, ariyo sammāsamādhi sa-upaniso iti pi saparikkhāro iti pi.}\footnote{M III 72, \emph{Mahācattārīsakasutta}}

What, monks, is noble right concentration with its supports and requisites? That is, right view, right intention, right speech, right action, right livelihood, right effort and right mindfulness -- that unification of mind equipped with these seven factors is called noble right concentration with its supports and requisites.
\end{quote}

So right concentration itself is the path. The singular number is used to denote the fact that it is accompanied by the requisite factors. Otherwise the plural \emph{maggaṅgā}, factors of the path, could have been used to name this category. The unitary notion has a significance of its own. It is suggestive of the fact that here we have a unification of all the forces built up by the participative factors of enlightenment.

In this discourse, the Buddha comes out with an explanation of certain other important aspects of this noble eightfold path. The fact that right view takes precedence is emphatically stated several times,

\begin{quote}
\emph{tatra, bhikkhave, sammā diṭṭhi pubbaṅgamā},

therein, monks, right view leads the way.
\end{quote}

It is also noteworthy that right view is declared as twofold,

\begin{quote}
\emph{Sammā diṭṭhiṁ pahaṁ dvayaṁ vadāmi.}

Even right view, I say, is twofold.

\emph{Atthi, bhikkhave, sammā diṭṭhi sāsavā puññabhāgiyā upadhivepakkā, atthi, bhikkhave, sammā diṭṭhi ariyā anāsavā lokuttarā maggaṅgā.}

There is right view, monks, that is affected by influxes, on the side of merit and maturing into assets, and there is right view, monks, that is noble, influx-free, supramundane, a factor of the path.
\end{quote}

The first type of right view, which is affected by influxes, on the side of merit and ripening in assets, is the one often met with in general in the analysis of the noble eightfold path, namely the ten-factored right view. It is known as the right view which takes \emph{kamma} as one's own, \emph{kammassakatā sammā diṭṭhi}. The standard definition of it runs as follows:

\begin{quote}
\emph{Atthi dinnaṁ, atthi yiṭṭhaṁ, atthi hutaṁ, atthi sukaṭadukkaṭānaṁ kammānaṁ phalaṁ vipāko, atthi ayaṁ loko, atthi paro loko, atthi mātā, atthi pitā, atthi sattā opapātikā, atthi loke samaṇabrāhmaṇā sammaggatā sammāpaṭipannā ye imañca lokaṁ parañca lokaṁ sayaṁ abhiññā sacchikatvā pavedenti.}

There is {[}an effectiveness{]} in what is given, what is offered and what is sacrificed, there is fruit and result of good and bad deeds, there is this world and the other world, there is mother and father, there are beings who are reborn spontaneously, there are in the world rightly treading and rightly practising recluses and Brahmins who have realized by themselves by direct knowledge and declare this world and the other world.
\end{quote}

This right view is still with influxes, it is on the side of merits and is productive of \emph{saṁsāric} assets. About this right view, this discourse has very little to say. In this sutta, the greater attention is focussed on that right view which is noble, influx-free, supramundane, and constitutes a factor of the path. It is explained as the right view that comes up at the supramundane path moment. It is noble, \emph{ariyā}, influx-free, \emph{anāsavā}, and conducive to transcendence of the world, \emph{lokuttarā}. It is defined as follows:

\begin{quote}
\emph{Yā kho, bhikkhave, ariyacittassa anāsavacittassa ariyamaggasamaṅgino ariyamaggaṁ bhāvayato paññā paññindriyaṁ paññābalaṁ dhammavicayasambojjhaṅgo sammādiṭṭhi maggaṅgā, ayaṁ, bhikkhave, sammādiṭṭhi ariyā anāsavā lokuttarā maggaṅgā.}

Monks that wisdom, that faculty of wisdom, that power of wisdom, that investigation of states enlightenment factor, that path factor of right view in one whose mind is noble, whose mind is influx-free, who has the noble path and is developing the noble path, that is the right view which is noble, influx-free and supramundane, a factor of the path.
\end{quote}

All these synonymous terms are indicative of that wisdom directed towards Nibbāna in that noble disciple. They are representative of the element of wisdom maintained from the faculty stage upwards in his systematic development of the enlightenment factors.

It is also noteworthy that, in connection with the supramundane aspect of the path factors, four significant qualifying terms are always cited, as, for instance, in the following reference to right view:

\begin{quote}
\emph{Idha, bhikkhave, bhikkhu sammādiṭṭhiṁ bhāveti vivekanissitaṁ virāganissitaṁ nirodhanissitaṁ vossaggapariṇāmiṁ.}\footnote{E.g. S V 2, \emph{Upaḍḍhasutta}}

Herein, monks, a monk develops right view which is based upon seclusion, dispassion and cessation, maturing in release.
\end{quote}

This is the higher grade of right view, which aims at Nibbāna. It implies the wisdom of the four noble truths, that noble wisdom which sees the rise and fall, \emph{udayatthagāminī paññā}.

The line of synonymous terms quoted above clearly indicates that the noble eightfold path contains, within it, all the faculties, powers and enlightenment factors so far developed. This is not a mere citation of apparent synonyms for an academic purpose. It brings out the fact that at the path moment the essence of all the wisdom that systematically got developed through the five faculties, the five powers and the seven enlightenment factors surfaces in the noble disciple to effect the final breakthrough.

The two-fold definition given by the Buddha is common to the first five factors of the path: right view, right thought, right speech, right action and right livelihood. That is to say, all these factors have an aspect that can be called `tinged with influxes', \emph{sa-āsava}, `on the side of merit', \emph{puññabhāgiya}, and `productive of \emph{saṁsāric} assets', \emph{upadhivepakka}, as well as an aspect that deserves to be called `noble', \emph{ariya}, `influx-free', \emph{anāsava}, `supramundane', \emph{lokuttara}, `a constituent factor of the path', \emph{maggaṅga}.

The usual definition of the noble eightfold path is well known. A question might arise as to the part played by right speech, right action and right livelihood at the arising of the supramundane path. Their role at the path moment is described as an abstinence from the four kinds of verbal misconduct, an abstinence from the three kinds of bodily misconduct, and an abstinence from wrong livelihood.

The element of abstinence therein implied is conveyed by such terms as \emph{ārati virati paṭivirati veramaṇī}, ``desisting from, abstaining, refraining, abstinence''. It is the very thought of abstaining that represents the three factors at the path moment and not their physical counterparts. That is to say, the act of refraining has already been accomplished.

So then we are concerned only with the other five factors of the path. Out of them, three factors are highlighted as running around and circling around each of these five for the purpose of their fulfilment, namely right view, right effort and right mindfulness. This running around and circling around, conveyed by the two terms \emph{anuparidhāvanti} and \emph{anuparivattanti}, is extremely peculiar in this context.

The role of these three states might be difficult for one to understand. Perhaps, as an illustration, we may take the case of a VIP, a very important person, being conducted through a crowd with much pomp. One ushers him in with his vanguard, another brings up the rear with his bandwagon while yet another is at hand as the bodyguard-cum-attendant. So also at the path moment right view shows the way, right effort gives the boost, while right mindfulness attends at hand.

These security forces keep the wrong side, \emph{micchā}, of the path factors in check. The precedence of right view is a salient feature of the noble eightfold path. The Buddha makes special mention of it, pointing out at the same time the inner consistency of its internal arrangement.

\begin{quote}
\emph{Tatra, bhikkhave, sammā diṭṭhi pubbaṅgamā hoti. Kathañca, bhikkhave, sammā diṭṭhi pubbaṅgamā hoti? Sammā diṭṭhissa, bhikkhave, sammā saṅkappo pahoti, sammā saṅkappassa sammā vācā pahoti, sammā vācassa sammā kammanto pahoti, sammā kammantassa sammā ājīvo pahoti, sammā ājīvassa sammā vāyāmo pahoti, sammā vāyāmassa sammā sati pahoti, sammā satissa sammā samādhi pahoti, sammā samādhissa sammā ñāṇam pahoti, sammā ñāṇassa sammā vimutti pahoti. Iti kho, bhikkhave, aṭṭhaṅgasamannāgato sekho pāṭipado, dasaṅgasamannāgato arahā hoti.}\footnote{M III 76, \emph{Mahācattārīsakasutta}}

Therein, monks, right view comes first. And how, monks, does right view come first? In one of right view, right intention arises. In one of right intention, right speech arises. In one of right speech, right action arises. In one of right action, right livelihood arises. In one of right livelihood, right effort arises. In one of right effort, right mindfulness arises. In one of right mindfulness, right concentration arises. In one of right concentration, right knowledge arises. In one of right knowledge, right deliverance arises. Thus, monks, the disciple in higher training possessed of eight factors becomes an \emph{arahant} when possessed of the ten factors.
\end{quote}

The fundamental importance of right view as the forerunner is highlighted by the Buddha in some discourses. In a particular discourse in the \emph{Aṅguttara Nikāya}, it is contrasted with the negative role of wrong view.

\begin{quote}
\emph{Micchādiṭṭhikassa, bhikkhave, purisapuggalassa yañceva kāyakammaṁ yathādiṭṭhi samattaṁ samādinnaṁ yañca vacīkammaṁ yathādiṭṭhi samattaṁ samādinnaṁ yañca manokammaṁ yathādiṭṭhi samattaṁ samādinnaṁ yā ca cetanā yā ca patthanā yo ca paṇidhi ye ca saṅkhārā sabbe te dhammā aniṭṭhaya akantāya amanāpāya ahitāya dukkhāya saṁvattanti. Taṁ kissa hetu? Diṭṭhi hi, bhikkhave, pāpikā.}\footnote{A I 32, \emph{Ekadhammapāḷi}}

Monks, in the case of a person with wrong view, whatever bodily deed he does accords with the view he has grasped and taken up, whatever verbal deed he does accords with the view he has grasped and taken up, whatever mental deed he does accords with the view he has grasped and taken up, whatever intention, whatever aspiration, whatever determination, whatever preparations he makes, all those mental states conduce to unwelcome, unpleasant, unwholesome, disagreeable and painful consequences. Why is that? The view, monks, is evil.
\end{quote}

Due to the evil nature of the view, all what follows from it partakes of an evil character. Then he gives an illustration for it.

\begin{quote}
\emph{Seyyathāpi, bhikkhave, nimbabījaṁ vā kosātakībījaṁ vā tittakalābubījaṁ vā allāya paṭhaviyā nikkhittaṁ yañceva paṭhavirasaṁ upādiyati yañca āporasaṁ upādiyati sabbaṁ taṁ tittakattāya kaṭukattāya asātattāya saṁvattati. Taṁ kissa hetu? Bījaṁ hi, bhikkhave, pāpakaṁ.}

Just as, monks, in the case of a margosa seed or a bitter gourd seed, or a long gourd seed thrown on wet ground, whatever taste of the earth it draws in, whatever taste of the water it draws in, all that conduces to bitterness, to sourness, to unpleasantness. Why is that? The seed, monks, is bad.
\end{quote}

Then he makes a similar statement with regard to right view.

\begin{quote}
\emph{Sammādiṭṭhikassa, bhikkhave, purisapuggalassa yañceva kāyakammaṁ yathādiṭṭhi samattaṁ samādinnaṁ yañca vacīkammaṁ yathādiṭṭhi samattaṁ samādinnaṁ yañca manokammaṁ yathādiṭṭhi samattaṁ samādinnaṁ yā ca cetanā yā ca patthanā yo ca paṇidhi ye ca saṅkhārā sabbe te dhammā iṭṭhaya kantāya manāpāya hitāya sukhāya saṁvattanti. Taṁ kissa hetu? Diṭṭhi hi, bhikkhave, bhaddikā.}

Monks, in the case of a person with right view, whatever bodily deed he does accords with the view he has grasped and taken up, whatever verbal deed he does accords with the view he has grasped and taken up, whatever mental deed he does accords with the view he has grasped and taken up, whatever intention, whatever aspiration, whatever determination, whatever preparations he makes, all those mental states conduce to welcome, pleasant, wholesome, agreeable and happy consequences. Why is that? The view, monks, is good.
\end{quote}

Then comes the illustration for it.

\begin{quote}
\emph{Seyyathāpi, bhikkhave, ucchubījaṁ vā sālibījaṁ vā muddikābījaṁ vā allāya paṭhaviyā nikkhittaṁ yañceva paṭhavirasaṁ upādiyati yañca āporasaṁ upādiyati sabbaṁ taṁ madhurattāya sātattāya asecanakattāya saṁvattati. Taṁ kissa hetu? Bījaṁ hi, bhikkhave, bhaddakaṁ.}

Just as, monks, in the case of a sugar cane seedling or a sweet paddy seed, or a grape seed thrown on wet ground, whatever taste of the earth it draws in, whatever taste of the water it draws in, all that conduces to sweetness, agreeableness and deliciousness. Why is that? The seed, monks, is excellent.
\end{quote}

This explains why the noble eightfold path begins with right view. This precedence of view is not to be found in the other groups of participative factors of enlightenment. The reason for this peculiarity is the fact that view has to come first in any total transformation of personality in an individual from a psychological point of view.

\enlargethispage{\baselineskip}

A view gives rise to thoughts, thoughts issue in words, words lead to actions, and actions mould a livelihood. Livelihood forms the basis for the development of other virtues on the side of meditation, namely right effort, right mindfulness and right concentration. So we find the precedence of right view as a unique feature in the noble eightfold path.

The fundamental importance of the noble eightfold path could be assessed from another point of view. It gains a high degree of recognition due to the fact that the Buddha has styled it as the middle path. For instance, in the \emph{Dhammacakkappavattanasutta}, the discourse on the turning of the wheel, the middle path is explicitly defined as the noble eightfold path. It is sufficiently well known that the noble eightfold path has been called the middle path by the Buddha. But the basic idea behind this definition has not always been correctly understood.

In the \emph{Dhammacakkappavattanasutta} the Buddha has presented the noble eightfold path as a middle path between the two extremes called \emph{kāmasukhallikānuyogo}, the pursuit of sensual pleasure, and \emph{attakilamathānuyogo}, the pursuit of self-mortification.\footnote{\href{https://suttacentral.net/sn56.11/pli/ms}{SN 56.11 / S V 421}, \emph{Dhammacakkappavattanasutta}}

The concept of a `middle' might make one think that the noble eightfold path is made up by borrowing fifty per cent from each of the two extremes, the pursuit of sense pleasures and the pursuit of self-mortification. But it is not such a piecemeal solution. There are deeper implications involved.

The \emph{Mahācattārīsakasutta} in particular brings out the true depth of this middle path. Instead of grafting half of one extreme to half of the other, the Buddha rejected the wrong views behind both those pursuits and, avoiding the pitfalls of both, presented anew a middle path in the form of the noble eightfold path.

By way of clarification, we may draw attention to the fact that one inclines to the pursuit of sense pleasures by taking one's stance on the annihilationist view. It amounts to the idea that there is no rebirth and that one can indulge in sense pleasures unhindered by ethical considerations of good and evil. It inculcates a nihilistic outlook characterized by a long line of negatives.

In contradistinction to it, we have the affirmative standpoint forming the lower grade of the right view referred to above, namely the right view which takes \emph{kamma} as one's own, \emph{kammasakatā sammā diṭṭhi}. The positive outlook in this right view inculcates moral responsibility and forms the basis for skilful or meritorious deeds. That is why it is called \emph{puññabhāgiya,} on the side of merits. By implication, the nihilistic outlook, on the other hand, is on the side of demerit, lacking a basis for skilful action.

In our analysis of the law of dependent arising, also, we happened to mention the idea of a middle path. But that is from the philosophical standpoint. Here we are concerned with the ethical aspect of the middle path. As far as the ethical requirements are concerned, a nihilistic view by itself does not entitle one to deliverance. Why? Because the question of influxes is there to cope with.

So long as the influxes of sensuality, \emph{kāmāsavā}, of becoming, \emph{bhavāsavā}, and of ignorance, \emph{avijjāsavā}, are there, one cannot escape the consequences of action merely by virtue of a nihilistic view. That is why the Buddha took a positive stand on those ten postulates. Where the nihilist found an excuse for indulgence in sensuality by negating, the Buddha applied a corrective by asserting. This affirmative stance took care of one extremist trend.

But the Buddha did not stop there. In the description of the higher grade of right view we came across the terms \emph{ariyā anāsavā lokuttarā maggaṅgā}. In the case of the lower grade it is \emph{sa-āsavā}, with influxes, here it is \emph{anāsavā}, influx-free. At whatever moment the mind develops that strength to withstand the influxes, one is not carried away by worldly conventions. That is why the right view at the supramundane path moment is called influx-free.

There is an extremely subtle point involved in this distinction. This noble influx-free right view, that is a constituent of the supramundane path, \emph{ariyā anāsavā lokuttarā maggaṅgā}, is oriented towards cessation, \emph{nirodha}. The right view that takes \emph{kamma} as one's own, \emph{kammasakatā sammā diṭṭhi}, on the other hand is oriented towards arising, \emph{samudaya}.

Due to the fact that the right view at the path moment is oriented towards cessation we find it qualified with the terms:

\begin{quote}
\emph{vivekanissitaṁ virāganissitaṁ nirodhanissitaṁ vossaggapariṇāmiṁ},

based upon seclusion, dispassion and cessation, maturing in release.
\end{quote}

It is this orientation towards Nibbāna that paves the way for the signless, \emph{animitta}, the undirected, \emph{appaṇihita}, and the void, \emph{suññata}. We have already discussed at length about them in our previous sermons.

Perhaps, while listening to them, some might have got scared at the thought: ``So then there is not even a mother or a father''. That is why the word \emph{suññatā}, voidness, drives terror into those who do not understand it properly. Here we see the depth of the Buddha's middle path. That right view with influxes, \emph{sa-āsavā}, is on the side of merits, \emph{puññabhāgiya}, not demerit, \emph{apuñña}.

If the Buddha sanctions demerit, he could have endorsed the nihilistic view that there is no this world or the other world, no mother or father. But due to the norm of \emph{kamma} which he explained in such terms as

\begin{quote}
\emph{kammassakā sattā kammadāyādā kammayonī kammabandhū},\footnote{M III 203, \emph{Cūḷakammavibhaṅgasutta}}

beings have \emph{kamma} as their own, they are inheritors of \emph{kamma}, \emph{kamma} is their matrix, \emph{kamma} is their relative.
\end{quote}

So long as ignorance and craving are there, beings take their stand on convention and go on accumulating \emph{kamma}. They have to pay for it. They have to suffer the consequences.

Though with influxes, \emph{sa-āsava}, that right view is on the side of merit, \emph{puññabhāgiya}, which mature into \emph{saṁsāric} assets, \emph{upadhivepakka}, in the form of the conditions in life conducive even to the attainment of Nibbāna. That kind of right view is preferable to the nihilistic view, although it is of a second grade.

But then there is the other side of the \emph{saṁsāric} problem. One cannot afford to stagnate there. There should be a release from it as a permanent solution. That is where the higher grade of right view comes in, the noble influx-free right view which occurs as a factor of the path. It is then that the terms \emph{animitta}, signless, \emph{appaṇihita}, the undirected and \emph{suññata}, the void, become meaningful.

When the mind is weaned away from the habit of grasping signs, from determining and from the notion of self-hood, the three doorways to deliverance, the signless, the undirected and the void, would open up for an exit from this \emph{saṁsāric} cycle. The cessation of existence is Nibbāna, \emph{bhavanirodho nibbānaṁ}. Here, then, we have the reason why the noble eightfold path is called the middle path.

In the life of a meditator, also, the concept of a middle path could sometimes give rise to doubts and indecision. One might wonder whether one should strive hard or lead a comfortable life. A midway solution between the two might be taken as the middle path. But the true depth of the middle path emerges from the above analysis of the twofold definition of the noble eightfold path.

It is because of this depth of the middle path that the Buddha made the following declaration in the \emph{Aggappasādasutta} of the \emph{Aṅguttara Nikāya}:

\begin{quote}
\emph{Yāvatā, bhikkhave, dhammā saṅkhatā, ariyo aṭṭhaṅgiko maggo tesaṁ aggam akkhāyati.}\footnote{\href{https://suttacentral.net/an4.34/pli/ms}{AN 4.34 / A II 34}, \emph{Aggappasādasutta}}

Monks, whatever prepared things there are, the noble eightfold path is called the highest among them.
\end{quote}

It is true that the noble eightfold path is something prepared and that is why we showed its relation to causes and conditions. Whatever is prepared is not worthwhile, and yet, it is by means of this prepared noble eightfold path that the Buddha clears the path to the unprepared.

This is an extremely subtle truth, which only a Buddha can discover and proclaim to the world. It is not easy to discover it, because one tends to confuse issues by going to one extreme or another. One either resorts to the annihilationist view and ends up by giving way to indulgence in sensuality, or inclines towards the eternalist view and struggles to extricate self by self-mortification.

\enlargethispage{\baselineskip}

In the Dhamma proclaimed by the Buddha one can see a marvellous middle way. We have already pointed it out in earlier sermons by means of such illustrations as sharpening a razor. There is a remarkable attitude of non-grasping about the middle path, which is well expressed by the term \emph{atammayatā}, non-identification. Relying on one thing is just for the purpose of eliminating another, as exemplified by the simile of the relay of chariots.

The key terms signifying the aim and purpose of this middle path are

\begin{quote}
\emph{vivekanissitaṁ virāganissitaṁ nirodhanissitaṁ vossaggapariṇāmiṁ},

based upon seclusion, dispassion and cessation, maturing in release.
\end{quote}

Placed in this \emph{saṁsāric} predicament, one cannot help resorting to certain things to achieve this aim. But care is taken to see that they are not grasped or clung to. It is a process of pushing away one thing with another, and that with yet another, a via media based on relativity and pragmatism. The noble eightfold path marks the consummation of this process, its systematic fulfilment. That is why we tried to trace a process of a gradual development among the thirty-seven participative factors of enlightenment.

Even the internal arrangement within each group is extraordinary. There is an orderly arrangement from beginning to end in an ascending order of importance. Sometimes, an analysis could start from the middle and extend to either side. Some groups portray a gradual development towards a climax. The noble eightfold path is exceptionally striking in that it indicates how a complete transformation of personality could be effected by putting right view at the head as the forerunner.

Perhaps the most impressive among discourses in which the Buddha highlighted the pervasive significance of the noble eightfold path is the \emph{Ākāsasutta}, `Sky Sutta', in the \emph{Magga Saṁyutta} of the \emph{Saṁyutta Nikāya}.\footnote{S V 49, \emph{Ākāsasutta}}

\enlargethispage{\baselineskip}

\begin{quote}
Just as, monks, various winds blow in the sky, easterly winds, westerly winds, northerly winds, southerly winds, dusty winds, dustless winds, cold winds and hot winds, gentle winds and strong winds; so too, when a monk develops and cultivates the noble eightfold path, for him the four foundations of mindfulness go to fulfilment by development, the four right efforts go to fulfilment by development, the four bases for success go to fulfilment by development, the five spiritual faculties go to fulfilment by development, the five powers go to fulfilment by development, the seven factors of enlightenment go to fulfilment by development.
\end{quote}

All these go to fulfilment by development only when the noble eightfold path is developed in the way described above, namely based upon seclusion, dispassion and cessation, maturing in release, \emph{vivekanissitaṁ virāganissitaṁ nirodhanissitaṁ vossaggapariṇāmiṁ}.

That is to say, with Nibbāna as the goal of endeavour. Then none of the preceding categories go astray. They all contribute to the perfection and fulfilment of the noble eightfold path. They are all enshrined in it. So well knitted and pervasive is the noble eightfold path.

Another discourse of paramount importance, which illustrates the pervasive influence of the noble eightfold path, is the \emph{Mahāsaḷāyatanikasutta} of the \emph{Majjhima Nikāya}. There the Buddha shows us how all the other enlightenment factors are included in the noble eightfold path.

In our discussion on Nibbāna, we happened to mention that the cessation of the six sense-spheres is Nibbāna. If Nibbāna is the cessation of the six sense-spheres, it should be possible to lay down a way of practice leading to Nibbāna through the six sense-spheres themselves. As a matter of fact, there is such a way of practice and this is what the \emph{Mahāsaḷāyatanikasutta} presents in summary form.

In this discourse, the Buddha first portrays how on the one hand the \emph{saṁsāric} suffering arises depending on the six-fold sense-sphere. Then he explains how on the other hand the suffering could be ended by means of a practice pertaining to the six-fold sense-sphere itself.

\begin{quote}
\emph{Cakkhuṁ, bhikkhave, ajānaṁ apassaṁ yathābhūtaṁ, rūpe ajānaṁ apassaṁ yathābhūtaṁ, cakkhuviññāṇaṁ ajānaṁ apassaṁ yathābhūtaṁ, cakkhusamphassaṁ ajānaṁ apassaṁ yathābhūtaṁ, yampidaṁ cakkhusamphassapaccayā uppajjati vedayitaṁ sukhaṁ vā dukkhaṁ vā adukkhamasukhaṁ vā tampi ajānaṁ apassaṁ yathābhūtaṁ, cakkhusmiṁ sārajjati, rūpesu sārajjati, cakkhuviññāṇe sārajjati, cakkhusamphasse sārajjati, yampidaṁ cakkhusamphassapaccayā uppajjati vedayitaṁ sukhaṁ vā dukkhaṁ vā adukkhamasukhaṁ vā tasmimpi sārajjati.}

\emph{Tassa sārattassa saṁyuttasa sammūḷhassa assādānupassino viharato āyatiṁ pañcupādānakkhandhā upacayaṁ gacchanti. Taṇhā cassa ponobhavikā nandirāgasahagatā tatratatrābhinandinī sā cassa pavaḍḍhati. Tassa kāyikāpi darathā pavaḍḍhanti, cetasikāpi darathā pavaḍḍhanti, kāyikāpi santāpā pavaḍḍhanti, cetasikāpi santāpā pavaḍḍhanti, kāyikāpi pariḷāhā pavaḍḍhanti, cetasikāpi pariḷāhā pavaḍḍhanti. So kāyadukkhampi cetodukkhampi paṭisaṁvedeti.}\footnote{\href{https://suttacentral.net/mn149/pli/ms}{MN 149 / M III 287}, \emph{Mahāsaḷāyatanikasutta}}

Monks, not knowing and not seeing the eye as it actually is, not knowing and not seeing forms as they actually are, not knowing and not seeing eye-consciousness as it actually is, not knowing and not seeing eye-contact as it actually is, whatever is felt as pleasant or unpleasant or neither-unpleasant-nor-pleasant, arising dependent on eye-contact, not knowing and not seeing that too as it actually is, one gets lustfully attached to the eye, to forms, to eye-consciousness, to eye-contact, and to whatever is felt as pleasant or unpleasant or neither-unpleasant-nor-pleasant, arising in dependence on eye-contact.

And for him, who is lustfully attached, fettered, infatuated, contemplating gratification, the five aggregates of grasping get accumulated for the future and his craving, which makes for re-becoming, which is accompanied by delight and lust, delighting now here now there, also increases, his bodily stresses increase, his mental stresses increase, his bodily torments increase, his mental torments increase, his bodily fevers increase, his mental fevers increase, and he experiences bodily and mental suffering.
\end{quote}

In this way, the Buddha first of all delineates how the entire \emph{saṁsāric} suffering arises in connection with the six-fold sense-sphere. We will discuss the rest of the discourse in our next sermon.
