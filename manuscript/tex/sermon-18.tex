\chapter{Sermon 18}

\NibbanaOpeningQuote

With the permission of the Most Venerable Great Preceptor and the assembly of the venerable meditative monks. This is the eighteenth sermon in the series of sermons on Nibbāna.

We happened to mention, in our last sermon, that many of the discourses dealing with the subject of Nibbāna, have been misinterpreted, due to a lack of appreciation of the fact that the transcendence of the world and crossing over to the farther shore of existence have to be understood in a psychological sense.

The view that the \emph{arahant} at the end of his life enters into an absolutely existing \emph{asaṅkhata dhātu}, or `unprepared element', seems to have received acceptance in the commentarial period. In the course of our last sermon, we made it very clear that some of the discourses cited by the commentators in support of that view deal, on the contrary, with some kind of realization the \emph{arahant} goes through here and now, in this very life, in this very world -- a realization of the cessation of existence, or the cessation of the six sense-spheres.

Even when the Buddha refers to the \emph{arahant} as the Brahmin who, having gone beyond, is standing on the farther shore,\footnote{E.g. It 57, \emph{Dutiyarāgasutta}: \emph{tiṇṇo pāraṁ gato thale tiṭṭhati brāhmaṇo}} he was speaking of the \emph{arahant} who has realized, in this very life, the influx-free deliverance of the mind and deliverance through wisdom, in his concentration of the fruit of \emph{arahanthood}.

Therefore, on the strength of this evidence, we are compelled to elicit a subtler meaning of the concept of `this shore' and the `farther shore' from these discourses dealing with Nibbāna than is generally accepted in the world. Our sermon today is especially addressed to that end.

As we mentioned before, if one is keen on getting a solution to the problems relating to Nibbāna, the discourses we are now taking up for discussion might reveal the deeper dimensions of that problem.

We had to wind up our last sermon while drawing out the implications of the last line in the \emph{Paramaṭṭhakasutta} of the \emph{Sutta Nipāta}: \emph{pāraṁgato na pacceti tādi.}\footnote{Sn 803, \emph{Paramaṭṭhakasutta}} We drew the inference that the steadfast one, the arahant, who is such-like, once gone to the farther shore, does not come back.

We find, however, quite a different idea expressed in a verse of the \emph{Nālakasutta} in the \emph{Sutta Nipāta}. The verse, which was the subject of much controversy among the ancients, runs as follows:

\begin{quote}
\emph{Uccāvāca hi paṭipadā,}\\
\emph{samaṇena pakāsitā,}\\
\emph{na pāraṁ diguṇaṁ yanti,}\\
\emph{na idaṁ ekaguṇaṁ mutaṁ.}\footnote{Sn 714, \emph{Nālakasutta}}

High and low are the paths,\\
Made known by the recluse,\\
They go not twice to the farther shore,\\
Nor yet is it to be reckoned a going once.
\end{quote}

The last two lines seem to contradict each other. There is no going twice to the farther shore, but still it is not to be conceived as a going once.

Now, as for the first two lines, the high and low paths refer to the modes of practice adopted, according to the grades of understanding in different character types.

For instances, the highest grade of persons attains Nibbāna by an easy path, being quick-witted, \emph{sukhā paṭipadā khippābhiññā}, whereas the lowest grade attains it by a difficult path, being relatively dull-witted, \emph{dukkhā paṭipadā dandhābhiññā}.\footnote{See e.g.~A II 149, \emph{Saṁkhittasutta}}

The problem lies in the last two lines. The commentary tries to tackle it by interpreting the reference to not going twice to the farther shore, \emph{na pāraṁ diguṇaṁ yanti}, as an assertion that there is no possibility of attaining Nibbāna by the same path twice, \emph{ekamaggena dvikkhattuṁ nibbānaṁ na yanti}.\footnote{Pj II 498} The implication is that the supramundane path of a stream-winner, a once-returner or a non-returner arises only once. Why it is not to be conceived as a going once is explained as an acceptance of the norm that requires not less than four supramundane paths to attain \emph{arahanthood}.

However, a deeper analysis of the verse in question would reveal the fact that it effectively brings up an apparent contradiction. The commentary sidetracks by resolving it into two different problems. The two lines simply reflect two aspects of the same problem.

They go not twice to the farther shore, and this not going twice, \emph{na idaṁ}, is however not to be thought of as a `going once' either. The commentary sidetracks by taking \emph{idaṁ}, `this', to mean the farther shore, \emph{pāraṁ}, whereas it comprehends the whole idea of not going twice. Only then is the paradox complete.

In other words, this verse concerns the such-like one, the \emph{arahant}, and not the stream-winner, the once-returner or the non-returner. Here we have an echo of the idea already expressed as the grand finale of the \emph{Paramaṭṭhakasutta}: \emph{pāraṁgato na pacceti tādi},\footnote{Sn 803, \emph{Paramaṭṭhakasutta}} the such-like one, ``gone to the farther shore, comes not back''.

It is the last line, however, that remains a puzzle. Why is this `not going twice,' not to be thought of as a `going once'? There must be something deep behind this riddle.

Now, for instance, when one says `I won't go there twice', it means that he will go only once. When one says `I won't tell twice', it follows that he will tell only once. But here we are told that the \emph{arahant} goes not twice, and yet it is not a going once.

The idea behind this riddle is that the influx-free \emph{arahant}, the such-like-one, gone to the farther shore, which is supramundane, does not come back to the mundane. Nevertheless, he apparently comes back to the world and is seen to experience likes and dislikes, pleasures and pains, through the objects of the five senses. From the point of view of the worldling, the \emph{arahant} has come back to the world. This is the crux of the problem.

Why is it not to be conceived of as a going once? Because the \emph{arahant} has the ability to detach himself from the world from time to time and re-attain to that \emph{arahattaphalasamādhi}.

It is true that he too experiences the objects of the five external senses, but now and then he brings his mind to dwell in that \emph{arahattaphalasamādhi,} which is like standing on the farther shore.

Here, then, we have an extremely subtle problem. When the \emph{arahant} comes back to the world and is seen experiencing the objects of the five senses, one might of course conclude that he is actually `in the world'.

This problematic situation, namely the question how the influx-free \emph{arahant}, gone to the farther shore, comes back and takes in objects through the senses, the Buddha resolves with the help of a simple simile, drawn from nature. For instance, we read in the \emph{Jarāsutta} of the \emph{Sutta Nipāta} the following scintillating lines.

\begin{quote}
\emph{Udabindu yathā pi pokkhare,}\\
\emph{padume vāri yathā na lippati,}\\
\emph{evaṁ muni nopalippati,}\\
\emph{yadidaṁ diṭṭhasutammutesu vā.}\footnote{Sn 812, \emph{Jarāsutta}}

Like a drop of water on a lotus leaf,\\
Or water that taints not the lotus petal,\\
So the sage unattached remains,\\
In regard to what is seen, heard and sensed.
\end{quote}

So the extremely deep problem concerning the relation between the supramundane and the mundane levels of experience, is resolved by the Buddha by bringing in the simile of the lotus petal and the lotus leaf.

Let us take up another instance from the \emph{Māgandiyasutta} of the \emph{Sutta Nipāta}.

\begin{quote}
\emph{Yehi vivitto vicareyya loke,}\\
\emph{na tāni uggayha vadeyya nāgo,}\\
\emph{elambujaṁ kaṇṭakaṁ vārijaṁ yathā,}\\
\emph{jalena paṁkena anūpalittaṁ,}\\
\emph{evaṁ munī santivādo agiddho,}\\
\emph{kāme ca loke ca anūpalitto.}\footnote{Sn 845, \emph{Māgandiyasutta}}

Detached from whatever views,\\
\vin the \emph{arahant} wanders in the world,\\
He would not converse, taking his stand on them,\\
Even as the white lotus, sprung up in the water,\\
Yet remains unsmeared by water and mud,\\
So is the sage,\\
\vin professing peace and free from greed,\\
Unsmeared by pleasures of sense\\
\vin and things of the world.''
\end{quote}

Among the Tens of the \emph{Aṅguttara Nikāya} we come across a discourse in which the Buddha answers a question put by Venerable Bāhuna. At that time the Buddha was staying near the pond Gaggara in the city of Campa. Venerable Bāhuna's question was:

\enlargethispage{\baselineskip}

\begin{quote}
\emph{Katīhi nu kho, bhante, dhammehi tathāgato nissaṭo visaṁyutto vippamutto vimariyādikatena cetasā viharati?}\footnote{A V 151, \emph{Bāhunasutta}}

Detached, disengaged and released from how many things does the Tathāgata dwell with an unrestricted mind?
\end{quote}

The Buddha's answer to the question embodies a simile, aptly taken from the pond, as it were.

\begin{quote}
\emph{Dasahi kho, Bāhuna, dhammehi tathāgato nissaṭo visaṁyutto vippamutto vimariyādikatena cetasā viharati. Katamehi dasahi? Rūpena kho, Bāhuna, Tathāgato nissaṭo visaṁyutto vippamutto vimariyādikatena cetasā viharati, vedanāya \ldots{} saññāya \ldots{} saṅkhārehi \ldots{} viññāṇena \ldots{} jātiyā \ldots{} jarāya \ldots{} maraṇena \ldots{} dukkhehi \ldots{} kilesehi kho, Bāhuna, Tathāgato nissaṭo visaṁyutto vippamutto vimariyādikatena cetasā viharati.}

\emph{Seyyathāpi, Bāhuna, uppalaṁ vā padumaṁ vā puṇḍarīkaṁ vā udake jātaṁ udake saṁvaḍḍhaṁ udakā accugamma tiṭṭhati anupalittaṁ udakena, evam eva kho Bāhuna Tathāgato imehi dasahi dhammehi nissaṭo visaṁyutto vippamutto vimariyādikatena cetasā viharati}.

Detached, disengaged and released from ten things, Bāhuna, does the Tathāgata dwell with a mind unrestricted. Which ten? Detached, disengaged and released from form, Bāhuna, does the Tathāgata dwell with a mind unrestricted; detached, disengaged and released from feeling \ldots{} from perceptions \ldots{} from preparations \ldots{} from consciousness \ldots{} from birth \ldots{} from decay \ldots{} from death \ldots{} from pains \ldots{} from defilements, Bāhuna, does the Tathāgata dwell with a mind unrestricted.

Just as, Bāhuna, a blue lotus, a red lotus, or a white lotus, born in the water, grown up in the water, rises well above the water and remains unsmeared by water, even so, Bāhuna, does the Tathāgata dwell detached, disengaged and released from these ten things with a mind unrestricted.
\end{quote}

This discourse, in particular, highlights the transcendence of the Tathāgata, though he seems to take in worldly objects through the senses. Even the release from the five aggregates is affirmed.

We might wonder why the Tathāgata is said to be free from birth, decay and death, since, as we know, he did grow old and pass away. Birth, decay and death, in this context, do not refer to some future state either. Here and now the Tathāgata is free from the concepts of birth, decay and death.

In the course of our discussion of the term \emph{papañca}, we had occasion to illustrate how one can be free from such concepts.\footnote{See \emph{Sermon 12}} If concepts of birth, decay and death drive fear into the minds of worldlings, such is not the case with the Tathāgata. He is free from such fears and forebodings. He is free from defilements as well.

The discourse seems to affirm that the Tathāgata dwells detached from all these ten things. It seems, therefore, that the functioning of the Tathāgata's sense-faculties in his every day life also should follow a certain extraordinary pattern of detachment and disengagement. In fact, Venerable Sāriputta says something to that effect in the \emph{Saḷāyatanasaṁyutta} of the \emph{Saṁyutta Nikāya}.

\begin{quote}
\emph{Passati Bhagavā cakkhunā rūpaṁ, chandarāgo Bhagavato natthi, suvimuttacitto Bhagavā.}\footnote{S IV 164, \emph{Koṭṭhikasutta}}

The Exalted One sees forms with the eye, but there is no desire or attachment in him, well freed in mind is the Exalted One.
\end{quote}

We come across a similar statement made by the brahmin youth Uttara in the \emph{Brahmāyusutta} of the \emph{Majjhima Nikāya}, after he had closely followed the Buddha for a considerable period to verify the good report of his extraordinary qualities.

\begin{quote}
\emph{Rasapaṭisaṁvedī kho pana so bhavaṁ Gotamo āhāraṁ āhāreti, no rasarāgapaṭisaṁvedī}.\footnote{M II 138, \emph{Brahmāyusutta}}

Experiencing taste Master Gotama takes his food, but not experiencing any attachment to the taste.
\end{quote}

It is indeed something marvellous. The implication is that there is such a degree of detachment with regard to things experienced by the tongue, even when the senses are taking in their objects. One can understand the difference between the mundane and the supramundane, when one reflects on the difference between experiencing taste and experiencing an attachment to taste.

Not only with regard to the objects of the five senses, but even with regard to mind-objects, the emancipated one has a certain degree of detachment. The \emph{arahant} has realized that they are not `such'. He takes in concepts, and even speaks in terms of `I' and `mine', but knows that they are false concepts, as in the case of a child's language.

There is a discourse among the Nines of the \emph{Aṅguttara Nikāya} which seems to assert this fact. It is a discourse preached by Venerable Sāriputta to refute a wrong viewpoint taken by a monk named Chandikāputta.

\begin{quote}
\emph{Evaṁ sammā vimuttacittassa kho, āvuso, bhikkhuno bhusā cepi cakkhuviññeyyā rūpā cakkhussa āpāthaṁ āgacchanti, nevassa cittaṁ pariyādiyanti, amissīkatamevassa cittaṁ hoti ṭhitaṁ āneñjappattaṁ, vayaṁ cassānupassati}. \emph{Bhusā cepi sotaviññeyyā saddā \ldots{} bhūsa cepi ghānaviññeyyā gandhā \ldots{} bhūsa cepi jivhāviññeyyā rasā \ldots{} bhūsa cepi kāyaviññeyyā phoṭṭhabbā \ldots{} bhūsa cepi manoviññeyyā dhammā manassa āpāthaṁ āgacchanti, nevassa cittaṁ pariyādiyanti, amissīkatamevassa cittaṁ hoti ṭhitaṁ āneñjappattaṁ, vayaṁ cassānupassati}.\footnote{A IV 404, \emph{Silāyūpasutta}}

Friend, in the case of a monk who is fully released, even if many forms cognizable by the eye come within the range of vision, they do not overwhelm his mind, his mind remains unalloyed, steady and unmoved, he sees its passing away. Even if many sounds cognizable by the ear come within the range of hearing \ldots{} even if many smells cognizable by the nose \ldots{} even if many tastes cognizable by the tongue \ldots{} even if many tangibles cognizable by the body \ldots{} even if many mind-objects cognizable by the mind come within the range of the mind, they do not overwhelm his mind, his mind remains unalloyed, steady and unmoved, he sees its passing away.
\end{quote}

So here we have the ideal of the emancipated mind. Generally, a person unfamiliar with the nature of a lotus leaf or a lotus petal, on seeing a drop of water on a lotus leaf or a lotus petal would think that the water drop smears them.

Earlier we happened to mention that there is a wide gap between the mundane and the supramundane. Some might think that this refers to a gap in time or in space. In fact it is such a conception that often led to various misinterpretations concerning Nibbāna. The supramundane seems so far away from the mundane, so it must be something attainable after death in point of time. Or else it should be far far away in outer space. Such is the impression made in general.

But if we go by the simile of the drop of water on the lotus leaf, the distance between the mundane and the supramundane is the same as that between the lotus leaf and the drop of water on it.

We are still on the problem of the hither shore and the farther shore. The distinction between the mundane and the supramundane brings us to the question of this shore and the other shore.

The \emph{arahant's} conception of this shore and the other shore differs from that of the worldling in general. If, for instance, a native of this island goes abroad and settles down there, he might even think of a return to his country as a `going abroad'. Similarly, as far as the emancipated sage is concerned, if he, having gone to the farther shore, does not come back, one might expect him to think of this world as the farther shore.

But it seems the \emph{arahant} has no such distinction. A certain \emph{Dhammapada} verse alludes to the fact that he has transcended this dichotomy:

\begin{quote}
\emph{Yassa pāraṁ apāraṁ vā,}\\
\emph{pārāpāraṁ na vijjati,}\\
\emph{vītaddaraṁ visaṁyuttaṁ,}\\
\emph{tam ahaṁ brūmi brāhmaṇaṁ.}\footnote{Dhp 385, \emph{Brāhmaṇavagga}}
\end{quote}

This is a verse we have quoted earlier too, in connection with the question of the verbal dichotomy.\footnote{See \emph{Sermon 5}}

\begin{quote}
\emph{Yassa pāraṁ apāraṁ vā, pārāpāraṁ na vijjati,}

to whom there is neither a farther shore, nor a hither shore, nor~both.
\end{quote}

That is to say, he has no discrimination between the two.

\begin{quote}
\emph{Vītaddaraṁ visaṁyuttaṁ, tam ahaṁ brūmi brāhmaṇaṁ},

who is free from pangs of sorrow and entanglements, him I call a Brahmin.
\end{quote}

This means that the \emph{arahant} is free from the verbal dichotomy, which is of relevance to the worldling. Once gone beyond, the emancipated one has no more use of these concepts. This is where the Buddha's dictum in the raft simile of the \emph{Alagaddūpamasutta} becomes meaningful.

Even the concepts of a `this shore' and a `farther shore' are useful only for the purpose of crossing over. If, for instance, the \emph{arahant}, having gone beyond, were to think `ah, this is my land', that would be some sort of a grasping. Then there will be an identification, \emph{tammayatā}, not a non-identification, \emph{atammayatā}.

As we had mentioned earlier, there is a strange quality called \emph{atammayatā}, associated with an \emph{arahant}.\footnote{See \emph{Sermon 14}} In connection with the simile of a man who picked up a gem, we have already stated the ordinary norm that prevails in the world.\footnote{See \emph{Sermon 9}}

If we possess something -- we are possessed by it.

If we grasp something -- we are caught by it.

This is the moral behind the parable of the gem. It is this conviction, which prompts the \emph{arahant} not to grasp even the farther shore, though he may stand there. `This shore' and the `other shore' are concepts, which have a practical value to those who are still on this side.

As it is stated in the \emph{Alagaddūpamasutta}, since there is no boat or bridge to cross over, one has to improvise a raft by putting together grass, twigs, branches and leaves, found on this shore. But after crossing over with its help, he does not carry it with him on his shoulder.

\begin{quote}
\emph{Evameva kho, bhikkhave, kullūpamo mayā dhammo desito nittharaṇatthāya no gahaṇatthāya. Kullūpamaṁ vo bhikkhave ājānantehi dhammā pi vo pahātabbā, pag'eva adhammā}.\footnote{M I 135, \emph{Alagaddūpamasutta}}

Even so, monks, have I preached to you a Dhamma that is comparable to a raft, which is for crossing over and not for grasping. Well knowing the Dhamma to be comparable to a raft, you should abandon even the good things, more so the bad~things.
\end{quote}

One might think that the \emph{arahant} is in the sensuous realm, when, for instance, he partakes of food. But that is not so. Though he attains to the realms of form and formless realms, he does not belong there. He has the ability to attain to those levels of concentration, but he does not grasp them egoistically, true to that norm of \emph{atammayatā}, or non-identification.

This indeed is something extraordinary. Views and opinions about language, dogmatically entertained by the worldlings, lose their attraction for him.

This fact is clearly illustrated for us by the \emph{Uragasutta} of the \emph{Sutta Nipāta}, the significance of which we have already stressed.\footnote{See \emph{Sermon 5}} We happened to mention that there is a refrain, running through all the seventeen verses making up that discourse. The refrain concerns the worn out skin of a snake. The last two lines in each verse, forming the refrain, are:

\begin{quote}
\emph{So bhikkhu jahāti orapāraṁ,}\\
\emph{urago jiṇṇamiva tacaṁ purāṇaṁ.}\footnote{Sn 1-17 , \emph{Uragasutta}}

That monk forsakes the hither and the thither,\\
Even as the snake its skin that doth wither.
\end{quote}

The term \emph{orapāraṁ} is highly significant in this context. \emph{Oraṁ} means `this shore' and \emph{paraṁ} is the `farther shore'. The monk, it seems, gives up not only this shore, but the other shore as well, even as the snake sloughs off its worn out skin. That skin has served its purpose, but now it is redundant. So it is sloughed off.

Let us now take up one more verse from the \emph{Uragasutta} which has the same refrain, because of its relevance to the understanding of the term \emph{papañca}.

The transcendence of relativity involves freedom from the duality in worldly concepts such as `good' and `evil'. The concept of a `farther shore' stands relative to the concept of a `hither shore'. The point of these discourses is to indicate that there is a freedom from worldly conceptual proliferations based on duality and relativity. The verse we propose to bring up is:

\begin{quote}
\emph{Yo nāccasārī na paccasārī,}\\
\emph{sabbaṁ accagamā imaṁ papañcaṁ,}\\
\emph{so bhikkhu jahāti orapāraṁ,}\\
\emph{urago jiṇṇamiva tacaṁ purāṇaṁ.}\footnote{Sn 8, \emph{Uragasutta}}

Who neither overreaches himself nor lags behind,\\
And has gone beyond all this proliferation,\\
That monk forsakes the hither and the thither,\\
Even as the snake its slough that doth wither.
\end{quote}

This verse is particularly significant in that it brings out some points of interest. The overreaching and lagging behind is an allusion to the verbal dichotomy. In the context of views, for instance, annihilationism is an overreaching and eternalism is a lagging behind.

We may give another illustration, easier to understand. Speculation about the future is an overreaching and repentance over the past is a lagging behind. To transcend both these tendencies is to get beyond proliferation, \emph{sabbaṁ accagamā imaṁ papañcaṁ}.

When a banknote is invalidated, cravings, conceits and views bound with it go down. Concepts current in the world, like banknotes in transaction, are reckoned as valid so long as cravings, conceits and views bound with them are there. They are no longer valid when these are gone.

We have defined \emph{papañca} with reference to cravings, conceits and views.\footnote{See \emph{Sermon 12}} Commentaries also speak of \emph{taṇhāpapañca, diṭṭhipapañca} and \emph{mānapapañca}.\footnote{E.g. Ps I 183, commenting on M I 40, \emph{Sallekhasutta}: \emph{n'etaṁ mama, n'eso ham asmi, na meso attā ti}} By doing away with cravings, conceits and views, one goes beyond all \emph{papañca}.

The term \emph{orapāraṁ}, too, has many connotations. It stands for the duality implicit in such usages as the `internal' and the `external', `one's own' and `another's', as well as `this shore' and the `farther shore'. It is compared here to the worn out skin of a snake. It is worn out by transcending the duality characteristic of linguistic usage through wisdom.

Why the Buddha first hesitated to teach this Dhamma was the difficulty of making the world understand.\footnote{M I 168, \emph{Ariyapariyesanasutta}} Perhaps it was the conviction that the world could easily be misled by those limitations in the linguistic medium.

We make these few observations in order to draw attention to the relativity underlying such terms as `this shore' and the `other shore' and to show how Nibbāna transcends even that dichotomy.

In this connection, we may take up for comment a highly controversial sutta in the \emph{Itivuttaka}, which deals with the two aspects of Nibbāna known as \emph{sa-upādisesā Nibbānadhātu} and \emph{anupādisesā Nibbānadhātu}. We propose to quote the entire sutta, so as to give a fuller treatment to the subject.

\begin{quote}
\emph{Vuttaṁ hetaṁ Bhagavatā, vuttam arahatā ti me suttaṁ:}

\emph{Dve-mā, bhikkhave, nibbānadhātuyo. Katame dve? Sa-upadisesā ca nibbānadhātu, anupādisesā ca nibbānadhātu.}

\emph{Katamā, bhikkhave, sa-upadisesā nibbānadhātu? Idha, bhikkhave, bhikkhu arahaṁ hoti khīṇāsavo vusitavā katakaraṇīyo ohitabhāro anuppattasadattho parikkhīṇabhavasaṁyojano sammadaññāvimutto. Tassa tiṭṭhanteva pañcindriyāni yesaṁ avighātattā manāpāmanāpaṁ paccanubhoti, sukhadukkhaṁ paṭisaṁvediyati. Tassa yo rāgakkhayo dosakkhayo mohakkhayo, ayaṁ vuccati, bhikkhave, sa-upadisesā nibbānadhātu.}

\emph{Katamā ca, bhikkhave,anupādisesā nibbānadhātu? Idha, bhikkhave, bhikkhu arahaṁ hoti khīṇāsavo vusitavā katakaraṇīyo ohitabhāro anuppattasadattho parikkhīṇabhavasaṁyojano sammadaññāvimutto. Tassa idheva sabbavedayitāni anabhinanditāni sītibhavissanti, ayaṁ vuccati, bhikkhave, anupādisesā nibbānadhātu.}

\emph{Etam atthaṁ Bhagavā avoca, tatthetaṁ iti vuccati:}

\emph{Duve imā cakkhumatā pakāsitā,}\\
\emph{nibbānadhātū anissitena tādinā,}\\
\emph{ekā hi dhātu idha diṭṭhadhammikā,}\\
\emph{sa-upadisesā bhavanettisaṅkhayā,}\\
\emph{anupādisesā pana samparāyikā,}\\
\emph{yamhi nirujjhanti bhavāni sabbaso.}

\emph{Ye etad-aññāya padaṁ asaṅkhataṁ,}\\
\emph{vimuttacittā bhavanettisaṅkhayā,}\\
\emph{te dhammasārādhigamā khaye ratā,}\\
\emph{pahaṁsu te sabbabhavāni tādino.}

\emph{Ayampi attho vutto Bhagavatā, iti me sutaṁ.}\footnote{It 38, \emph{Nibbānadhātusutta}}

This was said by the Exalted One, said by the Worthy One, so have I heard:

'Monks, there are these two Nibbāna elements. Which two? The Nibbāna element with residual clinging and the Nibbāna element without residual clinging.

And what, monks, is the Nibbāna element with residual clinging? Herein, monks, a monk is an \emph{arahant}, with influxes extinct, one who has lived the holy life to the full, done what is to be done, laid down the burden, reached one's goal, fully destroyed the fetters of existence and released with full understanding. His five sense faculties still remain and due to the fact that they are not destroyed, he experiences likes and dislikes, and pleasures and pains. That extirpation of lust, hate and delusion in him, that, monks, is known as the Nibbāna element with residual clinging.

And what, monks, is the Nibbāna element without residual clinging? Herein, monks, a monk is an \emph{arahant}, with influxes extinct, one who has lived the holy life to the full, done what is to be done, laid down the burden, reached one's goal, fully destroyed the fetters of existence and released with full understanding. In him, here itself, all what is felt will cool off, not being delighted in. This, monks, is the Nibbāna element without residual clinging.'

To this effect the Exalted One spoke and this is the gist handed down as `thus said'.

'These two Nibbāna elements have been made known,\\
By the one with vision, unattached and such,\\
Of relevance to the here and now is one element,\\
With residual clinging, yet with tentacles to becoming snapped,\\
But then that element\\
\vin without residual clinging is of relevance to the hereafter,\\
For in it surcease all forms of becoming.

They that comprehend fully this state of the unprepared,\\
Released in mind with tentacles to becoming snapped,\\
On winning to the essence of Dhamma\\
\vin they take delight in seeing to an end of it all,\\
So give up they, all forms of becoming,\\
\vin steadfastly such-like as they are.'
\end{quote}

The standard phrase summing up the qualification of an \emph{arahant} occurs in full in the definition of the \emph{sa-upādisesā Nibbānadhātu}. The distinctive feature of this Nibbāna element is brought out in the statement that the \emph{arahant's} five sense faculties are still intact, owing to which he experiences likes and dislikes, and pleasure and pain. However, to the extent that lust, hate and delusion are extinct in him, it is called the Nibbāna element with residual clinging.

In the definition of the Nibbāna element without residual clinging, the same standard phrase recurs, while its distinctive feature is summed up in just one sentence:

\clearpage

\begin{quote}
\emph{Tassa idheva sabbavedayitāni anabhinanditāni sītibhavissanti},

in him, here itself, all what is felt will cool off, not being delighted in.
\end{quote}

It may be noted that the verb is in the future tense and apart from this cooling off, there is no guarantee of a world beyond, as an \emph{asaṅkhata dhātu}, or `unprepared element', with no sun, moon or stars in it.

The two verses that follow purport to give a summary of the prose passage. Here it is clearly stated that out of the two Nibbāna elements, as they are called, the former pertains to the here and now, \emph{diṭṭhadhammika}, while the latter refers to what comes after death, \emph{samparāyika}.

The Nibbāna element with residual clinging, \emph{sa-upādisesā Nibbānadhātu}, has as its redeeming feature the assurance that the tentacular craving for becoming is cut off, despite its exposure to likes and dislikes, pleasures and pains, common to the field of the five senses.

As for the Nibbāna element without residual clinging, it is definitely stated that in it all forms of existence come to cease. The reason for it is none other than the crucial fact, stated in that single sentence, namely, the cooling off of all what is felt as an inevitable consequence of not being delighted in, \emph{anabhinanditāni}.

Why do they not take delight in what is felt at the moment of passing away? They take delight in something else, and that is: the very destruction of all what is felt, a foretaste of which they have already experienced in their attainment to that unshakeable deliverance of the mind, which is the very pith and essence of the Dhamma, \emph{dhammasāra}.

As stated in the \emph{Mahāsāropamasutta} of the \emph{Majjhima Nikāya}, the pith of the Dhamma is that deliverance of the mind,\footnote{M I 197, \emph{Mahāsāropamasutta}} and to take delight in the ending of all feelings, \emph{khaye ratā}, is to revert to the \emph{arahattaphalasamādhi} with which the \emph{arahant} is already familiar. That is how those such-like ones abandon all forms of existence, \emph{pahaṁsu te sabbabhavāni tādino}.

Let us now try to sort out the problems that are likely to be raised in connection with the interpretation we have given. First and foremost, the two terms \emph{diṭṭhadhammika} and \emph{samparāyika} have to be explained.

A lot of confusion has arisen, due to a misunderstanding of the meaning of these two terms in this particular context. The usual commentarial exegesis on the term \emph{diṭṭhadhammika} amounts to this: \emph{Imasmiṁ attabhāve bhavā vattamānā},\footnote{It-a I 167} ``in this very life, that is, in the present''. It seems all right. But then for \emph{samparāyika} the commentary has the following comment: \emph{samparāye khandhabhedato parabhāge}, ``\emph{samparāya} means after the breaking up of the aggregates''. The implication is that it refers to the \emph{arahant's} after death state.

Are we then to conclude that the \emph{arahant} gets half of his Nibbāna here and the other half hereafter? The terms \emph{diṭṭhadhammika} and \emph{samparāyika}, understood in their ordinary sense, would point to such a conclusion.

But let us not forget that the most distinctive quality of this Dhamma is associated with the highly significant phrase, \emph{diṭṭhevadhamme}, ``in this very life''. It is also conveyed by the expression \emph{sandiṭṭhika akālika}, ``here and now'' and ``timeless''.\footnote{In the standard formula for recollecting the Dhamma, e.g.~at D II 93}

The goal of endeavour, indicated by this Dhamma, is one that could be fully realized here and now, in this very life. It is not a piecemeal affair. Granting all that, do we find here something contrary to it, conveyed by the two terms \emph{diṭṭhadhammika} and \emph{samparāyika}? How can we reconcile these two passages?

In the context of \emph{kamma}, the meaning of the two terms in question can easily be understood.

For instance, that category of \emph{kamma} known as \emph{diṭṭhadhammavedanīya} refers to those actions which produce their results here and now.

\emph{Samparāyika} pertains to what comes after death, as for instance in the phrase \emph{samparāye ca duggati}, an ``evil bourn after death''.\footnote{S I 34, \emph{Maccharisutta}}

In the context of \emph{kamma} it is clear enough, then, that the two terms refer to what is experienced in this world and what comes after death, respectively.

Are we justified in applying the same criterion, when it comes to the so-called two elements of Nibbāna? Do the \emph{arahants} experience some part of Nibbāna here and the rest hereafter?

At this point, we have to admit that the term \emph{diṭṭhadhammika} is associated with \emph{sa-upādisesā Nibbānadhātu} while the term \emph{samparāyika} is taken over to refer to \emph{anupādisesā Nibbānadhātu}.

However, the fact that Nibbāna is explicitly defined elsewhere as the cessation of existence, \emph{bhavanirodho Nibbānaṁ},\footnote{A V 9, \emph{Sāriputtasutta}} must not be forgotten. If Nibbāna is the cessation of existence, there is nothing left for the \emph{arahant} to experience hereafter.

Nibbāna is solely the realization of the cessation of existence or the end of the process of becoming. So there is absolutely no question of a hereafter for the \emph{arahant}.

By way of clarification, we have to revert to the primary sense of the term Nibbāna. We have made it sufficiently clear that Nibbāna means `extinction' or `extinguishment', as of a fire.

All the commentarial jargon, equating \emph{vāna} to \emph{taṇhā}, is utterly irrelevant. If the idea of an extinguishment of a fire is brought in, the whole problem is solved. Think of a blazing fire. If no more firewood is added to it, the flames would subside and the embers would go on smouldering before turning into ashes. This is the norm. Now this is not an analogy we are superimposing on the Dhamma. It is only an echo of a canonical simile, picked up from the \emph{Nāgasutta} of the \emph{Aṅguttara Nikāya}. The relevant verse, we are quoting, recurs in the \emph{Udāyi Theragāthā} as well.

\begin{quote}
\emph{Mahāgini pajjalito,}\\
\emph{anāhārūpasammati,}\\
\emph{aṅgāresu ca santesu,,}\\
\emph{nibbuto ti pavuccati.}\footnote{A III 347, \emph{Nāgasutta} and Th 702, \emph{Udāyitheragāthā}}

As a huge blazing fire, with no more firewood added,\\
Goes down to reach a state of calm,\\
Embers smouldering, as they are, could be reckoned,\\
So long as they last, as almost `extinguished'.
\end{quote}

Though we opted to render the verse this way, there is a variant reading, which could lead to a different interpretation. As so often happens in the case of deep suttas, here too the correct reading is not easily determined.

Instead of the phrase \emph{aṅgāresu ca santesu}, attested as it is, many editions go for the variant reading \emph{saṅkhāresūpasantesu}. If that reading is adopted, the verse would have to be rendered as follows:

\begin{quote}
As a huge blazing fire, with no more fire wood added,\\
Goes down to reach a state of calm,\\
When \emph{saṅkhāras} calm down,\\
One is called `extinguished'.
\end{quote}

It maybe pointed out that this variant reading does not accord with the imagery of the fire presented by the first two lines of the verse. It is probably a scribe's error that has come down, due to the rhythmic similarity between the two phrases \emph{aṅgāresu ca santesu}, and \emph{saṅkhāresūpasantesu}.\footnote{The corresponding verse in the Chinese parallel, \emph{Madhyama Āgama} discourse 118 (Taishº I 608c27), does not mention \emph{saṅkhāra} at all. (Anālayo)} Between the reciter and the scribe, phrases that have a similar ring and rhythm, could sometimes bring about a textual corruption. Be that as it may, we have opted for the reading \emph{aṅgāresu ca santesu}, because it makes more sense.

From the particular context in which the verse occurs, it seems that this imagery of the fire is a restatement of the image of the lotus unsmeared by water. Though the embers are still smouldering, to the extent that they are no longer hungering for more fuel and are not emitting flames, they may as well be reckoned as `extinguished'.

We can draw a parallel between this statement and the definition of \emph{sa-upādisesā Nibbānadhātu} already quoted. As a full-fledged \emph{arahant}, he still experiences likes and dislikes and pleasures and pains, owing to the fact that his five sense-faculties are intact.

The assertion made by the phrase beginning with

\begin{quote}
\emph{tassa tiṭṭhanteva pañcindriyāni yesaṁ avighātattā \ldots{}},

his five senses do exist, owing to the non-destruction of which \ldots,
\end{quote}

rather apologetically brings out the limitations of the living \emph{arahant}. It is reminiscent of those smouldering embers in the imagery of the \emph{Nāgasutta}. However, in so far as flames of lust, hate and delusion are quenched in him, it comes to be called \emph{sa-upādisesā Nibbānadhātu}, even as in the case of those smouldering embers.

Craving is aptly called \emph{bhavanetti},\footnote{A II 1, \emph{Anubuddhasutta}} in the sense that it leads to becoming by catching hold of more and more fuel in the form of \emph{upādāna}. When it is under control, the functioning of the sense-faculties do not entail further rebirth. The inevitable residual clinging in the living \emph{arahant} does not precipitate a fresh existence.

This gives us a clue to the understanding of the term \emph{anupādisesa}. The element \emph{upādi} in this term is rather ambiguous.

In the \emph{Satipaṭṭhānasutta}, for instance, it is used as the criterion to distinguish the \emph{anāgāmi}, the `non-returner', from the \emph{arahant}, in the statement

\begin{quote}
\emph{diṭṭhevadhamme aññā, sati vā upādisese anāgāmitā},\footnote{M I 62, \emph{Satipaṭṭhānasutta}}

either full convincing knowledge of \emph{arahanthood} here and now, or the state of non-return in the case of residual clinging.
\end{quote}

But when it comes to the distinction between \emph{sa-upādisesa} and \emph{anupādisesa}, the element \emph{upādi} has to be understood in a more radical sense, in association with the word \emph{upādiṇṇa}. This body, as the product of past \emph{kamma}, is the `grasped' par excellence, which as an organic combination goes on functioning even in the \emph{arahant} until his last moment of life.

Venerable Sāriputta once declared that he neither delighted in death nor delighted in life, \emph{nābhinandāmi maraṇaṁ nābhinandāmi jīvitaṁ}.\footnote{Th 1001, \emph{Sāriputtatheragāthā}} So the embers go on smouldering until they become ashes. It is when the life span ends that the embers finally turn to ashes.

The popular interpretation of the term \emph{anupādisesā Nibbānadhātu} leaves room for some absolutist conceptions of an \emph{asaṅkhata dhātu}, unprepared element, as the destiny of the \emph{arahant}. After his \emph{parinibbāna}, he is supposed to enter this particular \emph{Nibbānadhātu}. But here, in this discourse, it is explained in just one sentence:

\begin{quote}
\emph{Tassa idheva, bhikkhave, sabbavedayitāni anabhinanditāni sītibhavissanti,}

in the case of him (that is the \emph{arahant}), O! monks, all what is felt, not having been delighted in, will cool off here itself.
\end{quote}

This cooling off happens just before death, without igniting another spark of life. When Māra comes to grab and seize, the \emph{arahant} lets go. The pain of death with which Māra teases his hapless victim and lures him into another existence, becomes ineffective in the case of the \emph{arahant}.

As he has already gone through the supramundane experience of deathlessness, in the \emph{arahattaphalasamādhi}, death loses its sting when at last it comes. The influx-free deliverance of the mind and the influx-free deliverance through wisdom enable him to cool down all feelings in a way that baffles Māra.

So the \emph{arahant} lets go of his body, experiencing ambrosial deathlessness. As in the case of Venerable Dabba Mallaputta, he would sometimes cremate his own body without leaving any ashes.\footnote{Ud 92, \emph{Paṭhamadabbasutta}} Outwardly it might appear as an act of self-immolation, which indeed is painful. But this is not so. Using his \emph{jhānic} powers, he simply employs the internal fire element to cremate the body he has already discarded.

This, then, is the Buddha's extraordinary solution to the problem of overcoming death, a solution that completely outwits Māra.
