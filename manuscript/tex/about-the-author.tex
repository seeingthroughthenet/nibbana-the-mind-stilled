\chapter{About the Author}

Venerable Kaṭukurunde Ñāṇananda was born in 1940 to a family of Buddhist parents in Galle, Sri Lanka. He received his school education at Mahinda College, Galle, where he imbibed the true Buddhist values. In 1962 he graduated from the University of Peradeniya and served as an Assistant Lecturer in Pali at the same University for a brief period. He renounced his post in 1967 to enter the Order of Buddhist monks at Island Hermitage, Dodanduwa.

Already during the first phase of his life as a monk at Island Hermitage, Ven. Ñāṇananda had written four books which were published by the Buddhist Publication Society in Kandy under the titles:

\begin{enumerate}
\item Concept and Reality in Early Buddhist Thought
\item Saṁyutta Nikāya -- An Anthology (Part II)
\item Ideal Solitude
\item The Magic of the Mind
\end{enumerate}

Then in 1972 he left Island Hermitage for Meetirigala Nissarana Vanaya, where he came under the tutelage of the late Ven. Mātara Sri Ñāṇārāma Mahāthera, a veteran teacher of Insight Meditation. The association of these two eminent disciples of the Buddha in a teacher-pupil relationship for about two decades, heralded a new era in the propagation of Dhamma through instructive books on Buddhist Meditation.

The signal contribution of this long association, however, was the set of 33 sermons on Nibbāna delivered by Ven. Ñāṇananda to his fellow resident monks at the invitation of the venerable Ñāṇārāma Mahāthera, during the period from August 1988 to January 1991. Inspired by these sermons, a group of lay enthusiasts initiated a Dhamma Publication Trust (D.G.M.B.) at the Public Trustee's Department to bring out the sermons in book form. The noble Dhammadāna aspiration of Ven. Ñāṇananda to give all books free to the readers provided an opportunity to the Buddhist public to contribute towards the publication of his books. This remarkable step had a spiritual dimension in reaffirming the age-old Buddhist values attached to Dhamamadāna, fast eroding before the hungry waves of commercialization. It has proved its worth by creating a healthy cultural atmosphere in which the readers shared the Dhamma-gift with others, thus moulding the links of salutary friendship (\emph{Kalyāna mittatā}) indispensable for the continuity of the Buddha Sāsana.

We are already convinced of the immense potentialities of this magnanimous venture, having witnessed the extraordinary response of the Buddhist public in sending their contributions to the Trust to enable the publication of books. Though usually the names of donors are shown at the end of each publication, some donations -- even sizeable ones -- are conspicuous by their anonymity. This exemplary trait is symbolic of the implicit confidence of the donor in the Trust.

Kaṭukurunde Ñāṇananda Sadaham Senasun Bhāraya (K.N.S.S.B) is bearing the burden of publication of Ven. Ñāṇananda's sermons and writings, while making available this Dhammadāna to a wider global audience through the new electronic technology. Recorded sermons on CDs are also being issued free as Dhammadāna by this Trust, while making available this Dhamma gift free through the internet.

\href{https://seeingthroughthenet.net/}{www.seeingthroughthenet.net}

\href{https://www.facebook.com/seeingthrough}{www.facebook.com/seeingthrough}
