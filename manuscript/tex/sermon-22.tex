\chapter{Sermon 22}

\NibbanaOpeningQuote

With the permission of the Most Venerable Great Preceptor and the assembly of the venerable meditative monks. This is the twentysecond sermon in the series of sermons on Nibbāna.

We made an attempt, in our last sermon, to explain that the comparison of the emancipated one in this dispensation to the great ocean has a particularly deep significance. We reverted to the simile of the vortex by way of explanation. Release from the \emph{saṁsāric} vortex, or the breach of the vortex of \emph{saṁsāra}, is comparable to the cessation of a whirlpool. It is equivalent to the stoppage of the whirlpool of \emph{saṁsāra}.

Generally, what is known as a vortex or a whirlpool, is a certain pervert, unusual or abnormal activity, which sustains a pretence of an individual existence in the great ocean with a drilling and churning as its centre. It is an aberration, functioning according to a duality, maintaining a notion of two things. As long as it exists, there is the dichotomy between a `here' and a `there', oneself and another. A vortex reflects a conflict between an `internal' and an `external' -- a `tangle within' and a `tangle without'. The~cessation of the vortex is the freedom from that duality. It is a solitude born of full integration.

We happened to discuss the meaning of the term \emph{kevalī} in our last sermon. The cessation of a vortex is at once the resolution of the conflict between an internal and an external, of the tangle within and without. When a vortex ceases, all those conflicts subside and a state of peace prevails. What remains is the boundless great ocean, with no delimitations of a `here' and a `there'. As is the great ocean, so is the vortex now.

This suchness itself indicates the stoppage, the cessation or the subsidence of the vortex. There is no longer any possibility of pointing out a `here' and a `there' in the case of a vortex that has ceased. Its `thusness' or `suchness' amounts to an acceptance of the reality of the great ocean. That `thus-gone' vortex, or the vortex that has now become `such', is in every respect worthy of being called \emph{tathāgata}.

The term \emph{tādī} is also semantically related to this suchness. The \emph{tathāgata} is sometimes referred to as \emph{tādī} or \emph{tādiso}, `such-like'. The `such-like' quality of the \emph{tathāgata} is associated with his unshakeable deliverance of the mind. His mind remains unshaken before the eight worldly vicissitudes.

Why the Buddha refused to give an answer to the tetralemma concerning the after-death state of the \emph{tathāgata}, should be clear to a great extent by those sutta quotations we brought up in our last sermon. Since the quotation

\begin{quote}
\emph{diṭṭheva dhamme saccato thetato tathāgate anupalabbhiyamāne,}\footnote{S III 118 and S IV 384, \emph{Anurādhasutta}}

when a \emph{tathāgata} is not to be found in truth and fact here in this very life,
\end{quote}

leads to the inference that a \emph{tathāgata} is not to be found in reality even while he is alive, we were forced to conclude that the question `what happens to the \emph{tathāgata} after his death?' is utterly meaningless.

It is also obvious from the conclusive statement,

\begin{quote}
\emph{pubbe cāhaṁ etarahi ca dukkhañceva paññāpemi dukkhassa ca nirodhaṁ}

formerly as well as now I make known just suffering and the cessation of suffering
\end{quote}

that the Buddha, in answering this question, completely put aside such conventional terms like `being' and `person', and solved the problem on the basis of the four noble truths, which highlight the pure quintessence of the Dhamma as it is.

We have to go a little deeper into this question of conventional terms like `being' and `person', because the statement that the \emph{tathāgata} does not exist in truth and fact is likely to drive fear into the minds of the generality of people. In our last sermon, we gave a clue to an understanding of the sense in which this statement is made, when we quoted an extraordinary new etymology, the Buddha had advanced, for the term \emph{satta} in the \emph{Rādhasaṁyutta}.

\begin{quote}
\emph{Rūpe kho, Rādha, yo chando yo rāgo yā nandī yā taṇhā, tatra satto, tatra visatto, tasmā `satto'ti vuccati.}\footnote{S III 190, \emph{Sattasutta}}

Rādha, that desire, that lust, that delight, that craving in form with which one is attached and thoroughly attached, therefore is one called a `being'.
\end{quote}

Here the Buddha has punned on the word \emph{satta}, to give a new orientation to its meaning, that is, \emph{rūpe satto visatto}, ``attached and thoroughly attached to form''.

From prehistoric times, the word \emph{satta} was associated with the idea of some primordial essence called \emph{sat}, which carried with it notions of permanent existence in the world. As derivatives from the present participle \emph{sant} and \emph{sat}, we get the two words \emph{satya} and \emph{sattva} in Sanskrit. \emph{Satya} means `truth', or what is `true'. \emph{Sattva} means a `being' or the `state of being'. We might even take \emph{sattva} as the place from which there is a positive response or an affirmation of a state of being.

Due to the semantic affinity between \emph{satya}, `truth', and \emph{sattva}, `being', an absolute reality had been granted to the term \emph{sattva} from ancient times.

But according to the new etymology advanced by the Buddha, the term \emph{sattva} is given only a relative reality within limits, that is to say, it is `real' only in a limited and a relative sense. The above quotation from the \emph{Rādhasaṁyutta} makes it clear that a being exists only so long as there is that desire, lust, delight and craving in the five aggregates.

Alternatively, when there is no desire, or lust, or delight, or craving for any of the five aggregates, there is no `being'. That is why we say that it is real only in a limited and relative sense.

When a thing is dependent on another thing, it is relative and for that very reason it has a limited applicability and is not absolute. Here, in this case, the dependence is on desire or attachment. As long as there is desire or attachment, there is a `being', and when it is not there, there is no `being'. So from this we can well infer that the \emph{tathāgata} is not a `being' by virtue of the very definition he had given to the term \emph{satta}.

The other day, we briefly quoted a certain simile from the \emph{Rādhasutta} itself, but could not explain it sufficiently. The Buddha gives this simile just after advancing the above new definition.

\begin{quote}
Suppose, Rādha, some little boys and girls are playing with sandcastles. So long as their lust, desire, love, thirst, passion and craving for those things have not gone away, they remain fond of them, they play with them, treat them as their property and call them their own. But when, Rādha, those little boys and girls have outgrown that lust, desire, love, thirst, passion and craving for those sandcastles, they scatter them with their hands and feet, demolish them, dismantle them and render them unplayable.
\end{quote}

When we reflect upon the meaning of this simile from the point of view of Dhamma, it seems that for those little boys and girls, sandcastles were real things, as long as they had ignorance and craving with regard to them. When they grew wiser and outgrew craving, those sandcastles became unreal. That is why they destroyed them.

The untaught worldling is in a similar situation. So long as he is attached to these five aggregates and has not comprehended their impermanent, suffering-fraught and not-self nature, they are real for him. He is bound by his own grasping.

The reality of the law of \emph{kamma}, of merit and demerit, follows from that very grasping. The dictum \emph{upādānapaccayā bhavo}, ``dependent on grasping is existence'', becomes meaningful in this context. There is an existence because there is grasping. But at whatever point of time wisdom dawned and craving faded away, all those things tend to become unreal and there is not even a `being', as there is no real `state of being'.

This mode of exposition receives support from the \emph{Kaccāyanagottasutta} of the \emph{Saṁyutta Nikāya}. The way the Buddha has defined right view in that discourse is highly significant. We have already discussed this sutta on an earlier occasion.\footnote{See \emph{Sermon 4}} Suffice it to remind ourselves of the basic maxim.

\begin{quote}
\emph{`Dukkham eva uppajjamānaṁ uppajjati, dukkhaṁ nirujjhamānaṁ nirujjhatī'ti na kaṅkhati na vicikicchati aparappaccayā ñāṇam ev' assa ettha hoti. Ettāvatā kho, Kaccāyana, sammā diṭṭhi hoti.}\footnote{S II 17, \emph{Kaccāyanagottasutta}}

It is only suffering that arises and suffering that ceases. Understanding thus, one does not doubt, one does not waver, and there is in him only the knowledge that is not dependent on another. It is in so far, Kaccāyana, that one has right view.
\end{quote}

What is called \emph{aparappaccayā ñāṇa} is that knowledge of realization by oneself for which one is not dependent on another. The noble disciple wins to such a knowledge of realization in regard to this fact, namely, that it is only a question of suffering and its cessation.

The right view mentioned in this context is the supramundane right view, and not that right view which takes \emph{kamma} as one's own, \emph{kammassakatā sammā diṭṭhi}, implying notions of `I' and `mine'.

This supramundane right view brings out the norm of Dhamma as it is. Being unable to understand this norm of Dhamma, contemporary ascetics and brahmins, and even some monks themselves, accused the Buddha of being an annihilationist. They brought up groundless allegations. There was also the opposite reaction of seeking refuge in a form of eternalism, through fear of being branded as annihilationists.

Sometimes the Buddha answered those wrong accusations in unmistakeable terms. We come across such an instance in the \emph{Alagaddūpama Sutta.} First of all the Buddha qualifies the emancipated one in his dispensation with the terms \emph{ariyo pannaddhajo pannabhāro visaṁyutto}.\footnote{M I 139, \emph{Alagaddūpamasutta}}

Once the conceit `am', \emph{asmimāna}, is abandoned, this noble one is called \emph{pannaddhajo}, ``one who has put down the flag of conceit''. He has `laid down the burden', \emph{pannabhāro}, and is `disjoined', \emph{visaṁyutto}, from the fetters of existence. About this emancipated one, he now makes the following declaration:

\begin{quote}
\emph{Evaṁ vimuttacittaṁ kho, bhikkhave, bhikkhuṁ sa-indā devā sa-pajāpatikā sa-brahmakā anvesaṁ nādhigacchanti: idaṁ nissitaṁ tathāgatassa viññāṇan'ti. Taṁ kissa hetu? Diṭṭhe vāhaṁ, bhikkhave, dhamme tathāgato ananuvejjo'ti vadāmi}.

\emph{Evaṁvādiṁ kho maṁ, bhikkhave, evam akkhāyiṁ eke samaṇabrāhmaṇā asatā tucchā musā abhūtena abbhācikkhanti: venayiko samaṇo Gotamo, sato sattassa ucchedaṁ vināsaṁ vibhavaṁ paññāpeti}.

A monk, thus released in mind, O! monks, gods including Indra, Pajāpati and Brahmā, are unable to trace in their search to be able to say of him: `the consciousness of this thus-gone-one is dependent on this'. And why is that so? Monks, I say, even here and now the Tathāgata is not to be found.

When I say thus, when I teach thus, some recluses and brahmins wrongly and falsely accuse me with the following unfounded allegation: `recluse Gotama is an annihilationist, he lays down an annihilation, a destruction and non-existence of a truly existing being'.
\end{quote}

As in the \emph{Anurādhasutta}, here too the Buddha concludes with the highly significant statement of his stance, \emph{pubbe cāhaṁ etarahi ca dukkhañceva paññāpemi dukkhassa ca nirodhaṁ}, ``formerly as well as now I make known just suffering and the cessation of suffering''.

Though the statements in the suttas follow this trend, it seems that the commentator himself was scared to bring out the correct position in his commentary. The fact that he sets out with some trepidation is clear enough from the way he tackles the term \emph{tathāgata} in his commentary to the above discourse in the \emph{Majjhima Nikāya.} In commenting on the word \emph{tathāgatassa} in the relevant context, he makes the following observation:

\begin{quote}
\emph{Tathāgatassā'ti ettha satto pi tathāgato'ti adhippeto, uttamapuggalo khīṇāsavo pi.}\footnote{Ps II 117}

Tathāgata's, herein, a being also is meant by the term \emph{tathāgata}, as well as the highest person, the influx-free \emph{arahant}.
\end{quote}

Just as he gives two meanings to the word \emph{tathāgata}, Venerable Buddhaghosa attributes two meanings to the word \emph{ananuvejjo} as well.

\begin{quote}
\emph{Ananuvejjo'ti asaṁvijjamāno vā avindeyyo vā. Tathāgato'ti hi satte gahite asaṁvijjamāno'ti attho vaṭṭati}, \emph{khīṇāsave gahite avindeyyo'ti attho vaṭṭati.}
\end{quote}

\begin{quote}
\emph{Ananuvejjo} -- `non-existing' or `untraceable'. When by the word \emph{tathāgata} a being is meant, the sense `non existing' is fitting; and when the influx-free one is meant, the sense `untraceable' is fitting.
\end{quote}

According to this exegesis, the term \emph{tathāgata} in contexts where it means a `being' is to be understood as non-existing, \emph{asamvijjamāno}, which is equivalent in sense to the expression \emph{anupalabbhiyamāne}, discussed above.

On the other hand, the other sense attributed to it is \emph{avindeyyo}, which somehow grants the existence but suggests that it is `untraceable'. In other words, the Tathāgata exists, but he cannot be traced or found out.

The commentator opines that the term in question has to be understood in two different senses, according to contexts. In order to substantiate his view, the commentator attributes the following apocryphal explanation to the Buddha.

\begin{quote}
\emph{Bhikkhave, ahaṁ diṭṭheva dhamme dharamānakaṁ yeva khīṇāsavaṁ viññāṇavasena indādīhi avindiyaṁ vadāmi. Na hi sa-indā devā sabrahmakā sapajāpatikā anvesantāpi khīṇāsavassa vipassanācittaṁ vā maggacittaṁ vā phalacittaṁ vā, idaṁ nāma ārammaṇaṁ nissāya vattatī'ti jānituṁ sakkonti. Te appaṭisandhikassa parinibbutassa kiṁ jānissanti?}

Monks, I say that even here and now the influx-free one, while he is alive, is untraceable by Indra and others in regard to his consciousness. Gods, including Indra, Brahmā and Pajāpati are indeed unable in their search to find out either the insight consciousness, or the path consciousness, or the fruition consciousness, to be able to say: `it is dependent on this object'. How then could they find out the consciousness of one who has attained \emph{parinibbāna} with no possibility of conception?
\end{quote}

Presumably, the argument is that, since the consciousness of the \emph{arahant} is untraceable by the gods while he is alive, it is all the more difficult for them to find it out when he has attained \emph{parinibbāna}. That is to say, the \emph{arahant} somehow exists, even after his \emph{parinibbāna}, only that he cannot be traced.

It is obvious from this commentarial trend that the commentator finds himself on the horns of a dilemma, because of his inability to grasp an extremely deep dimension of linguistic usage. The Buddha's forceful and candid declaration was too much for him. Probably, he demurred out of excessive faith, but his stance is not in accordance with the Dhamma. It falls short of right view.

Let us now recapitulate the correct position in the light of the above sutta passage. The Buddha declares at the very outset that the emancipated monk undergoes a significant change by virtue of the fact that he has abandoned the conceit `am'. That Tathāgata, that emancipated monk, who has put down the flag of conceit, laid down the burden of the five aggregates, and won release from the fetters to existence, defies definition and eludes categorization. Why is that?

As we pointed out earlier, the word \emph{asmi} constitutes the very basis of the entire grammatical structure.\footnote{See \emph{Sermons 10 and 13}} \emph{Asmi}, or `am', is the basic peg, which stands for the first person. The second person and the third person come later.

So \emph{asmi} is basic to the grammatical structure. When this basic peg is uprooted, the emancipated monk reaches that state of freedom from the vortex. There is no dichotomy to sustain a vortex, no two teams to keep up the vortical interplay. Where there is no turning round, there is no room for designation, and this is the implication of the phrase \emph{vaṭṭaṁ tesaṁ natthi paññāpanāya}, which we happened to quote on a previous occasion.\footnote{M I 141, \emph{Alagaddūpamasutta}; see \emph{Sermon 2} and \emph{Sermon 21}} For the \emph{arahants} there is no vortex whereby to designate.

That is why the Tathāgata, in this very life, is said to have transcended the state of a `being'. Only as a way of speaking in terms of worldly parlance one cannot help referring to him as a `being'. But in truth and fact, his position is otherwise.

Going by worldly usage, one might indiscriminately think of applying the four propositions of the tetralemma to the Tathāgata as well. But it is precisely in this context that the questioner's presumptions are fully exposed.

The fact that he has misconceived the implications of the terms \emph{satta} and Tathāgata is best revealed by the very question whether the Tathāgata exists after his death. It shows that he presumes the Tathāgata to be existing in truth and fact, and if so, he has either to go on existing or be annihilated after death. Here, then, we have an extremely deep dimension of linguistic usage.

The commentary says that gods and Brahmās cannot find the Tathāgata in point of his consciousness. The Tathāgata defies definition due to his abandonment of proliferations of cravings, conceits and views. Cravings, conceits and views, which bring in attachments, bindings and \mbox{entanglements} to justify the usage of terms like \emph{satta}, `being', and \emph{puggala}, `person', are extinct in the Tathāgata. That is why he is beyond reckoning.

In the \emph{Brahmajālasutta} of the \emph{Dīgha Nikāya} the Buddha makes the following declaration about himself, after refuting the sixty-two views, catching them all in one super-net.

\begin{quote}
\emph{Ucchinnabhavanettiko, bhikkhave, tathāgatassa kāyo tiṭṭhati. Yav'assa kāyo ṭhassati tāva naṁ dakkhinti devamanussā. Kāyassa bhedā uddhaṁ jīvitapariyādānā na naṁ dakkhinti devamanussā.}\footnote{D I 46, \emph{Brahmajālasutta}}

Monks, the Tathāgata's body stands with its leading factor in becoming cut off at the root. As long as his body stands, gods and men will see him. With the breaking up of his body, after the extinction of his life, gods and men will not see him.
\end{quote}

And then he follows up this promulgation with a simile.

\begin{quote}
\emph{Seyyathā pi, bhikkhave, ambapiṇḍiyā vaṇṭacchinnāya yāni kānici ambāni vaṇṭūpanibandhanāni, sabbāni tāni tad anvayāni bhavanti, evam eva kho, bhikkhave, ucchinnabhavanettiko tathāgatassa kāyo tiṭṭhati. Yav'assa kāyo ṭhassati tāva naṁ dakkhinti devamanussā. Kāyassa bhedā uddhaṁ jīvitapariyādānā na naṁ dakkhinti devamanussā.}

Just as, monks, in the case of a bunch of mangoes, when its stalk is cut off, whatever mangoes that were connected with the stalk would all of them be likewise cut off, even so, monks, stands the Tathāgata's body with its leading factor in becoming cut off at the root. As long as his body stands, gods and men will see him. With the breaking up of his body, after the extinction of his life, gods and men will not see him.
\end{quote}

The simile employed serves to bring out the fact that the Tathāgata's body stands with its leading factor in becoming eradicated. Here it is said that gods and men see the Tathāgata while he is alive. But the implications of this statement should be understood within the context of the similes given.

The reference here is to a tree uprooted, one that simply stands cut off at the root. In regard to each aggregate of the Buddha and other emancipated ones, it is clearly stated that it is cut off at the root, \emph{ucchinnamūlo}, that it is like a palm tree divested of its site \emph{tālāvatthukato}.\footnote{M I 139, \emph{Alagaddūpamasutta}}

In the case of a palm tree, deprived of its natural site but still left standing, anyone seeing it from afar would mistake it for an actual tree that is growing. It is the same idea that emerges from the simile of the bunch of mangoes. The Tathāgata is comparable to a bunch of mangoes with its stalk cut off.

What then is meant by the statement that gods and men see him? Their seeing is limited to the seeing of his body. For many, the concept of seeing the Tathāgata is just this seeing of his physical body. Of course, we do not find in this discourse any prediction that we can see him after five-thousand years.

Whatever it may be, here we seem to have some deep idea underlying this discourse. An extremely important clue to a correct understanding of this Dhamma, one that helps to straighten up right view, lies beneath this problem of the Buddha's refusal to answer the tetralemma concerning the Tathāgata. This fact comes to light in the \emph{Yamakasutta} of the \emph{Khandhasaṁyutta}.

A monk named Yamaka conceived the evil view, the distorted view,

\begin{quote}
\emph{tathāhaṁ bhagavatā dhammaṁ desitaṁ ājānāmi, yathā khīṇāsavo bhikkhu kāyassa bhedā ucchijjati vinassati, na hoti paraṁ maraṇā.}\footnote{S III 109, \emph{Yamakasutta}}

As I understand the Dhamma taught by the Exalted One, an influx-free monk, with the breaking up of his body, is annihilated and perishes, he does not exist after death.
\end{quote}

He went about saying that the Buddha had declared that the emancipated monk is annihilated at death. Other monks, on hearing this, tried their best to dispel his wrong view, saying that the Buddha had never declared so, but it was in vain. At last they approached Venerable Sāriputta and begged him to handle the situation.

Then Venerable Sāriputta came there, and after ascertaining the fact, proceeded to dispel Venerable Yamaka's wrong view by getting him to answer a series of questions. The first set of questions happened to be identical with the one the Buddha had put forward in Venerable Anurādha's case, namely a catechism on the three characteristics. We have already quoted it step by step, for facility of understanding.\footnote{See \emph{Sermon 21}}

Suffice it to mention, in brief, that it served to convince Venerable Yamaka of the fact that whatever is impermanent, suffering and subject to change, is not fit to be looked upon as `this is mine, this am I, and this is my self'.

The first step, therefore, consisted in emphasizing the not self characteristic through a catechism on the three signata. The next step was to get Venerable Yamaka to reflect on this not self characteristic in eleven ways, according to the standard formula.

\begin{quote}
\emph{Tasmātiha, āvuso Yamaka, yaṁ kiñci rūpaṁ atītānāgatapaccuppannaṁ ajjhattaṁ vā bahiddhā vā oḷārikaṁ va sukhumaṁ vā hīnaṁ vā panītaṁ vā yaṁ dūre santike vā, sabbaṁ rūpaṁ netaṁ mama neso 'ham asmi, na me so attā'ti evam etaṁ yathābhūtaṁ sammāpaññāya daṭṭhabbaṁ. Ya kāci vedanā \ldots{} ya kāci saññā \ldots{} ye keci saṅkhāra \ldots{} yaṁ kiñci viññāṇaṁ atītānāgatapaccuppannaṁ ajjhattaṁ vā bahiddhā vā oḷārikaṁ va sukhumaṁ vā hīnaṁ vā panītaṁ vā yaṁ dūre santike vā, sabbaṁ viññāṇaṁ netaṁ mama neso 'ham asmi, na me so attā'ti evam etaṁ yathābhūtaṁ sammāpaññāya daṭṭhabbaṁ.}

\emph{Evaṁ passaṁ, āvuso Yamaka, sutavā ariyasāvako rūpasmiṁ nibbindati, vedanāya nibbindati, saññāya nibbindati, saṅkhāresu nibbindati, viññāṇasmiṁ nibbindati. Nibbindam virajjati, virāgā vimuccati, vimuttasmiṁ vimuttam iti ñāṇaṁ hoti. Khīṇā jāti vusitaṁ brahmacariyaṁ kataṁ karaṇīyaṁ nāparaṁ itthattāyā'ti pajānāti.}

Therefore, friend Yamaka, any kind of form whatsoever, whether past, future or present, internal or external, gross or subtle, inferior or superior, far or near, all form must be seen as it really is with right wisdom thus: `this is not mine, this I am not, this is not my self'. Any kind of feeling whatsoever \ldots{} any kind of perception whatsoever \ldots{} any kind of preparations whatsoever \ldots{} any kind of consciousness whatsoever, whether past, future or present, internal or external, gross or subtle, inferior or superior, far or near, all consciousness must be seen as it really is with right wisdom thus: `this is not mine, this I am not, this is not my self'.

Seeing thus, friend Yamaka, the instructed noble disciple gets disgusted of form, gets disgusted of feeling, gets disgusted of perception, gets disgusted of preparations, gets disgusted of consciousness. Being disgusted, he becomes dispassionate, through dispassion his mind is liberated. When it is liberated, there comes the knowledge `it is liberated' and he understands: `extinct is birth, lived is the holy life, done is what had to be done, there is no more of this state of being'.
\end{quote}

As the third step in his interrogation of Venerable Yamaka, Venerable Sāriputta poses the same questions which the Buddha addressed to Venerable Anurādha.

\begin{quote}
``What do you think, friend Yamaka, do you regard form as the Tathāgata?'' ``No, friend.'' ``Do you regard feeling \ldots{} perception \ldots{} preparations \ldots{} consciousness as the Tathāgata?'' ``No, friend.''

``What do you think, friend Yamaka, do you regard the Tathāgata as in form?'' ``No, friend.'' ``Do you regard the Tathāgata as apart from form?'' ``No, friend.'' ``Do you regard the Tathāgata as in feeling?'' ``No, friend.'' ``Do you regard the Tathāgata as apart from feeling?'' ``No, friend.'' ``Do you regard the Tathāgata as in perception?'' ``No, friend.'' ``Do you regard the Tathāgata as apart from perception?'' ``No, friend.'' ``Do you regard the Tathāgata as in preparations?'' ``No, friend.'' ``Do you regard the Tathāgata as apart from preparations?'' ``No, friend.'' ``Do you regard the Tathāgata as in consciousness?'' ``No, friend.'' ``Do you regard the Tathāgata as apart from consciousness?'' ``No, friend.''

``What do you think, friend Yamaka, do you regard form, feeling, perception, preparations and consciousness as constituting the Tathāgata?'' ``No, friend.'' ``What do you think, friend Yamaka, do you regard the Tathāgata as one who is devoid of form, feeling, perception, preparations and consciousness?'' ``No, friend.''
\end{quote}

It was at this juncture that Venerable Sāriputta puts this conclusive question to Venerable Yamaka in order to drive the crucial point home.

\begin{quote}
``But then, friend Yamaka, now that for you a Tathāgata is not to be found in truth and fact here in this very life, is it proper for you to declare: `As I understand Dhamma taught by the Exalted One, an influx-free monk is annihilated and destroyed when the body breaks up and does not exist after death'?''
\end{quote}

At last, Venerable Yamaka confesses,

\begin{quote}
``Formerly, friend Sāriputta, I did hold that evil view, ignorant as I was. But now that I have heard this Dhamma sermon of the Venerable Sāriputta, I have given up that evil view and have gained an understanding of the Dhamma.''
\end{quote}

As if to get a confirmation of Venerable Yamaka's present stance, Venerable Sāriputta continues:

\begin{quote}
``If, friend Yamaka, they were to ask you the question: `Friend Yamaka, as to that monk, the influx-free \emph{arahant}, what happens to him with the breaking up of the body after death?' Being asked thus, what would you answer?''

``If they were to ask me that question, friend Sāriputta, I would answer in this way: Friends, form is impermanent, what is impermanent is suffering, what is suffering has ceased and passed away. Feeling \ldots{} perception \ldots{} preparations \ldots{} consciousness is impermanent, what is impermanent is suffering, what is suffering has ceased and passed away. Thus questioned, I would answer in such a way.''
\end{quote}

Be it noted that, in this conclusive answer, there is no mention whatsoever of a Tathāgata, a \emph{satta}, or a \emph{puggala}.

Now at this reply, Venerable Sāriputta expresses his approbation:

\begin{quote}
``Good, good, friend Yamaka, well then, friend Yamaka, I will bring up a simile for you that you may grasp this meaning all the more clearly.

Suppose, friend Yamaka, there was a householder or a householder's son, prosperous, with much wealth and property, protected by a bodyguard. Then some man would come by who wished to ruin him, to harm him, to imperil him, to deprive him of life. And it would occur to that man: `This householder or a householder's son is prosperous, with much wealth and property, he has his bodyguard, it is not easy to deprive him of his life by force. What if I were to get close to him and take his life?'

Then he would approach that householder or householder's son and say to him: `Would you take me on as a servant, sir?' Then the householder or householder's son would take him on as a servant. The man would serve him, rising up before him, going to bed after him, being at his beck and call, pleasing in his conduct, endearing in his speech. The householder or householder's son would regard him as a friend, an intimate friend, and would place trust in him. But once the man has ascertained that the householder or householder's son has trust in him, he waits for an opportunity to find him alone and kills him with a sharp knife.''
\end{quote}

Now this is the simile. Based on this deep simile, Venerable Sāriputta puts the following questions to Venerable Yamaka to see whether he has grasped the moral behind it.

\begin{quote}
``What do you think, friend Yamaka, when that man approached that householder or householder's son and said to him `would you take me on as a servant, sir?', wasn't he a murderer even then, though the householder or householder's son did not know him as `my murderer'? And when the man was serving him, rising up before him and going to bed after him, being at his beck and call, pleasing in his conduct and endearing in his speech, wasn't he a murderer then too, though the householder or householder's son did not know him as `my murderer'? And when the man, finding him alone, took his life with a sharp knife, wasn't he a murderer then too, though the other did not know him as `my murderer'?''
\end{quote}

Venerable Yamaka answers ``Yes, friend'', by way of assent to all these matter-of-fact questions.

It was then, that Venerable Sāriputta comes out with the full significance of this simile, portraying the uninstructed worldling in the same light as that naively unsuspecting and ignorant householder or householder's son.

\begin{quote}
``So too, friend Yamaka, the uninstructed worldling, who has no regard for the noble ones, and is unskilled and undisciplined in their Dhamma, who has no regard for good men and is unskilled and undisciplined in their Dhamma, regards form as self, or self as possessing form, or form as in self, or self as in form. He regards feeling as self \ldots{} perception as self \ldots{} preparations as self \ldots{} consciousness as self \ldots{}

He does not understand, as it really is, impermanent form as `impermanent form', impermanent feeling as `impermanent feeling', impermanent perception as `impermanent perception', impermanent preparations as `impermanent preparations', impermanent consciousness as `impermanent consciousness'.

He does not understand, as it really is, painful form as `painful form', painful feeling as `painful feeling', painful perception as `painful perception', painful preparations as `painful preparations', painful consciousness as `painful consciousness'.

He does not understand, as it really is, selfless form as `selfless form', selfless feeling as `selfless feeling', selfless perception as `selfless perception', selfless preparations as `selfless preparations', selfless consciousness as `selfless consciousness'.

He does not understand, as it really is, prepared form as `prepared form', prepared feeling as `prepared feeling', prepared perception as `prepared perception', prepared preparations as `prepared preparations', prepared consciousness as `prepared consciousness'.

He does not understand, as it really is, murderous form as `murderous form', murderous feeling as `murderous feeling', murderous perception as `murderous perception', murderous preparations as `murderous preparations', murderous consciousness as `murderous consciousness'.''
\end{quote}

This, then, is what the attitude of the uninstructed worldling amounts to. Venerable Sāriputta now goes on to describe the consequences of such an attitude for the worldling.

\begin{quote}
\emph{So rūpaṁ upeti upādiyati adhiṭṭhāti attā me'ti, vedanaṁ \ldots{} saññaṁ \ldots{} saṅkhāre \ldots{} viññāṇaṁ upeti upādiyati adhiṭṭhāti attā me'ti. Tassime pañcupādānakkhandhā upetā upādiṇṇā dīgharattaṁ ahitāya dukkhāya saṁvattanti.}

``He becomes committed to form, grasps it and takes a stand upon it as `my self'. He becomes committed to feeling \ldots{} to perception \ldots{} to preparations \ldots{} to consciousness, grasps it and takes a stand upon it as `my self'. These five aggregates of grasping, to which he becomes committed, and which he grasps, lead to his harm and suffering for a long time.''
\end{quote}

Then Venerable Sāriputta contrasts it with the standpoint of the instructed disciple.

\begin{quote}
``But, friend, the instructed noble disciple, who has regard for the noble ones, who is skilled and disciplined in their Dhamma, who has regard for good men and is skilled and disciplined in their Dhamma, does not regard form as self, or self as possessing form, or form as in self, or self as in form. He does not regard feeling as self \ldots{} perception as self \ldots{} preparations as self \ldots{} consciousness as self, or self as possessing consciousness, or consciousness as in self, or self as in consciousness.

He understands, as it really is, impermanent form as `impermanent form', impermanent feeling as `impermanent feeling', impermanent perception as `impermanent perception', impermanent preparations as `impermanent preparations', impermanent consciousness as `impermanent consciousness'.

He understands, as it really is, painful form as `painful form', painful feeling as `painful feeling', painful perception as `painful perception', painful preparations as `painful preparations', painful consciousness as `painful consciousness'.

He understands, as it really is, selfless form as `selfless form', selfless feeling as `selfless feeling', selfless perception as `selfless perception', selfless preparations as `selfless preparations', selfless consciousness as `selfless consciousness'.

He understands, as it really is, prepared form as `prepared form', prepared feeling as `prepared feeling', prepared perception as `prepared perception', prepared preparations as `prepared preparations', prepared consciousness as `prepared consciousness'.

He understands, as it really is, murderous form as `murderous form', murderous feeling as `murderous feeling', murderous perception as `murderous perception', murderous preparations as `murderous preparations', murderous consciousness as `murderous consciousness'.''

He does not become committed to form, does not grasp it, does not take a stand upon it as `my self'. He does not become committed to feeling \ldots{} to perception \ldots{} to preparations \ldots{} to consciousness, does not grasp it, does not take a stand upon it as `my self'. These five aggregates of grasping, to which he does not become committed, which he does not grasp, lead to his welfare and happiness for a long time.''
\end{quote}

What Venerable Sāriputta wanted to prove, was the fact that everyone of the five aggregates is a murderer, though the worldlings, ignorant of the true state of affairs, pride themselves on each of them, saying ``this is mine, this am I and this is my self''. As the grand finale of this instructive discourse comes the following wonderful declaration by Venerable Yamaka.

\begin{quote}
``Such things do happen, friend Sāriputta, to those venerable ones who have sympathetic and benevolent fellow monks in the holy life, like you, to admonish and instruct, so much so that, on hearing this Dhamma sermon of the Venerable Sāriputta, my mind is liberated from the influxes by non-grasping.''
\end{quote}

This might sound extremely strange in this age of scepticism regarding such intrinsic qualities of the Dhamma like \emph{sandiṭṭhika}, `visible here and now', \emph{akālika}, `timeless', and \emph{ehipassika}, `inviting to come and see'. But all the same we have to grant the fact that this discourse, which begins with a Venerable Yamaka who is bigoted with such a virulent evil view, which even his fellow monks found it difficult to dispel, concludes, as we saw, with this grand finale of a Venerable Yamaka joyfully declaring his attainment of \emph{arahanthood}.

This episode bears testimony to the fact that the tetralemma concerning the Tathāgata's after-death state has beneath it an extremely valuable criterion, proper to this Dhamma. There are some who are even scared to discuss this topic, perhaps due to unbalanced faith -- faith unwarranted by wisdom. The tetralemma, however, reveals on analysis a wealth of valuable Dhamma material that goes to purify one's right view. That is why the Venerable Yamaka ended up as an \emph{arahant}.

So this discourse, also, is further proof of the fact that the Buddha's solution to the problem of the indeterminate points actually took the form of a disquisition on voidness. Such expositions fall into the category called \emph{suññatapaṭisaṁyuttā suttantā}, ``discourses dealing with voidness''. This category of discourses avoids the conventional worldly usages, such as \emph{satta}, `being', and \emph{puggala}, `person', and highlights the teachings on the four noble truths, which bring out the nature of things `as they are'.

Generally, such discourses instil fear into the minds of worldlings, so much so that even during the Buddha's time there were those recorded instances of misconstruing and misinterpretation. It is in this light that we have to appreciate the Buddha's prediction that in the future there will be monks who would not like to listen or lend ear to those deep and profound discourses of the Buddha, pertaining to the supramundane and dealing with the void.

\begin{quote}
\emph{Puna ca paraṁ, bhikkhave, bhavissanti bhikkhū anāgatamaddhānaṁ abhāvitakāya abhāvitasīlā abhāvitacittā abhāvitapaññā, te abhāvitakāyā samānā abhāvitasīlā abhāvitacittā abhāvitapaññā ye te suttantā tathāgatabhāsitā gambhīrā gambhīratthā lokuttarā suññatāpaṭisaṁyuttā, tesu bhaññamānesu na sussūsanti, na sotaṁ odahissanti, na aññācittaṁ upaṭṭhapessanti, na ca te dhamme uggahetabbaṁ pariyāpuṇitabbaṁ maññissanti.}\footnote{A III 107, \emph{Tatiya-anāgatabhayasutta}; see also \href{https://suttacentral.net/sn20.7/pli/ms}{SN 20.7 / S II 267}, \emph{Āṇisutta}}

And moreover, monks, there will be in the future those monks who, being undeveloped in bodily conduct, being undeveloped in morality, being undeveloped in concentration, being undeveloped in wisdom, would not like to listen, to lend ear or to make an attempt to understand and deem it fit to learn when those discourses preached by the Tathāgata, which are deep, profound in meaning, supramundane and dealing with the void, are being recited.
\end{quote}

This brings us to an extremely deep dimension of this Dhamma. By way of clarification, we may allude to a kind of exorcism practiced by some traditional devil dancers. At the end of an all-night session of devil dancing, the mediating priest goes round, exorcising the spirits from the house with fistfuls of a highly inflammable incense powder. Blazing flames arise, as he sprinkles that powder onto the lighted torch, directing the flames at every nook and corner of the house. Some onlookers even get scared that he is trying to set the house on fire. But actually no harm is done.

Well, the Buddha, too, as the mediating priest of the three realms, had to conduct a similar exorcising ritual over linguistic conventions, aiming at some words in particular. It is true that he made use of conventional language in order to convey his teaching. But his Dhamma proper was one that transcended logic, \emph{atakkāvacaro}.\footnote{M I 167, \emph{Ariyapariyesanasutta}}

It happened to be a Dhamma that soared well above the limitations of grammar and logic, and analytically exposed their very structure. The marvel of the Dhamma is in its very inaccessibility to logic. That is why it defied the four-cornered logic of the tetralemma. It refused to be cornered and went beyond the concepts of a `being' or a `self'. The \emph{saṁsāric} vortex was breached and concepts themselves were transcended.

Now this is the exorcism the Buddha had to carry out. He smoked out the term \emph{attā}, `self', so dear to the whole world. Of course, he could not help making use of that word as such. In fact there is an entire chapter in the \emph{Dhammapada} entitled \emph{Attavagga}.\footnote{Dhp 157-166 make up the 12th chapter of Dhp, the \emph{Attavagga}} But it must be emphasized that the term in that context does not refer to a permanent self. It stands for `oneself'. Some who mistakenly rendered it as `self', ended up in difficulties. Take for instance the following verse.

\begin{quote}
\emph{Attā hi attano nātho,}\\
\emph{ko hi nātho paro siyā,}\\
\emph{attanā hi sudantena,}\\
\emph{nāthaṁ labhati dullabhaṁ.}\footnote{Dhp 160, \emph{Attavagga}}

Oneself, indeed, is one's own saviour,\\
What other saviour could there be?\\
Even in oneself, disciplined well,\\
One finds that saviour, so hard to find.
\end{quote}

Those who render the above verse literally, with a self-bias, would get stuck when confronted with the following verse in the \emph{Bālavagga}, the ``chapter of the fool''.

\begin{quote}
\emph{Puttā m'atthi, dhanam m'atthi,}\\
\emph{iti bālo vihaññati,}\\
\emph{attā hi attano natthi,}\\
\emph{kuto puttā, kuto dhanaṁ?}\footnote{Dhp 62, \emph{Bālavagga}}

`Sons I have, wealth I have',\\
So the fool is vexed,\\
Even oneself is not one's own,\\
Where then are sons, where is wealth?
\end{quote}

Whereas the former verse says \emph{attā hi attano nātho}, here we find the statement \emph{attā hi attano natthi}. If one ignores the reflexive sense and translates the former line with something like ``self is the lord of self'', one will be at a loss to translate the seemingly contradictory statement ``even self is not owned by self''.

At times, the Buddha had to be incisive in regard to some words, which the worldlings are prone to misunderstand and misinterpret. We have already discussed at length the significance of such terms as \emph{satta} and \emph{tathāgata}, with reference to their etymological background. \emph{Sakkāyadiṭṭhi}, or `personality view', masquerades even behind the term \emph{tathāgata}, and that is why they raise such ill-founded questions. That is also why one is averse to penetrate into the meanings of these deep discourses.

Like the term \emph{tathāgata}, the term \emph{loka} also had insinuations of a self-bias. The Buddha, as we saw, performed the same ritual of exorcism to smoke out those insinuations. His definition of the `world' with reference to the six sense-bases is a corrective to that erroneous concept.\footnote{S I 41, \emph{Lokasutta}, see also \emph{Sermon 4}; S IV 39, \emph{Samiddhisutta}, see also \emph{Sermon 20}}

Among the indeterminate points, too, we find questions relating to the nature of the world, such as \emph{sassato loko -- asassato loko}, ``the world is eternal -- the world is not eternal'', and \emph{antavā loko -- anantavā loko}, ``the world is finite -- the world is infinite''.\footnote{E.g. at M I 426, \emph{Mahāmālunkyasutta}} In all such contexts, the questioner had the prejudice of the conventional concept of the world. The commentaries refer to it as \emph{cakkavāḷaloka}, the common concept of `world system'.\footnote{Spk I 116} But the Buddha advanced a profound definition of the concept of the world with reference to the six bases of sense-contact.

In this connection, we come across a highly significant discourse in the \emph{Saḷāyatanavagga} of the \emph{Saṁyutta Nikāya}. There we find the Buddha making the following declaration to the monks.

\begin{quote}
\emph{Nāhaṁ, bhikkhave, gamanena lokassa antaṁ ñātayyaṁ, daṭṭhayyaṁ, patteyyan'ti vadāmi. Na ca panāhaṁ, bhikkhave, appatvā lokassa antaṁ dukkhassa antakiriyaṁ vadāmi.}\footnote{S IV 93, \emph{Lokakāmaguṇasutta}}

Monks, I do not say that by travelling one can come to know or see or reach the end of the world. Nor do I say that without reaching the end of the world one can put an end to suffering.
\end{quote}

After this riddle-like pronouncement, the Buddha gets up and retires to the monastery. We came across this kind of problematic situation earlier too. Most probably this is a device of the Buddha as the teacher to give his disciples an opportunity to train in the art of analytical exposition of the Dhamma.

After the Buddha had left, those monks, perplexed by this terse and tantalizing declaration, approached Venerable Ānanda and begged him to expound its meaning at length. With some modest hesitation, as usual, Venerable Ānanda agreed and came out with the way he himself understood the significance of the Buddha's declaration in the following words.

\begin{quote}
\emph{Yena kho, āvuso, lokasmiṁ lokasaññī hoti lokamānī, ayaṁ vuccati ariyassa vinaye loko. Kena c'āvuso lokasmiṁ lokasaññī hoti lokamānī?}

\emph{Cakkhunā kho, āvuso, lokasmiṁ lokasaññī hoti lokamānī, sotena \ldots{} ghānena \ldots{} jivhāya \ldots{} kāyena \ldots{} manena kho, āvuso, lokasmiṁ lokasaññī hoti lokamānī. Yena kho, āvuso, lokasmiṁ lokasaññī hoti lokamānī, ayaṁ vuccati ariyassa vinaye loko.}

Friends, that by which one has a perception of the world and a conceit of the world, that in this discipline of the noble ones is called `the world'. By what, friends, has one a perception of the world and a conceit of the world?

By the eye, friends, one has a perception of the world and a conceit of the world, by the ear \ldots{} by the nose \ldots{} by the tongue \ldots{} by the body \ldots{} by the mind, friends one has a perception of the world and a conceit of the world. That, friends, by which one has a perception of the world and a conceit of the world, that in this discipline of the noble ones is called `the world'.
\end{quote}

It seems, then, that the definition of the world in the discipline of the noble ones is one that accords with radical attention, \emph{yoniso manasikāra}, whereas the concept of the world as upheld in those indeterminate points is born of wrong attention, \emph{ayoniso manasikāra}.

In the present age, too, scientists, when they speak of an `end of the world', entertain presumptions based on wrong attention.

When those monks who listened to Venerable Ānanda's exposition reported it to the Buddha, he fully endorsed it. This definition, therefore, is as authentic as the word of the Buddha himself and conclusive enough. It is on the basis of the six sense-bases that the world has a perception of the `world' and a conceit of the `world'.

The conceit here meant is not pride as such, but the measuring characteristic of worldly concepts. For instance, there is this basic scale of measuring length: The inch, the span, the foot, the cubit and the fathom. These measurements presuppose this body to be a measuring rod.

In fact, all scales of measurement, in some way or other, relate to one or the other of the six sense-bases. That is why the above definition of the world is on the side of radical attention.

The worldling's concept of the world, conventionally so called, is the product of wrong or non-radical attention. It is unreal to the extent that it is founded on the notion of the compact, \emph{ghanasaññā}. The existence of the world, as a whole, follows the norm of arising and ceasing. It is by ignoring this norm that the notion of the compact receives acceptance.

Two persons are watching a magic kettle on display at a science exhibition. Water is endlessly flowing from the magic kettle to a basin. One is waiting until the kettle gets empty, while the other waits to see the basin overflowing. Neither of their wishes is fulfilled. Why? Because a hidden tube conducts the water in the basin back again to the kettle. So the magic kettle never gets emptied and the basin never overflows. This is the secret of the magic kettle.

The world also is such a magic kettle. Gigantic world systems contract and expand in cyclic fashion. In the ancient term for world systems, \emph{cakkavāḷa}, this cyclic nature is already insinuated. Taken in a broader sense, the existence or continuity of the world is cyclic, as indicated by the two terms \emph{saṁvaṭṭa} and \emph{vivaṭṭa}, `contraction' and `expansion'. In both these terms, the significant word \emph{vaṭṭa}, suggestive of `turning round', is seen to occur. It is as good as saying `rise and fall', \emph{udayabbaya}.

When one world system gets destroyed, another world system gets crystallized, as it were. We hear of Brahmā mansions emerging.\footnote{D I 17, \emph{Brahmajālasutta}} So the existence of the world is a continuous process of arising and ceasing. It is in a cycle. How can one find a point of beginning in a cycle? Can one speak of it as `eternal' or `non-eternal'? The question as a whole is fallacious.

On the other hand the Buddha's definition of the term \emph{loka}, based on the etymology \emph{lujjati}, \emph{palujjatī'ti loko}, is quite apt and meaningful.\footnote{S IV 52, \emph{Lokapañhāsutta}, see \emph{Sermon 20}}

The world is all the time in a process of disintegration. It is by ignoring this disintegrating nature and by overemphasizing the arising aspect that the ordinary uninstructed worldling speaks of a `world' as it is conventionally understood. The world is afflicted by this process of arising and passing away in every moment of its existence.

It is to be found in our breathing, too. Our entire body vibrates to the rhythm of this rise and fall. That is why the Buddha offered us a redefinition of the world. According to the terminology of the noble ones, the world is to be redefined with reference to the six bases of sense-contact. This includes mind and mind-objects as well. In fact, the range of the six bases of sense-contact is all comprehending. Nothing falls outside of it.
