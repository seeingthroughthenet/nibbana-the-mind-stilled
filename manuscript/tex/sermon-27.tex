\chapter{Sermon 27}

\NibbanaOpeningQuote

With the permission of the Most Venerable Great Preceptor and the assembly of the venerable meditative monks. This is the twenty-seventh sermon in the series of sermons on Nibbāna. In our last sermon, we brought up some similes and illustrations to explain why the suchness of the Tathāgata has been given special emphasis in the \emph{Kāḷakārāmasutta}.

Drawing inspiration from the Buddha's sermon, comparing consciousness to a magic show, we made an attempt to discover the secrets of a modern day magic show from a hidden corner of the stage. The parable of the magic show revealed us the fact that the direct and the indirect formulation of the Law of Dependent Arising, known as \emph{tathatā}, suchness, or \emph{idapaccayatā}, specific conditionality, is similar to witnessing a magic show from two different points of view. That is to say, the deluded point of view of the spectator in the audience and the discerning point of view of the wisdom-eyed critic, hidden in a corner of the stage.

\clearpage

The reason for the riddle-like outward appearance of the \emph{Kāḷakārāmasutta} is the problem of resolving the conflict between these two points of view. However, the fact that the Tathāgata resolved this conflict at a supramundane level and enjoyed the bliss of emancipation comes to light in the first three discourses of the \emph{Bodhivagga} in the \emph{Udāna}.\footnote{\href{https://suttacentral.net/ud1.1/pli/ms}{Ud 1.1-3 / Ud 1}-2, \emph{Bodhivagga}}

These three discourses tell us that, after the attainment of enlightenment, the Buddha spent the first week in the same seated posture under the Bodhi tree, and that on the last night of the week he reflected on the Law of Dependent Arising in the direct order in the first watch of the night, in the reverse order in the second watch, and both in direct and reverse order in the last watch.

These last-mentioned reflection, both in direct and reverse order, is like a compromise between the deluded point of view and the discerning point of view, mentioned above. Now, in a magic show to see \textbf{how} the magic is performed, is to get disenchanted with it, to make it fade away and cease, to free the mind from its spell. By seeing \textbf{how} a magician performs, one gets disgusted with \textbf{what} he performs. Similarly, seeing the arising of the six bases of sense-contact is the way to get disenchanted with them, to make them fade away and cease, to transcend them and be emancipated.

We come across two highly significant verses in the \emph{Soṇasutta} among the Sixes of the \emph{Aṅguttara Nikāya} with reference to the emancipation of the mind of an \emph{arahant}.

\begin{quote}
\emph{Nekkhammaṁ adhimuttassa,}\\
\emph{pavivekañca cetaso,}\\
\emph{abhyāpajjhādhimuttassa,}\\
\emph{upādānakkhayassa ca,}

\emph{taṇhakkhayādhimuttassa,}\\
\emph{asammohañca cetaso,}\\
\emph{disvā āyatanuppādaṁ,}\\
\emph{sammā cittaṁ vimuccati.}\footnote{A III 378, \emph{Soṇasutta}}

\clearpage

The mind of one who is fully attuned\\
To renunciation and mental solitude,\\
Who is inclined towards harmlessness,\\
Ending of grasping,

Extirpation of craving,\\
And non-delusion of mind,\\
On seeing the arising of sense-bases,\\
Is fully emancipated.
\end{quote}

To see how the sense-bases arise is to be released in mind. Accordingly we can understand how the magic consciousness of one who is enjoying a magic show comes to cease by comprehending it. Magic consciousness subsides. In other words, it is transformed into a non-manifestative consciousness, which no longer displays any magic.

That is the mental transformation that occurred in the man who watched the magic show from a hidden corner of the stage. This gives us a clue to the cessation of consciousness in the \emph{arahant} and the consequent non-manifestative consciousness attributed to him.

The \emph{Dvāyatanānupassanasutta} of the \emph{Sutta Nipāta} also bears testimony to this fact. The title itself testifies to the question of duality forming the theme of this discourse. Throughout the sutta we find a refrain-like distinction between the arising and the ceasing of various phenomena. It is like an illustration of the two aspects of the problem that confronted the Buddha. Now that we are concerned with the question of the cessation of consciousness, let us quote the relevant couplet of verses.

\begin{quote}
\emph{Yaṁ kiñci dukkhaṁ sambhoti,}\\
\emph{sabbaṁ viññāṇapaccayā,}\\
\emph{viññāṇassa nirodhena}\\
\emph{natthi dukkhassa sambhavo.}

\emph{Etam ādīnavaṁ ñatvā,}\\
\emph{`dukkhaṁ viññāṇapaccayā',}\\
\emph{viññāṇūpasamā bhikkhu,}\\
\emph{nicchāto parinibbuto.}\footnote{Sn 734, \emph{Dvāyatanānupassanasutta}}

Whatever suffering that arises,\\
All that is due to consciousness,\\
With the cessation of consciousness,\\
There is no arising of suffering.

Knowing this peril:\\
`This suffering dependent on consciousness',\\
By calming down consciousness, a monk\\
Is hunger-less and fully appeased.
\end{quote}

The comparison between the magic show and consciousness becomes more meaningful in the context of this discourse. As in the case of a magic show, the delusory character of the magic of consciousness is traceable to the perception of form. It is the perception of form which gives rise to the host of reckonings through cravings, conceits and views, which bring about a delusion.

Therefore, a monk intent on attaining Nibbāna has to get rid of the magical spell of the perception of form. The verse we cited from the \emph{Kalahavivādasutta} the other day has an allusion to this requirement. That verse, beginning with the words \emph{na saññasaññī,} is an attempt to answer the question raised in a previous verse in that sutta, posing the query:

\begin{quote}
\emph{Kathaṁ sametassa vibhoti rūpaṁ},\footnote{\href{https://suttacentral.net/snp4.11/pli/ms}{Snp 4.11 / Sn 873}, \emph{Kalahavivādasutta}}

to one, constituted in which manner, does form cease to exist?
\end{quote}

Let us remind ourselves of that verse.

\begin{quote}
\emph{Na saññasaññī, na visaññasaññī,}\\
\emph{no pi asaññī na vibhūtasaññī,}\\
\emph{evaṁ sametassa vibhoti rūpaṁ,}\\
\emph{saññānidānā hi papañcasaṅkhā.}

He is not conscious of normal perception,\\
\vin nor is he unconscious,\\
He is not devoid of perception,\\
\vin nor has he rescinded perception,

\clearpage

It is to one thus constituted\\
\vin that form ceases to exist,\\
For reckonings through prolificity\\
\vin have perception as their source.
\end{quote}

Here the last line states a crucial fact. Reckonings, designations and the like, born of prolificity, are traceable to perception in the last analysis. That is to say, all that is due to perception.

Another reason why form has received special attention here, is the fact that it is a precondition for contact. When there is form, there is the notion of resistance. That is already implicit in the question that comes in a verse at the beginning of the \emph{Kalahavivādasutta}:

\begin{quote}
\emph{Kismiṁ vibhūte na phusanti phassā},\footnote{\href{https://suttacentral.net/snp4.11/pli/ms}{Snp 4.11 / Sn 871}, \emph{Kalahavivādasutta}}

when what is not there, do touches not touch?
\end{quote}

The answer to that query is:

\begin{quote}
\emph{Rūpe vibhūte na phusanti phassā},

when form is not there, touches do not touch.
\end{quote}

We come across a phrase relevant to this point in the \emph{Saṅgītisutta} of the \emph{Dīgha Nikāya}, that is, \emph{sanidassanasappaṭighaṁ rūpaṁ}.\footnote{D III 217, \emph{Saṅgītisutta}}

Materiality, according to this phrase, has two characteristics. It has the quality of manifesting itself, \emph{sanidassana}; it also offers resistance, \emph{sappaṭigha}. Both these aspects are hinted at in a verse from the \emph{Jaṭāsutta} we had quoted at the very beginning of this series of sermons.

\begin{quote}
\emph{Yattha nāmañca rūpañca,}\\
\emph{asesaṁ uparujjhati,}\\
\emph{paṭighaṁ rūpasaññā ca,}\\
\emph{etthasā chijjate jaṭā.}\footnote{\href{https://suttacentral.net/sn1.23/pli/ms}{SN 1.23 / S I 13}, \emph{Jaṭāsutta}, see \emph{Sermon 1}}
\end{quote}

\clearpage

The \emph{Jaṭāsutta} tells us the place where the tangle within and the tangle without, \emph{antojaṭā bahijaṭā}, of this gigantic \emph{saṁsāric} puzzle is solved. And here is the answer:

\begin{quote}
Wherein name and form\\
As well as resistance and the perception of form\\
Are completely cut off,\\
It is there that the tangle gets snapped.
\end{quote}

The phrase \emph{paṭighaṁ rūpasaññā ca} is particularly significant. Not only the term \emph{paṭigha}, implying `resistance', but also the term \emph{rūpasaññā} deserves our attention, as it is suggestive of the connection between form and perception. It is perception that brings an image of form. Perception is the source of various reckonings and destinations.

The term \emph{saññā} has connotations of a `mark', a `sign', or a `token', as we have already pointed out.\footnote{See \emph{Sermon 12}} It is as if a party going through a forest is blazing a trail for their return by marking notches on the trees with an axe. The notion of permanence is therefore implicit in the term \emph{saññā}.

So it is this \emph{saññā} that gives rise to \emph{papañcasaṅkhā}, reckonings through prolificity. The compound term \emph{papañcasaññāsaṅkhā}, occurring in the \emph{Madhupiṇḍikasutta},\footnote{\href{https://suttacentral.net/mn18/pli/ms}{MN 18 / M I 109}, \emph{Madhupiṇḍikasutta}} is suggestive of this connection between \emph{saññā} and \emph{saṅkhā}.

Reckonings, definitions and designations, arising from prolific perception, are collectively termed \emph{papañcasaññāsaṅkhā}. The significance attached to \emph{saññā} could easily be guessed by the following dictum in the \emph{Guhaṭṭhakasutta} of the \emph{Sutta Nipāta}:

\begin{quote}
\emph{Saññāṁ pariññā vitareyya oghaṁ},\footnote{Sn 779, \emph{Guhaṭṭhaka Sutta}}

comprehend perception and cross the flood.
\end{quote}

Full comprehension of the nature of perception enables one to cross the four great floods of defilements in \emph{saṁsāra}. In other words, the penetrative understanding of perception is the way to deliverance.

Let us now go a little deeper into the connotations of the term \emph{saññā}. In the sense of `sign' or `token', it has to have something to signify or symbolize. Otherwise there is no possibility of designation. A sign can be significant only if there is something to signify. This is a statement that might need a lot of reflection before it is granted.

A sign properly so called is something that signifies, and when there is nothing to signify, it ceases to be a sign. So also is the case with the symbol. This is a norm which is well explained in the \emph{Mahāvedallasutta} of the \emph{Majjhima Nikāya}. In the course of a dialogue between Venerable Mahā Koṭṭhita and Venerable Sāriputta, we find in that sutta the following pronouncement made by Venerable Sāriputta:

\begin{quote}
\emph{Rāgo kho, āvuso, kiñcano, doso kiñcano, moho kiñcano, te khīnāsavassa bhikkhuno pahīnā ucchinnamūlā tālāvatthukatā anabhāvakatā āyatiṁ anuppādadhammā.}\footnote{M I 298, \emph{Mahāvedallasutta}}

Lust, friend, is something, hate is something, delusion is something. They have been abandoned in an influx-free monk, uprooted, made like a palm tree deprived of its site, made extinct and rendered incapable of sprouting again.
\end{quote}

So lust is a something, hate is a something, delusion is a something. Now a sign is significant and a symbol is symbolic only when there is something. Another statement that occurs a little later in that dialogue offers us a clarification.

\begin{quote}
\emph{Rāgo kho, āvuso, nimittakaraṇo, doso nimittakaraṇo, moho nimittakaraṇo},

lust, friend, is significative, hate is significative, delusion is significative.
\end{quote}

Now we can well infer that it is only so long as there are things like lust, hate and delusion that signs are significant. In other words, why the Tathāgata declared that there is no essence in the magic show of consciousness is because there is nothing in him that signs or symbols can signify or symbolize.

What are these things? Lust, hate and delusion. That is why the term \emph{akiñcana}, literally `thing-less', is an epithet for the \emph{arahant}. He is thing-less not because he no longer has the worldly possessions of a layman, but because the afore-said things lust, hate and delusion are extinct in him. For the Tathāgata, the magic show of consciousness has nothing substantial in it, because there was nothing in him to make the signs significant.

That man with discernment, who watched the magic show from a hidden corner of the stage, found it to be hollow and meaningless, since he had, in a limited and relative sense, got rid of attachment, aversion and delusion. That is to say, after discovering the tricks of the magician, he lost the earlier impulses to laugh, cry and fear. Now he has no curiosity, since the delusion is no more. At least temporarily, ignorance has gone down in the light of understanding. According to this norm, we can infer that signs become significant due to greed, hate and delusion in our own minds. Perceptions pander to these emotive tendencies.

The concluding verse of the \emph{Māgandiyasutta} of the \emph{Sutta Nipāta} is particularly important, in that it sums up the \emph{arahant's} detachment regarding perceptions and his release through wisdom.

\begin{quote}
\emph{Saññāvirattassa na santi ganthā,}\\
\emph{paññāvimuttassa na santi mohā,}\\
\emph{saññañca diṭṭhiñca ye aggahesuṁ,}\\
\emph{te ghaṭṭayantā vicaranti loke.}\footnote{\href{https://suttacentral.net/snp4.9/pli/ms}{Snp 4.9 / Sn 847}, \emph{Māgandiyasutta}}

To one detached from percepts there are no bonds,\\
To one released through wisdom there are no delusions,\\
Those who hold on to percepts and views,\\
Go about wrangling in this world.
\end{quote}

It is this state of detachment from perceptions and release through wisdom that is summed up by the phrase \emph{anāsavaṁ cetovimuttiṁ paññāvimuttiṁ} in some discourses. With reference to the \emph{arahant} it is said that he has realized by himself through higher knowledge in this very life that influx-free deliverance of the mind and deliverance through wisdom, \emph{anāsavaṁ cetovimuttiṁ paññāvimuttiṁ diṭṭhevadhamme sayaṁ abhiññā sacchikatvā}.\footnote{E.g. D I 156, \emph{Mahāli Sutta}}

So we could well infer that the \emph{arahant} is free from the enticing bonds of perceptions and the deceptive tricks of consciousness. It is this unshakeable stability that finds expression in the epithets \emph{anejo}, `immovable', and \emph{ṭhito}, `stable', used with reference to the \emph{arahant}.\footnote{Ud 27, \emph{Yasoja Sutta}}

The \emph{Āneñjasappāyasutta} of the \emph{Majjhima Nikāya} opens with the following exhortation by the Buddha:

\begin{quote}
\emph{Aniccā, bhikkhave, kāmā tucchā musā mosadhammā, māyākatam etaṁ, bhikkhave, bālalāpanaṁ. Ye ca diṭṭhadhammikā kāmā, ye ca samparāyikā kāmā, yā ca diṭṭhadhammikā kāmasaññā, yā ca samparāyikā kāmasañña, ubhayam etaṁ Māradheyyaṁ, Mārass'esa visayo, Mārass' esa nivāpo, Mārass' esa gocaro.}\footnote{M II 261, \emph{Āneñjasappāyasutta}}

Impermanent, monks, are sense pleasures, they are empty, false and deceptive by nature, they are conjuror's tricks, monks, tricks that make fools prattle. Whatever pleasures there are in this world, whatever pleasures that are in the other world, whatever pleasurable percepts there are in this world, whatever pleasurable percepts that are in the other world, they all are within the realm of Māra, they are the domain of Māra, the bait of Māra, the beat of Māra.
\end{quote}

This exhortation accords well with what was said above regarding the magic show. It clearly gives the impression that there is the possibility of attaining a state of mind in which those signs are no longer significant.

The comparison of consciousness to a magic show has deeper implications. The insinuation is that one has to comprehend perception for what it is, in order to become dispassionate towards it, \emph{saññaṁ pariññā vitareyya oghaṁ}, ``comprehend perception and cross the flood''. When perception is understood inside out, disenchantment sets in as a matter of course, since delusion is no more.

Three kinds of deliverances are mentioned in connection with the \emph{arahants}, namely \emph{animitta}, the signless, \emph{appaṇihita}, the undirected, and \emph{suññata}, the void.\footnote{Vin III 92, \emph{Pārājikakaṇḍa}} We spoke of signs being significant. Now where there is no signification, when one does not give any significance to signs, one does not direct one's mind to anything. \emph{Paṇidhi} means `direction of the mind', an `aspiration'. In the absence of any aspiration, there is nothing `essence-tial' in existence.

There is a certain interconnection between the three deliverances. \emph{Animitta}, the signless, is that stage in which the mind refuses to take a sign or catch a theme in anything. Where lust, hate and delusion are not there to give any significance, signs become ineffective. That is the signless. Where there is no tendency to take in signs, there is no aspiration, expectation or direction of the mind.

It is as if dejection in regard to the magic show has given rise to disenchantment and dispassion. When the mind is not directed to the magic show, it ceases to exist. It is only when the mind is continually there, directed towards the magic show or a film show, that they exist for a spectator. One finds oneself born into a world of magic only when one sees something substantial in it. A magic world is made up only when there is an incentive to exist in it.

Deeper reflection on this simile of the magic show would fully expose the interior of the magical illusion of consciousness. Where there is no grasping at signs, there is no direction or expectation, in the absence of which, existence ceases to appear substantial. That is why the three terms singless, \emph{animitta}, undirected, \emph{appaṇihita} and void \emph{suññata}, are used with reference to an \emph{arahant}. These three terms come up in a different guise in a discourse on Nibbāna we had discussed earlier. There they occur as \emph{appatiṭṭhaṁ, appavattaṁ} and \emph{anārammaṇaṁ}.\footnote{\href{https://suttacentral.net/ud8.2/pli/ms}{Ud 8.2 / Ud 80}, \emph{Paṭhamanibbānapaṭisaṁyuttasutta}; see \emph{Sermon 17}}

\emph{Appatiṭṭhaṁ} means `unestablished'. Mind gets established when there is desire or aspiration, \emph{paṇidhi}. Contemplation on the suffering aspect, \emph{dukkhānupassanā}, eliminates desire. So the mind is unestablished. Contemplation on not-self, \emph{anattānupassanā}, does away with the notion of substantiality, seeing nothing pithy or `essence-tial' in existence.

Pith is something that endures. A tree that has pith has something durable, though its leaves may drop off. Such notions of durability lose their hold on the \emph{arahant's} mind. The contemplation of impermanence, \emph{aniccānupassanā}, ushers in the signless, \emph{animitta}, state of the mind that takes no object, \emph{anārammaṇaṁ.}

The simile of the magic show throws light on all these aspects of deliverance. Owing to this detachment from perception, \emph{saññāviratta}, and release through wisdom, \emph{paññāvimutta}, an \emph{arahant's} point of view is totally different from the wordling's point of view. What appears as real for the worldling, is unreal in the estimation of the \emph{arahant}. There is such a wide gap between the two viewpoints. This fact comes to light in the two kinds of reflections mentioned in the \emph{Dvayatānupassanāsutta} of the \emph{Sutta Nipāta}.

\begin{quote}
\emph{Yaṁ, bhikkhave, sadevakassa lokassa samārakassa sabrahmakassa sassamaṇabrāhmaṇiyā pajāya sadevamanussāya `idaṁ saccan'ti upanijjhāyitaṁ, tadam ariyānaṁ `etaṁ musā'ti yathābhūtaṁ sammappaññāya suddiṭṭhaṁ -- ayaṁ ekānupassanā. Yaṁ, bhikkhave, sadevakassa lokassa samārakassa sabrahmakassa sassamaṇabrāhmaṇiyā pajāya sadevamanussāya `idaṁ musā'ti upanijjhāyitaṁ, tadam ariyānaṁ `etaṁ saccan'ti yathābhūtaṁ sammappaññāya suddiṭṭhaṁ -- ayaṁ dutiyānupassanā.}\footnote{(Prose before) \href{https://suttacentral.net/snp3.12/pli/ms}{Snp 3.12 / Sn 756}, \emph{Dvayatānupassanasutta}}

Monks, whatsoever in the world with its gods, Māras and Brahmas, among the progeny consisting of recluses, Brahmins, gods and men, whatsoever is pondered over as `truth', that by the \emph{ariyans} has been well discerned with right wisdom, as it is, as `untruth'. This is one mode of reflection. Monks, whatsoever in the world with its gods, Māras and Brahmas, among the progeny consisting of recluses, Brahmins, gods and men, whatsoever is pondered over as `untruth', that by the \emph{ariyans} has been well discerned with right wisdom, as it is, as `truth'. This is the second mode of reflection.
\end{quote}

From this, one can well imagine what a great difference, what a contrast exists between the two stand-points. The same idea is expressed in the verses that follow, some of which we had cited earlier too.

\begin{quote}
\emph{Anattani attamāniṁ,}\\
\emph{passa lokaṁ sadevakaṁ,}\\
\emph{niviṭṭhaṁ nāmarūpasmiṁ,}\\
\emph{idaṁ saccan'ti maññati.}

\emph{Yena yena hi maññanti,}\\
\emph{tato taṁ hoti aññathā,}\\
\emph{taṁ hi tassa musā hoti,}\\
\emph{mosadhammaṁ hi ittaraṁ.}

\emph{Amosadhammaṁ nibbānaṁ,}\\
\emph{tad ariyā saccato vidū,}\\
\emph{te ve saccābhisamayā,}\\
\emph{nicchātā parinibbutā.}\footnote{See \emph{Sermons 6 and 21}}

Just see the world, with all its gods,\\
Fancying a self where none exists,\\
Entrenched in name-and-form it holds\\
The conceit that this is real.

In whatever way they imagine,\\
Thereby it turns otherwise,\\
That itself is the falsity,\\
Of this puerile deceptive thing.

Nibbāna is unfalsifying in its nature,\\
That they understood as the truth,\\
And, indeed, by the higher understanding of that truth,\\
They have become hunger-less and fully appeased.
\end{quote}

Let us go for a homely illustration to familiarize ourselves with the facts we have related so far. Two friends are seen drawing something together on a board with two kinds of paints. Let us have a closer look. They are painting a chess board. Now the board is chequered. Some throw-away chunks of wood are also painted for the pieces. So the board and pieces are ready.

Though they are the best of friends and amicably painted the chessboard, the game of chess demands two sides -- the principle of duality. They give in to the demand and confront each other in a playful mood. A hazy idea of victory and defeat, another duality, hovers above them. But they are playing the game just for fun, to while away the time. Though it is for fun, there is a competition. Though there is a competition, it is fun.

While the chess-game is in progress, a happy-go-lucky benefactor comes by and offers a handsome prize for the prospective winner, to enliven the game. From now onwards, it is not just for fun or to while away the time that the two friends are playing chess. Now that the prospect of a prize has aroused greed in them, the innocuous game becomes a tussle for a prize.

Worthless pieces dazzle with the prospect of a prize. But just then, there comes a pervert killjoy, who shows a threatening weapon and adds a new rule to the game. The winner will get the prize all right, but the loser he will kill with his deadly weapon.

So what is the position now? The sportive spirit is gone. It is now a struggle for dear life. The two friends are now eying each other as an enemy. It is no longer a game, but a miserable struggle to escape death.

We do not know, how exactly the game ended. But let us hold the post mortem all the same. We saw how those worthless chunks of wood picked up to serve as pieces on the chessboard, received special recognition once they took on the paint. They represented two sides.

With the prospect of a prize, they got animated in the course of the game, due to cravings, conceits and views in the minds of the two players. Those impulses were so overwhelming that especially after the death knell sounded, the whole chess board became the world for these two friends. Their entire attention was on the board -- a life and death struggle.

But this is only one aspect of our illustration. The world, in fact, is a chessboard, where an unending chess game goes on. Let us look at the other aspect. Now, for the \emph{arahant}, the whole world appears like a chessboard. That is why the \emph{arahant} Adhimutta, when the bandits caught him while passing through a forest and got ready to kill him, uttered the following instructive verse, which we had quoted earlier too.

\begin{quote}
\emph{Tiṇakaṭṭhasamaṁ lokaṁ,}\\
\emph{yadā paññāya passati,}\\
\emph{mamattaṁ so asaṁvindaṁ,}\\
\emph{`natthi me'ti na socati.}\footnote{\href{https://suttacentral.net/thag16.1/pli/ms}{Thag 16.1 / Th 717}, \emph{Adhimutta Theragāthā}, see \emph{Sermon 8}}

When one sees with wisdom,\\
This world as comparable to grass and twigs,\\
Not finding anything worthwhile holding onto as mine,\\
One does not grieve, saying: `O! I have nothing!'
\end{quote}

Venerable Adhimutta's fearless challenge to the bandit chief was extraordinary -- you may kill me if you like, but the position is this: When one sees with wisdom the entire world, the world of the five aggregates, as comparable to grass and twigs, one does not experience any egoism and therefore does not grieve the loss of one's life.

Some verses uttered by the Buddha deepen our understanding of the \emph{arahant's} standpoint. The following verse of the \emph{Dhammapada}, for instance, highlights the conflict between victory and defeat.

\begin{quote}
\emph{Jayaṁ veraṁ passavati,}\\
\emph{dukkhaṁ seti parājito,}\\
\emph{upasanto sukhaṁ seti}\\
\emph{hitvā jayaparājayaṁ.}\footnote{Dhp 201, \emph{Sukhavagga}}

Victory breeds hatred,\\
In sorrow lies the defeated,\\
The one serene is ever at peace,\\
Giving up victory and defeat.
\end{quote}

As in the chess game, the idea of winning gives rise to hatred. The loser in the game has sorrow as his lot. But the \emph{arahant} is at peace, having given up victory and defeat. Isn't it enough for him to give up victory? Why is it said that he gives up both victory and defeat?

These two go as a pair. This recognition of a duality is a distinctive feature of this Dhamma. It gives, in a nutshell, the essence of this Dhamma. The idea of a duality is traceable to the vortex between consciousness and name-and-form. The same idea comes up in the following verse of the \emph{Attadaṇḍasutta} in the \emph{Sutta Nipāta}.

\begin{quote}
\emph{Yassa natthi `idaṁ me'ti}\\
\emph{`paresaṁ' vā pi kiñcanaṁ,}\\
\emph{mamattaṁ so asaṁvindaṁ,}\\
\emph{`natthi me'ti na socati.}\footnote{Sn 951, \emph{Attadaṇḍasutta}}

He who has nothing to call `this is mine',\\
Not even something to recognize as `theirs',\\
Finding no egoism within himself,\\
He grieves not, crying: `O! I have nothing!'
\end{quote}

So far in this series of sermons on Nibbāna, we were trying to explain what sort of a state Nibbāna is. We had to do so, because there has been quite a lot of confusion and controversy regarding Nibbāna as the aim of the spiritual endeavour in Buddhism. The situation today is no better. Many of those who aspire to Nibbāna today, aim not at the cessation of existence, but at some form of quasi existence as a surrogate Nibbāna.

If the aiming is wrong, will the arrow reach the target? Our attempt so far has been to clarify and highlight this target, which we call Nibbāna. If we have been successful in this attempt, the task before us now is to adumbrate the salient features of the path of practice.

Up to now, we have been administering a purgative, to dispel some deep-rooted wrong notions. If it has worked, it is time now for the elixir. In the fore-going sermons, we had occasion to bring up a number of key terms in the suttas, which have been more or less relegated into the limbo and rarely come up in serious Dhamma discussions.

We have highlighted such key terms as \emph{suññatā, dvayatā, tathatā, atammayatā, idappaccayatā, papañca,} and \emph{maññanā}. We have also discussed some aspects of their significance. But in doing so, our main concern was the dispelling of some misconceptions about Nibbāna as the goal.

The aim of this series of sermons, however, is not the satisfying of some curiosity at an academic level. It is to pave the way for an attainment of this goal, by rediscovering the intrinsic qualities of this Dhamma that is well proclaimed, \emph{svākkhāto}, visible here and now, \emph{sandiṭṭhiko}, timeless, \emph{akāliko}, inviting one to come and see, \emph{ehipassiko}, leading one onwards, \emph{opanayiko}, and realizable personally by the wise, \emph{paccattaṁ veditabbo viññūhi.} So the few sermons that will follow, might well be an elixir to the minds of those meditators striving hard day and night to realize Nibbāna.

\begin{quote}
\emph{Lobho, doso ca moho ca,}\\
\emph{purisaṁ pāpacetasaṁ,}\\
\emph{hiṁsanti attasambhūtā,}\\
\emph{tacasāraṁ va samphalaṁ.}\footnote{SN I 70, \emph{Purisasutta}}

Greed and hate and delusion too,\\
Sprung from within work harm on him\\
Of evil wit, as does its fruit\\
On the reed for which the bark is pith.
\end{quote}

The main idea behind this verse is that the three defilements -- greed, hatred and delusion -- spring up from within, that they are \emph{attasambhūta}, self-begotten. What is the provocation for such a statement?

It is generally believed that greed, hatred and delusion originate from external signs. The magic show and the chess game have shown us how signs become significant. They become significant because they find something within that they can signify and symbolize.

Now this is where the question of radical reflection, \emph{yoniso manasikāra}, comes in. What the Buddha brings up in this particular context, is the relevance of that radical reflection as a pre-requisite for treading the path.

The worldling thinks that greed, hatred and delusion arise due to external signs. The Buddha points out that they arise from within an individual and destroy him as in the case of the fruit of a reed or bamboo. It is this same question of radical reflection that came up earlier in the course of our discussion of the \emph{Madhupiṇḍikasutta}, based on the following deep and winding statement.

\begin{quote}
\emph{Cakkhuñc'āvuso paṭicca rūpe ca uppajjati cakkhuviññāṇaṁ, tiṇṇaṁ saṅgati phasso, phassapaccayā vedanā, yaṁ vedeti taṁ sañjānāti, yaṁ sañjānāti taṁ vitakketi, yaṁ vitakketi taṁ papañceti, yaṁ papañceti tatonidānaṁ purisaṁ papañcasaññāsaṅkhā samudācaranti atītānāgatapaccuppannesu cakkhuviññeyyesu rūpesu.}\footnote{MN I 111, \emph{Madhupiṇḍikasutta}, see \emph{Sermon 11}}

Dependent on eye and forms, friend, arises eye-consciousness; the concurrence of the three is contact; because of contact, feeling; what one feels, one perceives; what one perceives, one reasons about; what one reasons about, one proliferates; what one proliferates, owing to that, reckonings born of prolific perceptions overwhelm him in regard to forms cognizable by the eye relating to the past, the future and the present.
\end{quote}

Eye-consciousness, for instance, arises depending on eye and forms. The concurrence of these three is called contact. Depending on this contact arises feeling. What one feels, one perceives, and what one perceives, one reasons about. The reasoning about leads to a proliferation that brings about an obsession, as a result of which the reckonings born of prolific perceptions overwhelm the individual concerned.

The process is somewhat similar to the destruction of the reed by its own fruit. It shows how non-radical reflection comes about. Radical reflection is undermined when proliferation takes over. The true source, the matrix, is ignored, with the result an obsession follows, tantamount to an entanglement within and without, \emph{anto jaṭā bahi jaṭā}.\footnote{\href{https://suttacentral.net/sn1.23/pli/ms}{SN 1.23 / S I 13}, \emph{Jaṭāsutta}, see \emph{Sermon 1}}

The paramount importance of radical reflection is revealed by the \emph{Sūcilomasutta} found in the \emph{Sutta Nipāta}, as well as in the \emph{Sagāthakavagga} of the \emph{Saṁyutta Nikāya}. The \emph{yakkha} Sūciloma poses some questions to the Buddha in the following verse.

\begin{quote}
\emph{Rāgo ca doso ca kutonidānā,}\\
\emph{aratī ratī lomahaṁso kutojā,}\\
\emph{kuto samuṭṭhāya manovitakkā,}\\
\emph{kumārakā vaṁkam iv' ossajanti?}\footnote{Sn 270, \emph{Sūcilomasutta}, see also SN I 207}

Lust and hate, whence caused are they,\\
Whence spring dislike, delight and terror,\\
Whence arising do thoughts disperse,\\
Like children leaving their mother's lap?
\end{quote}

The Buddha answers those questions in three verses.

\begin{quote}
\emph{Rāgo ca doso ca itonidānā,}\\
\emph{aratī ratī lomahaṁso itojā,}\\
\emph{ito samuṭṭhāya manovitakkā,}\\
\emph{kumārakā vaṁkam iv' ossajanti.}

\emph{Snehajā attasambhūtā}\\
\emph{nigrodhasseva khandhajā,}\\
\emph{puthū visattā kāmesu}\\
\emph{māluvā va vitatā vane.}

\emph{Ye naṁ pajānanti yatonidānaṁ,}\\
\emph{te naṁ vinodenti, suṇohi yakkha,}\\
\emph{te duttaram ogham imaṁ taranti,}\\
\emph{atiṇṇapubbaṁ apunabbhavāya.}

It is hence that lust and hate are caused,\\
Hence spring dislike, delight and terror,\\
Arising hence do thoughts disperse,\\
Like children leaving their mother's lap.

Moisture-born and self-begotten,\\
Like the banyan's trunk-born runners\\
They cleave to diverse objects of sense,\\
Like the \emph{māluvā} creeper entwining the forest.

And they that know wherefrom it springs,\\
They dispel it, listen, O! Yakkha.\\
They cross this flood so hard to cross,\\
Never crossed before, to become no more.
\end{quote}

\clearpage

In explaining these verses, we are forced to depart from the commentarial trend. The point of controversy is the phrase \emph{kumārakā dhaṅkam iv' ossajanti}, recognized by the commentary as the last line of Sūciloma's verse.

We adopted the variant reading \emph{kumārakā vaṁkam iv' ossajanti}, found in some editions. Let us first try to understand how the commentary interprets this verse.

Its interpretation centres around the word \emph{dhaṅka}, which means a crow. In order to explain how thoughts disperse, it alludes to a game among village lads, in which they tie the leg of a crow with a long string and let it fly away so that it is forced to come back and fall at their feet.\footnote{Spk I 304} The commentary rather arbitrarily breaks up the compound term \emph{manovitakkā} in trying to explain that evil thoughts, \emph{vitakkā}, distract the mind, \emph{mano}.

If the variant reading \emph{kumārakā vaṁkam iv' ossajanti} is adopted, the element \emph{v} in \emph{vaṁkam iv' ossajanti} could be taken as a hiatus filler, \emph{āgama}, and then we have the meaningful phrase \emph{kumārakā aṁkam iv' ossajanti}, ``even as children leave the lap''.

Lust and hate, delight and terror, spring from within. Even so are thoughts in the mind, \emph{manovitakkā}. We take it as one word, whereas the commentary breaks it up into two words.

It is queer to find the same commentator analyzing this compound differently in another context. In explaining the term \emph{manovitakkā} occurring in the \emph{Kummasutta} of the \emph{Devatā Saṁyutta} in the \emph{Saṁyutta Nikāya}, the commentary says \emph{`manovitakke'ti manamhi uppannavitakke}: ``\emph{manovitakka}, this means thoughts arisen in the mind''.\footnote{Spk I 36, commenting on SN I 7, \emph{Kummasutta}}

The commentator was forced to contradict himself in the present context, because he wanted to justify the awkward simile of the game he himself had introduced. The simile of leaving the mother's lap, on the other hand, would make more sense, particularly in the light of the second verse uttered by the Buddha.

\clearpage

\begin{quote}
\emph{Snehajā attasambhūtā}\\
\emph{nigrodhasseva khandhajā,}\\
\emph{puthū visattā kāmesu}\\
\emph{māluvā va vitatā vane.}
\end{quote}

The verse enshrines a deep idea. \emph{Sneha} is a word which has such meanings as `moisture' and `affection'. In the simile of the banyan tree, the trunk-born runners are born of moisture. They are self-begotten.

Thoughts in the mind cleave to diverse external objects. Just as the runners of a banyan tree, once they take root would even conceal the main trunk, which gave them birth, so the thoughts in the mind, attached to external objects of sense, would conceal their true source and origin.

Non-radical reflection could easily come in. The runners are moisture-born and self-begotten from the point of view of the original banyan tree. The main trunk gets overshadowed by its own runners.

The next simile has similar connotations. The \emph{māluvā} creeper is a plant parasite. When some bird drops a seed of a \emph{māluvā} creeper into a fork of a tree, after some time a creeper comes up. As time goes on, it overspreads the tree, which gave it nourishment.

Both similes illustrate the nature of non radical reflection. Conceptual proliferation obscures the true source, namely the psychological mainsprings of defilements. Our interpretation of children leaving the mother's lap would be meaningful in the context of the two terms \emph{snehajā}, `born of affection', and \emph{attasambhūtā}, `self-begotten'. There is possibly a pun on the word \emph{sneha}. Children are affection-born and self-begotten, from a mother's point of view.

The basic theme running through these verses is the origin and source of things. The commentator's simile of the crow could ill afford to accommodate all the nuances of these pregnant terms. It distracts one from the main theme of these verses. The questions asked concern the origin, \emph{kuto nidānā, kutojā, kuto samuṭṭhāya}, and the answers are in full accord: \emph{ito nidānā, itojā, ito samuṭṭhāya}.

With reference to thoughts in the mind, the term \emph{snehajā} could even mean `born of craving', and \emph{attasambhūtā} conveys their origination from within. As in the case of the runners of the banyan tree and the \emph{māluvā} creeper, those defiling thoughts, arisen from within, once they get attached to sense objects outside, obscure their true source. The result is the pursuit of a mirage, spurred on by non-radical reflection.

The last verse is of immense importance. It says: But those who know from where all these mental states arise, are able to dispel them. It is they who successfully cross this flood, so hard to cross, and are freed from re-becoming.
