\chapter{Sermon 31}

\NibbanaOpeningQuote

With the permission of the assembly of the venerable meditative monks. This is the thirty-first sermon in the series of sermons on Nibbāna.

In our attempt to understand some subtle characteristics of the middle path leading to Nibbāna in our last sermon, we found some discourses like \emph{Saḷāyatanavibhangasutta, Oghataraṇasutta, Vitakkasanthānasutta, Māgandiyasutta, Rathavinītasutta} and \emph{Alagaddūpamasutta} particularly helpful. It became clear that the twin principle of pragmatism and relativity, underlying the norm of dependent arising, could be gleaned to a great extent from those discourses.

We also found that the course of practice leading to Nibbāna is not an accumulation or amassing, but a gradual process of attenuation or effacement, tending towards a realization of voidness, free from notions of `I' and `mine'.

It is for the purpose of emphasizing the twin principles of pragmatism and relativity that the Buddha compared the Dhamma to a raft in the \emph{Alagaddūpamasutta} of the \emph{Majjhima Nikāya}. In this series of sermons we made allusions to this simile in brief on several occasions, but let us now try to examine this simile in more detail. In order to present the parable of the raft, the Buddha addressed the monks and made the following declaration:

\begin{quote}
\emph{Kullūpamaṁ vo, bhikkhave, dhammaṁ desissāmi nittharaṇatthāya no gahaṇatthāya.}\footnote{\href{https://suttacentral.net/mn22/pli/ms}{MN 22 / M I 134}, \emph{Alagaddūpamasutta}}

``Monks, I shall preach to you the Dhamma comparable to a raft for crossing over and not for grasping.''
\end{quote}

With this introductory declaration, he goes on to relate the parable of the raft.

\begin{quote}
``Monks, suppose a man in the course of a long journey, saw a great expanse of water whose near shore was dangerous and fearful and whose further shore was safe and free from fear. But there was no ferry boat or bridge going to the far shore. Then he thought:

`There is this great expanse of water whose near shore is dangerous and fearful and whose further shore is safe and free from fear. But there is no ferry boat or bridge going to the far shore. Suppose I collect grass, sticks, branches and leaves and bind them together into a raft, and supported by the raft and making an effort with my hands and feet I were to get safely across to the far shore.'

And then the man collected grass, sticks, branches and leaves and bound them together into a raft, and supported by the raft and making an effort with his hands and feet he got safely across to the far shore. Then, when he got safely across and had arrived at the far shore he might think thus:

`This raft has been very helpful to me, supported by it and making an effort with my hands and feet I got safely across to the far shore. Suppose I were to hoist it on my head or load it on my shoulder and then go wherever I want.'

Now, monks, what do you think, by doing so would that man be doing what should be done with that raft?''

``No, Venerable Sir.''

``By doing what would that man be doing what should be done with that raft? Here, monks, when that man got across and had arrived at the far shore, he might think thus:

`This raft has been very helpful to me, since supported by it and making an effort with my hands and feet I got safely across to the far shore. Suppose I were to haul it on dry land or set it adrift in the water and then go wherever I want.'

Now it is by so doing that that man would be doing what should be done with the raft. Even so, monks, I have shown you how the Dhamma is similar to a raft, being for the purpose of crossing over, not for the purpose of grasping.''
\end{quote}

And the Buddha concludes with the significant statement:

\begin{quote}
\emph{Kullūpamaṁ vo, bhikkhave, ājānantehi dhammā pi vo pahātabbā, pageva adhammā.}

``Monks, when you know the Dhamma to be similar to a raft, you should abandon even good states, how much more so bad states.''
\end{quote}

So it seems, this raft simile has a very deep meaning. The building of the raft by the person wishing to cross symbolizes the pragmatic and relative values we highlighted in connection with the path of practice leading to Nibbāna.

The raft improvised with self effort is not for grasping or carrying on one's shoulder. As we have already pointed out with reference to such discourses like \emph{Saḷāyatanavibhaṅgasutta}, apart from the purpose of crossing, there is nothing worth holding on to or grasping. Why so? Because the aim of this holy life or this path of practice is non-grasping instead of grasping; non-identification, \emph{atammayatā}, instead of identification, \emph{tammayatā}; assetlessness, \emph{nirupadhi}, instead of assets, \emph{upadhi}.

The importance attached to this simile is so much that the Buddha reminds the monks of it in the \emph{Mahātaṇhāsaṅkhayasutta} also, with the following allusion:

\begin{quote}
\emph{Imaṁ ce tumhe, bhikkhave, diṭṭhiṁ evaṁ parisuddhaṁ evaṁ pariyodātaṁ allīyetha kelāyetha dhanāyetha mamāyetha, api nu tumhe, bhikkhave, kullūpamaṁ dhammaṁ desitaṁ ājāneyyatha nittharaṇatthāya no gahaṇatthāya?}

\emph{No h'etaṁ, bhante!}

\emph{Imaṁ ce tumhe, bhikkhave, diṭṭhiṁ evaṁ parisuddhaṁ evaṁ pariyodātaṁ na allīyetha na kelāyetha na dhanāyetha na mamāyetha, api nu tumhe, bhikkhave, kullūpamaṁ dhammaṁ desitaṁ ājāneyyatha nittharaṇatthāya no gahaṇatthāya?}

\emph{Evaṁ, bhante.}\footnote{M I 260, \emph{Mahātaṇhāsaṅkhayasutta}}

``Monks, purified and cleansed as this view is, if you adhere to it, cherish it, treasure it and treat it as a possession, would you then understand the Dhamma that has been taught as similar to a raft being for the purpose of crossing over and not for the purpose of grasping?''

``No, Venerable Sir!''

``Monks, purified and cleansed as this view is, if you do not adhere to it, cherish it, treasure it and treat it as a possession, would you then understand the Dhamma that has been taught as similar to a raft being for the purpose of crossing over and not for the purpose of grasping?''

``Yes, Venerable Sir!''
\end{quote}

This is an illustration of the relative validity of the constituents of the path. Instead of an accumulation and an amassing, we have here a setting in motion of a sequence of psychological states mutually interconnected according to the law of relativity. As in the simile of the relay of chariots, what we have here is a progression by relative dependence.

In this sequential progression, we see an illustration of the quality of leading onward, \emph{opanayika}, characteristic of this Dhamma.

The term \emph{opanayika} has been variously interpreted, but we get a clue to its correct meaning in the \emph{Udāyisutta} of the \emph{Bojjhaṅgasaṁyutta} in the \emph{Saṁyutta Nikāya}. Venerable Udāyi declares his attainment of the supramundane path in these words:

\begin{quote}
\emph{Dhammo ca me, bhante, abhisamito, maggo ca me paṭiladdho, yo me bhāvito bahulīkato tathā tathā viharantaṁ tathattāya upanessati.}\footnote{S V 90, \emph{Udāyisutta}}

The Dhamma has been well understood by me, Venerable Sir, and that path has been obtained which, when developed and cultivated, will lead me onwards to such states as I go on dwelling in the appropriate way.
\end{quote}

The implication is that the Dhamma has the intrinsic quality of leading onward whoever is dwelling according to it so that he attains states of distinction independent of another's intervention.

A clearer illustration of this intrinsic quality can be found in the \emph{Cetanākaraṇīyasutta} among the Tens of the \emph{Aṅguttara Nikāya}. In that discourse, the Buddha describes how a long sequence of mental states is interconnected in a subtle way, according to the principle of relativity, leading onwards as far as final deliverance itself. The following section of that long discourse might suffice as an illustration of the mutual interconnection between the mental states in the list.

\begin{quote}
\emph{Sīlavato, bhikkhave, sīlasampannassa na cetanāya karaṇīyaṁ `avippaṭisāro me uppajjatū'ti; dhammatā esā, bhikkhave, yaṁ sīlavato sīlasampannassa avippaṭisāro uppajjati.}

\emph{Avippaṭisārissa, bhikkhave, na cetanāya karaṇīyaṁ `pāmojjaṁ me uppajjatū'ti; dhammatā esā, bhikkhave, yaṁ avippaṭisārissa pāmojjaṁ jāyati.}

\emph{Pamuditassa, bhikkhave, na cetanāya karaṇīyaṁ `pīti me uppajjatū'ti; dhammatā esā, bhikkhave, yaṁ pamuditassa pīti uppajjati.}\footnote{A V 2, \emph{Cetanākaraṇīyasutta}}

To one who is virtuous, monks, who is endowed with virtue, there is no need for an act of will like: ``let remorselessness arise in me''; it is in the nature of Dhamma, monks, that remorselessness arises in one who is virtuous, who is endowed with virtue.

To one who is free from remorse, monks, there is no need for an act of will like: ``let gladness arise in me''; it is in the nature of Dhamma, monks, that gladness arises in one who is free from remorse.

To one who is glad, monks, there is no need for an act of will like: ``let joy arise in me''; it is in the nature of Dhamma, monks, that joy arises in one who is glad.
\end{quote}

In this way, the Buddha outlines the entire course of training leading up to knowledge and vision of deliverance, interlacing a long line of mental states in such a way as to seem an almost effortless flow. The profound utterance, with which the Buddha sums up this discourse, is itself a tribute to the quality of leading onward, \emph{opanayika}, in this Dhamma.

\begin{quote}
\emph{Iti kho, bhikkhave, dhammā va dhamme abhisandenti, dhammā va dhamme paripūrenti apārā pāraṁ gamanāya.}

Thus, monks, mere phenomena flow into other phenomena, mere phenomena fulfil other phenomena in the process of going from the not beyond to the beyond.
\end{quote}

So, then, in the last analysis, it is only a question of phenomena. There is no `I' or `mine' involved. That push, that impetus leading to Nibbāna, it seems, is found ingrained in the Dhamma itself.

Not only the term \emph{opanayika}, all the six terms used to qualify the Dhamma are highly significant. They are also interconnected in meaning. That is why very often in explaining one term others are dragged in.

Sometimes the questioner is concerned only about the meaning of the term \emph{sandiṭṭhika}, but the Buddha presents to him all the six qualities of the Dhamma.\footnote{S IV 41, \emph{Upavāṇasandiṭṭhikasutta}} In discourses like \emph{Mahātaṇhāsaṅkhayasutta} the emphasis is on the term \emph{opanayika}, but there, too, the Buddha brings in all the six terms, because they are associated in sense.

Let us now examine how these six epithets are associated in sense. The usual explanation of \emph{svākkhata}, `well preached', is that the Dhamma has been preached by the Buddha properly intoned with perfect symmetry as to the letter and to the spirit, excellent in the beginning, excellent in the middle and excellent in the end. But the true meaning of \emph{svākkhata} emerges when examined from the point of view of practice.

The quality of being visible here and now, \emph{sandiṭṭhika}, that is not found in an ill-preached doctrine, \emph{durakkhāta dhamma}, is to be found in this well-preached Dhamma. Whereas an ill-preached doctrine only promises a goal attainable in the next world, the well-preached Dhamma points to a goal attainable in this world itself. Therefore we have to understand the full import of the epithet \emph{svakkhāta} in relation to the next quality, \emph{sandiṭṭhika}, visible here and now.

We have already dealt with this quality to some extent in connection with an episode about General Sīha in an earlier sermon.\footnote{A III 39, \emph{Sīhasenāpattisutta}; see \emph{Sermon 19}} Briefly stated, the meaning of the term \emph{sandiṭṭhika} is ``visible here and now, in this very life'', as far as the results are concerned. The same idea is conveyed by the expression \emph{diṭṭheva dhamme} often cited with reference to Nibbāna in the standard phrase,

\begin{quote}
\emph{diṭṭheva dhamme sayam abhiññā sacchikatvā},\footnote{E.g.~M I 76, \emph{Mahāsīhanādasutta}}

having realized by one's own higher knowledge in this very life.
\end{quote}

Whereas \emph{samparāyika} stands for what comes after death, in another life, \emph{sandiṭṭhika} points to the attainability of results in this very life, here and now.

The term \emph{sandiṭṭhika} can be related to the next epithet \emph{akālika}. Since the results are attainable here and now, it does not involve an interval in time. It is, in other words, timeless, \emph{akālika}.

In our earlier sermons we brought in, as an illustration for this involvement with time, the period of suspense after an examination, these days, awaiting results. Nibbāna-examination, on the other hand, yields results then and there and produces the certificate immediately. So we see the quality ``visible here and now'' implicating a timelessness.

Unfortunately, however, the term \emph{akālika} also suffered by much commentarial jargon. Meanings totally foreign to the original sense came to be tagged on, so much so that it was taken to mean `true for all times' or `eternal'.

The \emph{Samiddhisutta} in the \emph{Devatāsaṁyutta} of the \emph{Saṁyutta Nikāya} clarifies for us the original meaning of the term \emph{akālika}. One day, Venerable Samiddhi had a bath at the hot springs in Tapodārāma and was drying his body outside in the sun. A deity seeing his handsome body gave him an advice contrary to the spirit of the Dhamma.

\begin{quote}
\emph{Bhuñja, bhikkhu, mānusake kāme, mā sandiṭṭhikaṁ hitvā kālikaṁ anudhāvi}.\footnote{S I 9, \emph{Samiddhisutta}}

Enjoy, monk, human sensual pleasures, do not abandon what is visible here and now and run after what takes time!
\end{quote}

Venerable Samiddhi met the challenge with the following explanatory reply:

\begin{quote}
\emph{Na kvhāhaṁ, āvuso, sandiṭṭhikaṁ hitvā kālikaṁ anudhāvāmi. Kālikañca khvāhaṁ, āvuso, hitvā sandiṭṭhikaṁ anudhāvāmi. Kālikā hi, āvuso, kāmā vuttā bhagavatā bahudukkhā bahupāyāsā, ādīnavo ettha bhiyyo. Sandiṭṭhiko ayaṁ dhammo akāliko ehipassiko opanayyiko paccattaṁ veditabbo viññūhi}.

It is not the case, friend, that I abandon what is visible here and now in order to run after what involves time. On the contrary, I am abandoning what involves time to run after what is visible here and now. For the Fortunate One has said that sensual pleasures are time involving, fraught with much suffering, much despair, and that more dangers lurk in them.

Visible here and now is this Dhamma, timeless, inviting one to come and see, leading one onwards, to be realized personally by the wise.
\end{quote}

This explanation makes it clear that the two terms \emph{sandiṭṭhika} and \emph{akālika} are allied in meaning. That is why \emph{sandiṭṭhika} is contrasted with \emph{kālika} in the above dialogue. What comes after death is \emph{kālika}, involving time. It may come or may not come, one cannot be certain about it. But of what is visible here and now, in this very life, one can be certain. There is no time gap. It is timeless.

The epithet \emph{akālika} is implicitly connected with the next epithet, \emph{ehipassika}. If the result can be seen here and now, without involving time, there is good reason for the challenge: `Come and see!' If the result can be seen only in the next world, all one can say is: `Go and see!'

As a matter of fact, it is not the Buddha who says: `Come and see!', it is the Dhamma itself that makes this challenge. That is why the term \emph{ehipassika} is regarded as an epithet of the Dhamma. Dhamma itself invites the wise to come and see.

Those who took up the challenge right in earnest have proved for themselves the realizable nature of the Dhamma, which is the justification for the last epithet, \emph{paccattaṁ veditabbo viññūhi}, ``to be experienced by the wise each one by oneself''.

The inviting nature of the Dhamma leads to personal experience and that highlights the \emph{opanayika} quality of leading onwards. True to the statement \emph{tathā tathā viharantaṁ tathattāya upanessati},\footnote{S V 90, \emph{Udāyisutta}} the Dhamma leads him onwards to appropriate states as he lives according to it.

Sometimes the Buddha sums up the entire body of Dhamma he has preached in terms of the thirty-seven participative factors of enlightenment. Particularly in the \emph{Mahāparinibbānasutta} we find him addressing the monks in the following memorable words:

\begin{quote}
\emph{Tasmātiha, bhikkhave, ye te mayā dhammā abhiññā desitā, te vo sādhukaṁ uggahetvā āsevitabbā bhāvetabbā bahulīkātabbā, yathayidaṁ brahmacariyaṁ addhaniyaṁ assa ciraṭṭhitikaṁ, tadassa bahujanahitāya bahujanasukhāya lokānukampāya atthāya hitāya sukhāya devamanussānaṁ.}

\emph{Katame ca te, bhikkhave, dhammā mayā abhiññā desitā ye vo sādhukaṁ uggahetvā āsevitabbā bhāvetabbā bahulīkātabbā, yathayidaṁ brahmacariyaṁ addhaniyaṁ assa ciraṭṭhitikaṁ, tadassa bahujanahitāya bahujanasukhāya lokānukampāya atthāya hitāya sukhāya devamanussānaṁ?}

\emph{Seyyathidaṁ cattāro satipaṭṭhāna cattāro sammappadhānā cattāro iddhipādā pañcindriyāni pañca balāni satta bojjhaṅgā ariyo aṭṭhaṅgiko maggo.}\footnote{D II 119, \emph{Mahāparinibbānasutta}}

Therefore, monks, whatever \emph{dhammas} I have preached with higher knowledge, you should cultivate, develop and practice thoroughly, so that this holy life would last long and endure for a long time, thereby conducing to the wellbeing and happiness of many, out of compassion for the world, for the benefit, the wellbeing and the happiness of gods and men.

And what, monks, are those \emph{dhammas} I have preached with higher knowledge that you should cultivate, develop and practice thoroughly, so that this holy life would last long and endure for a long time, thereby conducing to the wellbeing and happiness of many, out of compassion for the world, for the benefit, the wellbeing and the happiness of gods and men?

They are the four foundations of mindfulness, the four right endeavours, the four bases for success, the five faculties, the five powers, the seven factors of enlightenment, and the noble eightfold path.
\end{quote}

This group of \emph{dhammas,} collectively known as the thirty-seven participative factors of enlightenment illustrates the quality of leading onwards according to the twin principles of relativity and pragmatism.

It is customary in the present age to define the Dhamma from an academic point of view as constituting a set of canonical texts, but here in this context in the \emph{Mahāparinibbānasutta}, at such a crucial juncture as the final passing away, we find the Buddha defining the Dhamma from a practical point of view, laying emphasis on the practice. It is as if the Buddha is entrusting to the monks a tool-kit before his departure.

The thirty-seven participative factors of enlightenment are comparable to a tool-kit, or rather, an assemblage of seven tool-kits. Each of these seven is well arranged with an inner consistency. Let us now examine them.

First comes the four foundations of mindfulness. This group of \emph{dhammas} deserves pride of place due to its fundamental importance. The term \emph{satipaṭṭhāna} has been variously interpreted by scholars, some with reference to the term \emph{paṭṭhāna} and others connecting it with \emph{upaṭṭhāna}.

It seems more natural to associate it with the word \emph{paṭṭhāna}, `foundation', as the basis for the practice. \emph{Upaṭṭhita sati} is a term for one who has mastered mindfulness, based on the four foundations, as for instance in the aphorism:

\begin{quote}
\emph{upaṭṭhitasatissāyaṁ dhammo, nāyaṁ dhammo muṭṭhasatissa},\footnote{\href{https://suttacentral.net/dn34/pli/ms}{DN 34 / D III 287}, \emph{Dasuttarasutta}}

this Dhamma is for one who is attended by mindfulness, not for one who has lost it.
\end{quote}

The four foundations themselves exhibit an orderly arrangement. The four are termed:

\begin{enumerate}
\def\labelenumi{\arabic{enumi}.}
\tightlist
\item
  \emph{kāyānupassanā}, contemplation on the body,
\item
  \emph{vedanānupassanā}, contemplation on feelings,
\item
  \emph{cittānupassanā}, contemplation on the mind, and
\item
  \emph{dhammānupassanā}, contemplation on mind-objects.
\end{enumerate}

So here we have a basis for the exercise of mindfulness beginning with a gross object, gradually leading on to subtler objects. It is easy enough to contemplate on the body. As one goes on setting up mindfulness on the body, one becomes more aware of feelings and makes them, too, the object of mindfulness. This gradual process need not be interpreted as so many cut and dried separate stages. There is a subtle imperceptible interconnection between these four foundations themselves.

To one who has practiced contemplation on the body, not only pleasant and unpleasant feelings, but also neither-painful-nor-pleasant feeling, imperceptible to ordinary people, becomes an object for mindfulness. So also are the subtler distinctions between worldly, \emph{sāmisa}, and unworldly, \emph{nirāmisa}, feelings.

As one progresses to \emph{cittānupassanā}, contemplation on the mind, one becomes aware of the colour-light system of the mind in response to feelings, the alternations between a lustful mind, \emph{sarāgaṁ cittaṁ}, a hateful mind, \emph{sadosaṁ cittaṁ}, and a deluded mind, \emph{samohaṁ cittaṁ}, as well as their opposites.

Further on in his practice he becomes conversant with the wirings underlying this colour-light system of the mind and the know-how necessary for controlling it. With \emph{dhammānupassanā} he is gaining the skill in avoiding and overcoming negative mental states and encouraging and stabilizing positive mental states.

Let us now see whether there is any connection between the four foundations of mindfulness and the four right endeavours. For purposes of illustration we may take up the subsection on the hindrances, included under \emph{dhammānupassanā}, contemplation on mind-objects. There we read:

\begin{quote}
\emph{Yathā ca anuppannassa kāmacchandassa uppādo hoti, tañ ca pajānāti; yathā ca uppannassa kāmacchandassa pahānaṁ hoti tañ ca pajānāti.}\footnote{M I 60, \emph{Satipaṭṭhānasutta}}

And he also understands how there comes to be the arising of unarisen sensual desire, and how there comes to be the abandoning of arisen sensual desire.
\end{quote}

These two statements in the subsection on the hindrances could be related to the first two out of the four right endeavours:

\begin{quote}
\emph{Anuppannānaṁ pāpakānaṁ akusalānaṁ dhammānaṁ anuppādāya chandaṁ janeti vāyamati viriyaṁ ārabhati cittaṁ paggaṇhāti padahati; uppannānaṁ pāpakānaṁ akusalānaṁ dhammānaṁ pahānāya chandaṁ janeti vāyamati viriyaṁ ārabhati cittaṁ paggaṇhāti padahati.}\footnote{E.g.~D III, 221, \emph{Saṅgītisutta}}

For the non-arising of unarisen evil unskilful mental states he arouses a desire, makes an effort, puts forth energy, makes firm the mind and endeavours; for the abandoning of arisen evil unskilful mental states he arouses a desire, makes an effort, puts forth energy, makes firm the mind and endeavours.
\end{quote}

The understanding of the hindrances is the pre-condition for this right endeavour. What we have in the \emph{Satipaṭṭhānasutta} is a statement to the effect that one comprehends, \emph{pajānāti}, the way hindrances arise as well as the way they are abandoned. Right endeavour is already implicated. With mindfulness and full awareness one sees what is happening. But that is not all. Right endeavour has to step in.

Just as the first two right endeavours are relevant to the subsection on the hindrances, the next two right endeavours could be related to the following two statements in the subsection on the enlightenment factors in the \emph{Satipaṭṭhānasutta}.

\begin{quote}
\emph{Yathā ca anuppannassa satisambojjhaṅgassa uppādo hoti, tañ ca pajānāti; yathā ca uppannassa satisambojjhaṅgassa bhāvanāpāripūrī hoti tañ ca pajānāti.}\footnote{M I 62, \emph{Satipaṭṭhānasutta}}

And he also understands how there comes to be the arising of the unarisen mindfulness enlightenment factor, and how the arisen mindfulness enlightenment factor comes to fulfilment by development.
\end{quote}

One can compare these two aspects of the \emph{dhammānupassanā} section in the \emph{Satipaṭṭhānasutta} with the two right endeavours on the positive side.

\begin{quote}
\emph{Anuppannānaṁ kusalānaṁ dhammānaṁ uppādāya chandaṁ janeti vāyamati viriyaṁ ārabhati cittaṁ paggaṇhāti padahati; uppannānaṁ kusalānaṁ dhammānaṁ ṭhitiyā asammosāya bhiyyobhāvāya vepullāya bhāvanāya pāripūriyā chandaṁ janeti vāyamati viriyaṁ ārabhati cittaṁ paggaṇhāti padahati.}\footnote{E.g.~D III, 221, \emph{Saṅgītisutta}}

For the arising of unarisen skilful mental states he arouses a desire, makes an effort, puts forth energy, makes firm the mind and endeavours; for the stability, non-remiss, increase, amplitude and fulfilment by development of arisen skilful mental states he arouses a desire, makes an effort, puts forth energy, makes firm the mind and endeavours.
\end{quote}

This is the right endeavour regarding skilful mental states. Why we refer to this aspect in particular is that there is at present a tendency among those who recommend \emph{satipaṭṭhāna} meditation to overemphasize the role of attention. They seem to assert that bare attention or noticing is all that is needed. The reason for such an attitude is probably the attempt to specialize in \emph{satipaṭṭhāna} in isolation, without reference to the rest of the thirty-seven participative factors of enlightenment.

These seven tool-kits are interconnected. From the \emph{satipaṭṭhāna} tool-kit, the \emph{sammappadhāna} tool-kit comes out as a matter of course. That is why bare attention is not the be all and end all of it.

Proper attention is actually the basis for right endeavour. Even when a machine is out of order, there is a need for tightening or loosening somewhere. But first of all one has to mindfully scan or scrutinize it. That is why there is no explicit reference to effort in the \emph{Satipaṭṭhānasutta}. But based on that scrutiny, the four right endeavours play their role in regard to unskilful and skilful mental states. So we see the close relationship between the four foundations of mindfulness and the four right endeavours.

It is also interesting to examine the relationship between the four right endeavours and the four paths to success. We have already quoted a phrase that is commonly used with reference to all the four right endeavours, namely:

\begin{quote}
\emph{chandaṁ janeti vāyamati viriyaṁ ārabhati cittaṁ paggaṇhāti padahati},

arouses a desire, makes an effort, puts forth energy, makes firm the mind and endeavours.
\end{quote}

Here we have a string of terms suggestive of striving, systematically arranged in an ascending order.

\emph{Chandaṁ janeti} refers to the interest or the desire to act.

\emph{Vāyamati} suggests effort or exercise.

\emph{Viriyaṁ ārabhati} has to do with the initial application of energy.

\emph{Cittaṁ paggaṇhāti} stands for that firmness of resolve or grit.

\emph{Padahati} signifies the final all out effort or endeavour.

These terms more or less delineate various stages in a progressive effort. One who practices the four right endeavours in course of time specializes in one or the other of the four bases for success, \emph{iddhipāda}. That is why the four bases for success are traceable to the four right endeavours.

To illustrate the connection between the right endeavours and the four bases for success, let us take up a simile. Suppose there is a rock which we want to get out of our way. We wish to topple it over. Since our wishing it away is not enough, we put some kind of lever underneath it and see whether it responds to our wish. Even if the rock is unusually obstinate, we at least give our shoulders an exercise, \emph{vāyamati}, in preparation for the effort.

Once we are ready, we heave slowly slowly, \emph{viriyaṁ ārabhati}. But then it looks as if the rock is precariously balanced, threatening to roll back. So we grit our teeth and make a firm resolve, \emph{cittaṁ paggaṇhāti}.

Now comes the last decisive spurt. With one deep breath, well aware that it could be our last if the rock had its own way, we push it away with all our might. It is this last all out endeavour that in the highest sense is called \emph{sammappadhāna} or right endeavour.

In the context of the right endeavour for enlightenment it is called \emph{caturaṅgasamannāgata viriya} ``effort accompanied by four factors'',\footnote{E.g.~Ps III 194} which is worded as follows:

\begin{quote}
\emph{Kāmaṁ taco ca nahāru ca aṭṭhi ca avasissatu, sarīre upasussatu maṁsalohitaṁ, yaṁ taṁ purisathāmena purisaviriyena purisaparakkamena pattabbaṁ na taṁ apāpuṇitvā viriyassa saṇṭhānaṁ bhavissati}.\footnote{M I 481, \emph{Kīṭāgirisutta}}

Verily let my skin, sinews and bones remain, and let the flesh and blood dry up in my body, but I will not relax my energy so long as I have not attained what can be attained by manly strength, by manly energy, by manly exertion.
\end{quote}

Though as an illustration we took an ordinary worldly object, a rock, one can substitute for it the gigantic mass of suffering to make it meaningful in the context of the Dhamma.

It is the formula for the toppling of this mass of suffering that is enshrined in the phrase \emph{chandaṁ janeti vāyamati viriyaṁ ārabhati cittaṁ paggaṇhāti padahati}, ``arouses a desire, makes an effort, puts forth energy, makes firm the mind and endeavours''.

The four bases for success, \emph{iddhipāda}, namely \emph{chanda}, `desire'; \emph{viriya}, `energy'; \emph{citta}, `mind'; and \emph{vīmaṁsā}, `investigation', to a great extent are already implicit in the above formula.

Clearly enough, \emph{chandaṁ janeti} represents \emph{chanda-iddhipāda}; \emph{vāyamati} and \emph{viriyaṁ ārabhati} together stand for \emph{viriya-iddhipāda}; while \emph{cittaṁ paggaṇhāti} stands for the power of determination implied by \emph{citta-iddhipāda}.

Apparently investigation, \emph{vīmaṁsā}, as an \emph{iddhipāda}, has no representative in the above formula. However, in the process of mindfully going over and over again through these stages in putting forth effort one becomes an adept in the art of handling a situation. In fact, \emph{vīmaṁsā}, or investigation, is \emph{paññā}, or wisdom, in disguise.

Even toppling a rock is not a simple task. One has to have the knowhow in order to accomplish it. So then, all the four bases for success emerge from the four right endeavours.

What is meant by \emph{iddhipāda}? Since the word \emph{iddhi} is associated with psychic power,\footnote{S V 276, \emph{Bhikkhusutta}; S V 286, \emph{Ānandasutta 1 and 2}; S V 287, \emph{Bhikkhusutta 1 and 2}} it is easy to mistake it as a base for psychic power.

But the basic sense of \emph{iddhi} is `success' or `proficiency'. For instance, \emph{samiddhi} means `prosperity'. It is perhaps more appropriate to render it as a `base for success', because for the attainment of Nibbāna, also, the development of the \emph{iddhipādas} is recommended. Going by the illustration given above, we may say in general that for all mundane and supramundane accomplishments, the four bases hold good to some extent or other.

In the \emph{Iddhipādasaṁyutta} these four bases for success are described as four ways to accomplish the task of attaining influx-free deliverance of the mind and deliverance by wisdom.\footnote{S V 266, \emph{Pubbesutta}}

With the experience gathered in the course of practising the fourfold right endeavour, one comes to know one's strongpoint, where one's forte lies. One might recognize \emph{chanda}, desire or interest, as one's strongpoint and give it first place. In the case of the bases for success, it is said that even one would do, as the others fall in line.

According to the commentaries, Venerable Raṭṭhapāla of the Buddha's time belonged to the \emph{chanda}-category, and Venerable Mogharāja had \emph{vīmaṁsa} as his forte, excelling in wisdom.\footnote{Sv II 642, which further mentions Venerable Soṇa as an example for energy and Venerable Sambhūta as an example for the category of the mind.}

Someone might get so interested in a particular course of action and get an intense desire and tell himself: ``Somehow I must do it.'' To that wish the others -- energy, determination and investigation -- become subservient.

Another might discover that his true personality emerges in the thick of striving. So he would make energy the base for success in his quest for Nibbāna.

Yet another has, as his strong point, a steel determination. The other three fall in line with it.

One who belongs to the wisdom category is never tired of investigation. He, even literally, leaves no stone unturned if he gets curious to see what lies underneath.

The fact that there is a normative tendency for \emph{iddhipādas} to work in unison comes to light in the description of \emph{iddhipāda} meditation in the \emph{Saṁyutta Nikāya}. For instance, in regard to \emph{chanda-iddhipāda}, we find the descriptive initial statement.

\begin{quote}
\emph{Idha, bhikkhave, bhikkhu chandasamādhipadhānasaṅkhāra-\\ samannāgataṁ iddhipādaṁ bhāveti},\footnote{E.g. SN V 255, \emph{Aparāsutta}}

herein, monks, a monk develops the base for success that is equipped with preparations for endeavour, arising from desire-concentration.
\end{quote}

Now what is this \emph{chandasamādhi} or `desire-concentration'? This strange type of concentration, not to be found in other contexts, is explained in the \emph{Chandasutta} itself as follows:

\begin{quote}
\emph{Chandaṁ ce, bhikkhave, bhikkhu nissāya labhati samādhiṁ labhati cittassa ekaggataṁ, ayaṁ vuccati chandasamādhi.}\footnote{S V 268, \emph{Chandasutta}}

If by relying on desire, monks, a monk gets concentration, gets one-pointedness of mind, this is called `desire-concentration'.
\end{quote}

Due to sheer interest or desire, a monk might reach a steady state of mind, like some sort of concentration. With that as his basis, he applies himself to the four right endeavours:

\begin{quote}
\emph{So anuppannānaṁ pāpakānaṁ akusalānaṁ dhammānaṁ anuppādāya chandaṁ janeti vāyamati viriyaṁ ārabhati cittaṁ paggaṇhāti padahati; uppannānaṁ pāpakānaṁ akusalānaṁ dhammānaṁ pahānāya chandaṁ janeti vāyamati viriyaṁ ārabhati cittaṁ paggaṇhāti padahati; anuppannānaṁ kusalānaṁ dhammānaṁ uppādāya chandaṁ janeti vāyamati viriyaṁ ārabhati cittaṁ paggaṇhāti padahati; uppannānaṁ kusalānaṁ dhammānaṁ ṭhitiyā asammosāya bhiyyobhāvāya vepullāya bhāvanāya pāripūriyā chandaṁ janeti vāyamati viriyaṁ ārabhati cittaṁ paggaṇhāti padahati.}

For the non-arising of unarisen evil unskilful mental states he arouses a desire, makes an effort, puts forth energy, makes firm the mind and endeavours; for the abandoning of arisen evil unskilful mental states he arouses a desire, makes an effort, puts forth energy, makes firm the mind and endeavours; for the arising of unarisen skilful mental states he arouses a desire, makes an effort, puts forth energy, makes firm the mind and endeavours; for the stability, non-remiss, increase, amplitude and fulfilment by development of arisen skilful mental states he arouses a desire, makes an effort, puts forth energy, makes firm the mind and endeavours.
\end{quote}

So here, again, the standard definition of the four right endeavours is given. The implication is that, once the base for success is ready, the four right endeavours take off from it. The four bases for success are therefore so many ways of specializing in various aspects of striving, with a view to wielding the four right endeavours all the more effectively. All the constituents of right endeavour harmoniously fall in line with the four bases for success.

Here, then, we have a concept of four types of concentrations as bases for right endeavour, \emph{chandasamādhi}, desire-concentration; \emph{viriyasamādhi}, energy-concentration; \emph{cittasamādhi}, mind-concentration; and \emph{vīmaṁsasamādhi}, investigation-concentration.

Now what is meant by \emph{padhānasaṅkhārā}, ``preparations for right endeavour''? It refers to the practice of the four right endeavours with one or the other base as a solid foundation. \emph{Padhāna} is endeavour or all out effort. \emph{Saṅkhārā} are those preparations directed towards it. Finally, the Buddha analyses the long compound to highlight its constituents.

\begin{quote}
\emph{Iti ayaṁ ca chando, ayaṁ ca chandasamādhi, ime ca padhānasaṅkhārā; ayaṁ vuccati, bhikkhave, chandasamādhipadhānasaṅkhāra-\\ samannāgato iddhipādo.}

Thus this desire, and this desire-concentration, and these preparations for endeavour; this is called the base for success that is equipped with preparations for endeavour, arising from desire-concentration.
\end{quote}

So we see how the four bases for success come out of the four right endeavours.

The relation between the four bases for success and the next tool-kit, the five faculties, \emph{pañcindriya}, may not be so clear. But there is an implicit connection which might need some explanation.

The five faculties here meant are faith, \emph{saddhā}; energy, \emph{viriya}; mindfulness, \emph{sati}; concentration, \emph{samādhi}; and wisdom, \emph{paññā}.

The four bases for success provide the proper environment for the arising of the five faculties. The term \emph{indriya}, faculty, has connotations of dominance and control. When one has specialized in the bases for success, it is possible to give predominance to certain mental states.

\emph{Saddhā}, or faith, is \emph{chanda}, desire or interest, in disguise. It is in one who has faith and confidence that desire and interest arise. With keen interest in skilful mental states one is impelled to take an initiative. The Buddha gives the following description of \emph{saddhindriya}:

\begin{quote}
\emph{Kattha ca, bhikkhave, saddhindriyaṁ daṭṭhabbaṁ? Catusu sotāpattiyaṅgesu.}\footnote{S V 196, \emph{Daṭṭhabbasutta}}

Where, monks, is the faculty of faith to be seen? In the four factors of stream-entry.
\end{quote}

The four factors of stream-entry, briefly stated, are as follows:

\begin{enumerate}
\def\labelenumi{\arabic{enumi}.}
\tightlist
\item
  \emph{buddhe aveccappasādena samannāgato,}\\
  He is endowed with confidence born of understanding in the Buddha;
\item
  \emph{dhamme aveccappasādena samannāgato,}\\
  he is endowed with confidence born of understanding in the Dhamma;
\item
  \emph{saṅghe aveccappasādena samannāgato,}\\
  he is endowed with confidence born of understanding in the Saṅgha;
\item
  \emph{ariyakantehi sīlehi samannāgato.}\footnote{S V 343, \emph{Rājasutta}}\\
  he is endowed with virtues dear to the Noble Ones.
\end{enumerate}

The stream-winner has a deep faith in the Buddha, in the Dhamma and in the Saṅgha that is born of understanding. His virtue is also of a higher order, since it is well based on that faith. So in the definition of the faculty of faith we have an echo of \emph{chanda-iddhipāda}.

It can also be inferred that \emph{viriyindriya}, the faculty of energy, also takes off from the energy base for success. We are told:

\begin{quote}
\emph{Kattha ca, bhikkhave, viriyindriyaṁ daṭṭhabbaṁ? Catusu sammapadhānesu.}\footnote{S V 196, \emph{Daṭṭhabbasutta}}

And where, monks, is the faculty of energy to be seen? In the four right endeavours.
\end{quote}

The faculty of energy is obviously nurtured by the four right endeavours and the four bases for success.

The antecedents of \emph{satindriya}, the faculty of mindfulness, may not be so obvious. But from the stage of \emph{satipaṭṭhāna} onwards it has played its silent role impartially throughout almost unseen. Here, too, it stands in the middle of the group of leaders without taking sides. In fact, its role is the preserving of the balance of power between those who are on either side, the balancing of faculties.

About the place of \emph{satindriya} the Buddha says:

\begin{quote}
\emph{Kattha ca, bhikkhave, satindriyaṁ daṭṭhabbaṁ? Catusu satipaṭṭhānesu.}

And where, monks, is the faculty of mindfulness to be seen? In the four foundations of mindfulness.
\end{quote}

It is the same four foundations, now reinforced by greater experience in vigilance.

Then comes the faculty of concentration, \emph{samādhindriya}. We already had a glimpse of it at the \emph{iddhipāda}-stage as \emph{chandasamādhi}, desire-concentration; \emph{viriyasamādhi}, energy-concentration; \emph{cittasamādhi}, mind-concentration; and \emph{vīmaṁsasamādhi}, investigation-concentration.

But it was only a steadiness or stability that serves as a make shift launching pad for concentrated effort. But here in this context \emph{samādhindriya} has a more refined sense. It is formally defined with reference to the four \emph{jhānic} attainments.

\begin{quote}
\emph{Kattha ca, bhikkhave, samādhindriyaṁ daṭṭhabbaṁ? Catusu jhānesu.}

And where, monks, is the faculty of concentration to be seen? In the four absorptions.
\end{quote}

Sometimes, rather exceptionally, another definition is also given:

\begin{quote}
\emph{Idha, bhikkhave, ariyasāvako vossaggārammanaṁ karitvā labhati samādhiṁ labhati cittass'ekaggataṁ.}\footnote{S V 197, \emph{Vibhaṅgasutta}}

Herein, monks, a noble disciple gains concentration, gains one-pointedness of mind, having made release its object.
\end{quote}

However, it is by the development of the bases for success that concentration emerges as a full-fledged faculty.

Lastly, there is the faculty of wisdom, \emph{paññindriya}. Though it has some relation to \emph{vīmaṁsā} or investigation as a base for success, it is defined directly with reference to the four noble truths.

\begin{quote}
\emph{Kattha ca, bhikkhave, paññindriyaṁ daṭṭhabbaṁ? Catusu ariyasaccesu.}\footnote{S V 196, \emph{Daṭṭhabbasutta}}

And where, monks, is the faculty of wisdom to be seen? In the four noble truths.
\end{quote}

Nevertheless, in the four noble truths, too, we see some parallelism with the illustration for \emph{iddhipādas} we picked up. Suffering, its arising, its cessation and the path to its cessation is comparable to our reactions to our encounter with that stumbling block -- the rock.

In the context of insight, \emph{paññindriya} is defined in terms of the knowledge of rise and fall, \emph{udayatthagāmini paññā}.\footnote{S V 197, \emph{Vibhaṅgasutta}}

The sharpness of faculties may vary from person to person, according to their \emph{saṁsāric} background. The Buddha, who could see this difference between persons, \emph{puggalavemattatā}, was able to tame them easily.

As we have already mentioned, mindfulness is in the middle of this group of faculties. Being the main stay of the entire \emph{satipaṭṭhāna} practice, it renders a vigilant service in silence here too, as the arbiter in the struggle for power between the two factions on either side.

Now that they have the dominance, \emph{saddhā}, faith, and \emph{paññā}, wisdom, drag to either side, wishing to go their own way. Mindfulness has to strike a balance between them. Likewise \emph{viriya}, energy, and \emph{samādhi}, concentration, left to themselves tend to become extravagant and mindfulness has to caution them to be moderate. So in this tool-kit of faculties, \emph{sati} is the spanner for tightening or loosening, for relaxing or gripping.

Alternatively one can discern another orderly arrangement among these five faculties. In the \emph{Indriyasaṁyutta} Venerable Sāriputta extols the wonderful inner coherence between these faculties before the Buddha in the following words:

\begin{quote}
\emph{Saddhassa hi, bhante, ariyasāvakassa etaṁ pāṭikaṅkhaṁ yaṁ āraddhaviriyo viharissati akusalānaṁ dhammānaṁ pahānāya, kusalānaṁ dhammānaṁ upasampadāya, thāmava daḷhaparakkamo anikkhittadhuro kusalesu dhammesu. Yaṁ hissa, bhante, viriyaṁ tadassa viriyindriyaṁ.}

\emph{Saddhassa hi, bhante, ariyasāvakassa āraddhaviriyassa etaṁ pāṭikaṅkhaṁ yaṁ satimā bhavissati, paramena satinepakkena samannāgato, cirkatampi cirabhāsitampi saritā anussaritā. Yā hissa, bhante, sati tadassa satindriyaṁ.}

\emph{Saddhassa hi, bhante, ariyasāvakassa āraddhaviriyassa upaṭṭhitasatino etaṁ pāṭikaṅkhaṁ yaṁ vossaggārammaṇaṁ kartivā labhissati samādhiṁ labhissati cittassa ekaggataṁ. Yo hissa, bhante, samādhi tadassa samādhindriyaṁ.}

\emph{Saddhassa hi, bhante, ariyasāvakassa āraddhaviriyassa upaṭṭhitasatino samāhitacittassa etaṁ pāṭikaṅkhaṁ yaṁ evaṁ pajānissati:}

\emph{``Anamataggo kho saṁsāro, pubbā koṭi na paññāyati avijjānīvaraṇānaṁ sattānaṁ taṇhāsaṁyojanānaṁ sandhāvataṁ saṁsarataṁ. Avijjāya tveva tamokāyassa asesavirāganirodho santam etaṁ padaṁ paṇītam etaṁ padaṁ, yadidaṁ sabbasaṅkhārasamatho sabbūpadhipaṭinissaggo taṇhakkhayo virāgo nirodho nibbānaṁ.''}

\emph{Yā hissa, bhante, paññā tadassa paññindriyaṁ.}\footnote{S V 225, \emph{Āpaṇasutta}}

It could indeed be expected, Venerable Sir, of a noble disciple who has faith that he will dwell with energy put forth for the abandoning of unskilful states and the arising of skilful states, that he will be steady, resolute in exertion, not shirking the burden of fulfilling skilful states. That energy of his, Venerable Sir, is his faculty of energy.

It could indeed be expected, Venerable Sir, of that noble disciple who has faith and who has put forth energy that he will be mindful, endowed with supreme adeptness in mindfulness, one who remembers and recollects what was done and said even long ago. That mindfulness of his, Venerable Sir, is his faculty of mindfulness.

It could indeed be expected, Venerable Sir, of that noble disciple who has faith, who has put forth energy and who is attended by mindfulness that he will gain concentration, will gain one-pointedness of mind, having made release the object. That concentration of his, Venerable Sir, is his faculty of concentration.

It could indeed be expected, Venerable Sir, of that noble disciple who has faith, who has put forth energy, who is attended by mindfulness and whose mind is concentrated that he will understand thus:

``This \emph{saṁsāra} is without a conceivable beginning, a first point is not discernable of beings roaming and wandering, hindered by ignorance and fettered by craving. But the remainderless fading away and cessation of ignorance, the mass of darkness, this is the peaceful state, this is the excellent state, that is, the stilling of all preparations, the relinquishment of all assets, the destruction of craving, detachment, cessation, extinction.''

That wisdom of his, Venerable Sir, is his faculty of wisdom.
\end{quote}
