\chapter{Sermon 20}

\NibbanaOpeningQuote

With the permission of the Most Venerable Great Preceptor and the assembly of the venerable meditative monks. This is the twentieth sermon in the series of sermons on Nibbāna.

In our last sermon we described, as something of a marvel in the attainment of Nibbāna, the very possibility of realizing, in this very life, as \emph{diṭṭhadhammika}, one's after death state, which is \emph{samparāyika}. The phrase \emph{diṭṭheva dhamme sayaṁ abhiññā sacchikatvā}, ``having realized here and now by one's own higher knowledge'',\footnote{E.g. at \href{https://suttacentral.net/mn6/pli/ms}{MN 6 / M I 35}, \emph{Ākaṅkheyyasutta}} occurs so often in the discourses because the emancipated one ascertains his after death state as if by seeing with his own eyes.

\enlargethispage{\baselineskip}

\emph{Natthidāni punabbhavo}, `there is no re-becoming now',\footnote{E.g. at \href{https://suttacentral.net/mn26/pli/ms}{MN 26 / M I 167}, \emph{Ariyapariyesanasutta}} \emph{khīṇā jāti}, `extinct is birth',\footnote{E.g. at \href{https://suttacentral.net/mn4/pli/ms}{MN 4 / M I 23}, \emph{Bhayabheravasutta}} are some of the joyous utterances of the Buddha and the \emph{arahants}, which were inspired by the realization of the cessation of existence in this very life.

Through that realization itself, they experience a bliss devoid of feeling, which is called `the cooling off of feelings'. That is why Nibbāna as such is known as \emph{avedayita sukha}, a `bliss devoid of feeling'.\footnote{Ps III 115, \emph{aṭṭhakathā} on \href{https://suttacentral.net/mn59/pli/ms}{MN 59} \emph{Bahuvedanīyasutta}}

At the end of their lives, at the moment when death approaches, those emancipated ones, the \emph{arahants}, put forward their unshakeable deliverance of the mind, \emph{akuppā cetomivutti} (which remains unshaken even in the face of death), and become deathless well before their death, not after it.

On many an occasion the Buddha has spoken highly of this unshakeable deliverance of the mind, describing it as the supreme bliss, the supreme knowledge and the supreme freedom from death. For instance, among the Sixes of the \emph{Aṅguttara Nikāya}, we come across the following two verses:

\begin{quote}
\emph{Tassa sammā vimuttassa,}\\
\emph{ñāṇaṁ ce hoti tādino,}\\
\emph{`akuppā me vimuttī'ti,}\\
\emph{bhavasaṁyojanakkhaye.}

\emph{Etaṁ kho paramaṁ ñāṇaṁ,}\\
\emph{etaṁ sukhamanuttaraṁ,}\\
\emph{asokaṁ virajaṁ khemaṁ,}\\
\emph{etaṁ ānaṇyamuttamaṁ.}\footnote{\href{https://suttacentral.net/an6.45/pli/ms}{AN 6.45 / A III 354}, \emph{Iṇasutta}}

To that such like one, who is fully released,\\
There arises the knowledge:\\
`Unshakeable is my deliverance',\\
Upon his extinction of fetters to existence.

This is the highest knowledge,\\
This is the unsurpassed bliss,\\
This sorrow-less, taintless security,\\
Is the supreme debtless-ness.
\end{quote}

\emph{Arahants} are said to be debtless in regard to the four requisites offered by the laity out of faith, but when Nibbāna is regarded as a debtless-ness, it seems to imply something deeper.

\emph{Saṁsāra} or reiterated existence is itself a debt, which one can never pay off. When one comes to think of \emph{kamma} and its result, it is a debt that keeps on gathering an interminable interest, which can never be paid off.

But even from this debt the \emph{arahants} have won freedom by destroying the seeds of \emph{kamma}, by rendering them infertile. They are made ineffective beyond this life, as there is no rebirth. The meaningful line of the \emph{Ratanasutta},

\begin{quote}
\emph{khīṇaṁ purāṇaṁ, navaṁ natthi sambhavaṁ},\footnote{\href{https://suttacentral.net/snp2.1/pli/ms}{Snp 2.1 / Sn 235}, \emph{Ratanasutta}}
\end{quote}

\begin{quote}
whatever is old is extinct and there is no arising anew,
\end{quote}

has to be understood in that sense. The karmic debt is paid off and there is no fresh incurring.

All this is in praise of that unshakeable deliverance of the mind. It is a kind of extraordinary knowledge, almost unimaginable, a `real'-ization of one's own after death state.

In almost all serious discussions on Nibbāna, the subtlest moot point turns out to be the question of the after death state of the emancipated one. A brief answer, the Buddha had given to this question, we already brought up in our last sermon, by quoting the two concluding verses of the \emph{Udāna}, with which that collection of inspired utterances ends with a note of exceptional grandeur. Let us recall them.

\begin{quote}
\emph{Ayoghanahatass'eva,}\\
\emph{jalato jātavedaso,}\\
\emph{anupubbūpasantassa,}\\
\emph{yathā na ñāyate gati.}

\emph{Evaṁ sammāvimuttānaṁ,}\\
\emph{kāmabandhoghatārinaṁ,}\\
\emph{paññāpetuṁ gati natthi,}\\
\emph{pattānaṁ acalaṁ sukhaṁ.}\footnote{\href{https://suttacentral.net/ud8.10/pli/ms}{Ud 8.10 / Ud 93}, \emph{Dutiyadabbasutta}}

\clearpage

Just as in the case of a fire,\\
Blazing like a block of iron in point of compactness,\\
When it gradually calms down,\\
No path it goes by can be traced.

Even so, of those who are well released,\\
Who have crossed over the flux of shackles of sensuality,\\
And reached bliss unshaken,\\
There is no path to be pointed out.
\end{quote}

The last two lines are particularly significant. There is no path to be pointed out of those who have reached bliss unshaken. \emph{Acalaṁ sukhaṁ}, or `unshakeable bliss', is none other than that unshakeable deliverance of the mind.

\emph{Akuppa} means `unassailable' or `unshakeable'. Clearly enough, what the verse says is that after their death the emancipated ones leave no trace of a path gone by, even as the flames of a raging fire.

The flame may appear as something really existing due to the perception of the compact, \emph{ghanasaññā}, but when it goes down and disappears, no one can say that it went in such and such a direction.

Though this is the obvious meaning, some try to attribute quite a different meaning to the verse in question. The line \emph{paññāpetuṁ gati natthi}, ``there is no path to be pointed out'', is interpreted even by the commentators (who take the word \emph{gati} to mean some state of existence) as an assertion that, although such a bourne cannot be pointed out, the \emph{arahants} pass away into some non-descript realm.

This kind of interpretation is prompted by an apprehension of the charge of annihilation. A clear instance of this tendency is revealed in the commentary to the following verse in the \emph{Dhammapada}:

\begin{quote}
\emph{Ahiṁsakā ye munayo,}\\
\emph{niccaṁ kāyena saṁvutā,}\\
\emph{te yanti accutaṁ ṭhānaṁ,}\\
\emph{yattha gantvā na socare.}\footnote{\href{https://suttacentral.net/dhp221-234/pli/ms}{Dhp 225}, \emph{Kodhavagga}}

Innocent are the sages,\\
That are ever restrained in body,\\
They go to that state unshaken,\\
Wherein they grieve no more.
\end{quote}

The commentator, in paraphrasing, brings in the word \emph{sassataṁ}, `eternal', for \emph{accutaṁ}, thereby giving the idea that the \emph{arahants} go to an eternal place of rest.\footnote{Dhp-a III 321} Because the verb \emph{yanti}, `go', occurs there, he must have thought that this state unshaken, \emph{accutaṁ}, is something attainable after death.

But we can give another instance in support of our explanation of the term \emph{accutaṁ}. The following verse in the \emph{Hemakamāṇavapucchā} of the \emph{Pārāyanavagga} in the \emph{Sutta Nipāta} clearly shows what this \emph{accutaṁ} is:

\begin{quote}
\emph{Idha diṭṭhasutamutaviññātesu,}\\
\emph{piyarūpesu Hemaka,}\\
\emph{chandarāgavinodanaṁ,}\\
\emph{nibbānapadaṁ accutaṁ.}\footnote{\href{https://suttacentral.net/snp5.9/pli/ms}{Snp 5.9 / Sn 1086}, \emph{Hemakamāṇavapucchā}}

The dispelling here in this world of desire and lust,\\
In pleasurable things,\\
Seen, heard, sensed and cognized,\\
Is the unshaken state of Nibbāna, O Hemaka.
\end{quote}

This is further proof of the fact that there is no eternal immortal rest awaiting the \emph{arahants} after their demise.

The reason for such a postulate is probably the fear of falling into the annihilationist view. Why this chronic fear? To the worldlings overcome by craving for existence any teaching that leads to the cessation of existence appears dreadful.

That is why they put forward two new parables, following the same commentarial trend. The other day we mentioned about those two parables, the parable of the tortoise and the parable of the frog.\footnote{See \emph{Sermon 19}} When the fish and the toad living in water ask what sort of a thing land is, the tortoise and the frog are forced to say `no, no' to every question they put. Likewise the Buddha, so it is argued, was forced to give a string of negative terms in his discourses on Nibbāna.

But we have pointed out that this argument is fallacious and that those discourses have to be interpreted differently. The theme that runs through such discourses is none other than the cessation of existence.

In the \emph{Alagaddūpamasutta} of the \emph{Majjhima Nikāya} the Buddha declares in unmistakeable terms that some recluses and brahmins, on hearing him preaching the Dhamma for the cessation of existence, wrongly accuse him with the charge of being an annihilationist, \emph{sato sattassa ucchedaṁ vināsaṁ vibhavaṁ paññāpeti}, ``he is showing the way to the annihilation, destruction and non-existence of a truly existing being''.\footnote{\href{https://suttacentral.net/mn22/pli/ms}{MN 22 / M I 140}, \emph{Alagaddūpamasutta}}

He clearly states that some even grieve and lament and fall into despair, complaining \emph{ucchijjissāmi nāma su, vinassissāmi nāma su, na su nāma bhavissāmi}, ``so it seems I shall be annihilated, so it seems I shall perish, so it seems I shall be no more''.\footnote{\href{https://suttacentral.net/mn22/pli/ms}{MN 22 / M I 137}, \emph{Alagaddūpamasutta}}

Even during the lifetime of the Buddha there were various debates and controversies regarding the after death state of the emancipated person among recluses and brahmins. They were of the opinion that the after death state of the emancipated one in any particular religious system has to be explained according to a fourfold logic, or tetralemma. A paradigm of that tetralemma occurs quite often in the discourses. It consists of the following four propositions:

\begin{enumerate}
\def\labelenumi{\arabic{enumi}.}
\tightlist
\item
  \emph{hoti tathāgato paraṁ maraṇā},\\
  ``The Tathāgata exists after death''
\item
  \emph{na hoti tathāgato paraṁ maraṇā},\\
  ``The Tathāgata does not exist after death''
\item
  \emph{hoti ca na ca hoti tathāgato paraṁ maraṇā},\\
  ``The Tathāgata both exists and does not exist after death''
\item
  \emph{n'eva hoti na na hoti tathāgato paraṁ maraṇā},\footnote{E.g. at \href{https://suttacentral.net/mn72/pli/ms}{MN 72 / M I 484}, \emph{Aggivacchagottasutta}}\\
  ``The Tathāgata neither exists nor does not exist after death''.
\end{enumerate}

This four-cornered logic purports to round up the four possible alternatives in any situation, or four possible answers to any question.

The dilemma is fairly well known, where one is caught up between two alternatives. The tetralemma, with its four alternatives, is supposed to exhaust the universe of discourse in a way that one cannot afford to ignore~it.

When it comes to a standpoint regarding a particular issue, one is compelled to say `yes' or `no', or at least to assert both standpoints or negate them altogether. The contemporary recluses and brahmins held on to the view that the Tathāgata's after death state has to be predicated in accordance with the four-cornered logic.

When we hear the term Tathāgata, we are immediately reminded of the Buddha. But for the contemporary society, it was a sort of technical term with a broader meaning. Those recluses and brahmins used the term Tathāgata to designate the perfected individual in any religious system, whose qualifications were summed up in the thematic phrase \emph{uttamapuriso, paramapuriso, paramapattipatto},\footnote{\href{https://suttacentral.net/sn22.86/pli/ms}{SN 22.86 / S III 116}, \emph{Anurādhasutta}} ``the highest person, the supreme person, the one who has attained the supreme state''.

This fact is clearly borne out by the \emph{Kutūhalasālāsutta} in the \emph{Avyākata Saṁyutta} of the \emph{Saṁyutta Nikāya}. In that discourse we find the wandering ascetic Vacchagotta coming to the Buddha with the following report.

Recently there was a meeting of recluses, brahmins and wandering ascetics in the debating hall. In that assembly, the following chance talk arose:

\begin{quote}
Now there is this teacher, Pūraṇa Kassapa, who is widely acclaimed and who has a large following. When an ordinary disciple of his passes away, he predicates his destiny. So also in the case of a disciple who has attained the highest state of perfection in his religious system. Other well known teachers like Makkhali Gosāla, Nigaṇṭha Nātaputta, Sañjaya Belaṭṭhiputta, Pakudha Kaccāyana and Ajita Kesakambali do the same. They all declare categorically the after death state of both types of their disciples.

But as for this ascetic Gotama, who also is a teacher widely acclaimed with a large following, the position is that he clearly declares the after death state of an ordinary disciple of his, but in the case of a disciple who has attained the highest state of perfection, he does not predicate his destiny according to the above mentioned tetralemma. Instead he makes such a declaration about him as the following:

\emph{Acchecchi taṇhaṁ, vāvattayi saññojanaṁ, sammā mānābhisamayā antam akāsi dukkhassa},\footnote{\href{https://suttacentral.net/sn44.9/pli/ms}{SN 44.9 / S IV 399}, \emph{Kutūhalasālāsutta}}

``he cut off craving, disjoined the fetter and, by rightly understanding conceit for what it is, made an end of suffering''.
\end{quote}

Vacchagotta concludes this account with the confession that he himself was perplexed and was in doubt as to how the Dhamma of the recluse Gotama has to be understood. The Buddha grants that Vacchagotta's doubt is reasonable, with the words

\begin{quote}
\emph{alañhi te, Vaccha, kaṅkhituṁ, alaṁ vicikicchituṁ, kaṅkhaniye ca pana te ṭhāne vicikicchā uppannā},

it behoves you to doubt, Vaccha, it behoves you to be perplexed, for doubt has arisen in you on a dubious point.
\end{quote}

Then the Buddha comes out with the correct standpoint in order to dispel Vacchagotta's doubt.

\begin{quote}
\emph{Sa-upādānassa kvāhaṁ, Vaccha, upapattiṁ paññāpemi, no anupādānassa},

it is for one with grasping, Vaccha, that I declare there is an occurrence of birth, not for one without grasping.
\end{quote}

He gives the following simile by way of illustration.

\begin{quote}
\emph{Seyyathāpi, Vaccha, aggi sa-upādāno jalati no anupādāno, evam eva kvāhaṁ, Vaccha, sa-upādānassa upapattiṁ paññāpemi, no anupādānassa},

just as a fire burns when it has fuel to grasp and not when it has no fuel, even so, Vaccha, I declare that there is an occurrence of birth for one with grasping, not for one without grasping.
\end{quote}

As we have mentioned before, the word \emph{upādāna} has two meanings, it means both grasping as well as fuel. In fact fuel is just what the fire `grasps'. Just as the fire depends on grasping in the form of fuel, so also the individual depends on grasping for his rebirth.

Within the context of this analogy, Vacchagotta now raises a question that has some deeper implications:

\begin{quote}
\emph{Yasmiṁ pana, bho Gotama, samaye acci vātena khittā dūrampi gacchati, imassa pana bhavaṁ Gotamo kim upādānasmiṁ paññāpeti},

Master Gotama, at the time when a flame flung by the wind goes even far, what does Master Gotama declare to be its object of grasping or fuel?
\end{quote}

The Buddha's answer to that question is:

\begin{quote}
\emph{Yasmiṁ kho, Vaccha, samaye acci vātena khittā dūrampi gacchati, tamahaṁ vātupādānaṁ vadāmi; vāto hissa, Vaccha, tasmiṁ samaye upādānaṁ hoti},

at the time, Vaccha, when a flame flung by the wind goes even far, that, I say, has wind as its object of grasping. Vaccha, at that time wind itself serves as the object of grasping.
\end{quote}

Now this is only an analogy. Vaccha raises the question proper only at this point:

\begin{quote}
\emph{Yasmiñca pana, bho Gotama, samaye imañca kāyaṁ nikkhipati satto ca aññataraṁ kāyam anuppatto hoti, imassa pana bhavaṁ Gotamo kim upādānasmiṁ paññāpeti},

at the time, Master Gotama, when a being lays down this body and has reached a certain body, what does Master Gotama declare to be a grasping in his case?
\end{quote}

The Buddha replies:

\begin{quote}
\emph{Yasmiñca pana, Vaccha, samaye imañca kāyaṁ nikkhipati satto ca aññataraṁ kāyam anuppatto hoti, tam ahaṁ taṇhupādānaṁ vadāmi; taṇhā hissa, Vaccha, tasmiṁ samaye upādānaṁ hoti},

at the time, Vaccha, when a being lays down this body and has reached a certain body, I say, he has craving as his grasping. At that time, Vaccha, it is craving that serves as a grasping for him.
\end{quote}

With this sentence the discourse ends abruptly, but there is an intricate point in the two sections quoted above. In these two sections, we have adopted the reading \emph{anuppatto}, `has reached', as more plausible in rendering the phrase \emph{aññataraṁ kāyam anuppatto}, ``has reached a certain body''.\footnote{This suggestion finds support in the Chinese parallel to the \emph{Kutūhalasālāsutta}, \emph{Saṁyukta Āgama} discourse 957 (Taishº II 244b2), which speaks of the being that has passed away as availing himself of a mind-made body. (Anālayo)}

The commentary, however, seeks to justify the reading \emph{anupapanno}, `is not reborn', which gives quite an opposite sense, with the following explanation \emph{cutikkhaṇeyeva paṭisandhicittassa anuppannattā anuppanno hoti},\footnote{Spk III 114} ``since at the death moment itself, the rebirth consciousness has not yet arisen, he is said to be not yet reborn''.

Some editors doubt whether the correct reading should be \emph{anuppatto}.\footnote{Feer, L. (ed.): \emph{Saṁyutta Nikāya}, PTS 1990 (1894), p 400 n 2} The doubt seems reasonable enough, for even syntactically, \emph{anuppatto} can be shown to fit into the context better than \emph{anuppanno}. The word \emph{aññataraṁ} provides us with the criterion. It has a selective sense, like `a certain', and carries definite positive implications. To express something negative a word like \emph{aññaṁ}, `another', has to be used instead of the selective \emph{aññataraṁ}, `a certain'.

On the other hand, the suggested reading \emph{anuppatto} avoids those syntactical difficulties. A being lays down this body and has reached a certain body. Even the simile given as an illustration is in favour of our interpretation. The original question of Vaccha about the flame flung by the wind, reminds us of the way a forest fire, for instance, spreads from one tree to another tree some distance away. It is the wind that pushes the flame for it to catch hold of the other tree.

The commentarial explanation, however, envisages a situation in which a being lays down this body and is not yet reborn in another body. It is in the interim that craving is supposed to be the grasping or a fuel. Some scholars have exploited this commentarial explanation to postulate a theory of \emph{antarābhava}, or interim existence, prior to rebirth proper.

Our interpretation, based on the reading \emph{anuppatto}, rules out even the possibility of an \emph{antarābhava}. Obviously enough, Vacchagotta's question is simple and straightforward. He is curious to know what sort of a grasping connects up the being that lays down the body and the being that arises in another body. That is to say, how the apparent gap could be bridged.

The answer given by the Buddha fully accords with the analogy envisaged by the premise. Just as the wind does the work of grasping in the case of the flame, so craving itself, at the moment of death, fulfils the function of grasping for a being to reach another body.

That is precisely why craving is called \emph{bhavanetti}, ``the guide in becoming''.\footnote{E.g. \href{https://suttacentral.net/sn23.3/pli/ms}{SN 23.3 / S III 190}, \emph{Bhavanettisutta}} Like a promontory, it juts out into the ocean of \emph{saṁsāra}. When it comes to rebirth, it is craving that bridges the apparent gap. It is the invisible combustible fuel that keeps the raging \emph{saṁsāric} forest fire alive.

All in all, what transpired at the debating hall (\emph{Kutūhalasālā}) reveals one important fact, namely that the Buddha's reluctance to give a categorical answer regarding the after death state of the emancipated one in his dispensation had aroused the curiosity of those recluses and brahmins. That is why they kept on discussing the subject at length.

However, it was not the fact that he had refused to make any comment at all on this point. Only, that the comment he had made appeared so strange to them, as we may well infer from Vacchagotta's report of the discussion at the debating hall.

The Buddha's comment on the subject, which they had quoted, was not based on the tetralemma. It was a completely new formulation.

\begin{quote}
\emph{Acchecchi taṇhaṁ, vāvattayi saññojanaṁ, sammā mānābhisamayā antamakāsi dukkhassa},

he cut off craving, disjoined the fetter and, by rightly understanding conceit for what it is, made an end of suffering.
\end{quote}

This then, is the correct answer, and not any one of the four corners of the tetralemma. This brief formula is of paramount importance. When craving is cut off, the `guide-in-becoming', which is responsible for rebirth, is done away with. It is as if the fetter binding to another existence has been unhooked.

The term \emph{bhavasaṁyojanakkhaya}, ``destruction of the fetter to existence'', we came across earlier, conveys the same sense.\footnote{\href{https://suttacentral.net/iti62/pli/ms}{Iti 62 / It 53}, \emph{Indriyasutta}; see \emph{Sermon 16},}

The phrase \emph{sammā mānābhisamaya} is also highly significant. With the dispelling of ignorance, the conceit `am', \emph{asmimāna}, is seen for what it is. It disappears when exposed to the light of understanding and that is the end of suffering as well. The concluding phrase \emph{antam akāsi dukkhassa}, ``made an end of suffering'', is conclusive enough. The problem that was there all the time was the problem of suffering, so the end of suffering means the end of the whole problem.

In the \emph{Aggivacchagottasutta} of the \emph{Majjhima Nikāya} the Buddha's response to the question of the after death state of the \emph{arahant} comes to light in greater detail. The question is presented there in the form of the tetralemma, beginning with \emph{hoti tathāgato paraṁ maraṇā}.\footnote{\href{https://suttacentral.net/mn72/pli/ms}{MN 72 / M I 484}, \emph{Aggivacchagottasutta}}

While all the other recluses and brahmins held that the answer should necessarily take the form of one of the four alternatives, the Buddha put them all aside, \emph{ṭhapitāni,} rejected them, \emph{patikkhittāni}, refused to state his view categorically in terms of them, \emph{avyākatāni}.

This attitude of the Buddha puzzled not only the ascetics of other sects, but even some of the monks like Māluṅkyāputta. In very strong terms, Māluṅkyāputta challenged the Buddha to give a categorical answer or else confess his ignorance.\footnote{\href{https://suttacentral.net/mn63/pli/ms}{MN 63 / M I 427}, \emph{Cūḷamāluṅkyāputtasutta}}

As a matter of fact there are altogether ten such questions, which the Buddha laid aside, rejected and refused to answer categorically. The first six take the form of three dilemmas, while the last four constitute the tetralemma already mentioned. Since an examination of those three dilemmas would reveal some important facts, we shall briefly discuss their significance as well.

The three sets of views are stated thematically as follows:

\begin{enumerate}
\def\labelenumi{\arabic{enumi}.}
\tightlist
\item
  \emph{sassato loko,} ``the world is eternal''
\item
  \emph{asassato loko,} ``the world is not eternal''
\item
  \emph{antavā loko}, ``the world is finite''
\item
  \emph{anantavā loko}, ``the world is infinite''
\item
  \emph{taṁ jīvaṁ taṁ sarīraṁ}, ``the soul and the body are the same''
\item
  \emph{aññaṁ jīvaṁ aññaṁ sarīraṁ}, ``the soul is one thing and the body another''.
\end{enumerate}

These three dilemmas, together with the tetralemma, are known as \emph{abyākatavatthūni}, the ten undetermined points.\footnote{The expression \emph{abyākatavatthu} occurs e.g.~at \href{https://suttacentral.net/an7.54/pli/ms}{AN 7.54 / A IV 68}, \emph{Abyākatasutta}} Various recluses and brahmins, as well as king Pasenadi Kosala, posed these ten questions to the Buddha, hoping to get categorical answers.

Why the Buddha laid them aside is a problem to many scholars. Some, like Māluṅkyāputta, would put it down to agnosticism. Others would claim that the Buddha laid them aside because they are irrelevant to the immediate problem of deliverance, though he could have answered them. That section of opinion go by the \emph{Siṁsapāvanasutta} in the \emph{Saccasaṁyutta} of the \emph{Saṁyutta Nikāya}.\footnote{\href{https://suttacentral.net/sn56.31/pli/ms}{SN 56.31 / S V 437}, \emph{Sīsapāvanasutta}}

Once while dwelling in a \emph{siṁsapā} grove, the Buddha took up some \emph{siṁsapā} leaves in his hands and asked the monks:

\begin{quote}
``What do you think, monks, which is more, these leaves in my hand or those in the \emph{siṁsapā} grove?''

The monks reply that the leaves in the hand are few and those in the \emph{siṁsapā} grove are greater in number. Then the Buddha makes a declaration to the following effect:

``Even so, monks, what I have understood through higher knowledge and not taught you is far more than what I have taught you''.
\end{quote}

If we rely on this simile, we would have to grant that the questions are answerable in principle, but that the Buddha preferred to avoid them because they are not relevant. But this is not the reason either.

All these ten questions are based on wrong premises. To take them seriously and answer them would be to grant the validity of those premises. The dilemmas and the tetralemma seek arbitrarily to corner anyone who tries to answer them. The Buddha refused to be cornered that way.

The first two alternatives, presented in the form of a dilemma, are \emph{sassato loko,} ``the world is eternal'', and \emph{asassato loko,} ``the world is not eternal''. This is an attempt to determine the world in temporal terms. The next set of alternatives seeks to determine the world in spatial terms.

Why did the Buddha refuse to answer these questions on time and space? It is because the concept of `the world' has been given quite a new definition in this dispensation.

Whenever the Buddha redefined a word in common usage, he introduced it with the phrase \emph{ariyassa vinaye}, ``in the discipline of the noble ones''.

We have already mentioned on an earlier occasion that according to the discipline of the noble ones, `the world' is said to have arisen in the six sense-spheres, \emph{chasu loko samuppanno}.\footnote{\href{https://suttacentral.net/sn1.70/pli/ms}{SN 1.70 / S I 41}, \emph{Lokasutta}; see \emph{Sermon 4}} In short, the world is redefined in terms of the six spheres of sense. This is so fundamentally important that in the \emph{Saḷāyatanasaṁyutta} of the \emph{Saṁyutta Nikāya} the theme comes up again and again.

For instance, in the \emph{Samiddhisutta} Venerable Samiddhi poses the following question to the Buddha:

\begin{quote}
'\emph{Loko, loko'ti, bhante, vuccati. Kittāvatā nu kho, bhante, loko vā assa lokapaññatti vā?}\footnote{\href{https://suttacentral.net/sn35.68/pli/ms}{SN 35.68 / S IV 39}, \emph{Samiddhisutta}}

`The world, the world', so it is said Venerable sir, but how far, Venerable sir, does this world or the concept of the world go?
\end{quote}

The Buddha gives the following answer:

\begin{quote}
\emph{Yattha kho, Samiddhi, atthi cakkhu, atthi rūpā, atthi cakkhuviññāṇaṁ, atthi cakkhuviññāṇaviññātabbā dhammā, atthi tattha loko vā lokapaññatti vā},

Where there is the eye, Samiddhi, where there are forms, where there is eye-consciousness, where there are things cognizable by eye-consciousness, there exists the world or the concept of the world.
\end{quote}

A similar statement is made with regard to the other spheres of sense, including the mind. That, according to the Buddha, is where the world exists. Then he makes a declaration concerning the converse:

\begin{quote}
\emph{Yattha ca kho, Samiddhi, natthi cakkhu, natthi rūpā, natthi cakkhuviññāṇaṁ, natthi cakkhuviññāṇaviññātabbā dhammā, natthi tattha loko vā lokapaññatti vā},

Where there is no eye, Samiddhi, where there are no forms, where there is no eye-consciousness, where there are no things cognizable by eye-consciousness, there the world does not exist, nor any concept of the world.
\end{quote}

From this we can well infer that any attempt to determine whether there is an end of the world, either in temporal terms or in spatial terms, is misguided. It is the outcome of a wrong view, for there is a world so long as there are the six spheres of sense. That is why the Buddha consistently refused to answer those questions regarding the world.

There are a number of definitions of the world given by the Buddha. We shall cite two of them. A certain monk directly asked the Buddha to give a definition of the world:

\begin{quote}
\emph{`Loko, loko'ti bhante, vuccati. Kittāvatā nu kho, bhante, `loko'ti vuccati?}

`The world, the world', so it is said. In what respect, Venerable sir, is it called a world?
\end{quote}

Then the Buddha makes the following significant declaration:

\begin{quote}
\emph{`Lujjatī'ti kho, bhikkhu, tasmā `loko'ti vuccati. Kiñca lujjati? Cakkhu kho, bhikkhu, lujjati, rūpā lujjanti, cakkhuviññāṇaṁ lujjati, cakkhusamphasso lujjati, yampidaṁ cakkhusamphassapaccayā uppajjati vedayitaṁ sukhaṁ vā dukkhaṁ vā adukkhamasukhaṁ vā tampi lujjati. `Lujjatī'ti kho, bhikkhu, tasmā `loko'ti vuccati.}\footnote{\href{https://suttacentral.net/sn35.82/pli/ms}{SN 35.82 / S IV 52}, \emph{Lokapañhāsutta}}

It is disintegrating, monk, that is why it is called `the world'. And what is disintegrating? The eye, monk, is disintegrating, forms are disintegrating, eye-consciousness is disintegrating, eye-contact is disintegrating, and whatever feeling that arises dependent on eye-contact, be it pleasant, or painful, or neither-pleasant-nor-painful, that too is disintegrating. It is disintegrating, monk, that is why it is called `the world'.
\end{quote}

Here the Buddha is redefining the concept of the world, punning on the verb \emph{lujjati}, which means to `break up' or `disintegrate'. To bring about a radical change in outlook, in accordance with the Dhamma, the Buddha would sometimes introduce a new etymology in preference to the old. This definition of `the world' is to the same effect.

Venerable Ānanda, too, raises the same question, soliciting a redefinition for the well-known concept of the world, and the Buddha responds with the following answer:

\begin{quote}
\emph{Yaṁ kho, Ānanda, palokadhammaṁ, ayaṁ vuccati ariyassa vinaye loko.}\footnote{\href{https://suttacentral.net/sn35.84/pli/ms}{SN 35.84 / S IV 53}, \emph{Palokadhammasutta}}

Whatever, Ānanda, is subject to disintegration that is called `the world' in the noble one's discipline.
\end{quote}

He even goes on to substantiate his statement at length:

\begin{quote}
\emph{Kiñca, Ānanda, palokadhammaṁ? Cakkhuṁ kho, Ānanda, palokadhammaṁ, rūpā palokadhammā, cakkhuviññāṇaṁ palokadhammaṁ, cakkhusamphasso palokadhammo, yampidaṁ cakkhusamphassapaccayā uppajjati vedayitaṁ sukhaṁ vā dukkhaṁ vā adukkhamasukhaṁ vā tampi palokadhammaṁ. Yaṁ kho, Ānanda, palokadhammaṁ, ayaṁ vuccati ariyassa vinaye loko}.

And what, Ānanda, is subject to disintegration? The eye, Ānanda, is subject to disintegration, forms are subject to disintegration, eye-consciousness is subject to disintegration, eye-contact is subject to disintegration, and whatever feeling that arises dependent on eye-contact, be it pleasant, or painful, or neither-pleasant-nor-painful, that too is subject to disintegration. Whatever is subject to disintegration, Ānanda, is called `the world' in the noble one's discipline.
\end{quote}

In this instance, the play upon the word \emph{loka} is vividly apt in that it brings out the transciency of the world. If the world by definition is regarded as transient, it cannot be conceived substantially as a unit. How then can an eternity or infinity be predicated about it? If all the so-called things in the world, listed above, are all the time disintegrating, any unitary concept of the world is fallacious.

Had the Buddha answered those misconceived questions, he would thereby concede to the wrong concept of the world current among other religious groups. So then we can understand why the Buddha refused to answer the first four questions.

Now let us examine the next dilemma, \emph{taṁ jīvaṁ taṁ sarīraṁ, aññaṁ jīvaṁ aññaṁ sarīraṁ}, ``the soul and the body are the same, the soul is one thing and the body another''. To these questions also, the other religionists insisted on a categorical answer, either `yes' or `no'.

There is a `catch' in the way these questions are framed. The Buddha refused to get caught by them. These two questions are of the type that clever lawyers put to a respondent these days. They would sometimes insist strictly on a `yes' or `no' as answer and ask a question like: ``Have you now given up drinking?''

If the respondent happens to be a teetotaller, he would be in a quandary, since both answers tend to create a wrong impression.

So also in the case of these two alternatives, ``the soul and the body are the same, the soul is one thing and the body another''. Either way there is a presumption of a soul, which the Buddha did not subscribe to. The Buddha had unequivocally declared that the idea of soul is the outcome of an utterly foolish view, \emph{kevalo paripūro bāladhammo}.\footnote{\href{https://suttacentral.net/mn22/pli/ms}{MN 22 / M I 138}, \emph{Alagaddūpamasutta}} That is why the Buddha rejected both standpoints.

A similar `catch', a similar misconception, underlies the tetralemma concerning the after death state of the Tathāgata. It should be already clear to some extent by what we have discussed so far.

For the Buddha, the term Tathāgata had a different connotation than what it meant for those of other sects. The latter adhered to the view that both the ordinary disciple as well as the perfected individual in their systems of thought had a soul of some description or other.

The Buddha never subscribed to such a view. On the other hand, he invested the term Tathāgata with an extremely deep and subtle meaning. His definition of the term will emerge from the \emph{Aggivacchagottasutta}, which we propose to discuss now.

In this discourse we find the wandering ascetic Vacchagotta trying to get a categorical answer to the questionnaire, putting each of the questions with legal precision one by one, as a lawyer would at the courts of law.

\begin{quote}
\emph{Kiṁ nu kho, bho Gotamo, `sassato loko, idam eva saccaṁ, mogham aññan'ti, evaṁ diṭṭhi bhavaṁ Gotamo?}\footnote{\href{https://suttacentral.net/mn72/pli/ms}{MN 72 / M I 484}, \emph{Aggivacchagottasutta}}

``Now, Master Gotama, `the world is eternal, this only is true, all else is false', are you of this view, Master Gotama?''
\end{quote}

The Buddha replies: \emph{na kho ahaṁ, Vaccha, evaṁ diṭṭhi}, ``no, Vaccha, I am not of this view''.

Then Vacchagotta puts the opposite standpoint, which too the Buddha answers in the negative. To all the ten questions the Buddha answers `no', thereby rejecting the questionnaire in toto. Then Vacchagotta asks why, on seeing what danger, the Buddha refuses to hold any of those views. The Buddha gives the following explanation:

\begin{quote}
\emph{`Sassato loko'ti kho, Vaccha, diṭṭhigatam etaṁ diṭṭhigahanaṁ diṭṭhikantāraṁ diṭṭhivisūkaṁ diṭṭhivipphanditaṁ diṭṭhisaṁyojanaṁ sadukkhaṁ savighātaṁ sa-upāyāsaṁ sapariḷāhaṁ, na nibbidāya na virāgāya na nirodhāya na upasamāya na abhiññāya na sambodhāya na nibbānāya saṁvattati}.

Vaccha, this speculative view that the world is eternal is a jungle of views, a desert of views, a distortion of views, an aberration of views, a fetter of views, it is fraught with suffering, with vexation, with despair, with delirium, it does not lead to disenchantment, to dispassion, to cessation, to tranquillity, to higher knowledge, to enlightenment, to Nibbāna.
\end{quote}

So with regard to the other nine views.

Now here we find both the above-mentioned reasons. Not only the fact that these questions are not relevant to the attainment of Nibbāna, but also the fact that there is something wrong in the very statement of the problems. What are the dangers that he sees in holding any of these views?

Every one of them is just a speculative view, \emph{diṭṭhigataṁ}, a jungle of views, \emph{diṭṭhigahanaṁ}, an arid desert of views, \emph{diṭṭhikantāraṁ}, a mimicry or a distortion of views, \emph{diṭṭhivisūkaṁ}, an aberration of views, \emph{diṭṭhivipphanditaṁ}, a fetter of views, \emph{diṭṭhisaṁyojanaṁ}.

They bring about suffering, \emph{sadukkhaṁ}, vexation, \emph{savighātaṁ}, despair, \emph{sa-upāyāsaṁ}, delirium, \emph{sapariḷāhaṁ}.

They do not conduce to disenchantment, \emph{na nibbidāya}, to dispassion, \emph{na virāgāya}, to cessation, \emph{na nirodhāya}, to tranquillity, \emph{na upasamāya}, to higher knowledge, \emph{na abhiññāya}, to enlightenment, \emph{na sambodhāya}, to extinguishment, \emph{na nibbānāya}.

From this declaration it is obvious that these questions are ill founded and misconceived. They are a welter of false views, so much so that the Buddha even declares that these questions simply do not exist for the noble disciple, who has heard the Dhamma. They occur as real problems only to the untaught worldling. Why is that?

Whoever has a deep understanding of the four noble truths would not even raise these questions. This declaration should be enough for one to understand why the Buddha refused to answer them.

Explaining that it is because of these dangers that he rejects them in toto, the Buddha now makes clear what his own stance is. Instead of holding any of those speculative views, he has seen for himself the rise, \emph{samudaya}, and fall, \emph{atthagama}, of the five aggregates as a matter of direct experience, thereby getting rid of all `I'-ing and `my'-ing and latencies to conceits, winning ultimate release.

Even after this explanation Vacchagotta resorts to the fourfold logic to satisfy his curiosity about the after death state of the monk thus released in mind.

\begin{quote}
\emph{Evaṁ vimuttacitto pana, bho Gotamo, bhikkhu kuhiṁ uppajjati?}

When a monk is thus released in mind, Master Gotama, where is he reborn?
\end{quote}

The Buddha replies:

\begin{quote}
\emph{Uppajjatī'ti kho, Vaccha, na upeti},

To say that he is reborn, Vaccha, falls short of a reply.
\end{quote}

Then Vacchagotta asks:

\begin{quote}
\emph{Tena hi, bho Gotama, na uppajjati?}

If that is so, Master Gotama, is he not reborn?

\emph{Na uppajjatī'ti kho, Vaccha, na upeti},

To say that he is not reborn, Vaccha, falls short of a reply.

\emph{Tena hi, bho Gotama, uppajjati ca na ca uppajjati}?

If that is so, Master Gotama, is he both reborn and is not reborn?

\emph{Uppajjati ca na ca uppajjatī'ti kho, Vaccha, na upeti},

To say that he is both reborn and is not reborn, Vaccha, falls short of a reply.

\emph{Tena hi, bho Gotama, neva uppajjati na na uppajjati?}

If that is so, Master Gotama, is he neither reborn nor is not reborn?

\emph{Neva uppajjati na na uppajjatī'ti kho, Vaccha, na upeti},

To say that he is neither reborn nor is not reborn, Vaccha, falls short of a reply.
\end{quote}

At this unexpected response of the Buddha to his four questions, Vacchagotta confesses that he is fully confused and bewildered. The Buddha grants that his confusion and bewilderment are understandable, since this Dhamma is so deep and subtle that it cannot be plumbed by logic, \emph{atakkāvacaro}.

However, in order to give him a clue to understand the Dhamma point of view, he gives an illustration in the form of a catechism.

\begin{quote}
\emph{Taṁ kiṁ maññasi, Vaccha, sace te purato aggi jaleyya, jāneyyāsi tvaṁ `ayaṁ me purato aggi jalatī'ti?}

What do you think, Vaccha, suppose a fire were burning before you, would you know `this fire is burning before me'?

\emph{Sace me, bho Gotama, purato aggi jaleyya, jāneyyāhaṁ `ayaṁ me purato aggi jalatī'ti.}

If, Master Gotama, a fire were burning before me, I would know `this fire is burning before me'.

\emph{Sace pana taṁ, Vaccha, evaṁ puccheyya `yo te ayaṁ purato aggi jalati, ayaṁ aggi kiṁ paṭicca jalatī'ti, evaṁ puṭṭho tvaṁ, Vaccha, kinti byākareyyāsi?}

If someone were to ask you, Vaccha, `what does this fire that is burning before you burns in dependence on', being asked thus, Vaccha, what would you answer?

\emph{Evaṁ puṭṭho ahaṁ, bho Gotama, evaṁ byākareyyaṁ `yo me ayaṁ purato aggi jalati, ayaṁ aggi tiṇakaṭṭhupādānaṁ paṭicca jalatī'ti}.

Being asked thus, Master Gotama, I would answer `this fire burning before me burns in dependence on grass and sticks'.

\emph{Sace te, Vaccha, purato so aggi nibbāyeyya, jāneyyāsi tvaṁ `ayaṁ me purato aggi nibbuto'ti?}

If that fire before you were to be extinguished, Vaccha, would you know `this fire before me has been extinguished'?

\emph{Sace me, bho Gotamo, purato so aggi nibbāyeyya, jāneyyāhaṁ `ayaṁ me purato aggi nibbuto'ti.}

If that fire before me were to be extinguished, Master Gotama, I would know `this fire before me has been extinguished'.

\emph{Sace pana taṁ, Vaccha, evaṁ puccheyya `yo te ayaṁ purato aggi nibbuto, so aggi ito katamaṁ disaṁ gato, puratthimaṁ vā dakkhiṇaṁ vā pacchimaṁ vā uttaraṁ vā'ti, evaṁ puṭṭho tvaṁ, Vaccha, kinti byākareyyāsi?}

If someone were to ask you, Vaccha, when that fire before you were extinguished, `to which direction did it go, to the east, the west, the north or the south', being asked thus, what would you answer?

\emph{Na upeti, bho Gotama, yañhi so, bho Gotama, aggi tiṇakaṭṭhupādānaṁ paṭicca jalati, tassa ca pariyādānā aññassa ca anupahārā anāhāro nibbuto tveva saṅkhaṁ gacchati.}

That wouldn't do as a reply, Master Gotama, for that fire burnt in dependence on its fuel of grass and sticks. That being used up and not getting any more fuel, being without fuel, it is reckoned as extinguished.
\end{quote}

At this point a very important expression comes up, which we happened to discuss earlier too, namely \emph{saṅkhaṁ gacchati}.\footnote{See \emph{Sermons 1, 12 and 13}} It means `to be reckoned', or `to be known as', or `to be designated'. So the correct mode of designation in this case is to say that the fire is reckoned as `extinguished', and not to say that it has gone somewhere.

If one takes mean advantage of the expression `fire has gone out' and insists on locating it, it will only be a misuse or an abuse of linguistic usage. It reveals a pervert tendency to misunderstand and misinterpret. Therefore, all that can be said by way of predicating such a situation, is \emph{nibbuto tveva saṅkhaṁ gacchati}, ``it is reckoned as `extinguished'\,''.

Now comes a well-timed declaration in which the Buddha, starting right from where Vacchagotta leaves off, brings the whole discussion to a climactic end.

\begin{quote}
\emph{Evameva kho, Vaccha, yena rūpena tathāgataṁ paññāpayamāno paññāpeyya, taṁ rūpaṁ tathāgatassa pahīnaṁ ucchinnamūlaṁ tālāvatthukataṁ anabhāvakataṁ āyatiṁ anuppādadhammaṁ. Rūpasaṅkhāvimutto kho, Vaccha, tathāgato, gambhīro appameyyo duppariyogāho, seyyathāpi mahāsamuddo. Uppajjatī'ti na upeti, na uppajjatī'ti na upeti, uppajjati ca na ca uppajjatī'ti na upeti, neva uppajjati na na uppajjatī'ti na upeti.}

Even so, Vaccha, that form by which one designating the Tathāgata might designate him, that has been abandoned by him, cut off at the root, made like an uprooted palm tree, made non-existent and incapable of arising again. The Tathāgata is free from reckoning in terms of form, Vaccha, he is deep, immeasurable and hard to fathom, like the great ocean. To say that he is reborn falls short of a reply, to say that he is not reborn falls short of a reply, to say that he is both reborn and is not reborn falls short of a reply, to say that he is neither reborn nor is not reborn falls short of a reply.
\end{quote}

This declaration, which a fully convinced Vacchagotta now wholeheartedly hailed and compared to the very heartwood of a \emph{Sāla} tree, enshrines an extremely profound norm of Dhamma.

It was when Vacchagotta had granted the fact that it is improper to ask in which direction an extinguished fire has gone, and that the only proper linguistic usage is simply to say that `it is extinguished', that the Buddha came out with this profound pronouncement concerning the five aggregates.

In the case of the Tathāgata, the aggregate of form, for instance, is abandoned, \emph{pahīnaṁ}, cut off at the root, \emph{ucchinnamūlaṁ}, made like an uprooted palm tree divested from its site, \emph{tālāvatthukataṁ}, made non existent, \emph{anabhavakataṁ}, and incapable of arising again, \emph{āyatiṁ anuppādadhammaṁ}.

Thereby the Tathāgata becomes free from reckoning in terms of form, \emph{rūpasaṅkhāvimutto kho tathāgato}. Due to this very freedom, he becomes deep, immeasurable and unfathomable like the great ocean. Therefore he cannot be said to be reborn, or not to be reborn, or both or neither. The abandonment of form, referred to above, comes about not by death or destruction, but by the abandonment of craving.

The fact that by the abandonment of craving itself, form is abandoned, or eradicated, comes to light from the following quotation from the \emph{Rādhasaṁyutta} of the \emph{Saṁyutta Nikāya}.

\begin{quote}
\emph{Rūpe kho, Rādha, yo chando yo rāgo yā nandī yā taṇhā, taṁ pajahatha. Evaṁ taṁ rūpaṁ pahīnaṁ bhavissati ucchinnamūlaṁ tālāvatthukataṁ anabhāvakataṁ āyatiṁ anuppādadhammaṁ.}\footnote{\href{https://suttacentral.net/sn23.9/pli/ms}{SN 23.9 / S III 193}, \emph{Chandarāgasutta}}

Rādha, you give up that desire, that lust, that delight, that craving for form. It is thus that form comes to be abandoned, cut off at the root, made like an uprooted palm tree, made non-existent and incapable of arising again.
\end{quote}

Worldlings are under the impression that an \emph{arahant's} five aggregates of grasping get destroyed at death. But according to this declaration, an \emph{arahant} is like an uprooted palm tree. A palm tree uprooted but left standing, divested of its site, might appear as a real palm tree to one who sees it from a distance. Similarly, an untaught worldling thinks that there is a being or person in truth and fact when he hears the term Tathāgata, even in this context too.

This is the insinuation underlying the above quoted pronouncement. It has some profound implications, but time does not permit us to go into them today.
