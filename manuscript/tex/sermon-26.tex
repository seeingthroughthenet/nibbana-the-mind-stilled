\chapter{Sermon 26}

\NibbanaOpeningQuote

With the permission of the Most Venerable Great Preceptor and the assembly of the venerable meditative monks. This is the twenty-sixth sermon in the series of sermons on Nibbāna.

Even from what we have so far explained, it should be clear that the \emph{Kāḷakārāmasutta} enshrines an extremely deep analysis of the concepts of truth and falsehood, generally accepted by the world. We had to clear up a lot of jungle to approach this discourse, which has suffered from neglect to such an extent, that it has become difficult to determine the correct one out of a maze of variant readings.

But now we have exposed the basic ideas underlying this discourse through semantic and etymological explanations, which may even appear rather academic. The task before us now is to assimilate the deep philosophy the Buddha presents to the world by this discourse in a way that it becomes a vision.

The Tathāgata who had an insight into the interior mechanism of the six-fold sense-base, which is the factory for producing dogmatic views that are beaten up on the anvil of logic, \emph{takkapariyāhata}, was confronted with the problem of mediation with the worldlings, who see only the exterior of the six-fold sense-base.

In order to facilitate the understanding of the gravity of this problem, we quoted the other day an extract from the \emph{Pheṇapiṇḍūpamasutta} of the \emph{Khandhasaṁyutta} where consciousness is compared to a magical illusion.

\begin{quote}
\emph{Seyyathāpi, bhikkhave, māyākāro vā māyākārantevāsī vā cātum-\\ mahāpathe māyaṁ vidaṁseyya, tam enaṁ cakkhumā puriso passeyya nijjhāyeyya yoniso upaparikkheyya. Tassa taṁ passato nijjhāyato yoniso upaparikkhato rittakaññ'eva khāyeyya tucchakaññ'eva khāyeyya asārakaññ'eva khāyeyya. Kiñhi siyā, bhikkhave, māyāya sāro.}

\emph{Evameva kho, bhikkhave, yaṁ kiñci viññāṇaṁ atītānāgatapaccuppannaṁ, ajjhattaṁ vā bahiddhā vā, oḷārikaṁ vā sukhumaṁ vā, hīnaṁ vā paṇītaṁ vā, yaṁ dūre santike vā, taṁ bhikkhu passati nijjhāyati yoniso upaparikkhati. Tassa taṁ passato nijjhāyato yoniso upaparikkhato rittakaññ'eva khāyati tucchakaññ'eva khāyati asārakaññ'eva khāyati. Kiñhi siyā, bhikkhave, viññāṇe sāro.}\footnote{S III 142, \emph{Pheṇapiṇḍūpamasutta}}

Suppose, monks, a magician or a magician's apprentice should hold a magic show at the four crossroads and a keen sighted man should see it, ponder over it and reflect on it radically. Even as he sees it, ponders over it and reflects on it radically, he would find it empty, he would find it hollow, he would find it void of essence. What essence, monks, could there be in a magic show?

Even so, monks, whatever consciousness, be it past, future or present, in oneself or external, gross or subtle, inferior or superior, far or near, a monk sees it, ponders over it and reflects on it radically. Even as he sees it, ponders over it and reflects on it radically, he would find it empty, he would find it hollow, he would find it void of essence. What essence, monks, could there be in consciousness?
\end{quote}

So for the Buddha, consciousness is comparable to a magic show. This is a most extraordinary exposition, not to be found in any other philosophical system, because the soul theory tries to sit pretty on consciousness when all other foundations are shattered. But then, even this citadel itself the Buddha has described in this discourse as essenceless and hollow, as a magical illusion. Let us now try to clarify for ourselves the full import of this simile of the magic show.

A certain magician is going to hold a magic show in some hall or theatre. Among those who have come to see the magic show, there is a witty person with the wisdom eye, who tells himself: ``Today I must see the magic show inside out!''

With this determination he hides himself in a corner of the stage, unseen by others. When the magic show starts, this person begins to discover, before long, the secrets of the magician, his deceitful stock-in-trade -- counterfeits, hidden strings and buttons, secret pockets and false bottoms in his magic boxes. He observes clearly all the secret gadgets that the audience is unaware of. With this vision, he comes to the conclusion that there is no magic in any of those gadgets.

Some sort of disenchantment sets in. Now he has no curiosity, amazement, fright or amusement that he used to get whenever he watched those magic shows. Instead he now settles into a mood of equanimity. Since there is nothing more for him to see in the magic show, he mildly turns his attention towards the audience. Then he sees the contrast. The entire hall is a sea of craned necks, gaping mouths and goggle-eyes with `Ahs' and `Ohs' and whistles of speechless amazement. At this sorry sight, he even feels remorseful that he himself was in this same plight before. So in this way he sees through the magic show -- an `insight' instead of a `sight'.

When the show ends, he steps out of the hall and tries to slink away unseen. But he runs into a friend of his, who also was one of the spectators. Now he has to listen to a vivid commentary on the magic show. His friend wants him to join in his appreciation, but he listens through with equanimity. Puzzled by this strange reserved attitude, the friend asks:

\clearpage

``Why, you were in the same hall all this time, weren't you?''\\
``Yes, I was.''\\
``Then were you sleeping?''\\
``Oh, no.''\\
``You weren't watching closely, I suppose.''\\
``No, no, I was watching it all right, maybe I was watching too closely.''\\
``You say you were watching, but you don't seem to have seen the show.''\\
``No, I saw it. In fact I saw it so well that I missed the show.''

The above dialogue between the man who watched the show with discernment and the one who watched with naive credulity should give a clue to the riddle-like proclamations of the Buddha in the \emph{Kāḷakārāmasutta}. The Buddha also was confronted with the same problematic situation after his enlightenment, which was an insight into the magic show of consciousness.

That man with discernment hid himself in a corner of the stage to get that insight. The Buddha also had to hide in some corner of the world stage for his enlightenment. The term \emph{paṭisallāna}, `solitude', has a nuance suggestive of a hide-away. It is in such a hide-away that the Buddha witnessed the interior of the six-fold sense-base. The reason for his equanimity towards conflicting views about truth and falsehood in the world, as evidenced by this discourse, is the very insight into the six sense-bases.

First of all, let us try to compare our parable with the discourse proper. Now the Buddha declares:

\begin{quote}
\emph{Yaṁ, bhikkhave, sadevakassa lokassa samārakassa sabrahmakassa sassamaṇabrāhmaṇiyā pajāya sadevamanussāya diṭṭhaṁ sutaṁ mutaṁ viññātaṁ pattaṁ pariyesitaṁ anuvicaritaṁ manasā, tam ahaṁ jānāmi.}

\emph{Yaṁ, bhikkhave, sadevakassa lokassa samārakassa sabrahmakassa sassamaṇabrāhmaṇiyā pajāya sadevamanussāya diṭṭhaṁ sutaṁ mutaṁ viññātaṁ pattaṁ pariyesitaṁ anuvicaritaṁ manasā, tam ahaṁ abhaññāsiṁ. Taṁ tathāgatasssa viditaṁ, taṁ tathāgato na upaṭṭhāsi.}\footnote{A II 25, \emph{Kāḷakārāmasutta}}

Monks, whatsoever in the world, with its gods, Māras and Brahmas, among the progeny consisting of recluses and Brahmins, gods and men, whatsoever is seen, heard, sensed, cognized, sought after and pondered over by the mind, all that do I know.

Monks, whatsoever in the world, with its gods, Māras and Brahmas, among the progeny consisting of recluses and Brahmins, gods and men, whatsoever is seen, heard, sensed, cognized, sought after and pondered over by the mind, that have I fully understood. All that is known to the Tathāgata, but the Tathāgata has not taken his stand upon it.
\end{quote}

Here the Buddha does not stop after saying that he knows all that, but goes on to declare that he has fully understood all that and that it is known to the Tathāgata. The implication is that he has seen through all that and discovered their vanity, hollowness and essencelessness. That is to say, he not only knows, but he has grown wiser. In short, he has seen the magic show so well as to miss the show.

Unlike in the case of those worldly spectators, the released mind of the Tathāgata did not find anything substantial in the magic show of consciousness. That is why he refused to take his stand upon the sense-data, \emph{taṁ tathāgato na upaṭṭhāsi}, ``the Tathāgata has not taken his stand upon it''. In contrast to the worldly philosophers, the \emph{tathāgatas} have no entanglement with all that, \emph{ajjhositaṁ natthi tathāgatānaṁ.}

The dialogue we have given might highlight these distinctions regarding levels of knowledge. It may also throw more light on the concluding statement that forms the gist of the discourse.

\begin{quote}
\emph{Iti kho, bhikkhave, tathāgato diṭṭhā daṭṭhabbaṁ diṭṭhaṁ na maññati, adiṭṭhaṁ na maññati, daṭṭhabbaṁ na maññati, daṭṭhāraṁ na maññati. Sutā sotabbaṁ sutaṁ na maññati, asutaṁ na maññati, sotabbaṁ na maññati, sotāraṁ na maññati. Mutā motabbaṁ mutaṁ na maññati, amutaṁ na maññati, motabbaṁ na maññati, motāraṁ na maññati. Viññātā viññātabbaṁ viññātaṁ na maññati, aviññātaṁ na maññati, viññātabbaṁ na maññati, viññātāraṁ na maññati.}

Thus, monks, a Tathāgata does not imagine a visible thing as apart from seeing, he does not imagine an unseen, he does not imagine a thing worth seeing, he does not imagine a seer. He~does not imagine an audible thing as apart from hearing, he does not imagine an unheard, he does not imagine a thing worth hearing, he does not imagine a hearer. He does not imagine a thing to be sensed as apart from sensation, he does not imagine an unsensed, he does not imagine a thing worth sensing, he does not imagine one who senses. He does not imagine a cognizable thing as apart from cognition, he does not imagine an uncognized, he does not imagine a thing worth cognizing, he does not imagine one who cognizes.
\end{quote}

It is like the hesitation of that man with discernment who, on coming out of the hall, found it difficult to admit categorically that he had seen the magic show. Since the Tathāgata had an insight into the mechanism of the six-fold sense-base, that is to say, its conditioned nature, he understood that there is no one to see and nothing to see -- only a seeing is there.

The dictum of the \emph{Bāhiyasutta} ``in the seen just the seen'', \emph{diṭṭhe diṭṭhamattaṁ},\footnote{Ud 8, \emph{Bāhiyasutta}} which we cited the other day, becomes more meaningful now. Only a seeing is there. Apart from the fact of having seen, there is nothing substantial to see. There is no magic to see. \emph{Diṭṭhā daṭṭhabbaṁ diṭṭhaṁ na maññati,} he does not imagine a sight worthwhile apart from the seen. There is no room for a conceit of having seen a magic show.

On the other hand, it is not possible to deny the fact of seeing, \emph{adiṭṭhaṁ na maññati}. He does not imagine an unseen. Now that friend was curious whether this one was asleep during the magic show, but that was not the case either.

\emph{Daṭṭhabbaṁ na maññati}, the Tathāgata does not imagine a thing worthwhile seeing. The equanimity of that witty man was so much that he turned away from the bogus magic show to have a look at the audience below. This way we can understand how the Tathāgata discovered that there is only a seen but nothing worthwhile seeing.

Likewise the phrase \emph{daṭṭhāraṁ na maññati}, he does not imagine a seer, could also be understood in the light of this parable. All those who came out of that hall, except this discerning one, were spectators. He was not one of the audience, because he had an insight into the magic show from his hiding place on the stage.

The statement \emph{tam ahaṁ `na jānāmī'ti vadeyyaṁ, taṁ mama assa musā}, ``if I were to say, that I do not know, it would be a falsehood in me'', could similarly be appreciated in the light of the dialogue after the magic show.

The discerning one could not say that he was not aware of what was going on, because he was fully awake during the magic show. Nor can he say that he was aware of it in the ordinary sense. An affirmation or negation of both standpoints would be out of place. This gives us a clue to understand the two statements of the Tathāgata to the effect that he is unable to say that he both knows and does not know, \emph{jānāmi ca na ca jānāmi}, and neither knows nor does not know, \emph{n'eva jānāmi na na jānāmi}.

All this is the result of his higher understanding, indicated by the word \emph{abhaññāsiṁ}. The Tathāgata saw the magic show of consciousness so well as to miss the show, from the point of view of the worldlings.

Now we come to the conclusive declaration:

\begin{quote}
\emph{Iti kho, bhikkhave, tathāgato diṭṭha-suta-muta-viññātabbesu dhammesu tādī yeva tādī, tamhā ca pana tādimhā añño tādī uttaritaro vā paṇītataro vā natthī'ti vadāmi.}

Thus, monks, the Tathāgata, being such in regard to all phenomena, seen, heard, sensed and cognized, is such. Moreover than he who is such there is none other higher or more excellent, I declare.
\end{quote}

The other day we discussed the implications of the term \emph{tādī}.\footnote{See \emph{Sermon 25}} The term is usually explained as signifying the quality of remaining unshaken before the eight worldly vicissitudes. But in this context, it has a special significance. It implies an equanimous attitude towards dogmatic views and view-holders. This attitude avoids categorical affirmation or negation regarding the question of truth and falsehood. It grants a relative reality to those viewpoints.

This is the moral behind the hesitation to give clear-cut answers to that inquisitive friend in our pithy dialogue. It is not the outcome of a dilly-dally attitude. There is something really deep. It is the result of an insight into the magic show. The reason for this suchness is the understanding of the norm of dependent arising, known as \emph{tathatā}.

It is obvious from the expositions of the norm of dependent arising that there are two aspects involved, namely, \emph{anuloma}, direct order, and \emph{paṭiloma}, indirect order.

The direct order is to be found in the first half of the twelve linked formula, beginning with the word \emph{avijjāpaccayā saṅkhārā}, ``dependent on ignorance, preparations'', while the indirect order is given in the second half with the words, \emph{avijjāya tveva asesavirāganirodhā} etc., ``with the remainderless fading away and cessation of ignorance'' etc.

The implication is that where there is ignorance, aggregates of grasping get accumulated, which, in other words, is a heaping up of suffering. That is a fact. But then, when ignorance fades away and ceases, they do not get accumulated.

Now, with this magic show as an illustration, we can get down to a deeper analysis of the law of dependent arising. In a number of earlier sermons, we have already made an attempt to explain a certain deep dimension of this law, with the help of illustrations from the dramatic and cinematographic fields. The magic show we have brought up now is even more striking as an illustration.

In the case of the cinema, the background of darkness we compared to the darkness of ignorance. Because of the surrounding darkness, those who go to the cinema take as real whatever they see on the screen and create for themselves various moods and emotions.

In the case of the magic show, the very ignorance of the tricks of the magician is what accounts for the apparent reality of the magic performance. Once the shroud of ignorance is thrown off, the magic show loses its magic for the audience. The magician's secret stock-in-trade gave rise to the \emph{saṅkhāras} or preparations with the help of which the audience created for themselves a magic show.

To that discerning man, who viewed the show from his hiding place on the stage, there were no such preparations. That is why he proverbially missed the show.

The same principle holds good in the case of the magical illusion, \emph{māyā}, that is consciousness. A clear instance of this is the reference in the \emph{Mahāvedallasutta} of the \emph{Majjhima Nikāya} to \emph{viññāṇa}, consciousness, and \emph{paññā}, wisdom, as two conjoined psychological states.

They cannot be separated one from the other, \emph{saṁsaṭṭhā no visaṁsaṭṭhā}.\footnote{M I 292, \emph{Mahāvedallasutta}} But they can be distinguished functionally. Out of them, wisdom is to be developed, while consciousness is to be comprehended, \emph{paññā bhāvetabbā, viññāṇaṁ pariññeyyaṁ}.

The development of wisdom is for the purpose of comprehending consciousness and comprehended consciousness proves to be empty, essenceless and hollow. It is such a transformation that took place within the person who watched the magic show with discernment. He watched it too closely, so much so, that the preparations, \emph{saṅkhārā}, in the form of the secret stock-in-trade of the magician, became ineffective and nugatory.

This makes clear the connection between ignorance, \emph{avijjā}, and preparations, \emph{saṅkhārā}. That is why ignorance takes precedence in the formula of dependent arising. Preparations owe their effectiveness to ignorance. They are dependent on ignorance. To understand preparations for what they are is knowledge. Simultaneous with the arising of that knowledge, preparations become mere preparations, or pure preparations, \emph{suddha saṅkhārā}.

This gives us the clue to unravel the meaning of the verse in the \emph{Adhimutta Theragāthā}, quoted earlier.

\begin{quote}
\emph{Suddhaṁ dhammasamuppādaṁ,}\\
\emph{suddhaṁ saṅkhārasantatiṁ,}\\
\emph{passantassa yathābhūtaṁ,}\\
\emph{na bhayaṁ hoti gāmani.}\footnote{Th 716, \emph{Adhimutta Theragāthā}; see also \emph{Sermon 8}}

To one who sees\\
The arising of pure \emph{dhammas}\\
And the sequence of pure preparations, as they are,\\
There is no fear, oh headman.
\end{quote}

In a limited sense, we can say that graspings relating to a magic show did not get accumulated in the mind of that discerning person, while his friend was gathering them eagerly. The latter came out of the hall as if coming out of the magic world. He had been amassing graspings proper to a magic world due to his ignorance of those preparations.

From this one may well infer that if at any point of time consciousness is comprehended by wisdom, preparations, \emph{saṅkhārā}, become mere preparations, or pure preparations. Being influx-free, they do not go to build up a prepared, \emph{saṅkhata}. They do not precipitate an amassing of grasping, \emph{upādāna}, to bring about an existence, \emph{bhava.} This amounts to a release from existence.

One seems to be in the world, but one is not of the world. That man with discernment was in the hall all that time, but it was as if he was not there.

Let us now go deeper into the implications of the term \emph{tādī}, `such', with reference to the law of dependent arising, known as \emph{tathatā}, `suchness'. From the dialogue that followed the magic show, it is clear that there are two points of view. We have here a question of two different points of view. If we are to explain these two viewpoints with reference to the law of dependent arising, we may allude to the distinction made for instance in the \emph{Nidāna Saṁyutta} between the basic principle of dependent arising and the phenomena dependently arisen. We have already cited the relevant declaration.

\begin{quote}
\emph{Paṭiccasamuppādañca vo, bhikkhave, desessāmi paṭiccasamuppanne ca dhamme.}\footnote{\href{https://suttacentral.net/sn12.20/pli/ms}{SN 12.20 / S II 25}, \emph{Paccayasutta}; see \emph{Sermon 2}}

Monks, I shall preach to you dependent arising and things that are dependently arisen.
\end{quote}

Sometimes two significant terms are used to denote these two aspects, namely \emph{hetu} and \emph{hetusamuppannā dhammā}.

About the \emph{ariyan} disciple, be he even a stream-winner, it is said that his understanding of dependent arising covers both these aspects, \emph{hetu ca sudiṭṭho hetusamuppannā ca dhammā}.\footnote{\href{https://suttacentral.net/an6.95/pli/ms}{AN 6.95 / A III 440}, \emph{Catuttha-abhabbaṭṭhānasutta}} The cause, as well as the things arisen from a cause, are well seen or understood by him.

As we pointed out in our discussion of the hill-top festival in connection with the Upatissa and Kolita episode,\footnote{See \emph{Sermon 5}} the disenchantment with the hill-top festival served as a setting for their encounter with the venerable Assaji. As soon as venerable Assaji uttered the significant pithy verse:

\begin{quote}
\emph{Ye dhammā hetuppabhavā,}\\
\emph{tesaṁ hetuṁ tathāgato āha,}\\
\emph{tesañca yo nirodho,}\\
\emph{evaṁ vādī mahāsamaṇo.}\footnote{\href{https://suttacentral.net/pli-tv-kd1/pli/ms}{Vin I 40}, \emph{Mahāvagga}}

Of things that proceed from a cause,\\
Their cause the Tathāgata has told,\\
And also their cessation,\\
Thus teaches the great ascetic.
\end{quote}

The wandering ascetic Upatissa, who was to become venerable Sāriputta later, grasped the clue to the entire \emph{saṁsāric} riddle then and there, and discovered the secret of the magic show of consciousness, even by the first two lines. That was because he excelled in wisdom.

As soon as he heard the lines ``of things that proceed from a cause, their cause the Tathāgata has told'', he understood the basic principle of dependent arising, \emph{yaṁ kiñci samudayadhammaṁ, sabbaṁ taṁ nirodhadhammaṁ}, ``whatever is of a nature to arise, all that is of a nature to cease''. The wandering ascetic Kolita, however, became a stream-winner only on hearing all four lines.

This pithy verse has been variously interpreted. But the word \emph{hetu} in this verse has to be understood as a reference to the law of dependent arising. When asked what \emph{paṭicca samuppāda} is, the usual answer is a smattering of the twelve-linked formula in direct and reverse order. The most important normative prefatory declaration is ignored:

\begin{quote}
\emph{Imasmiṁ sati idaṁ hoti,}\\
\emph{imassa uppādā idaṁ upajjati,}\\
\emph{imasmiṁ asati idaṁ na hoti,}\\
\emph{imassa nirodhā idaṁ nirujjhati.}

This being, this comes to be;\\
With the arising of this, this arises;\\
This not being, this does not come to be;\\
With the cessation of this, this ceases.
\end{quote}

This statement of the basic principle of dependent arising is very often overlooked. It is this basic principle that finds expression in that pithy verse.

The line \emph{ye dhammā hetuppabhavā}, ``of things that proceed from a cause'', is generally regarded as a reference to the first link \emph{avijjā}. But this is not the case. All the twelve links are dependently arisen, and \emph{avijjā} is no exception. Even ignorance arises with the arising of influxes, \emph{āsavasamudayā avijjāsamudayo}.\footnote{\href{https://suttacentral.net/mn9/pli/ms}{MN 9 / M I 54}, \emph{Sammādiṭṭhisutta}} Here we have something extremely deep.

The allusion here is to the basic principle couched in the phrases \emph{imasmiṁ sati idaṁ hoti} etc. In such discourses as the \emph{Bahudhātukasutta} the twelve-linked formula is introduced with a set of these thematic phrases, which is then related to the formula proper with the conjunctive ``that is to say'', \emph{yadidaṁ}.\footnote{\href{https://suttacentral.net/mn115/pli/ms}{MN 115 / M III 63}, \emph{Bahudhātukasutta}}

This conjunctive clearly indicates that the twelve-linked formula is an illustration. The twelve links are therefore things dependently arisen, \emph{paṭicca samuppannā dhammā}. They are all arisen from a cause, \emph{hetuppabhavā dhammā}.

So even ignorance is not the cause. The cause is the underlying principle itself. This being, this comes to be. With the arising of this, this arises. This not being, this does not come to be. With the cessation of this, this ceases. This is the norm, the suchness, \emph{tathatā}, that the Buddha discovered.

That man with discernment at the magic show, looking down at the audience with commiseration, had a similar sympathetic understanding born of realization: ``I too have been in this same sorry plight before''.

Due to ignorance, a sequence of phenomena occurs, precipitating a heaping of graspings. With the cessation of ignorance, all that comes to cease. It is by seeing this cessation that the momentous inner transformation took place. The insight into this cessation brings about the realization that all what the worldlings take as absolutely true, permanent or eternal, are mere phenomena arisen from the mind. \emph{Manopubbangamā dhammā}, mind is the forerunner of all mind-objects.\footnote{Dhp 1, \emph{Yamakavagga}} One comes to understand that all what is arisen is bound to cease, and that the cessation can occur here and~now.

In discussing the formula of \emph{paṭicca samuppāda}, the arising of the six sense-bases is very often explained with reference to a mother's womb. It is the usual practice to interpret such categories as \emph{nāma-rūpa}, name-and-form, and \emph{saḷāyatana}, six sense-bases, purely in physiological terms. But for the Buddha the arising of the six sense-bases was not a stage in the growth of a foetus in the mother's womb.

It was through wisdom that he saw the six bases of sense-contact arising then and there, according to the formula beginning with \emph{cakkhuñca paṭicca rūpe ca uppajjati cakkhuviññāṇaṁ}, ``dependent on eye and forms arises eye-consciousness'' etc. They are of a nature of arising and ceasing, like that magic show. Everything in the world is of a nature to arise and cease.

The words \emph{ye dhammā hetuppabhavā}, ``of things that proceed from a cause'' etc., is an enunciation of that law. Any explanation of the law of dependent arising should rightly begin with the basic principle \emph{imasmiṁ sati idaṁ hoti,} ``this being, this comes to be'' etc.

This confusion regarding the way of explaining \emph{paṭicca samuppāda} is a case of missing the wood for the trees. It is as if the Buddha stretches his arm and says: ``That is a forest'', and one goes and catches hold of a tree, exclaiming: ``Ah, this is the forest''. To rattle off the twelve links in the hope of grasping the law of \emph{paṭicca samuppāda} is like counting the number of trees in order to see the forest.

The subtlest point here is the basic principle involved. ``This being, this comes to be. With the arising of this, this arises. This not being, this does not come to be. With the cessation of this, this ceases''.

Let us now examine the connection between the law of dependent arising, \emph{paṭicca samuppāda}, and things dependently arisen, \emph{paṭiccasamuppannā dhammā}.

Worldings do not even understand things dependently arisen as `dependently arisen'. They are fully involved in them. That itself is \emph{saṁsāra}. One who has seen the basic principle of \emph{paṭicca samuppāda} understands the dictum, \emph{avijjāya sati saṅkhārā honti}, preparations are there only when ignorance is there.\footnote{S II 78, \emph{Ariyasāvakasutta}} So he neither grasps ignorance, nor does he grasp preparations.

In fact, to dwell on the law of dependent arising is the way to liberate the mind from the whole lot of dependently arisen things. Now why do we say so? Everyone of those twelve links, according to the Buddha, is impermanent, prepared, dependently arisen, of a nature to wither away, wear away, fade away and cease, \emph{aniccaṁ, saṅkhataṁ, paṭicca samuppannaṁ, khayadhammaṁ, vayadhammaṁ, virāgadhammaṁ, nirodhadhammaṁ}.\footnote{S II 27, \emph{Paccayasutta}}

The very first link \emph{avijjā} is no exception. They are impermanent because they are made up or prepared, \emph{saṅkhata}. The term \emph{saṅkhataṁ} has nuances of artificiality and spuriousness. All the links are therefore unreal in the highest sense. They are dependent on contact, \emph{phassa}, and therefore dependently arisen. It is in their nature to wither away, wear away, fade away and cease.

When one has understood this as a fact of experience, one brings one's mind to rest, not on the things dependently arisen, but on the law of dependent arising itself.

There is something extraordinary about this. One must not miss the wood for the trees. When the Buddha stretches his arm and says: ``That is a forest'', he does not expect us to go and grasp any of the trees, or to go on counting them, so as to understand what a forest is. One has to get a synoptic view of it from here itself. Such a view takes into account not only the trees, but also the intervening spaces between them, all at one synoptic glance.

In order to get a correct understanding of \emph{paṭicca samuppāda} from a pragmatic point of view, one has to bring one's mind to rest on the norm that obtains between every two links. But this is something extremely difficult, because the world is steeped in the notion of duality. It grasps either this end, or the other end. Hard it is for the world to understand the stance of the \emph{arahant} couched in the cryptic phrase:

\begin{quote}
\emph{nev'idha na huraṁ na ubhayam antare},\footnote{Ud 8, \emph{Bāhiyasutta}}

neither here nor there nor in between the two.
\end{quote}

The worldling is accustomed to grasp either this end or the other end. For instance, one may grasp either ignorance, \emph{avijjā}, or preparations, \emph{saṅkhārā}. But here we have neither. When one dwells on the interrelation between them, one is at least momentarily free from ignorance as well as from the delusive nature of preparations.

Taking the magic show itself as an illustration, let us suppose that the magician is performing a trick, which earlier appeared as a miracle. But now that one sees the counterfeits, hidden strings and secret bottoms, one is aware of the fact that the magical effect is due to the evocative nature of those preparations. So he does not take seriously those preparations. His ignorance is thereby reduced to the same extent.

This is how each of those links gets worn out, as the phrase \emph{khayadhammaṁ, vayadhammaṁ, virāgadhammaṁ, nirodhadhammaṁ} suggests. All the links are of a nature to wither away, wear away, fade away and cease. So, then, preparations are there only when ignorance is there. The preparations are effective only so long as ignorance is there. With the arising of ignorance, preparations arise. When ignorance is not there, preparations lose their provenance. With the complete fading away and cessation of ignorance, preparations, too, fade away and cease without residue. This, then, is the relationship between those two links.

Let us go for another instance to illustrate this point further. \emph{Saṅkhārapaccayā viññāṇaṁ}, ``dependent on preparations is consciousness''. Generally, the worldlings are prone to take consciousness as a compact unit. They regard it as their self or soul. When everything else slips out from their grasp, they grasp consciousness as their soul, because it is invisible.

Now if someone is always aware that consciousness arises dependent on preparations, that with the arising of preparations consciousness arises -- always specific and never abstract -- consciousness ceases to appear as a monolithic whole.

This particular eye-consciousness has arisen because of eye and forms. This particular ear-consciousness has arisen because of ear and sound, and so on. This kind of reflection and constant awareness of the part played by preparations in the arising of consciousness will conduce to the withering away, wearing away and fading away of consciousness. Disgust, disillusionment and dejection in regard to consciousness is what accounts for its complete cessation, sooner or later.

Consciousness is dependent on preparations, and name-and-form, \emph{nāma-rūpa}, is dependent on consciousness. The worldling does not even recognize \emph{nāma-rūpa} as such. We have already analyzed the mutual relationship between name-and-form as a reciprocity between nominal form and formal name.\footnote{See \emph{Sermon 1}} They always go together and appear as a reflection on consciousness. Here is a case of entanglement within and an entanglement without, \emph{anto jaṭā bahi jaṭā}.\footnote{\href{https://suttacentral.net/sn1.23/pli/ms}{SN 1.23 / S I 13}, \emph{Jaṭāsutta}}

We brought in a simile of a dog on a plank to illustrate the involvement with name-and-form. When one understands that this name-and-form, which the world takes as real and calls one's own, is a mere reflection on consciousness, one does not grasp it either.

To go further, when one attends to the fact that the six sense-bases are dependent on name-and-form, and that they are there only as long as name-and-form is there, and that with the cessation of name-and-form the six sense-bases also cease, one is attuning one's mind to the law of dependent arising, thereby weaning one's mind away from its hold on dependently arisen things.

Similarly, contact arises in dependence on the six sense-bases. Generally, the world is enslaved by contact. In the \emph{Nandakovādasutta} of the \emph{Majjhima Nikāya} there is a highly significant dictum, stressing the specific character of contact as such.

\begin{quote}
\emph{Tajjaṁ tajjaṁ, bhante, paccayaṁ paṭicca tajjā tajjā vedanā uppajjanti; tajjassa tajjassa paccayassa nirodhā tajjā tajjā vedanā nirujjhanti.}\footnote{M III 273, \emph{Nandakovādasutta}}

Dependent on each specific condition, venerable sir, specific feelings arise, and with the cessation of each specific condition, specific feelings cease.
\end{quote}

The understanding that contact is dependent on the six sense-bases enables one to overcome the delusion arising out of contact. Since it is conditioned and limited by the six sense-bases, with their cessation it has to cease. Likewise, to attend to the specific contact as the cause of feeling is the way of disenchantment with both feeling and contact.

Finally, when one understands that this existence is dependent on grasping, arising out of craving, one will not take existence seriously. Dependent on existence is birth, \emph{bhavapaccayā jāti}. While the magic show was going on, the spectators found themselves in a magic world, because they grasped the magic in it. Even so, existence, \emph{bhava}, is dependent on grasping, \emph{upādāna}.

Just as one seated on this side of a parapet wall might not see what is on the other side, what we take as our existence in this world is bounded by our parents from the point of view of birth. What we take as death is the end of this physical body. We are ignorant of the fact that it is a flux of preparations, \emph{saṅkhārasantati}.\footnote{Th 716, \emph{Adhimutta Theragāthā}} Existence is therefore something prepared or made up. Birth is dependent on existence.

Sometimes we happen to buy from a shop an extremely rickety machine deceived by its paint and polish, and take it home as a brand new thing. The very next day it goes out of order. The newly bought item was born only the previous day, and now it is out of order, to our disappointment.

So is our birth with its unpredictable vicissitudes, taking us through decay, disease, sorrow, lamentation, pain, grief and despair. This is the price we pay for this brand new body we are blessed with in this existence.

In this way we can examine the relation between any two links of the formula of dependent arising. It is the insight into this norm that constitutes the understanding of \emph{paṭicca samuppāda}, and not the parrot-like recitation by heart of the formula in direct and reverse order.

Of course, the formulation in direct and reverse order has its own special significance, which highlights the fact that the possibility of a cessation of those twelve links lies in their arising nature itself. Whatever is of a nature to arise, all that is of a nature to cease, \emph{yaṁ kiñci samudayadhammaṁ, sabbaṁ taṁ nirodhadhammaṁ}. As for the \emph{arahant}, he has realized this fact in a way that the influxes are made extinct.

To go further into the significance of the formula, we may examine why ignorance, \emph{avijjā}, takes precedence in it. This is not because it is permanent or uncaused. The deepest point in the problem of release from \emph{saṁsāra} is traceable to the term \emph{āsavā}, or influxes. Influxes are sometimes reckoned as fourfold, namely those of sensuality, \emph{kāmāsavā}, of existence, \emph{bhavāsavā}, of views, \emph{diṭṭhāsavā}, and of ignorance, \emph{avijjāsavā}.

But more often, in contexts announcing the attainment of \emph{arahanthood}, the standard reference is to three types of influxes, \emph{kāmāsavā pi cittaṁ vimuccati, bhavāsavā pi cittaṁ vimuccati, āvijjāsavā pi cittaṁ vimuccati,} the mind is released from influxes of sensuality, existence and ignorance. This is because the influxes of ignorance could easily include those of views as well.

The term \emph{āsavā} implies those corrupting influences ingrained in beings due to \emph{saṁsāric} habits. They have a tendency to flow in and tempt beings towards sensuality, existence and ignorance.

It might be difficult to understand why even ignorance is reckoned as a kind of influxes, while it is recognized as the first link in the chain of dependent arising. Ignorance or ignoring is itself a habit. There is a tendency in \emph{saṁsāric} beings to grope in darkness and dislike light. They have a tendency to blink at the light and ignore. It is easy to ignore and forget. This forgetting trait enables them to linger long in \emph{saṁsāra}.

Ignorance as a kind of influxes is so powerful that even the keenest in wisdom cannot attain \emph{arahanthood} at once. The wheel of Dhamma has to turn four times, hence the fourfold distinction as stream-winner, once returner, non-returner and \emph{arahant}. The difficulty of combating this onslaught of influxes is already insinuated by the term \emph{sattakkhattuparama}, `seven more lives at the most',\footnote{E.g. A V 120, \emph{Niṭṭhaṅgatasutta}} designating a stream-winner, and the term \emph{sakadāgāmī}, `once-returner'.

The way to cut off these influxes is the very insight into the law of dependent arising. Sometimes the path is defined as the law of dependent arising itself. That doesn't mean the ability to rattle off the twelve links by heart, but the task of bringing the mind to rest on the norm of \emph{paṭicca samuppāda} itself.

\begin{quote}
\emph{Imasmiṁ sati idaṁ hoti,}\\
\emph{imassa uppādā idaṁ upajjati,}\\
\emph{imasmiṁ asati idaṁ na hoti,}\\
\emph{imassa nirodhā idaṁ nirujjhati.}

This being, this comes to be;\\
With the arising of this, this arises;\\
This not being, this does not come to be;\\
With the cessation of this, this ceases.
\end{quote}

It is an extremely difficult task, because the mind tends to slip off. The habitual tendency is to grasp this one or the other. The worldling, for the most part, rests on a duality. Not to cling even to the middle is the ideal of an \emph{arahant}. That is the implication of the conclusive statement in the advice to Bāhiya, \emph{nev'idha na huraṁ na ubhayam antarena}, ``neither here, nor there, no in between the two''.\footnote{Ud 8, \emph{Bāhiyasutta}}

For clarity's sake, let us quote the relevant section in full:

\begin{quote}
\emph{Yato tvaṁ Bāhiya na tena, tato tvaṁ Bāhiya na tattha. Yato tvaṁ Bāhiya na tattha, tato tvaṁ Bāhiya nev'idha na huraṁ na ubhayamantarena. Es' ev' anto dukkhassa}.

And when, Bāhiya, you are not by it, then, Bāhiya, you are not in it. And when, Bāhiya, you are not in it, then, Bāhiya, you are neither here nor there nor in between. This, itself, is the end of suffering.
\end{quote}

So one who has fully understood the norm of \emph{paṭicca samuppāda} is not attached to ignorance, nor is he attached to preparations, since he has seen the relatedness between them. He is attached neither to preparations nor to consciousness, having seen the relatedness between them. The insight into this dependent arising and ceasing promotes such a detached attitude.

It is this insight that inculcated in the Tathāgata that supreme and excellent suchness. His neutral attitude was not the result of any lack of knowledge, or tactical eel wriggling, as in the case of Sañjaya Belaṭṭhiputta.

Why does the Tathāgata not declare the sense-data categorically as true or false? He knows that, given ignorance, they are true, and that they are falsified only when ignorance fades away in one who sees the cessation. It is for such a person that the sense-bases appear as false and consciousness appears as a conjurer's trick.

Fortified with that understanding, he does not categorically assert the sense-data as true, nor does he reprimand those who assert them as the truth. That is why the Buddha advocates a tolerant attitude in this discourse. This is the typical attitude of an understanding elder to the questions put by an inquisitive toddler.

Generally, the dogmatists in the world are severally entrenched in their own individual viewpoints, as the line \emph{paccekasaccesu puthū niviṭṭhā} suggests.\footnote{Sn 824, \emph{Pasūrasutta}, see \emph{Sermon 25}} We explained the term \emph{sayasaṁvuta} as on a par with the phrase \emph{paccekasaccesu}. The problematic term \emph{sayasaṁvuta} is suggestive of virulent self-opinionatedness. Why are they committed and limited by their own views? Our quotation from the \emph{Cūḷa-Viyūhasutta} holds the answer.

\begin{quote}
\emph{Na h'eva saccāni bahūni nānā,}\\
\emph{aññatra saññāya niccāni loke,}\footnote{Sn 886, \emph{Cūḷa-Viyūhasutta}}

There are no several and various truths,\\
That are permanent in the world, apart from perception.
\end{quote}

According to one's level of perception, one forms a notion of reality. To those in the audience the tricks of the magician remained concealed. It is that ignorance which aroused preparations, \emph{saṅkhārā}, in them.

A typical illustration of individual truths, \emph{paccekasacca}, is found in the chapter titled \emph{Jaccandha}, `congenitally blind', in the \emph{Udāna}. There the Buddha brings up a parable of the blind men and the elephant.\footnote{Ud 67, \emph{Paṭhamanānātitthiyasutta}}

A certain king got a crowd of congenitally blind men assembled, and having made them touch various limbs of an elephant, asked them what an elephant looks like. Those who touched the elephant's head compared the elephant to a pot, those who touched its ears compared it to a winnowing basket, those who touched its tusk compared it to a ploughshare and so forth.

The dogmatic views in the world follow the same trend. All that is due to contact, \emph{phassapaccayā}, says the Buddha in the \emph{Brahmajālasutta} even with reference to those who have supernormal knowledges, \emph{abhiññā}.\footnote{D I 42, \emph{Brahmajālasutta}} Depending on name-and-form, which they grasped, they evolved dogmatic theories, based on their perceptions, spurred on by sense-contact. Their dogmatic involvement is revealed by the thematic assertion \emph{idam eva saccaṁ, mogham aññaṁ}, ``this alone is true, all else is false''.

The Buddha had no dogmatic involvement, because he had seen the cessation of consciousness. Even the mind ceases, and mind-objects fade away. That is why the Buddha was tolerantly neutral. On many such issues, silence happens to be the answer.

This brings us to an extremely deep dimension of this Dhamma. Just as that man with discerning wisdom at the magic show had difficulties in coming to terms with the naive magic fan, so the Buddha, too, had to face situations where problems of communication cropped up.

We come across such an instance in the \emph{Mahāparinibbānasutta}. On his way to Kusinārā, to attain \emph{parinibbāna}, the Buddha happened to rest under a tree for a while, to overcome fatigue. Pukkusa of Malla, a disciple of Āḷāra Kālāma, who was coming from Kusinārā on his way to Pāvā, saw the Buddha seated there and approached him. After worshipping him he made the following joyful utterance: \emph{Santena vata, bhante, pabbajitā vihārena viharanti}, ``Venerable Sir, those who have gone forth are indeed living a peaceful life''.\footnote{D II 130, \emph{Mahāparinibbānasutta}}

Though it was apparently a compliment for the Buddha, he came out with an episode, which was rather in praise of his teacher Āḷāra Kālāma, who had attained to the plane of nothingness, \emph{ākiñcaññāyatana}.

\begin{quote}
While on a long journey, my teacher Āḷāra Kālāma sat under a wayside tree for noonday siesta. Just then five-hundred carts were passing by. After the carts had passed that spot, the man who was following them walked up to Āḷāra Kālāma and asked him:

``Venerable sir, did you see about five-hundred carts passing by?''\\
``No, friend, I didn't see.''

``But, Venerable sir, didn't you even hear the sound?''\\
``No, friend, I didn't hear the sound.''

``Venerable sir, were you asleep, then?''\\
``No, friend, I was not asleep.''

``Were you conscious, then, Venerable sir?''\\
``Yes, friend.''

``So, then, venerable sir, while being conscious and awake, you neither saw nor heard as many as five-hundred carts passing by. All the same your double robe is bespattered with mud.''

``Yes, friend.''
\end{quote}

And then, Venerable Sir, that man was highly impressed by it, and paid the following compliment to Āḷāra Kālāma:

\begin{quote}
``It is a wonder, it is a marvel, what a peaceful life those who have gone forth are leading, so much so that one being conscious and awake would neither see nor hear as many as five-hundred carts passing by.''
\end{quote}

When Pukkusa cited this incident in praise of Āḷāra Kālāma, the Buddha asked him:

\begin{quote}
``What do you think, Pukkusa, which of these two feats is more difficult to accomplish, that one being conscious and awake would neither see nor hear as many as five-hundred carts passing by, or that while being conscious and awake, one would not see or hear the streaks of lightening and peals of thunder in the midst of a torrential downpour?''
\end{quote}

When Pukkusa grants that the latter feat is by far the more difficult to accomplish, the Buddha comes out with one of his past experiences.

\begin{quote}
``At one time, Pukkusa, I was staying in a chaff house at Ātumā, and there was a torrential downpour, with streaks of lightening and peals of thunder, during the course of which two farmers -- brothers -- and four bulls were struck down dead. A big crowd of people had gathered at the spot. Coming out of the chaff house, I was pacing up and down in open air when a man from that crowd walked up to me and worshipped me, and respectfully stood on one side. Then I asked him:

``Friend, why has this big crowd gathered here?''

``Just now, Venerable Sir, while it was raining in torrents with streaks of lightening and peals of thunder, two farmers -- brothers -- and four bulls were struck down dead. That is why a big crowd has gathered here. But where were you, Venerable Sir?''

``I was here itself, friend.''

``But didn't you see it, Venerable Sir?''

``No, friend, I didn't see it.''

``But didn't you hear the sound, Venerable Sir?''

``No, friend, I did not hear the sound.''

``But, then, Venerable Sir, were you asleep?''

``No, friend, I was not asleep.''

``But, Venerable Sir, were you conscious (\emph{saññī})?''

``Yes, friend.''
\end{quote}

And then, Pukkusa, that man expressed his surprise in the words:

\begin{quote}
``It is a wonder, it is a marvel, what a peaceful life those who have gone forth are leading, so much so that while being conscious and awake one would neither see nor hear the streaks of lightening and peals of thunder in the midst of a torrential downpour.''
\end{quote}

\begin{quote}
``With that he came out with his fervent faith in me, worshipped me, reverentially circumambulated me and left.''
\end{quote}

Some interpret this incident as an illustration of the Buddha's attainment to the cessation of perceptions and feelings. But if it had been the case, the words \emph{saññī samāno jāgaro}, ``while being conscious and awake'', would be out of place.

That man expressed his wonder at the fact that the Buddha, while being conscious and awake, had not seen or heard anything, though it was raining in torrents with streaks of lightening and peals of thunder. Nor can this incident be interpreted as a reference to the realm of nothingness, \emph{ākiñcaññāyatana}, in the context of the allusion to Āḷārā Kālāma and his less impressive psychic powers.

The true import of this extraordinary psychic feat has to be assessed with reference to the \emph{arahattaphalasamādhi}, we have already discussed.\footnote{See \emph{Sermons 16-19}}

The incident had occurred while the Buddha was seated in \emph{arahattaphalasamādhi}, experiencing the cessation of the six sense-spheres, equivalent to the cessation of the world. He had gone beyond the world -- that is why he didn't see or hear.

We are now in a position to appreciate meaningfully that much-vexed riddle-like verse we had quoted earlier from the \emph{Kalahavivādasutta.}

\begin{quote}
\emph{Na saññasaññī, na visaññasaññī,}\\
\emph{no pi asaññī na vibhūtasaññī,}\\
\emph{evaṁ sametassa vibhoti rūpaṁ,}\\
\emph{saññānidānā hi papañcasaṅkhā.}\footnote{Sn 874, \emph{Kalahavivādasutta}}

He is not conscious of normal perception,\\
\vin nor is he unconscious,\\
He is not devoid of perception,\\
\vin nor has he rescinded perception,\\
It is to one thus constituted\\
\vin that form ceases to exist,\\
For reckonings through prolificity\\
\vin have perception as their source.
\end{quote}

Perception is the source of all prolific reckonings, such as those that impelled the audience at the magic show to respond with the `Ahs', and `Ohs' and whistles. One is completely free from that prolific perception when one is in the \emph{arahattaphalasamādhi}, experiencing the cessation of the six sense-spheres.

As we had earlier cited:

\begin{quote}
\ldots{} one is neither percipient of earth in earth, nor of water in water, nor of fire in fire, nor of air in air, nor is one conscious of a `this world' in this world, nor of `another world' in another world \ldots{}
\end{quote}

and so on, but all the same `one is percipient', \emph{saññī ca pana assa.}\footnote{A V 318, \emph{Saññāsutta}, see \emph{Sermon 16}} Of what is he percipient or conscious? That is none other than what comes up as the title of these series of sermons, namely:

\begin{quote}
\emph{Etaṁ santaṁ, etaṁ paṇītaṁ, yadidaṁ sabbasaṅkhārasamatho sabbūpadhipaṭinissaggo taṇhakkhayo virāgo nirodho nibbānaṁ.}\footnote{\href{https://suttacentral.net/mn64/pli/ms}{MN 64 / M I 436}, \emph{Mahāmālunkyasutta}}

This is peaceful, this is excellent, namely the stilling of all preparations, the relinquishment of all assets, the destruction of craving, detachment, cessation, extinction.
\end{quote}
