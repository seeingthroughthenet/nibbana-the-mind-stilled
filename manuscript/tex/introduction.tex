\chapter{Introduction}

Nibbāna -- the ultimate goal of the Buddhist, has been variously understood and interpreted in the history of Buddhist thought. One who earnestly takes up the practice of the Noble Eightfold Path for the attainment of this goal, might sometimes be dismayed to find this medley of views confronting him. Right View, as the first factor of that path, has always to be in the vanguard in one's practice. In the interests of this Right View, which one has to progressively `straighten-up', a need for clarification before purification might sometimes be strongly felt. It was in such a context that the present series of 33 sermons on Nibbāna came to be delivered.

The invitation for this series of sermons came from my revered teacher, the late Venerable Mātara Sri Ñāṇārāma Mahāthera, who was the resident meditation teacher of Meetirigala Nissarana Vanaya Meditation Centre. Under his inspiring patronage these sermons were delivered once every fortnight before the group of resident monks of Nissarana Vanaya, during the period from the New Moon uposatha of 1988 Aug.~12th to the Full Moon uposatha of 1991 Jan.~30th.

The sermons, which were originally circulated on cassettes, began issuing in book-form only in 1997, when the first volume of the Sinhala series titled \emph{Nivane Niveema} came out, published by the \emph{Dharma Grantha Mudrana Bhāraya} (Dhamma Publications Trust) setup for the purpose in the Department of the Public Trustee, Sri Lanka. The series is scheduled to comprise 11 volumes, of which so far 9 have come out. The entire series is for free distribution as \emph{Dhamma dāna} -- `the gift of truth that excels all other gifts'. The sister series to come out in English will comprise 7 volumes of 5 sermons each, which will likewise be strictly for free distribution since Dhamma is price-less.

In these sermons I have attempted to trace the original meaning and significance of the Pāli term Nibbāna (Skt. \emph{Nirvāna}) based on the evidence from the discourses of the Pāli Canon. This led to a detailed analysis and a re-appraisal of some of the most controversial suttas on Nibbāna often quoted by scholars in support of their interpretations. The findings, however, were not presented as a dry scholastic exposition of mere academic interest. Since the sermons were addressed to a meditative audience keen on \emph{realizing Nibbāna}, edifying similes, metaphors and illustrations had their place in the discussion. The gamut of 33 sermons afforded sufficient scope for dealing with almost all the salient teachings in Buddhism from a practical point of view.

The present translation, in so far as it is faithful to the original, will reflect the same pragmatic outlook. While the findings could be of interest even to the scholar bent on \emph{theorizing on Nibbāna}, it is hoped that the mode of presentation will have a special appeal for those who are keen on \emph{realizing}~it.

I would like to follow up these few prefatory remarks with due acknowledgements to all those who gave their help and encouragement for bringing out this translation:

To Venerable Anālayo for transcribing the tape recorded translations and the meticulous care and patience with which he has provided references to the P.T.S. editions.

To Mr.~U. Mapa, presently the Ambassador for Sri Lanka in Myanmar, for his yeoman service in taking the necessary steps to establish the Dhamma Publications Trust in his former capacity as the Public Trustee of Sri Lanka.

To Mr.~G.T. Bandara, Director, Royal Institute, 191, Havelock Road, Colombo 5, for taking the lead in this Dhammadāna movement with his initial donation and for his devoted services as the `Settler' of the Trust.

\clearpage

And last but not least --

To, Mr.~Hideo Chihashi, Director, Green Hill Meditation Institute, Tokyo, Japan, and to his group of relatives, friends and pupils for their munificence in sponsoring the publication of the first volume of \emph{Nibbāna -- The Mind Stilled}.

\begin{quote}
\emph{Nibbānaṁ paramaṁ sukhaṁ}

Nibbāna is the supreme bliss
\end{quote}

-- Bhikkhu Kaṭukurunde Ñāṇananda

Pothgulgala Aranyaya\\
`Pahankanuwa'\\
Kandegedara\\
Devalegama\\
Sri Lanka

August 2002 (B.E. 2546)
